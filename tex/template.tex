\documentclass[a5paper, 12pt, twoside, openany]{ltjtbook} %ltjbook% ltjtbook %declare oneside / or twoside with "openany" will prevent blank pages
\usepackage{luatexja}
\usepackage{luatexja-ruby}
%\usepackage[abspath]{currfile}
\usepackage{luatexja-fontspec}
\usepackage[a5paper, scale = 0.75, bottom = 0.8in, top = 1.2in, centering]{geometry}
\usepackage{moresize}
\usepackage{tocloft}
\usepackage[hidelinks, plainpages = false, unicode = true]{hyperref}
\usepackage{pdfpages}
\usepackage{bookmark}
\usepackage{endnotes}
\usepackage{zhnumber}
\usepackage{xcolor}
\usepackage{float}
\usepackage{lipsum}

%\setmainfont{Arno Pro}
%\setsansfont{Myriad Pro}

\setmainjfont[BoldFont = SourceHanSerifSC-Bold.otf]{STZHONGS.otf}
\setsansjfont[BoldFont = SourceHanSansSC-Bold.otf]{STZHONGS.otf}

%\newjfontfamily{\titlejfont}{A-OTF-OutaiKaiStd-Light.otf}
\newjfontfamily{\kaishu}{AdobeKaitiStd-Regular.otf}
\newjfontfamily{\markfont}{STZHONGS.otf}

\setlength\columnsep{3\Cwd}
\setlength\parindent{2em}
\setlength{\parskip}{0.5em} 

\def\enoteheading{\subsection*{\notesname}%
  \mbox{}\par\vskip -2.5\baselineskip\noindent%
  %\rule{.5\textwidth}{0.4pt}%
  \rule{3em}{0.5pt}%
  \par\vskip 0.0\baselineskip}
\makeatother

%use customized quotation: 
%\let\quotationold\quotation
%\let\endquotationold\endquotation
%\renewenvironment{quotation}{\kaishu\quotationold}{\endquotationold}

\newenvironment{myquoteold}[1][\kaishu]{\par%
	\vspace{0.5em}#1%
	\rightskip=1em%
	\setlength{\parskip}{0em}%
	\everypar{\noindent\setlength{\hangindent}{4em}\hangafter=0\raggedright}%
	}
{\par%
	%\vspace{0.5em}
}

\newenvironment{myquote0}{%
\vspace{0.5em}\setlength{\parskip}{0em}\everypar{\noindent\setlength{\hangindent}{0em}\hangafter=0}\kaishu\raggedright}
{\par%
%\vspace{0.5em}
}

\newenvironment{declareqianyan}{%
  \vspace{-4em}%
  \flushright\kaishu%
}{\vspace{4em}\relax\par}

\definecolor{mydarkgray}{RGB}{96, 96, 96}

%https://tex.stackexchange.com/questions/300340/topsep-itemsep-partopsep-and-parsep-what-does-each-of-them-mean-and-wha
%https://tex.stackexchange.com/questions/4959/how-do-i-indent-all-lines-of-a-paragraph-so-that-it-looks-like-a-blockquote/4970#4970
\newenvironment{myquote}[1][\kaishu]{\par%
	\setlength{\parskip}{0em}%
	\setlength{\parsep}{0em}%
	%\setlength{\topsep}{0em}%
	#1%
   	\list{}{%
		\setlength{\topsep}{0.5em}%
   		\noindent%
   	 	\setlength{\parskip}{0em}%
		\setlength{\parsep}{0em}%
		\setlength\itemsep{0em}%
   	 	\leftmargin4em%
   	 	\rightmargin1em%
	}%
   	\raggedright\item\relax
}{\endlist\selectfont\vspace{-0.5em}\par}

\newenvironment{myquote2}[1][\markfont]{\par%
	\setlength{\parskip}{0em}%
	\setlength{\parsep}{0em}%
	%\setlength{\topsep}{0em}%
	#1%
   	\list{}{%
		\setlength{\topsep}{0.5em}%
   		\noindent%
   	 	\setlength{\parskip}{0em}%
		\setlength{\parsep}{0.5em}%
		\setlength\itemsep{0em}%
   	 	\leftmargin4em%
   	 	\rightmargin1em%
	}%
   	\raggedright\item\relax
}{\endlist\selectfont\vspace{-0.5em}\par}

\renewcommand{\contentsname}{目錄}
\renewcommand{\cftchapfont}{}
\renewcommand{\cftchappagefont}{}
\renewcommand{\notesname}{註  釋}

\newcommand{\innerzhushi}{\kaishu}{\small}{\selectfont}
\newcommand{\marktext}{\markfont}{\small}{\selectfont}
\newcommand{\rightquadmargin}{\quad\ }

\newcommand{\titlename}{《新刻繡像批評金瓶梅》}

%--------------------------------------------------------------------------
%--------------------------------------------------------------------------
\makeatletter	%begin definition with @

\def\ps@headings{\let\ps@jpl@in\ps@headnombre
  \let\@oddfoot\@empty\let\@evenfoot\@empty
  \def\@evenhead{\footnotesize\thepage\hfil\leftmark}%
  \def\@oddhead{\footnotesize{\rightmark}\hfil\thepage}%
  \let\@mkboth\markboth
\def\chaptermark##1{\markboth{%
   \ifnum \c@secnumdepth >\m@ne
     \if@mainmatter
       \@chapapp\thechapter\@chappos\hskip1\zw
     \fi
   \fi
   ##1}{}}%
\def\sectionmark##1{\markright{%
   \ifnum \c@secnumdepth >\z@ \thesection.\hskip1\zw\fi
   ##1}}%
}
  
%copied from file "ltjsbook.cls" which can customize chapter font size:
\newcommand{\headfont}{\gtfamily\sffamily}	%without this line, "makeschapterhead" can't compile

\def\@makeschapterhead#1{%
  %\vspace*{2\Cvs}  %original value
  \vskip 2\Cvs
  {\parindent \z@ \raggedright
    \normalfont
    \interlinepenalty\@M
    \LARGE \headfont #1\par\nobreak  %LARGE: 25pt / huge: 28pt / Huge: 33pt / 
    \vskip 2\Cvs}
    \thispagestyle{empty}      %no page number shows on head or foot
    \setcounter{endnote}{0}    %reset endnote counter
    \setcounter{footnote}{0}}  %reset footnote counter 
    
\let\old@endpart\@endpart
\renewcommand\@endpart{{\hskip5em\vskip1em}{\insertflower}\old@endpart}

\newcommand{\tableofcontentslocal}{%
  \if@twocolumn\@restonecoltrue\onecolumn
  \else\@restonecolfalse\fi
  \chapter*{\contentsname
    \@mkboth{\titlename}{\contentsname}%
  }\@starttoc{toc}%
  \if@restonecol\twocolumn\fi
}

%https://github.com/rf-latex/endnotes/blob/master/endnotes.sty
\renewcommand{\theendnote}{(\zhdig{endnote})}

%\def\@makeenmark{\hbox{\@textsuperscript{\normalfont\@theenmark}}}
\def\@makeenmark{\hbox{\kaishu\scriptsize\color{gray}{\@theenmark}}}

%this for the endnote mark shows at the end of chapter: 
%this for the endnote mark shows at the end of chapter: 
\def\makeenmarktwo{\hbox{\kaishu\small{\@theenmark}\hspace{0.5em}}}

\def\enoteformat{\rightskip\z@ \leftskip\z@ \parindent\z@
  \leavevmode{\makeenmarktwo}\leftskip 1em\rightskip 1em\setlength{\hangindent}{2.5em}\kaishu\normalsize}
    	 
%\renewcommand\cftchapafterpnum{\vskip0pt}
\renewcommand\cftbeforepartskip{4pt}
\renewcommand\cftbeforechapskip{1pt}

\makeatother	%end definition with @
%--------------------------------------------------------------------------
%--------------------------------------------------------------------------

%http://latexcolor.com/
\definecolor{ballblue}{rgb}{0.13, 0.67, 0.8}
\definecolor{cobalt}{rgb}{0.0, 0.28, 0.67}
\definecolor{darkblue}{rgb}{0.0, 0.0, 0.55}

\newcommand{\colormeipi}{\color{ballblue}}
\newcommand{\colorjiapi}{\color{cobalt}}
\newcommand{\colorpangpi}{\color{darkblue}}

\newcommand{\meipi}{\kaishu\footnotesize\colormeipi}
\newcommand{\jiapi}{\kaishu\footnotesize\colorjiapi}
\newcommand{\pangpi}{\kaishu\footnotesize\colorpangpi}

\newcommand{\insertflower}{\vspace{-\baselineskip}%    
    \begin{figure}[H]%
    \vspace*{-0.5em}%
    \hspace*{1cm}%
    \scalebox{1}[-1]{\includegraphics[keepaspectratio=true,scale=0.7, angle=90]{flower.jpg}}%
    %\includegraphics[keepaspectratio=true,scale=0.7, angle = -90]{flower.jpg}%
    \end{figure}%
}
  
\newcommand{\insertauthorlogo}{\vspace{-3.2\baselineskip}%    
    \begin{figure}[H]%
    \hspace*{7.2cm}%
    \includegraphics[keepaspectratio=true,scale=0.05, angle=90]{qrcode.png}%
    \end{figure}%
}
  
\title{\Huge{\titlename}}
%\author{蘭陵笑笑生原著\\\kaishu{校對排版:細雨如煙}\\\kaishu{簡繁校對:老而彌堅劉仁軌}}
\author{會校本\quad{重訂版}\\\kaishu{蘭陵笑笑生\quad{著}}}
%\author{\kaishu{會校本{\quad{重訂版}}\\蘭陵笑笑生 著}}
\date{}

\newcommand\mydot[1]{\color{lightgray}\scalebox{#1}{.}}
\renewcommand\cftdot{\mydot{2}}
\renewcommand\cftdotsep{2}
%\renewcommand\cftpartnumwidth{6em}
\renewcommand{\cftchapleader}{\hspace{0.2em}{\cftdotfill{\cftdotsep}}}
%\renewcommand{\cftpnumalign}{l}

%control the right margin of toc-dots:
\makeatletter
\def\@pnumwidth{3.2em}
\makeatother

\begin{document}
\maketitle

\pagestyle{headings}
\pagenumbering{zhdig}

\pdfbookmark{書名頁}{}
\pdfbookmark{封面}{}
\includepdf[pages={1},fitpaper=false]{cover.pdf}

\cleardoublepage
\tableofcontentslocal
\pdfbookmark{\contentsname}{toc}
%prevent the 1st page of TOC generate foot page number:
\addtocontents{toc}{\protect\thispagestyle{empty}}
\mainmatter

\pagenumbering{zhdig}
\input{sub/split_002}
\chapter*{金瓶梅序}
\addcontentsline{toc}{chapter}{東吴弄珠客《金瓶梅序》}
\markboth{\titlename}{金瓶梅序}


《金瓶梅》,穢書也。袁石公亟稱之,亦自寄其牢騷耳,非有取於《金瓶梅》也。然作者亦自有意,蓋為世戒,非為世勸也。如諸婦多矣,而獨以潘金蓮、李瓶兒、春梅命名者,亦楚《檮杌》之意也。蓋金蓮以姦死,瓶兒以孽死,春梅以淫死,較諸婦為更慘耳。借西門慶以描畫世之大淨,應伯爵以描繪世之小丑,諸淫婦以描畫世之醜婆、淨婆,令人讀之汗下。蓋為世戒,非為世勸也。余嘗曰:「讀《金瓶梅》而生憐憫心者,菩薩也;生畏懼心者,君子也;生歡喜心者,小人也;生傚法心者,乃禽獸耳。」余友人褚孝秀偕一少年同赴歌舞之筵,衍至霸王夜宴,少年垂涎曰:「男兒何可不如此!」褚孝秀曰:「也只為這烏江設此一着耳。」同座聞之,嘆為有道之言。若有人識得此意,方許他讀《金瓶梅》也。不然,石公幾為導淫宣慾之尤矣。奉勸世人,勿為西門之後車可也。

\begin{quotation}
\raggedleft{東吳弄珠客題\rightquadmargin}
\end{quotation}


\part*{附錄}
\addcontentsline{toc}{part}{附錄}

\chapter*{金瓶梅詞話序}
\addcontentsline{toc}{chapter}{欣欣子《金瓶梅詞話序》}
\markboth{\titlename}{金瓶梅詞話序}


竊謂蘭陵笑笑生作《金瓶梅傳》,寄意於時俗,蓋有謂也。人有七情,憂鬱為甚。上智之士,與化俱生,霧散而氷裂,是故不必言矣。次焉者,亦知以理自排,不使為累。惟下焉者,既不能了於心胸,又無詩書道腴可以撥遣,然則不致於坐病者幾希!吾友笑笑生為此,爰罄平日所蘊者,著斯傳,凡一百囘。其中語句新奇,膾炙人口。無非明人倫、戒淫奔、分淑慝、化善惡,知盛衰消長之機,取報應輪迴之事,如在目前;始終如脈絡貫通,如萬絲迎風而不亂也。使觀者庶幾可以一哂而忘憂也。其中未免語涉俚俗,氣含脂粉。余則曰:不然。《關雎》之作,楽而不淫,哀而不傷。富與貴,人之所慕也,鮮有不至於淫者;哀與怨,人之所惡也,鮮有不至於傷者。吾嘗觀前代騷人,如盧景暉之《剪燈新話》、元微之之《鶯鶯傳》、趙君弼之《效顰集》、羅貫中之《水滸傳》、丘瓊山之《鍾情麗集》、盧梅湖之《懷春雅集》、周靜軒之《秉燭清談》,其後《如意傳》、《于湖記》,其間語句文確,讀者往往不能暢懷,不至終篇而掩棄之矣。此一傳者,雖市井之常談,閨房之碎語,使三尺童子聞之,如飫天漿而拔鯨牙,洞洞然易曉。雖不比古之集理趣,文墨綽有可觀。其它關繋世道風化,懲戒善惡,滌慮洗心,不無小補。譬如房中之事,人皆好之,人非堯舜聖賢,鮮不為所耽。富貴善良,人皆惡之,是以搖動人心,蕩其素志。觀其高堂大廈,雲窗霧閣,何深沉也;金屏綉褥,何羙麗也;鬢雲斜軃,春酥滿胸,何嬋娟也;雄鳳雌凰迭舞,何慇懃也;錦衣玉食,何侈費也;佳人才子,嘲風咏月,何綢繆也;鷄舌含香,唾圓流玉,何溢度也;一雙玉腕綰復綰,兩隻金蓮顛倒顛,何猛浪也。旣其樂矣,然楽極必悲生:如離别之機將興,憔悴之容必見者,所不能免也;折梅逢驛使,尺素寄魚書,所不能無也;患難迫切之中,顛沛流離之頃,所不能脫也;陷命於刀劔,所不能逃也;陽有王灋,幽有鬼神,所不能逭也。至於淫人妻子,妻子淫人,祸因惡積,福緣善慶,種種皆不出循環之機。故天有春夏秋冬,人有悲歡離合,莫怪其然也。合天時者,遠則子孫悠久,近則安享終身;逆天時者,身名罹喪,祸不旋踵。人之䖏世,雖不出乎世運代謝,然不經凶祸,不蒙耻辱者,亦幸矣。吾故曰:笑笑生作此傳者,蓋有所謂也。

\begin{quotation}
\raggedleft{欣欣子書於明賢里之軒\rightquadmargin}
\end{quotation}


\chapter*{跋}
\addcontentsline{toc}{chapter}{廿公《跋》}
\markboth{\titlename}{跋}


《金瓶梅》,傳為世廟時一鉅公寓言,蓋有所刺也。然曲盡人間醜態,其亦先師不删鄭衛之旨乎?中間䖏處埋伏因果,作者亦大慈悲矣。今後流行此書,功德無量矣。不知者竟目為淫書,不惟不知作者之旨,併亦寃却流行者之心矣!特為白之。

\begin{quotation}
\raggedleft{廿公書\rightquadmargin}
\end{quotation}


\chapter*{謝肇淛《金瓶梅》跋}
\addcontentsline{toc}{chapter}{謝肇淛《金瓶梅》跋}
\markboth{\titlename}{謝肇淛《金瓶梅》跋}

《金瓶梅》一書,不著作者名代。相傳永陵中有金吾戚裡,憑怙奢汰,淫縱無度,而其門客病之,採摭日逐行事,彙以成編,而托之西門慶也。書凡數百萬言,為卷二十,始末不過數年事耳。

其中朝野之政務,官私之晉接,閨闥之蝶語,市裡之猥談,與夫勢交利合之態,心輸背笑之局,桑中濮上之期,尊罍枕蓆之語,駔驗之機械意智,粉黛之自媚爭妍,狎客之從臾逢迎,奴怡之嵇唇淬語,窮極境像,駥意快心。譬之范工摶泥,妍媸老少,人鬼萬殊,不徒肖其貌,且並其神傳之。信稗官之上乘,爐錘之妙手也。其不及《水滸傳》者,以其猥瑣淫蝶,無關名理。而或以為過之者,彼猶機軸相放,而此之面目各別,聚有自來,散有自去,讀者竟想不到,唯恐易盡。此豈可與褒儒俗士見哉。此書向無鏤版,抄寫流傳,參差散失。唯弇州家藏者最為完好。余於袁中郎得其十三,於丘諸城得其十五,稍為釐正,而闕所未備,以俟他日。有嗤余誨淫者,余不敢知。然溱洧之音,聖人不刪,則亦中郎帳中必不可無之物也。仿此者有《玉嬌麗》,然而乖彞敗度,君子無取焉。
  
\begin{quotation}\begin{flushright}謝肇淛。\end{flushright}\end{quotation}


\chapter*{新鐫金瓶梅詞話}
\addcontentsline{toc}{chapter}{新刻金瓶梅詞話·詞曰·四貪詞}
\markboth{\titlename}{新刻金瓶梅詞話·詞曰·四貪詞}


\section*{詞曰}

\begin{myquote0}
閬苑瀛洲,金谷陵樓,筭不如茅舍清幽。野花綉地,莫也風流。也宜春,也宜夏,也宜秋。

酒熟堪あ,客至須留,更無榮無辱無憂。退閒一步,着甚來由。但倦時眠,渴時飲,醉時謳。

短短横牆,矮矮疎窻,忔い兒小小池塘。高低疊峯,綠水邊傍。也有些風,有些月,有些凉。

日用家常,竹几藤床,靠眼前水色山光。客來無酒,清話何妨。但細烹茶,熱烘盞,淺澆湯。

水竹之居,吾愛吾盧,石磷磷床砌堦除。軒窻隨意,小巧規模。却也清幽,也瀟灑,也寬舒。

懶散無拘,此等何如:倚闌干臨水觀魚。風花雪月,贏得功夫,好炷心香,說些話,讀些書。

淨掃塵埃,惜取蒼苔,任門前紅葉鋪堦。也堪圖畫,還也奇哉。有數株松,數竿竹,數枝梅。

花木栽培,取次教開,明朝事天自安排。知他富貴幾時來。且優遊,且隨分,且開懷。
\end{myquote0}

\newpage\section*{四貪詞}
%\addcontentsline{toc}{chapter}{四貪詞}
%\markboth{\titlename}{四貪詞}

\hspace*{1em}酒

\begin{myquote0}
酒損精神破喪家,語言無狀鬧喧嘩。疎親慢友多由你,背義忘恩盡是他。

切須戒,飲流霞。若能依此寳無差。失却萬事皆因此,今後逄賔只待茶。
\end{myquote0}

\hspace*{1em}色

\begin{myquote0}
休愛綠髩羙朱顔,少貪紅粉翠花鈿。損身害命多嬌態,傾國傾城色更鮮。

莫戀此,養丹田。人能寡慾壽長年。従今罷却閒風月,紙帳梅花獨自眠。
\end{myquote0}

\hspace*{1em}財

\begin{myquote0}
錢帛金珠籠内收,若非公道少貪求。親朋道義因財失,父子懷情為利休。

急縮手,且抽頭。免使身心晝夜愁。兒孫自有兒孫福,莫與兒孫作遠憂。
\end{myquote0}

\hspace*{1em}氣

\begin{myquote0}
莫使強梁逞技能,揮拳捰袖弄精神。一時怒發無明火,到後憂煎祸及身。

莫太過,免災迍。勸君凡事放寬情。合撒手時須撒手,得饒人處且饒人。
\end{myquote0}



\input{sub/chapter_001}
\input{sub/chapter_002}
\input{sub/chapter_003}
\input{sub/chapter_004}
\input{sub/chapter_005}
\input{sub/chapter_006}
\input{sub/chapter_007}
\input{sub/chapter_008}
\input{sub/chapter_009}
\input{sub/chapter_010}
\input{sub/chapter_011}
\input{sub/chapter_012}
\input{sub/chapter_013}
\input{sub/chapter_014}
\input{sub/chapter_015}
\input{sub/chapter_016}
\input{sub/chapter_017}
\input{sub/chapter_018}
\input{sub/chapter_019}
\input{sub/chapter_020}
\input{sub/chapter_021}
\input{sub/chapter_022}
\input{sub/chapter_023}
\input{sub/chapter_024}
\input{sub/chapter_025}
\input{sub/chapter_026}
\input{sub/chapter_027}
\input{sub/chapter_028}
\input{sub/chapter_029}
\input{sub/chapter_030}
\input{sub/chapter_031}
\input{sub/chapter_032}
\input{sub/chapter_033}
\input{sub/chapter_034}
\input{sub/chapter_035}
\input{sub/chapter_036}
\input{sub/chapter_037}
\input{sub/chapter_038}
\input{sub/chapter_039}
\input{sub/chapter_040}
\input{sub/chapter_041}
\input{sub/chapter_042}
\input{sub/chapter_043}
\input{sub/chapter_044}
\input{sub/chapter_045}
\input{sub/chapter_046}
\input{sub/chapter_047}
\input{sub/chapter_048}
\input{sub/chapter_049}
\input{sub/chapter_050}
\input{sub/chapter_051}
\input{sub/chapter_052}
\input{sub/chapter_053}
\input{sub/chapter_054}
\input{sub/chapter_055}
\input{sub/chapter_056}
\input{sub/chapter_057}
\input{sub/chapter_058}
\input{sub/chapter_059}
\input{sub/chapter_060}
\input{sub/chapter_061}
\input{sub/chapter_062}
\input{sub/chapter_063}
\input{sub/chapter_064}
\input{sub/chapter_065}
\input{sub/chapter_066}
\input{sub/chapter_067}
\input{sub/chapter_068}
\input{sub/chapter_069}
\input{sub/chapter_070}
\input{sub/chapter_071}
\input{sub/chapter_072}
\input{sub/chapter_073}
\input{sub/chapter_074}
\input{sub/chapter_075}
\input{sub/chapter_076}
\input{sub/chapter_077}
\input{sub/chapter_078}
\input{sub/chapter_079}
\input{sub/chapter_080}
\input{sub/chapter_081}
\input{sub/chapter_082}
\input{sub/chapter_083}
\input{sub/chapter_084}
\input{sub/chapter_085}
\input{sub/chapter_086}
\input{sub/chapter_087}
\input{sub/chapter_088}
\input{sub/chapter_089}
\input{sub/chapter_090}
\input{sub/chapter_091}
\input{sub/chapter_092}
\input{sub/chapter_093}
\input{sub/chapter_094}
\input{sub/chapter_095}
\input{sub/chapter_096}
\input{sub/chapter_097}
\input{sub/chapter_098}
\input{sub/chapter_099}
\input{sub/chapter_100}
\chapter*{製作說明}
\addcontentsline{toc}{chapter}{製作說明}
\markboth{\titlename}{製作說明}


◎本書是根據齊魯書社1989年齊煙、王汝梅校點出版的《新刻繡像批評金瓶梅》足本作為校對底本。此版本是《金瓶梅》在大陸出版唯一的未刪減版本,出版後,一時洛陽紙貴。目前因多種原因,導致實體版已炒至上萬元,而且大多數正版由個人收藏,有價無市,電子版更是稀缺。本人以夢梅齋製作的PDF為基礎,對照秦修容1998年中華書局出版的《會評會校金瓶梅》版本,修正了缺字若干,加入胡也佛繪《金瓶梅祕戲圖》、戴敦邦繪《金瓶梅人物譜》61張、《金瓶梅全書圖》40張、《日本內閣文庫藏本·新鐫繡像批評金瓶梅·插圖》200張。
後老而彌堅劉仁軌以松影拂雲製作的mobi文件為基礎,仿照原刊本格式,重新進行了豎行排版,將原電子文本的簡體重新繁化,採用康熙字典字體,以求儘量在電子設備再現原刊本的樣貌。原刊本各卷時常簡繁互用(如:礼—禮、响—響),一字多形(如:個—箇、由—繇),幷包含大量異體字和生僻字。種種非標準情況都儘量在本書內予以還原。其本人對照掃描刊本,進行了校對修正補充。本矢量版PDF以而彌堅劉仁軌製作的豎排電子檔為基礎,僅做少量修改。採用{\LaTeX}製版。

◎繪者:

戴敦邦(1938年-),中國著名國畫家,擅人物,工寫兼長,多以古典題材及古裝人物入畫,所作氣魄宏大,筆墨雄健豪放,形象生動傳神,畫風雅俗共賞,主要作品《水滸人物一百零八圖》、《戴敦邦水滸人物譜》、《紅樓夢人物百圖》、《戴敦邦新繪紅樓夢》、《戴敦邦古典文學名著畫集》等;

胡也佛(1908年-1980年)本名國華,工書畫,學宗仇十洲,擅作仕女,間寫宋元一路山水,雋逸過人。

◎版本:新刻繡像批評《金瓶梅》

◎出版社:齊魯書社

◎出版時間:1989年

◎模板原始程式碼設計:林某人/林姓匹夫

◎製作說明頁原始程式碼設計:雲軒閣閣主

◎拓展字集提供並指導:萌哦萌

◎插圖《人物譜》攝影者:vc2270

◎校對排版:細雨如煙

◎特別申明:此書為金學交流使用,切勿用於商業。版權歸原作者所有。

\begin{quotation}
\raggedleft\small\kaishu\color{gray}{庚子歲正月製書於米國山景城}
\end{quotation}

{\insertauthorlogo}
\includepdf[pages={1-},fitpaper=false]{additional_pictures.pdf}



\end{document}
