\chapter*{《金瓶梅》校後記(代序)}
\addcontentsline{toc}{chapter}{《金瓶梅》校後記(代序)}
\markboth{\titlename}{《金瓶梅》校後記(代序)}

\begin{declareqianyan}
校者\qquad\ 
\end{declareqianyan}	

金瓶梅之於古典文學,其創造了數個第一,這是明白無誤了的。這內中至少就有兩點──第一部文人獨立創作的長篇小說;第一部以中下層民眾日常生活為描寫對象的長篇小說。其他各種,若一一舉之,不但有附會之嫌,亦難免貼金之議。

我初開始接觸金瓶梅,算比較晚了。蓋彼時受主客觀各種因素,我還只知四大名著而不知四大奇書的名號。囿於大山之深處,游於題海之堆中,雖孜孜於「課外讀物」,而所讀不過三國西遊而已,即便算上泛泛而讀之者,也不過水滸,紅樓,今古奇觀等等,其中於三國西遊二書,著力尤多。自數遍乃至十數遍之多。乃吾深愛二書至此耶?曰:篋中書甚少,唯此二好讀。

我究竟何時第一次耳聞金瓶梅的大名,今已不可確記,但大致可以推想,池魚游於淵,始知池之小,山人出深山,方曉天地寬,古人謂,讀萬卷書,行萬里路,信然。而余讀書,離萬卷還差得遠。行路萬里,更是遙遙不及,故一旦徜徉書海。還來不及生出「吾生也有涯,而知也無涯」的感受來,先「躲進書館成一統,哪管外面夏與冬」。或者就在這個時候,無意發現了金瓶梅,也未可知,不過,我更心疑,是哪位博學的仁兄賢弟,帶我入的坑,都說男生寢室兩個永恆的話題:性與遊戲,孔孟之書,也道「食色性也」,蓋男生於食字上,並不十分要緊──這又是女生永恆的話題之一了。而於色字上,便大有「為伊消得人憔悴,衣帶漸寬終不悔」(此處原序是「衣帶漸寬終不悔,為伊消得人憔悴」)的意境。

但當是時,只覺金瓶梅滿紙瑣屑,無非吃吃喝喝,用張子竹坡的話,叫做「計賬簿」,即今所謂流水賬是也。不忍卒讀,遂一擱數年。

但受著不讀紅樓先讀蔣勳說紅樓(此書褒貶不一,但以余論之,不妨作為還不太看得進去紅樓夢原著而又想窺其一斑的同志的入門讀物)的啟發,余守著I207那幾架。一揀就是四五本,再看已然兩更天。因此讀了不少金瓶梅研究和評論的書,其中,對岸侯文詠先生的《沒有神的所在》一書,「害」余甚苦,我之掉入OCR制書大坑。便和他有關,想必有OCR經歷的同志,必能理解,非深愛一書,不去OCR也,有著這個基礎,當有人欲製作金瓶梅時,我便承攬下校對的活計,下面重點說說與之相關的事情(可知上面洋洋灑灑,下筆千言,即不儘是廢話,也大半是廢話也,此吾固有自知之明也)。

金瓶梅一書,分為兩個系統,即金瓶梅詞話,和金瓶梅。又已以出版年代劃分,前者叫萬曆本,後者叫崇禎本。或者從特點上分,因後者刊刻時每回附有精美插圖兩幅,故稱繡像本,而前者內中有大量的詞話(詞曲說話),故稱詞話本,兩書若從宏觀上分析,大同小異,但著眼細處,不一樣的地方卻非常多,最明顯的就是兩本書的第一回是十分不同的,光回目即可知其大概,一個叫「景陽崗武松打虎,潘金蓮嫌夫賣風月」,一個叫「西門慶熱結十弟兄,武二郎冷遇親哥嫂」,這也作為辨別詞話本與繡像本,最簡單粗暴的法子,屢屢教授給初讀者。

我無意,也無那學識能力,來臧否二者優劣,更無資格──因為我只讀過繡像本,而未全文閱讀詞話本,但個人有一點小小的意見,初讀者還是先讀繡像本,因為詞話本,既名「詞話」,內中便有大量詞話,這一點前文已敘及,而內中於情節的推動,人物的塑造,並不是不可或缺的,反而使文章的流暢性,打了折扣,猶如西遊中,孫悟空豬八戒沙和尚三人,每遇妖怪不識之,便開始長篇大論,將自己一生甚至自己的污點,如孫悟空之官封弼馬,豬八戒之酒後調嫦,沙和尚之失手打盞,一絲不藏,盡數說與妖怪聽,那妖怪卻也耐心,面對動輒數百千言的嘮叨,既不打斷,也不插嘴,有時甚或自己也來上一段,兩人陣前大打「嘴炮」。樂此不疲。當然,這個比喻或許不太恰當,也不妨事,求同存異嘛。

另外,公認繡像本情節更加嚴謹,語言更加易讀,繡像本的改寫者做了大量的方言詞改通語的工作(雖然有因不理解詞義而改成一個不符合作者本意的詞,但瑕不掩瑜),內容更加連貫。又有一點,自繡像本通行後,詞話本名漸不顯,慘遭雪藏,而後世的張子竹坡批評,更以繡像本為底本,而竹坡的第一奇書本,刊印通行後,更後來居上,連繡像本都一發退避三舍了,所以後人批評金瓶梅,都以繡像本為底,而詞話本,直到民國二十一年才在山西介休被發現。距離出版之時,已數百年矣。

所以我的校對,也是以繡像本為底,外加張子竹坡的批評,計劃倘待此完成後,再加入繡像本原評及文龍的評語,蓋張子竹坡萬般皆好,有兩點為我所不甚喜,一是利用同音字,進行大量的過度解讀,如潘之為蓮,孟之為杏,月娘為月,陳敬濟為陳莖芰,薛嫂為雪等,不厭其煩,未免給人主觀臆測之感,二是狠鬥凶批月娘,斥之為「奸險好人」,並且不放過任何一個機會,來「黑」一把月娘,一味苛責月娘,不能敦正西門家風,勸導西門向善,豈不聞先人有言,「使堯在上,雖十桀不能亂,使桀在上,雖十堯不能治。」如此怪法,不正與殷亡怪妲己,周亡怪褒姒,安史之亂起,全因楊貴妃相埒嗎?吾不解張子高才,卻囿於此陳腐舊說而不悟,何也!反觀繡像評語,對月娘貪財虛偽處,該批評即批評,對其能守節不移,正色抗拒撩撥強暴,對西門多有襄助勸諫,對下人多有保護善待,亦不吝讚美之辭,而文龍之批評又大有與張子唱反調之趣味,且其批語亦多亮點,魏徵引古人云「偏聽則暗,兼聽則明」。誠萬世眞理,使三家批語點綴文中,猶花之添錦上也。

閒話休敘,說說校對幾點:

一、所謂以繡像本為底本,即倘有字句幾個版本有不同的,即使此說可通,亦棄而不用,改成和繡像本一致,但是由於明刊繡像本。PDF甚是龐大,又沒目錄,所以我並非盯著他對校,而是讀校,讀到有不通順,不可解處,找到中華書局和天地書局排印本對照,若仍舊不解或兩者有衝突,則再對照明刊本,所以余不敢狂言百分百契合繡像本。

二、先期做簡體字本,所以在簡化總表裡有的字,一例簡化,沒有的絕不用「類推簡化」的辦法,生造出不倫不類的簡化字,一仍其原貌,但彼時所用的異體字俗體字不在此列,盡量不改動,如耽閣(有改成耽擱的)、卓兒(桌兒)、丁當(玎璫)等。

三、因為是讀校而非對校,所以肯定有漏網之魚,這點就有待於同樣對金瓶梅有興趣的同志來玉成了,我發這篇拙文之意也正在此。大約一人之力有限,而眾生智慧無窮。吾拋磚瓦,以待璋玉,以俟同好,是作斯文。

\begin{quotation}
\raggedleft{丙申十月庚戌。雲中。\rightquadmargin}
\end{quotation}
