\includepdf[pages={11,12},fitpaper=false]{tst.pdf}
\chapter*{第六囘 何九受賄瞞天 王婆幫閑遇雨}
\addcontentsline{toc}{chapter}{第六囘 何九受賄瞞天 王婆幫閑遇雨}
\markboth{{\titlename}卷之一}{第六囘 何九受賄瞞天 王婆幫閑遇雨}


詞曰:

\begin{myquote} 
別後誰知,珠分玉剖。忘海誓山盟天共久。偶戀着山雞,輒棄鸞儔。從此簫郎淚暗流。過秦樓,幾空囘首。縱新人勝舊,也應須一別,灑淚登舟。

\raggedleft{——右調《懶畫眉》\rightquadmargin}
\end{myquote} 

卻說西門慶去了。到天大明,王婆拏銀子買了棺材冥器,又買些香燭紙錢之類,歸來就于武大靈前點起一盞隨身燈。隣舍街坊都來看望,那婦人虛掩着粉臉假哭。衆街坊問道:「大郎得何病患便死了?」那婆娘答道:「因害心疼,不想一日日越重了,看看不能勾好。不幸昨夜三更鼓死了,好是苦也!」又哽哽咽咽假哭起來。衆隣舍明知道此人死的不明,不好只顧問他。衆人盡勸道:「死是死了,活的自要安穩過。娘子省煩惱,天氣暄熱。」那婦人只得假意兒謝了,衆人各自散去。王婆擡了棺材來,去請仵作團頭何九。但是入殮用的都買了,並家裡一應物件也都買了。就於報恩寺叫了兩個禪和子,晚夕伴靈拜懺。不多時,何九先撥了幾個火家整頓。且說何九到巳牌時分,慢慢的走來,到紫石街巷口,迎見西門慶。叫道:「老九何徃?」何九答道:「小人只去前面,殮這賣炊餅的武大郎屍首。」西門慶道:「且停一步說話。」何九跟着西門慶,來到轉角頭一個小酒店裡坐下。在閣兒內,西門慶道:「老九請上坐。」何九道:「小人是何等人,敢對大官人一處坐的!」西門慶道:「老九何故見外?且請坐。」二人讓了一囘,坐下。西門慶分付酒保:「取瓶好酒來。」酒保一面鋪下菜蔬菓品按酒之類,一面燙上酒來。何九心中疑忌,想道:「西門慶自來不曾和我吃酒,今日這盃酒必有蹊蹺。」兩個飲勾多時,只見西門慶向袖子裡摸出一錠雪花銀子,放在面前說道:「老九休嫌輕微,明日另有酬謝。」何九叉手道:「小人無半點效力之處,如何敢受大官人見賜銀兩!若是大官人有使令,小人也不敢辭。」西門慶道:「老九休要見外,請收過了。」何九道:「大官人便說不妨。」西門慶道:「別無甚事。少刻他家自有些辛苦錢。只是如今殮武大的屍首,凡百事周全,一床錦被遮蓋則個。」何九道:「我道何事!這些小事,有甚打緊,如何敢受大官人銀兩?」西門慶道:「你若不受時,便是推卻。」何九自來懼西門慶是個把持官府的人,只得收了銀子。又吃了幾盃酒,西門慶呼酒保來:「記了帳目,明日來我鋪子內支錢。」兩個下樓,一面出了店門。臨行,西門慶道:「老九是必記心,不可泄漏。改日另有補報。」{\pangpi{伏後張本。}}分付罷,一直去了。何九接了銀子,自忖道:「其中緣故那卻是不須提起的了。只是這銀子,恐怕武二來家有說話,留着倒是個見證。」一面又忖道:「這兩日倒要些銀子攪纏,且落得用了,到其間再做理會便了。」於是一直到武大門首。只見那幾個火家正在門首伺候。王婆也等的心裡火發。何九一到,便問火家:「這武大是甚病死了?」火家道:「他家說害心疼病死了。」何九入門,揭起簾子進來。王婆接着道:「久等多時了,陰陽也來了半日,老九如何這咱纔來?」何九道:「便是有些小事絆住了脚,來遲了一步。」只見那婦人穿着一件素淡衣裳,白布鬏髻,從裡面假哭出來。何九道:「娘子省煩惱,大郎已是歸天去了。」那婦人虛掩着淚眼道:「說不得的苦!我夫心疼病症,幾個日子便把命丟了。撇得奴好苦!」這何九一面上上下下看了婆娘的模樣,心裡暗道:「我從來只聽得人說武大娘子,不曾認得他。原來武大郎討得這個老婆在屋裡。西門慶這十兩銀子使着了!」一面走向靈前,看武大屍首。陰陽宣念經畢,揭起千秋幡,扯開白絹,定睛看時,見武大指甲青,唇口紫,面皮黃,眼皆突出,就知是中惡。旁邊那兩個火家說道:「怎的臉也紫了,口唇上有牙痕,口中出血?」何九道:「休得胡說!兩日天氣十分炎熱,如何不走動些!」一面七手八脚糊塗提殮了,裝入棺材內,兩下用長命釘釘了。王婆一力攛掇,拏出一弔錢來與何九,打發衆火家去了,就問:「幾時出去?」王婆道:「大娘子說只三日便出殯,城外燒化。」何九也便起身。那婦人當夜擺着酒請人,第二日請四個僧念經。第三日早五更,衆火家都來扛擡棺材,也有幾個隣舍街坊,弔孝相送。那婦人帶上孝,坐了一乘轎子,一路上口內假哭「養家人」。來到城外化人場上,便教舉火燒化棺材。不一時燒得乾乾淨淨,把骨殖撒在池子裡,原來齋堂管待,一應都是西門慶出錢整頓。那婦人歸到家中,樓上設個靈牌,上寫「亡夫武大郎之靈」。{\meipi{凡此都可不必。}}靈床子前點一盞琉璃燈,裡面貼些經幡錢紙、金銀錠之類。那日卻和西門慶做一處,打發王婆家去,二人在樓上任意縱橫取樂,不比先前在王婆家茶房裡,只是偷雞盜狗之歡。如今武大已死,家中無人,兩個肆意停眠整宿。初時西門慶恐隣舍瞧破,先到王婆那邊坐一囘,落後帶着小厮竟從婦人家後門而入。自此和婦人情沾意密,常時三五夜不歸去,把家中大小丟得七顛八倒,都不歡喜。正是:

\begin{myquote} 
色膽如天不自由,情深意密兩綢繆。\\貪歡不管生和死,溺愛誰將身體脩。\\只為恩深情鬱鬱,多因愛闊恨悠悠。\\要將吳越冤仇解,地老天荒難歇休。
\end{myquote} 

光陰迅速,日月如梭,西門慶刮剌那婦人將兩月有餘。一日,將近端陽佳節,但見:

\begin{myquote} 
綠楊嬝嬝垂絲碧,海榴點點胭脂赤。\\微微風動幔,颯颯涼侵扇。\\處處過端陽,家家共舉觴。
\end{myquote} 

卻說西門慶自岳廟上囘來,到王婆茶坊裡坐下。那婆子連忙點一盞茶來,便問:「大官人徃那裡來?怎的不過去看看大娘子?」西門慶道:「今日徃廟上走走。大節間記掛着,來看看六姐。」婆子道:「今日他娘潘媽媽在這裡,怕還未去哩。等我過去看看,囘大官人。」這婆子走過婦人後門看時,婦人正陪潘媽媽在房裡吃酒,見婆子來,連忙讓坐。婦人笑道:「乾娘來得正好,請陪俺娘且吃個進門盞兒,到明日養個好娃娃!」婆子笑道:「老身又沒有老伴兒,那裡得養出來?你年小少壯,正好養哩!」婦人道:「常言小花不結老花兒結。」婆子便看着潘媽媽嘈道:「你看你女兒,這等傷我,說我是老花子。到明日還用着我老花子哩!」{\meipi{王婆妙舌,應是曼倩一流人。}}說罷,潘媽道:「他從小是這等快嘴,乾娘休要和他一般見識。」王婆道:「你家這姐姐,端的百伶百俐,不枉了好個婦女。到明日,不知什麼有福的人受的他起。」潘媽媽道:「乾娘既是撮合山,全靠乾娘作成則個!」一面安下鍾筯,婦人斟酒在他面前。婆子一連陪了幾盃酒,吃得臉紅紅的,又怕西門慶在那邊等候,連忙丟了個眼色與婦人,告辭歸家。婦人知西門慶來了,因一力攛掇他娘起身去了。將房中收拾乾淨,燒些異香,從新把娘吃的殘饌撇去,另安排一席齊整酒餚預備。西門慶從後門過來,婦人接着到房中,道個萬福坐下。原來婦人自從武大死後,怎肯帶孝!把武大靈牌丟在一邊,{\meipi{帶孝不出真心,虛文原是可省。}}用一張白紙蒙着,羹飯也不揪採。每日只是濃粧艷抹,穿顏色衣服,打扮嬌樣。因見西門慶兩日不來,就罵:「負心的賊,如何撇閃了奴,又徃那家另續上心甜的了?把奴冷丟,不來揪採。」西門慶道:「這兩日有些事,今日徃廟上去,替你置了些首飾珠翠衣服之類。」那婦人滿心歡喜。西門慶一面喚過小厮玳安來,毡包內取出,一件件把與婦人。婦人方纔拜謝收了。小女迎兒,尋常被婦人打怕的,以此不瞞他,令他拏茶與西門慶吃。一面婦人安放桌兒,陪西門慶吃茶。西門慶道:「你不消費心,我已與了乾娘銀子買東西去了。大節間,正要和你坐一坐。」婦人道:「此是待俺娘的,奴存下這桌整菜兒。等到乾娘買來,且有一囘耽擱,咱且吃着。」婦人陪西門慶臉兒相貼,腿兒相壓,並肩一處飲酒。且說婆子提着個籃兒,走到街上打酒買肉。那時正值五月初旬天氣,大雨時行。只見紅日當天,忽被黑雲遮掩,俄而大雨傾盆。但見:

\begin{myquote} 
烏雲生四野,黑霧鎖長空。刷剌剌漫空障日飛來,一點點選得芭蕉聲碎。狂風相助,侵天老檜掀翻;霹靂交加,泰華嵩喬震動。洗炎驅暑,潤澤田苗。正是:江淮河濟添新水,翠竹紅榴洗濯清。
\end{myquote} 

那婆子正打了一瓶酒,買了一籃菜蔬菓品之類,在街上遇見這大雨,慌忙躲在人家房簷下,用手帕裹着頭,把衣服都淋濕了。等了一歇,那雨脚慢了些,大步雲飛來家。進入門來,把酒肉放在廚房下,走進房來,看婦人和西門慶飲酒,笑嘻嘻道:「大官人和大娘子好飲酒!你看把婆子身上衣服都淋濕了,到明日就教大官人賠我!」西門慶道:「你看老婆子,就是個賴精。」婆子道:「也不是賴精,大官人少不得賠我一疋大海青。」婦人道:「乾娘,你且飲盞熱酒兒。」那婆子陪着飲了三盃,說道:「老身徃廚下烘衣裳去也。」一面走到廚下,把衣服烘乾,那雞鵝嗄飯切割安排停當,用盤碟盛了菓品之類,都擺在房中,燙上酒來。西門慶與婦人重斟美酒,交盃疊股而飲。西門慶飲酒中間,看見婦人壁上掛着一面琵琶,便道:「久聞你善彈,今日好夕彈個曲兒我下酒。」婦人笑道:「奴自幼粗學一兩句,不十分好,你卻休要笑恥。」西門慶一面取下琵琶來,摟婦人在懷,看着他放在膝兒上,輕舒玉筍,款弄氷弦,慢慢彈着,低聲唱道:

\begin{myquote} 
冠兒不帶懶梳粧,髻挽青絲雲鬂光,金釵斜插在烏雲上。喚梅香,開籠箱,穿一套素縞衣裳,打扮的是西施模樣。出繡房,「梅香,你與我捲起簾兒,燒一炷兒夜香。」{\meipi{只一詞,便見金蓮自寓百種嫵媚。}}
\end{myquote} 

西門慶聽了,歡喜的沒入脚處,一手摟過婦人粉頸來,就親了個嘴,稱誇道:「誰知姐姐有這段兒聰明!就是小人在構欄三街兩巷相交唱的,也沒你這手好彈唱!」婦人笑道:「蒙官人擡舉,奴今日與你百依百順,是必過後休忘了奴家。」西門慶一面捧着他香腮,說道:「我怎肯忘了姐姐!」兩個殢雨尤雲,調笑玩耍。少頃,西門慶又脫下他一隻繡花鞋兒,擎在手內,放一小盃酒在內,吃鞋盃耍子。{\meipi{何福能消。}}婦人道:「奴家好小脚兒,你休要笑話。」不一時,二人吃得酒濃,掩閉了房門,解衣上床玩耍。王婆把大門頂着,和迎兒在廚房中坐地。二人在房內顛鸞倒鳳,似水如魚。那婦人枕邊風月,比娼妓尤甚,百般奉承。西門慶亦施逞槍法打動。兩個女貌郎才,俱在妙齡之際。

\begin{myquote} 
寂靜蘭房簟枕涼,佳人才子意何長。\\方纔枕上澆紅燭,忽又偷來火隔墻。\\粉蝶探香花萼顫,蜻蜓戲水徃來狂。\\情濃樂極猶餘興,珍重檀郎莫相忘。
\end{myquote} 

那日西門慶在婦人家盤桓至晚,欲囘家,留了幾兩散碎銀子與婦人做盤纏。婦人再三挽留不住。西門慶帶上眼罩,出門去了。婦人下了簾子,關上大門,又和王婆吃了一囘酒,纔散。正是:

\begin{myquote} 
倚門相送劉郎去,烟水桃花去路迷。
\end{myquote} 

