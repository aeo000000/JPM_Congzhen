\includepdf[pages={19,20},fitpaper=false]{tst.pdf}
\chapter*{第十囘 義士充配孟州道 妻妾玩賞芙蓉亭}
\addcontentsline{toc}{chapter}{第十囘 義士充配孟州道 妻妾玩賞芙蓉亭}
\markboth{{\titlename}卷之一}{第十囘 義士充配孟州道 妻妾玩賞芙蓉亭}


詞曰:

\begin{myquote}
八月中秋,涼飆微逗,芙蓉卻是花時候。誰家姊妹鬪新粧,園林散步攜手。折得花枝,寶瓶隨後,歸來玩賞全憑酒。三盃酩酊破愁城,醒時愁緒應還又。

\raggedleft{——右調《踏莎行》\rightquadmargin}
\end{myquote}

話說武二被地方保甲拏去縣裡見知縣,不題。且表西門慶跳下樓窓,扒伏在人家院裡藏了。原來是行醫的胡老人家。只見他家使的一個大胖丫頭,走來毛廁裡淨手,蹶着大屁股,猛可見一個漢子扒伏在院墻下,徃前走不迭,大叫:「有賊了!」{\meipi{劈空點綴,令人絕倒。}}慌的胡老人急進來。看見,認得是西門慶,便道:「大官人,且喜武二尋你不着,把那人打死了。地方拏他縣中見官去了。這一去定是死罪。大官人歸家去,料無事矣。」西門慶拜謝了胡老人,搖擺來家,一五一十對潘金蓮說,二人拍手喜笑,以為除了患害。{\meipi{世事徃徃如此。}}婦人叫西門慶上下多使些錢,務要結果了他,休要放他出來。西門慶一面差心腹家人來旺兒,餽送了知縣一副金銀酒器、五十兩銀子,上下吏典也使了許多錢,只要休輕勘了武二。知縣受了賄賂,到次日陞廳。地方押着武松並酒保、唱的一班人,當廳跪下。縣主翻了臉,便叫:「武松!你這厮昨日誣告平人,我已再三寬你,如何不遵法度,今又平白打死人?」武松道:「小人本與西門慶有仇,尋他厮打,不料撞遇此人。他隱匿西門慶不說,小人一時怒起,誤將他打死。只望相公與小人做主,拏西門慶正法,與小人哥哥報這一段冤仇。小人情願償此人誤傷之罪。」知縣道:「這厮胡說,你豈不認得他是縣中皁隸!今打殺他,定別有緣故,為何又纏到西門慶身上?不打如何肯招!」喝令左右加刑。兩邊閃三四個皁隸,把武松拖翻,雨點般打了二十。打得武二口口聲冤道:「小人也有與相公效勞用力之處,相公豈不憐憫?相公休要苦刑小人!」知縣聽了此言,越發惱了,道:「你這厮親手打死了人,尚還口強,抵賴那個?」喝令:「好生與我拶起來!」當下又拶了武松一拶,敲了五十杖子,教取面長枷帶了,收在監內。一干人寄監在門房裡。內中縣丞、佐二官也有和武二好的,念他是個義烈漢子,有心要周旋他,爭奈都受了西門慶賄賂,粘住了口,做不的主張。又見武松只是聲冤,延挨了幾日,只得朦朧取了供招,喚當該吏典並仵作、隣里人等,押到獅子街,檢驗李外傳身屍,填寫屍單格目。委的被武松尋問他索討,分錢不均,酒醉怒起,一時鬪毆,拳打脚踢,撞跌身死。左肋、面門、心坎、腎囊,俱有青赤傷痕不等。檢驗明白,囘到縣中。一日,做了文書申詳,解送東平府來,詳允發落。這東平府尹,姓陳雙名文昭,乃河南人氏,極是個清廉的官,聽的報來,隨即陞廳。但見他:

\begin{myquote}
平生正直,秉性賢明。幼年向雪案攻書,長大在金鑾對策。常懷忠孝之心,每發仁慈之政。戶口登,錢糧辦,黎民稱頌滿街衢;詞頌減,盜賊休,父老赞歌喧市井。正是:名標青史播千年,聲振黃堂傳萬古。賢良方正號青天,正直清廉民父母。
\end{myquote}

這府尹陳文昭陞了廳,便教押過這干犯人,就當廳先把清河縣申文看了,又把各人供狀招擬看過,端的上面怎生寫着?文曰:

\begin{myquote}[\markfont]
東平府清河縣,為人命事呈稱:犯人武松,年二十八歲,系陽谷縣人氏。因有膂力,本縣叅做都頭。因公差囘還,祭奠亡兄,見嫂潘氏不守孝滿,擅自嫁人。是日,鬆在巷口緝聽,不合在獅子街王鸞酒樓上撞遇李外傳。因酒醉,索討前借錢三百文,{\meipi{招卷之不得情實,古今如此。}}外傳不與;又不合因而鬪毆,相互不服,揪打踢撞傷重,當時身死。比有唱婦牛氏、包氏見證,致被地方保甲捉獲。委官前至屍所,拘集仵作、裡甲人等,檢驗明白,取供具結,填圖解繳前來,覆審無異。擬武松合依鬪毆殺人,不問手足、他物、金兩,律絞。酒保王鸞並牛氏、包氏,俱供明無罪。今合行申到案發落,請允施行。

政和三年八月日

\raggedleft{知縣李達天,縣丞樂和安,\rightquadmargin}

\raggedleft{主簿華荷祿,典史夏恭基,司吏錢勞。\rightquadmargin}
\end{myquote}

府尹看了一遍,將武松叫過面前,問道:「你如何打死這李外傳?」那武松只是朝上磕頭告道:「青天老爺!小的到案下,得見天日。容小的說,小的敢說。」府尹道:「你只顧說來。」武松遂將西門慶奸娶潘氏,並哥哥捉姦,踢中心窩,後來縣中告狀不準,前後情節細說一遍,道:「小的本為哥哥報仇,因尋西門慶厮打,不料誤打死此人。委是小的負屈含冤,奈西門慶錢大,禁他不得。小人死不足惜,但只是小人哥哥武大含冤地下,枉了性命。」府尹道:「你不消多言,我已盡知了。」因把司吏錢勞叫來,痛責二十板,說道:「你那知縣也不待做官,何故這等任情賣法?」於是將一干人衆,一一審錄過,用筆將武松供招都改了,因向佐二官說道:「此人為兄報仇,誤打死這李外傳,也是個有義的烈漢,比故殺平人不同。」一面開啟他長枷,換了一面輕罪枷枷了,下在牢裡。一干人等都發囘本縣聽候。一面行文書着落清河縣,添提豪惡西門慶,並嫂潘氏、王婆、小厮鄆哥、仵作何九,一同從公根勘明白,奏請施行。武松在東平府監中,人都知道他是條好漢,因此押牢禁子都不要他一文錢,到把酒食與他吃。

早有人把這件事報到清河縣。西門慶知道了,慌了手脚。陳文昭是個清廉官,不敢來打點他。只得走去央求親家陳宅心腹,並使家人來旺,{\pangpi{伏。}}星夜徃東京下書與楊提督。提督轉央內閣蔡太師。太師又恐怕傷了李知縣名節,{\meipi{好個愛賢宰相。}}連忙齎了一封密書,特來東平府下與陳文昭,擴音西門慶、潘氏。這陳文昭原系大理寺寺正,陞東平府府尹,又系蔡太師門生,又見楊提督乃是朝廷面前說得話的官,以此人情兩盡,只把武松免死,問了個脊杖四十,刺配二千里充軍。況武大已死,屍傷無存,事涉疑似,勿論。其餘一干人犯釋放甯家。申詳過省院,文書到日,即便施行。陳文昭從牢中取出武松來,當堂讀了朝廷明降,開了長枷,免不得脊杖四十,取一具七斤半鐵葉團頭枷釘了,臉上刺了兩行金字,迭配孟州牢城。其餘發落已完,當堂府尹押行公文,差兩個防送公人,領了武松解赴孟州交割。當日武松與兩個公人出離東平府,來到本縣家中,將家活多變賣了,打發那兩個公人路上盤費,央托左隣姚二郎看管迎兒:「倘遇朝廷恩典,赦放還家,恩有重報,不敢有忘。」街坊隣舍,上戶人家,見武二是個有義的漢子,不幸遭此,都資助他銀兩,也有送酒食錢米的。武二到下處,問土兵要出行李包裹來,即日離了清河縣上路,迤邐徃孟州大道而行。有詩為證:

\begin{myquote}
府尹推詳秉至公,武松垂死又疏通。\\今朝刺配牢城去,病草萋萋遇暖風。
\end{myquote}

這裡武二徃孟州充配去了,不題。

且說西門慶打聽他上路去了,一塊石頭方落地,心中如去了痞一般,十分自在。於是家中分付家人來旺、來保、來興兒,收拾打掃後花園芙蓉亭乾淨,鋪設圍屏,掛起錦障,安排酒席齊整,叫了一起樂人,吹彈歌舞。請大娘子吳月娘、第二李嬌兒、第三孟玉樓、第四孫雪娥、第五潘金蓮,合家歡喜飲酒。家人媳婦、丫鬟使女兩邊侍奉。但見:

\begin{myquote}
香焚寶鼎,花插金瓶。器列象州之古玩,簾開合浦之明珠。水晶盤內,高堆火棗交梨;碧玉盃中,滿泛瓊漿玉液。烹龍肝,炮鳳腑,果然下筯了萬錢;黑熊掌,紫駝蹄,酒後獻來香滿座。碾破鳳團,白玉甌中分白浪;斟來瓊液,紫金壺內噴清香。畢竟壓賽孟嘗君,只此敢欺石崇富。
\end{myquote}

當下西門慶與吳月娘居上,其餘多兩傍列坐,傳盃弄盞,花簇錦攢。飲酒間,只見小厮玳安領下一個小厮、一個小女兒,纔頭髮齊眉,生得乖覺,拏着兩個盒兒,說道:「隔壁花家,送花兒來與娘們戴。」走到西門慶、月娘衆人跟前,都磕了頭,立在傍邊,說:「俺娘使我送這盒兒點心並花兒與西門大娘戴。」揭開盒兒看,一盒是朝廷上用的菓餡椒鹽金餅,一盒是新摘下來鮮玉簪花。月娘滿心歡喜,說道:「又叫你娘費心。」一面看菜兒,打發兩個吃了點心。月娘與了那小丫頭一方汗巾兒,與了小厮一百文錢,說道:「多上覆你娘,多謝了。」因問小丫頭兒:「你叫什麼名字?」他囘言道:「我叫綉春。小厮便是天福兒。」打發去了。月娘便向西門慶道:「咱這花家娘子兒,倒且是好,常時使小厮丫頭送東西與我們。我並不曾囘些禮兒與他。」西門慶道:「花二哥娶了這娘子兒,今不上二年光景。他自說娘子好個性兒。不然房裡怎生得這兩個好丫頭。」{\pangpi{字字綿裡裹針。}}{\meipi{似為李瓶兒出笋,卻又暗伏收春梅,機緣線索之妙,令人不測。}}月娘道:「前者他家老公公死了出殯時,我在山頭會他一面。生得五短身材,團面皮,細灣灣兩道眉兒,且是白淨,好個溫克性兒。年紀還小哩,不上二十四五。」西門慶道:「你不知,他原是大名府梁妾,晚嫁花家子虛,帶一分好錢來。」月娘道:「他送盒兒來,咱休差了禮數,到明日也送些禮物囘答他。」

看官聽說:原來花子虛渾家姓李,因正月十五所生,那日人家送了一對魚瓶兒來,就小字喚做瓶姐。先與大名府梁為妾。梁乃東京蔡太師女婿,夫人性甚嫉妒,婢妾打死者多埋在後花園中。這李氏只在外邊書房內住,有養娘伏侍。只因政和三年正月上元之夜,梁同夫人在翠雲樓上,李逵殺了全家老小,{\pangpi{照應}}梁與夫人各自逃生。這李氏帶了一百顆西洋大珠,{\pangpi{伏。}}二兩重一對鴉青寶石,與養娘走上東京投親。那時花太監由御前班直陞廣南鎮守,因姪男花子虛沒妻室,就使媒婆說親,娶為正室。太監到廣南去,也帶他到廣南,住了半年有餘。不幸花太監有病,告老在家,因是清河縣人,在本縣住了。如今花太監死了,一分錢多在子虛手裡。每日同朋友在院中行走,與西門慶都是前日結拜的弟兄。終日與應伯爵、謝希大一班十數個,每月會在一處,叫些唱的,花攢錦簇頑耍。衆人又見花子虛乃是內臣家勤兒,手裡使錢撒漫,哄着他在院中請婊子,整三五夜不歸。正是:

\begin{myquote}
紫陌春光好,紅樓醉管絃。\\人生能有幾?不樂是徒然。
\end{myquote}

此事表過不題。

且說當日西門慶率同妻妾,合家歡樂,在芙蓉亭上飲酒,至晚方散。歸來潘金蓮房中,已有半酣,乘着酒興,要和婦人雲雨。婦人連忙薰香打鋪,和他解衣上床。西門慶且不與他雲雨,明知婦人第一好品簫,於是坐在青紗帳內,令婦人馬爬在身邊,雙手輕籠金釧,捧定那話,徃口裡吞放。西門慶垂首玩其出入之妙,嗚咂良久,淫情倍增,因呼春梅進來遞茶。{\pangpi{未必無心。}}婦人恐怕丫頭看見,連忙放下帳子來。西門慶道:「怕怎麼的?」因說起:「隔壁花二哥房裡到有兩個好丫頭。今日送花來的,是小丫頭;還有一個也有春梅年紀,也是花二哥收用過了。但見他娘在門首站立,{\pangpi{不丟開,寫出貪心。}}他跟出來,卻是生得好模樣兒。誰知這花二哥年紀小小的,房裡恁般用人!」{\meipi{牽枝扯葉,語語含,卻語語露,何物文人,摹寫至此。}}婦人聽了,瞅了他一眼,說道:「恠行貨子,我不好罵你,你心裡要收這個丫頭,{\pangpi{解心人。}}收他便了,如何遠打週折,指山說磨,拏人家來比奴。奴不是那樣人,他又不是我的丫頭!既然如此,明日我徃後邊坐一囘,騰個空兒,你自在房中叫他來,收他便了。」{\meipi{金蓮亦有心擡舉春梅,故一說便肯。}}西門慶聽了,歡喜道:「我的兒,你會這般解趣,怎教我不愛你!」二人說得情投意洽,更覺美愛無加,慢慢的品簫過了,方纔抱頭交股而寢。正是:自有內事迎郎意,殷勤快把紫簫吹。有《西江月》為證:

\begin{myquote}
紗帳香飄蘭麝,娥眉慣把簫吹。雪瑩玉體透房幃,禁不住魂飛魄碎。玉腕款籠金釧,兩情如醉如癡。纔郎情動囑奴知,慢慢多咂一會。
\end{myquote}

到次日,果然婦人徃孟玉樓房中坐了。西門慶叫春梅到房中,收用了這妮子。正是:

\begin{myquote}
春點杏桃紅綻蕊,風欺楊柳綠翻腰。
\end{myquote}

潘金蓮自此一力擡舉他起來,不令他上鍋抹竈,只叫他在房中鋪床疊被,遞茶水,衣服首飾揀心愛的與他,纏得兩隻脚小小的。原來春梅比秋菊不同,性聰慧,喜謔浪,善應對,生的有幾分顏色,西門慶甚是寵他。秋菊為人濁蠢,不諳事體,婦人常常打的是他。正是:

\begin{myquote}
燕雀池塘語話喧,蜂柔蝶嫩總堪憐。\\雖然異數同飛鳥,貴賤高低不一般。
\end{myquote}

