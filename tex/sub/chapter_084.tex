\includepdf[pages={167,168},fitpaper=false]{tst.pdf}
\chapter*{第八十四囘 吳月娘大鬧碧霞宮 曾靜師化緣雪澗洞}
\addcontentsline{toc}{chapter}{第八十四囘 吳月娘大鬧碧霞宮 曾靜師化緣雪澗洞}
\markboth{\titlename}{第八十四囘 吳月娘大鬧碧霞宮 曾靜師化緣雪澗洞}


詩曰:

一自當年折鳳凰,至今情緒幾惶惶。蓋棺不作橫金婦,入地還從折桂郎。彭澤曉烟歸宿夢,瀟湘夜雨斷愁腸。新詩寫向空山寺,高掛雲帆過豫章。

說話一日,吳月娘請將吳大舅來商議,要徃泰安州頂上與娘娘進香,因西門慶病重之時許的願心。吳大舅道:「既要去,須是我同了你去。」一面備辦香燭紙馬祭品之物,玳安、來安兒跟隨,顧了三個頭口,月娘便坐一乘暖轎,分付孟玉樓、潘金蓮、孫雪娥、西門大姐:「好生看家,同奶子如意兒、衆丫頭好生看孝哥兒。後邊儀門無事早早關了,休要出外邊去。」又分付陳敬濟:「休要那去,同傅夥計大門首看顧。我約莫到月盡就來家了。」{\meipi{托家緣幼子與一班異心之人,而遠出燒香,月娘殊亦愚而多事。}}十五日早辰燒紙通訊,晚夕辭了西門慶靈,與衆姊妹置酒作別,把房門、各庫門房鑰匙交付與小玉拏着。次日早五更起身,離了家門,一行人奔大路而去。那秋深時分,天寒日短,一日行程六七十里之地。未到黃昏,投客店村房安歇,次日再行。一路上,秋雲淡淡,寒雁悽悽,樹木凋落,景物荒涼,不勝悲愴。

話休饒舌。一路無詞,行了數日,到了泰安州,望見泰山,端的是天下第一名山,根盤地脚,頂接天心,居齊魯之邦,有巖巖之氣象。吳大舅見天晚,投在客店歇宿一宵。次日早起上山,望岱嶽廟來。那岱嶽廟就在山前,乃累朝祀典,歷代封禪,為第一廟貌也。但見:

廟居岱嶽,山鎮乾坤,為山嶽之尊,乃萬福之領袖。山頭倚檻,直望弱水蓬萊;絕頂攀松,都是濃雲薄霧。樓臺森聳,金烏展翅飛來;殿宇稜層,玉兔騰身走到。雕梁畫棟,碧瓦朱簷,鳳扉亮槅映黃紗,龜背繡簾垂錦帶。遙觀聖像,九獵舞舜目堯眉;近觀神顏,袞龍袍湯肩禹背。御香不斷,天神飛馬報丹書;祭祀依時,老幼望風祈護福。嘉寧殿祥雲香靄,正陽門瑞氣盤旋。

正是:

萬民朝拜碧霞宮,四海皈依神聖帝。

吳大舅領月娘到了岱嶽廟,正殿上進了香,瞻拜了聖像,廟祝道士在旁宣念了文書。然後兩廊都燒化了紙錢,吃了些齋食。然後領月娘上頂,登四十九盤,攀藤攬葛上去。娘娘金殿在半空中雲烟深處,約四五十里,風雲雷雨都望下觀看。{\meipi{此山奇峻,只八字寫出。}}月娘衆人從辰牌時分岱嶽廟起身,登盤上頂,至申時已後方到。娘娘金殿上朱紅牌扁,金書「碧霞宮」三字。進入宮內,瞻礼娘娘金身。怎生模樣?但見:

頭綰九龍飛鳳髻,身穿金縷絳綃衣。藍田玉帶曳長裾,白玉圭璋檠彩袖。臉如蓮萼,天然眉目映雲鬟;唇似金朱,自在規模端雪體。猶如王母宴瑤池,卻似嫦娥離月殿。正大仙雲描不就,威嚴形象畫難成。

月娘瞻拜了娘娘仙容,香案邊立着一個廟祝道士,約四十年紀,生的五短身材,三溜髭鬚,明眸皓齒,頭戴簪冠,身披絳服,足登雲履,向前替月娘宣讀了還願文疏,金爐內炷了香,焚化了紙馬金銀,令小童收了祭供。

原來這廟祝道士,也不是個守本分的,乃是前邊岱嶽廟裡金住持的大徒弟,姓石,雙名伯才,極是個貪財好色之輩,趨時攬事之徒。這本地有個殷太歲,姓殷,雙名天錫,乃是本州知州高廉的妻弟。常領許多不務本的人,或張弓挾彈,牽架鷹犬,在這上下二宮,專一睃看四方燒香婦女,{\meipi{據泰山而觀天下婦女,亦是奇人。}}人不敢惹他。這道士石伯才,專一藏奸蓄詐,替他撰誘婦女到方丈,任意姦淫,取他喜歡。因見月娘生的姿容非俗,戴着孝冠兒,若非官戶娘子,定是豪家閨眷;又是一位蒼白髭髯老子跟隨,兩個家童,{\pangpi{便不足畏。}}不免向前稽首,收謝神福:「請二位施主方丈一茶。」吳大舅便道:「不勞生受,還要趕下山去。」伯才道:「就是下山也還早哩。」{\pangpi{款得賊。}}不一時,請至方丈,裡面糊的雪白,正面放一張芝麻花坐床,桺黃錦帳,香几上供養一幅洞賓戲白牡丹圖畫,{\pangpi{絕妙招牌。}}左右一對聯,大書着:「兩袖清風舞鶴,一軒明月談經。」伯才問吳大舅上姓,大舅道:「在下姓吳,這個就是舍妹吳氏,因為夫主來還香願,不當取擾上宮。」伯才道:「既是令親,俱延上坐。」他便主位坐了,便叫徒弟看茶。原來他手下有兩個徒弟,一個叫郭守清,一個名郭守礼,皆十六歲,生得標致,頭上戴青段道髻,身穿青絹道服,脚上涼鞋淨襪,渾身香氣襲人。{\meipi{以人誘人之法。}}客至則遞茶遞水,斟酒下菜。到晚來,背地便拏他解饞填餡。不一時,守清、守礼安放桌兒,就擺齋上來,都是美口甜食,蒸堞餅饊,各樣菜蔬,擺滿春臺。每人送上甜水好茶,吃了茶,收下家伙去。就擺上案酒。大盤大碗餚饌,都是雞鵝魚鴨上來。用琥珀鑲盞,滿泛金波。吳月娘見酒來,就要起身,叫玳安近前,用紅漆盤托出一疋大布、二兩白金,與石道士作致謝之礼。吳大舅便說:「不當打攪上宮,這些微礼致謝仙長。不勞見賜酒食,天色晚來,如今還要趕下山去。」慌的石伯才致謝不已,說:「小道不才,娘娘福蔭,在本山碧霞宮做個住持,仗賴四方錢糧,不管待四方財主,作何項下使用?今聊備粗齋薄饌,倒反勞見賜厚礼,使小道卻之不恭,受之有愧。」辭謝再三,方令徒弟收下去。一面留月娘、吳大舅坐:「好歹坐片時,略飲三盃,盡小道一點薄情而已。」吳大舅見款留懇切,不得已和月娘坐下。

不一時,熱下飯上來。石道士分付徒弟:「這個酒不中吃,另開啟昨日徐知府老爺送的那一罈透瓶香荷花酒來,與你吳老爹用。」不一時,徒弟另用熱壺篩熱酒上來。先滿斟一盃,雙手遞與月娘,{\meipi{先奉月娘,微露注意。}}月娘不肯接。吳大舅道:「舍妹他天性不用酒。」伯才道:「老夫人一路風霜,用些何害?好歹淺用些。」一面倒去半鍾,遞上去與月娘接了。又斟一盃遞與吳大舅,說:「吳老爹,你老人家試用此酒,其味如何?」吳大舅飲了一口,覺香甜絕美,其味深長,說道:「此酒甚好。」伯才道:「不瞞你老人家說,此是青州徐知府老爹送與小道的酒。他老夫人、小姐、公子,年年來岱嶽廟燒香建醮,與小道相交極厚。他小姐;衙內又寄名在娘娘位下。見小道立心平淡,殷勤香火,一味至誠,甚是敬愛小道。常年,這岱嶽廟上下二宮錢糧,有一半徵收入庫。近年多虧了我這恩主徐知府老爹題奏過,也不徵收,都全放常住用度,侍奉娘娘香火,餘者接待四方香客。」{\meipi{說老爺卻夾出夫人、小姐,說感恩卻全是自贊,又使勢,又攤眼,又奉承,語語綿裡裹針,婦女稍不見慣,未有不墜其術中者。賊道,賊道。}}這裡說話,下邊玳安、來安、跟從轎伕,下邊自有坐處,湯飯點心,大盤大碗酒肉,都吃飽了。

吳大舅飲了幾盃,見天晚要起身。伯才道:「日色將落,晚了趕不下山去。{\meipi{先說早,後說晚,絕妙騙法。}}倘不棄,在小道方丈權宿一宵,明早下山從容些。」吳大舅道:「爭奈有些小行李在店內,誠恐一時小人羅唣。」伯才笑道:「這個何須掛意!決無絲毫差池。聽得是我這裡進香的,不拘村坊店面,聞風害怕,好不好把店家拏來本州來打,就教他尋賊人下落。」{\meipi{豈道士之言,明眼人便當看破。}}吳大舅聽了,就坐住了。伯才拏大鐘斟上酒來。吳大舅見酒利害,便推醉更衣,{\pangpi{此處還有主意。}}遂徃後邊閣上觀看隨喜去了。這月娘覺身子乏困,便在床上側側兒。這石伯才一面把房門拽上,外邊去了。月娘方纔床上𢱉着,忽聽裡面响喨了一聲,床背後紙門內跳出一個人來,淡紅面貌,三桺髭鬚,約三十年紀,頭戴滲青巾,身穿紫錦袴衫,雙手抱住月娘,說道:「小生殷天錫,乃高太守妻弟。久聞娘子乃官豪宅眷,天然國色,思慕如渴。今既接英標,乃三生有幸,倘蒙見憐,死生難忘也。」{\meipi{沒頭沒腦,說得親親切切,亦大可笑。想見一輩交淺言深者,與此相類。}}一面按着月娘在床上求歡。月娘唬的慌做一團,高聲大叫:「清平世界,朗朗乾坤,沒事把良人妻室,強𢺞攔在此做甚!」就要奪門而走。被天錫抵死攔擋不放,便跪下說:「娘子禁聲,下顧小生,懇求憐允。」{\pangpi{不象太歲。}}那月娘越高聲叫的緊了,口口大叫:「救人!」平安、玳安聽見是月娘聲音,慌慌張張走去後邊閣上,叫大舅說:「大舅快去,我娘在方丈和人合口哩。」這吳大舅慌的兩步做一步奔到方丈推門,那裡推得開。只見月娘高聲:「清平世界,攔燒香婦女在此做甚麼?」這吳大舅便叫:「姐姐休慌,我來了!」一面拏石頭把門砸開。那殷天錫見有人來,撇開手,打床背後一溜烟走了。原來這石道士床背後都有出路。吳大舅砸開方丈門。問月娘道:「姐姐,那厮玷汙不曾?」月娘道:「不曾玷汙。那厮打床背後走了。」吳大舅尋道士,那石道士躲去一邊,只教徒弟來支調。大舅大怒,喝令手下跟隨玳安、來安兒把道士門窓戶壁都打碎了。一面保月娘出離碧霞宮,上了轎子,便趕下山來。

約黃昏時分起身,走了半夜,方到山下客店內。{\meipi{一婦人、一老子,半夜在泰山頂上逃難,危甚,險甚。此是燒香下場頭。}}如此這般,告店小二說。小二叫苦連聲,說:「不合惹了殷太歲,他是本州知州相公妻弟,有名殷太歲。你便去了,俺開店之家,定遭他淩辱,怎肯干休!」吳大舅便多與他一兩店錢,取了行李,保定月娘轎子,急急奔走。後面殷天錫氣不捨,率領二三十閑漢,各執腰刀短棍,趕下山來。

吳大舅一行人,兩程做一程,約四更時分,趕到一山凹裡。遠遠樹木叢中有燈光,走到跟前,卻是一座石洞,裡面有一老僧秉燭念經。吳大舅問:「老師,我等頂上燒香,被強人所趕,奔下山來,天色昏黑,迷蹤失路至此。敢問老師,此處是何地名?從那條路囘得清河縣去?」老僧說:「此是岱嶽東峰,這洞名喚雪澗洞。貧僧就叫雪洞禪師,法名普靜,在此修行二三十年。你今遇我,實乃有緣。休徃前去,山下狼蟲虎豹極多。明日早行,一直大道就是你清河縣了。」吳大舅道:「只怕有人追趕。」老師把眼一觀說:「無妨,那強人趕至半山,已囘去了。」因問月娘姓氏。吳大舅道:「此乃吾妹,西門慶之妻。因為夫主,來此進香。得遇老師搭救,恩有重報,不敢有忘。」於是在洞內歇了一夜。

次日,天不亮,月娘拏出一疋大布謝老師。老師不受,說:「貧僧只化你親生一子作個徒弟,你意下何如?」{\meipi{似說破,又似不說破,此書妙處,只是一冷。}}吳大舅道:「吾妹止生一子,指望承繼家業。若有多餘,就與老師作徒弟。」月娘道:「小兒還小,今纔不到一週歲兒,如何來得?」老師道:「你只許下,我如今不問你要,過十五年纔問你要哩。」月娘口中不言,過十五年再作理會,遂含糊許下老師。一面作辭老師,竟奔清河縣大道而來。正是:

世上只有人心歹,萬物還教天養人。但交方寸無諸惡,狼虎叢中也立身。

