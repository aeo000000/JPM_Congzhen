\part*{{\titlename}卷之一}
\addcontentsline{toc}{part}{{\titlename}卷之一}


\includepdf[pages={1,2},fitpaper=false]{tst.pdf}
\chapter*{第一囘 西門慶熱結十弟兄 武二郎冷遇親哥嫂}
\addcontentsline{toc}{chapter}{第一囘 西門慶熱結十弟兄 武二郎冷遇親哥嫂}
\markboth{{\titlename}卷之一}{第一囘 西門慶熱結十弟兄 武二郎冷遇親哥嫂}


詩曰:

豪華去後行人絕,簫箏不響歌喉咽。雄劍無威光彩沉,寶琴零落金星滅。玉堦寂寞墜秋露,月照當時歌舞處。當時歌舞人不囘,化為今日西陵灰。{\meipi{一部炎涼境況,盡此數語中。}}

又詩曰:

二八佳人體似酥,腰間仗劍斬愚夫。{\pangpi{不是愚夫,盡是貪夫。}}雖然不見人頭落,暗裡教君骨髓枯。

這一首詩,是昔年大唐國時,一箇修真練性的英雄,入聖超凡的豪傑,到後來位居紫府,名列仙班,率領上八洞群仙,救拔四部洲沉苦一位仙長,姓呂名巖,道號純陽子祖師所作。單道世上人,營營逐逐,急急巴巴,跳不出七情六慾關頭,打不破酒色財氣圈子。到頭來同歸於盡,着甚要緊!雖是如此說,只這酒色財氣四件中,唯有「財色」二者更為利害。怎見得他的利害?假如一箇人到了那窮苦的田地,受盡無限淒涼,耐盡無端懊惱,晚來摸一摸米甕,苦無隔宿之炊,早起看一看廚前,愧無半星烟火,妻子飢寒,一身凍餒,{\meipi{情景逼真,酸徠談此,能不雪涕!}}就是那粥飯尚且艱難,那討餘錢沽酒!{\pangpi{酒因財缺。}}更有一種可恨處,親朋白眼,面目寒酸,便是淩雲志氣,分外消磨,怎能勾與人爭氣!{\pangpi{氣以財弱。}}正是:

一朝馬死黃金盡,親者如同陌路人。

到得那有錢時節,揮金買笑,一擲鉅萬。思飲酒,真箇瓊漿玉液,{\pangpi{酒需財美。}}不數那琥珀盃流;要鬪氣,錢可通神,果然是頤指氣使。{\pangpi{氣用財伸。}}趨炎的壓脊挨肩,附勢的吮癰舐痔,真所謂得勢疊肩而來,失勢掉臂而去。古今炎冷惡態,莫有甚於此者。這兩等人,豈不是受那財的利害處!如今再說那色的利害。{\meipi{引起三段,格法一變,更見靈活。}}請看如今世界,你說那坐懷不亂的桺下惠,閉門不納的魯男子,與那秉燭達旦的關雲長,古今能有幾人?至如三妻四妾,買笑追歡的,又當別論。還有那一種好色的人,見了箇婦女畧有幾分顏色,便百計千方偷寒送暖,一到了着手時節,只圖那一瞬歡娛,也全不顧親戚的名分,也不想朋友的交情。起初時,不知用了多少濫錢,費了幾遭酒食。{\pangpi{酒。}}正是:

三盃花作合,兩盞色媒人。

到後來情濃事露,甚而鬪狠殺傷,{\pangpi{氣。}}性命不保,妻孥難顧,事業成灰。就如那石季倫潑天豪富,為綠珠命䘮囹圄;楚霸王氣概拔山,因虞姬頭懸垓下。真說謂「生我之門死我戶,看得破時忍不過」。這樣人,豈不是受那色的利害處?說便如此說,這「財色」二字,從來只沒有看得破的。{\meipi{說的世情氷冷,須從蒲團面壁十年纔辨。}}若有那看得破的,便見得堆金積玉,是棺材內帶不去的瓦礫泥沙;貫朽粟紅,是皮囊內裝不盡的臭淤糞土。高堂廣廈,玉宇瓊樓,是墳山上起不得的享堂;錦衣繡襖,狐服貂裘,是骷髏上裹不了的敗絮。即如那妖姬艷女,獻媚工妍,看得破的,卻如交鋒陣上將軍叱吒獻威風;硃唇皓齒,掩袖囘眸,懂得來時,便是閻羅殿前鬼判夜叉增惡態。羅襪一彎,金蓮三寸,是砌墳時破土的鍬鋤;{\pangpi{尖穎異常。}}枕上綢繆,被中恩愛,是五殿下油鍋中生活。只有那《金剛經》上兩句說得好,他說道:「如夢幻泡影,如電覆如露。」見得人生在世,一件也少不得,到了那結束時,一件也用不着。隨着你舉鼎盪舟的神力,到頭來少不得骨軟觔麻;繇着你銅山金谷的奢華,正好時卻又要氷消雪散。假饒你閉月羞花的容貌,一到了垂眉落眼,人皆掩鼻而過之;比如你陸賈、隋何的機鋒,若遇着齒冷唇寒,吾末如之何也已。{\meipi{生公說法,石應肯首。}}到不如削去六根清淨,{\pangpi{伏脈。}}披上一領袈裟,叅透了空色世界,打磨穿生滅機關,直超無上乘,不落是非窠,倒得箇清閉自在,不向火坑中翻觔斗也。正是:

三寸氣在千般用,一日無常萬事休。

說話的,為何說此一段酒色財氣的緣故?只為當時有一箇人家,先前恁地富貴,到後來煞甚淒涼,權謀術智,一毫也用不着,親友兄弟,一箇也靠不着,享不過幾年的榮華,倒做了許多的話靶。內中又有幾箇鬪寵爭強,迎奸賣俏的,起先好不妖嬈嫵媚,到後來也免不得屍橫燈影,血染空房。正是:

善有善報,惡有惡報;天網恢恢,疏而不漏。

話說大宋徽宗皇帝政和年間,山東省東平府清河縣中,有一箇風流子弟,生得狀貌魁梧,性情瀟灑,饒有幾貫家資,年紀二十六七。這人覆姓西門,單諱一箇慶字。他父親西門達,原走川廣販藥材,就在這清河縣前開着一箇大大的生藥鋪。現住着門面五間到底七進的房子。家中呼奴使婢,騾馬成群,雖算不得十分富貴,卻也是清河縣中一箇殷實的人家。只為這西門達員外夫婦去世的早,單生這箇兒子,卻又百般愛惜,聽其所為,所以這人不甚讀書,{\pangpi{四字是一生病痛。}}終日閑遊浪蕩。一自父母亡後,專一在外眠花宿桺,惹草招風,學得些好拳棒,又會賭博,雙陸、象棋、抹牌、道字,無不通曉。結識的朋友,也都是些幫閑抹嘴,不守本分的人。第一箇最相契的,姓應名伯爵,表字光侯,原是開紬段鋪應員外的第二箇兒子,落了本錢,跌落下來,專在本司三院幫嫖貼食,因此人都起他一箇渾名,叫做應花子。又會一腿好氣毬,雙陸、棋子,件件皆通。{\meipi{叙得錯綜變化。}}第二箇姓謝名希大,字子純,乃清河衛千戶官兒應襲子孫,自幼父母雙亡,遊手好閑,把前程丟了,亦是幫閑勤兒,會一手好琵琶。自這兩箇與西門慶甚合得來。其餘還有幾箇,都是些破落戶,沒名器的。一箇叫做祝實念,表字貢誠。一箇叫做孫天化,表字伯修,綽號孫寡嘴。一箇叫做吳典恩,乃是本縣陰陽生,因事革退,專一在縣前與官吏保債,以此與西門慶徃來。還有一箇雲叅將的兄弟叫做雲理守,字非去。一箇叫做常峙節,表字堅初。一箇叫做卜志道。一箇叫做白賚光,表字光湯。說這白賚光,衆人中也有道他名字取的不好聽的,他卻自己解說道:「不然我也改了,只為當初取名的時節,原是一箇門館先生,說我姓白,當初有一箇什麼故事,是白魚躍入武王舟。又說有兩句書是『周有大賚,於湯有光』,取這箇意思,所以表字就叫做光湯。我因他有這段故事,也便不改了。」{\meipi{磊落寫來,於結處獨以此段瀠洄,便覺鬚眉生動。}}說這一干共十數人,見西門慶手裡有錢,又撒漫肯使,所以都亂撮哄着他耍錢飲酒,嫖賭齊行。正是:

把盞啣盃意氣深,兄兄弟弟抑何親。一朝平地風波起,此際相交纔見心。

說話的,這等一箇人家,生出這等一箇不肖的兒子,又搭了這等一班無益有損的朋友,隨你怎的豪富也要窮了,還有甚長進的日子!卻有一箇緣故,只為這西門慶生來秉性剛強,作事機深詭譎,又放官吏債,就是那朝中高、楊、童、蔡四大奸臣,他也有門路與他浸潤。{\meipi{好針線。}}所以專在縣裡管些公事,與人把攪說事過錢,因此滿縣人都懼怕他。因他排行第一,人都叫他是西門大官人。這西門大官人先頭渾家陳氏早逝,身邊只生得一箇女兒,叫做西門大姐,就許與東京八十萬禁軍楊提督的親家陳洪的兒子陳敬濟為室,尚未過門。只為亡了渾家,無人管理家務,新近又娶了本縣清河左衛吳千戶之女,填房為繼室。這吳氏年紀二十五六,是八月十五生的,小名叫做月姐,後來嫁到西門慶家,都順口叫他月娘。卻說這月娘秉性賢能,夫主面上百依百隨。{\meipi{如此賢婦,世上有幾?}}房中也有三四箇丫鬟婦女,都是西門慶收用過的。又嘗與勾欄內李嬌兒打熱,也娶在家裡做了第二房娘子。南街又佔着窠子卓二姐,名卓丟兒,包了些時,也娶來家做了第三房。只為卓二姐身子瘦怯,時常三病四痛,他卻又去飄風戲月,調弄人家婦女。正是:

東家歌笑醉紅顏,又向西隣開玳宴。幾日碧桃花下臥,牡丹開處總堪憐。

話說西門慶一日在家閑坐,對吳月娘說道:「如今是九月廿五日了,出月初三日,卻是我兄弟們的會期。到那日也少不的要整兩席齊整的酒席,叫兩箇唱的姐兒,自恁在咱家與兄弟們好生玩耍一日。你與我料理料理。」吳月娘便道:「你也便別要說起這幹人,那一箇是那有良心和行貨!無過每日來勾使的遊魂撞屍。我看你自搭了這起人,幾時曾着箇家哩!{\meipi{數語可配名臣諫疏。}}現今卓二姐自恁不好,我勸你把那酒也少要吃了。」西門慶道:「你別的話倒也中聽。今日這些說話,我卻有些不耐煩聽他。依你說,這些兄弟們沒有好人,別的倒也罷了,自我這應二哥,這一箇人本心又好,又知趣着人,{\pangpi{溺愛者智昏,不止西門一箇。}}使着他,沒有一箇不依順的,做事又十分停當。就是那謝子純這箇人,也不失為箇伶俐能事的好人。咱如今是這等計較罷,只管恁會來會去,終不着箇切實。咱不如到了會期,都結拜了兄弟罷,明日也有箇靠傍些。」吳月娘接過來道:「結拜兄弟也好。只怕後日還是別箇靠你的多哩。若要你去靠人,提傀儡兒上戲場——還少一口氣兒哩。」西門慶笑道:「咱恁長把人靠得着,卻不更好了。咱只等應二哥來,與他說這話罷。」正說着話,只見一箇小厮兒,生得眉清目秀,伶俐乖覺,原是西門慶貼身伏侍的,喚名玳安兒,走到面前來說:「應二叔和謝大叔在外見爹說話哩。」西門慶道:「我正說他,他卻兩箇就來了。」一面走到廳上來,只見應伯爵頭上戴一頂新盔的玄羅帽兒,身上穿一件半新不舊的天青夾縐紗褶子,脚下絲鞋淨襪,坐在上首。下首坐的便是姓謝的謝希大。見西門慶出來,一齊立起身來,邊忙作揖道:「哥在家,連日少看。」西門慶讓他坐下,一面喚茶來吃,說道:「你們好人兒,這幾日我心裡不耐煩,不出來走跳,你們通不來傍箇影兒。」伯爵向希大道:「何如?我說哥要說哩。」因對西門慶道:「哥,你恠的是。連咱自也不知道成日忙些什麼!自咱們這兩隻脚,還趕不上一張嘴哩。」西門慶因問道:「你這兩日在那裡來?」伯爵道:「昨日在院中李家瞧了箇孩子兒,就是哥這邊二嫂子的姪女兒、桂卿的妹子,叫做桂姐兒。幾時兒不見他,就出落的好不標致了。到明日成人的時候,還不知怎的樣好哩!昨日他媽再三向我說:『二爹,千萬尋箇好子弟梳籠他。』敢怕明日還是哥的貨兒哩。」{\pangpi{伏脈。}}西門慶道:「有這等事!等咱空閑了去瞧瞧。」謝希大接過來道:「哥不信,委的生得十分顏色。」西門慶道:「昨日便在他家,前幾日卻在那裡去來?」伯爵道:「便是前日卜志道兄弟死了,咱在他家幫着亂了幾日,傳送他出門。{\pangpi{伏脈。}}他嫂子再三向我說,叫我拜上哥,承哥這裡送了香楮奠礼去,因他沒有寬轉地方兒,晚夕又沒甚好酒席,不好請哥坐的,甚是過不意去。」西門慶道:「便是我聞得他不好得沒多日子,就這等死了。我前日承他送我一把真金川扇兒,我正要拏甚答謝答謝,不想他又作了故人!」

謝希大便嘆了一口氣道:「咱會中兄弟十人,卻又少他一箇了。」因向伯爵說:「出月初三日,又是會期,咱每少不得又要煩大官人這裡破費,兄弟們頑耍一日哩。」西門慶便道:「正是,我剛纔正對房下說來,咱兄弟們似這等會來會去,無過只是吃酒頑耍,不着一箇切實,倒不如尋一箇寺院裡,寫上一箇疏頭,結拜做了兄弟,到後日彼此扶持,有箇傍靠。到那日,咱少不得要破些銀子,買辦三牲,衆兄弟也便隨多少各出些分資。不是我科派你們,這結拜的事,各人出些,也見些情分。」伯爵連忙道:「哥說的是。婆兒燒香當不的老子念佛,各自要儘自的心。只是俺衆人們,老鼠尾巴生瘡兒——有膿也不多。」西門慶笑道:「恠狗才,誰要你多來!你說這話。」謝希大道:「結拜須得十箇方好。如今卜志道兄弟沒了,卻教誰補?」西門慶沉吟了一囘,說道:「咱這間壁花二哥,原是花太監姪兒,手裡肯使一股濫錢,常在院中走動。他家後邊院子與咱家只隔着一層壁兒,{\pangpi{伏脈。}}與我甚說得來,咱不如叫小厮邀他邀去。」應伯爵拍着手道:「敢就是在院中包着吳銀兒的花子虛麼?」西門慶道:「正是他!」伯爵笑道:「哥,快叫那箇大官兒邀他去。{\pangpi{等不得了。}}與他徃來了,咱到日後,敢又有一箇酒碗兒。」西門慶笑道:「傻花子,你敢害饞癆痞哩,說着的是吃。」大家笑了一囘。西門慶旋叫過玳安兒來說:「你到間壁花家去,對你花二爹說,如此這般:『俺爹到了出月初三日,要結拜十兄弟,敢叫我請二爹上會哩。』看他怎的說,你就來囘我話。你二爹若不在家,就對他二娘說罷。」玳安兒應諾去了。伯爵便道:「到那日還在哥這裡是,還在寺院裡好?」希大道:「咱這裡無過只兩箇寺院,僧家便是永福寺,道家便是玉皇廟。{\pangpi{又伏永福寺、玉皇廟。}}這兩箇去處,隨分那裡去罷。」西門慶道:「這結拜的事,不是僧家管的,那寺裡和尚,我又不熟,倒不如玉皇廟吳道官與我相熟,他那裡又寬展又幽靜。」伯爵接過來道:「哥說的是,敢是永福寺和尚倒和謝家嫂子相好,故要薦與他去的。」希大笑罵道:「老花子,一件正事,說說就放出屁來了。」

正說笑間,只見玳安兒轉來了,因對西門慶說道:「他二爹不在家,俺對他二娘說來。二娘聽了,好不歡喜,{\pangpi{伏脈。}}說道:『既是你西門爹攜帶你二爹做兄弟,那有箇不來的。{\meipi{只恐攜帶二爹,便要插戴二娘。}}等來家我與他說,至期以定攛掇他來,多拜上爹。』又與了小的兩件茶食來了。」{\pangpi{閑處都韻。}}西門慶對應、謝二人道:「自這花二哥,倒好箇伶俐標致娘子兒。」{\pangpi{伏脈。}}

說畢,又拏一盞茶吃了,二人一齊起身道:「哥,別了罷,咱好去通知衆兄弟,糾他分資來。哥這裡先去與吳道官說聲。」西門慶道:「我知道了,我也不留你罷。」於是一齊送出大門來。應伯爵走了幾步,迴轉來道:「那日可要叫唱的?」西門慶道:「這也罷了,弟兄們說說笑笑,到有趣些。」說畢,伯爵舉手,和希大一路去了。

話休饒舌,撚指過了四五日,卻是十月初一日。西門慶早起,剛在月娘房裡坐的,只見一箇纔留頭的小厮兒,手裡拏着箇描金退光拜匣,走將進來,向西門慶磕了一箇頭兒,立起來站在傍邊說道:「俺是花家,俺爹多拜上西門爹。

那日西門爹這邊叫大官兒請俺爹去,俺爹有事出門了,不曾當面領教的。聞得爹這邊是初三日上會,俺爹特使小的先送這些分資來,說爹這邊胡亂先用着,等明日爹這裡用過多少派開,該俺爹多少,再補過來便了。」西門慶拏起封袋一看,簽上寫着「分資一兩」,便道:「多了,不消補的。到後日叫爹莫徃那去,起早就要同衆爹上廟去。」那小厮兒應道:「小的知道。」剛待轉身,被吳月娘喚住,{\pangpi{臨去秋波。}}叫大丫頭玉簫在食籮裡揀了兩件蒸酥菓餡兒與他。因說道:「這是與你當茶的。你到家拜上你家娘,{\pangpi{想必要結姊妹。}}你說西門大娘說,遲幾日還要請娘過去坐半日兒哩。」那小厮接了,又磕了一箇頭兒,應着去了。

西門慶纔打發花家小厮出門,只見應伯爵家應寶夾着箇拜匣,玳安兒引他進來見了,磕了頭,說道:「俺爹糾了衆爹們分資,叫小的送來,爹請收了。」西門慶取出來看,共總八封,也不拆看,都交與月娘,道:「你收了,到明日上廟,好湊着買東西。」說畢,打發應寶去了。立起身到那邊看卓二姐。剛走到坐下,只見玉簫走來,說道:「娘請爹說話哩。」{\pangpi{餘波。}}西門慶道:「怎的起先不說來?」隨即又到上房,看見月娘攤着些紙包在面前,指着笑道:「你看這些分子,止有應二的是一錢二分八成銀子,其餘也有三分的,也有五分的,都是些紅的、黃的,倒象金子一般。咱家也曾沒見這銀子來,收他的也汙箇名,不如掠還他罷。」西門慶道:「你也耐煩,丟着罷,咱多的也包補,在乎這些?」說着一直徃前去了。

到了次日初二日,西門慶稱出四兩銀子,叫家人來興兒買了一口豬、一口羊、五六罈金華酒和香燭紙劄、雞鴨案酒之物,又封了五錢銀子,旋叫了大家人來保和玳安兒、來興三箇:「送到玉皇廟去,對你吳師父說,俺爹明日結拜兄弟,要勞師父做紙疏辭,晚夕就在師父這裡散福。煩師父與俺爹預備預備,俺爹明早便來。」只見玳安兒去了一會,來囘說:「已送去了,吳師父說知道了。」

須臾,過了初二,次日初三早,西門慶起來梳洗畢,叫玳安兒:「你去請花二爹,到咱這裡吃早飯,一同好上廟去。一發到應二叔家,叫他催催衆人。」

玳安應諾去,剛請花子虛到來,只見應伯爵和一班兄弟也來了,卻正是前頭所說的這幾箇人。為頭的便是應伯爵,謝希大、孫天化、祝念實、吳典恩、雲理守、常峙節、白賚光,連西門慶、花子虛共成十箇。進門來一齊籮圈作了一箇揖。伯爵道:「咱時候好去了。」西門慶道:「也等吃了早飯着。」便叫:「拏茶來。」一面叫:「看菜兒。」須臾,吃畢早飯,西門慶換了一身衣服,打選衣帽光鮮,一齊徑徃玉皇廟來。不到數裡之遙,早望見那座廟門,造得甚是雄峻。但見:

殿宇嵯峨,宮墻高聳。正面前,起着一座墻門八字,一帶都粉赭色紅泥;進裡邊,列着三條甬道川紋,四方都砌水痕白石。正殿上金碧輝煌,兩廊下簷阿峻峭。三清聖祖莊嚴寶相列中央,太上老君背倚青牛居後殿。

進入第二重殿後,轉過一重側門,卻是吳道官的道院。進的門來,兩下都是些瑤草琪花,蒼松翠竹。西門慶擡頭一看,只見兩邊門楹上貼着一副對聯道:

洞府無窮歲月,壺天別有乾坤。

上面三間敞廳,卻是吳道官朝夕做作功課的所在。當日鋪設甚是齊整,上面掛的是昊天金闕玉皇上帝,兩邊列着的紫府星官,側首掛着便是馬、趙、溫、關四大元帥。{\pangpi{伏脈。}}當下吳道官卻又在經堂外躬身迎接。西門慶一起人進入裡邊,獻茶已罷,衆人都起身,四圍觀看。白賚光攜着常峙節手兒,從左邊看將過來,一到馬元帥面前,見這元帥威風凜凜,相貌堂堂,面上畫着三隻眼睛,便叫常峙節道:「哥,這卻是怎的說?如今世界,開隻眼閉隻眼兒便好,還經得多出隻眼睛看人破綻哩!」應伯爵聽見,走過來道:「呆兄弟,他多隻眼兒看你倒不好麼?」{\pangpi{雋。}}衆人笑了。常峙節便指着下首溫元帥道:「二哥,這箇通身藍的,卻也古恠,敢怕是盧杞的祖宗。」伯爵笑着猛叫道:「吳先生你過來,我與你說箇笑話兒。」那吳道官真箇走過來聽他。伯爵道:「一箇道家死去,見了閻王,閻王問道:『你是什麼人?』道者說:『是道士。』閻王叫判官查他,果系道士,且無罪孽。『這等,放他還魂。』只見道士轉來,路上遇着一箇染房中的博士,原認得的,那博士問道:『師父,怎生得轉來?』道者說:『我是道士,所以放我轉來。』那博士記了,見閻王時也說是道士。

那閻王叫查他身上,只見伸出兩隻手來,是藍的。問其何故,那博士打着宣科的聲音道:『曾與溫元帥搔胞。』」說的衆人大笑。一面又轉過右首來,見下首供着箇紅臉的,卻是關帝。上首又是一箇黑麵的,是趙元壇元帥,身邊畫着一箇大老虎。白賚光指着道:「哥,你看這老虎,難道是吃素的,隨着人不妨事麼?」伯爵笑道:「你不知,這老虎是他一箇親隨的伴當兒哩。」謝希大聽得走過來,伸出舌頭道:「這等一箇伴當隨着,我一刻也成不的。我不怕他要吃我麼?」伯爵笑着向西門慶道:「這等,虧他怎地過來!」西門慶道:「卻怎的說?」伯爵道:「子純一箇要吃他的伴當隨不的,似我們這等七八箇要吃你的隨你,卻不嚇死了你罷了。」{\pangpi{趣。}}說着,一齊正大笑時,吳道官走過來,說道:「官人們講這老虎,{\meipi{落脈無痕,手筆入化。}}只俺這清河縣,這兩日好不受這老虎的虧!徃來的人也不知吃了多少,就是獵戶,也害死了十來人。」

西門慶問道:「是怎的來?」吳道官道:「官人們還不知道。不然我也不曉的,只因日前一箇小徒,到滄州橫海郡柴大官人那裡去化些錢糧,{\pangpi{照應。}}整整住了五七日,纔得過來。俺這清河縣近着滄州路上,有一條景陽岡,岡上新近出了一箇弔睛白額老虎,時常出來吃人。客商過徃,好生難走,必須要成群結夥而過。如今縣裡現出着五十兩賞錢,要拏他,白拏不得。可憐這些獵戶,不知吃了多少限棒哩!」白賚光跳起來道:「咱今日結拜了,明日就去拏他,也得些銀子使。」西門慶道:「你性命不值錢麼?」白賚光笑道:「有了銀子,要性命怎的!」衆人齊笑起來。應伯爵道:「我再說箇笑話你們聽:一箇人被虎啣了,他兒子要救他,拏刀去殺那虎。這人在虎口裡叫道:『兒子,你省可而的砍,怕砍壞了虎皮。』」{\meipi{這纔是要錢不要命。}}說着衆人哈哈大笑。

只見吳道官打點牲礼停當,來說道:「官人們燒紙罷。」一面取出疏紙來,說:「疏已寫了,只是那位居長?那位居次?排列了,好等小道書寫尊諱。」衆人一齊道:「這自然是西門大官人居長。」{\pangpi{怎見得?}}西門慶道:「這還是叙齒,應二哥大如我,是應二哥居長。」伯爵伸着舌頭道:「爺,可不折殺小人罷了!{\meipi{小人一幅行樂圖。}}如今年時,只好叙些財勢,那裡好叙齒!{\pangpi{可憐!可嘆!}}若叙齒,這還有大如我的哩。且是我做大哥,有兩件不妥:第一不如大官人有威有德,{\pangpi{要緊話。}}衆兄弟都服你;第二我原叫做應二哥,如今居長,卻又要叫應大哥,倘或有兩箇人來,一箇叫『應二哥』,一箇叫『應大哥』,我還是應『應二哥』,應『應大哥』呢?」西門慶笑道:「你這搊斷腸子的,單有這些閑說的!」謝希大道:「哥,休推了。」西門慶再三謙讓,被花子虛、應伯爵等一干人逼勒不過,只得做了大哥。第二便是應伯爵,第三謝希大,第四讓花子虛,有錢做了四哥。其餘挨次排列。吳道官寫完疏紙,於是點起香燭,衆人依次排列。吳道官伸開疏紙,朗聲讀道:

維大宋國山東東平府清河縣信士西門慶、應伯爵、謝希大、花子虛、孫天化、祝念實、雲理守、吳典恩、常峙節、白賚光等,是日沐手焚香請旨。伏為桃園義重,衆心仰慕而敢效其風;管鮑情深,各姓追維而欲同其志。況四海皆可兄弟,豈異姓不如骨肉?是以涓今政和年月日,營備豬羊牲礼,鸞馭金資,端叩齋壇,虔誠請禱,拜投昊天金闕玉皇上帝,五方值日功曹,本縣城隍社令,過徃一切神衹,仗此真香,普同鑑察。伏念慶等生雖異日,死冀同時,期盟言之永固;安樂與共,顛沛相扶,思締結以常新。必富貴常念貧窮,乃始終有所依倚。情共日徃以月來,誼若天高而地厚。伏願自盟以後,相好無尤,更祈人人增有永之年,戶戶慶無疆之福。凡在時中,全叨覆庇,謹疏。

政和年月日文疏

吳道官讀畢,衆人拜神已罷,依次又在神前交拜了八拜。然後送神,焚化錢紙,收下福礼去。不一時,吳道官又早叫人把豬羊卸開,雞魚菓品之類整理停當,俱是大碗大盤,擺下兩桌,西門慶居於首席,其餘依次而坐,吳道官側席相陪。須臾,酒過數巡,衆人猜枚行令,耍笑鬨堂,不必細說。正是:

纔見扶桑日出,又看曦馭啣山。醉後倩人扶去,樹梢新月纔彎。

飲酒熱鬧間,只見玳安兒來,附西門慶耳邊說道:「娘叫小的接爹來了,說三娘今日發昏哩,請爹早些家去。」西門慶隨即立起來說道:「不是我搖席破座,委的我第三箇小妾十分病重,咱先去休。」只見花子虛道:「咱與哥同路,咱兩箇一搭兒去罷。」伯爵道:「你兩箇財主的都去了,{\pangpi{口吻極肖。}}丟下俺們怎的?花二哥你再坐囘去。」西門慶道:「他家無人,俺兩箇一搭裡去的是,省和他嫂子疑心。」玳安兒道:「小的來時,二娘也叫天福兒備馬來了。」

只見一箇小厮走近前,向子虛道:「馬在這裡,娘請爹家去哩。」於是二人一齊起身,向吳道官致謝打攪,與伯爵等舉手道:「你們自在耍耍,我們去也。」

說着出門上馬去了。單留下這幾箇嚼倒泰山不謝土的,在廟流連痛飲,不題。

卻表西門慶到家,與花子虛別了,進來問吳月娘:「卓二姐怎的發昏來?」月娘道:「我說一箇病人在家,恐怕你搭了這起人,又纏到那裡去了,故此叫玳安兒恁地說。只是一日日覺得重來,你也要在家看他的是。」西門慶聽了,徃那邊去看,連日在家守着,不題。

卻說光陰過隙,又早是十月初十外了。一日,西門慶正使小厮請太醫診視卓二姐病症,剛走到廳上,只見應伯爵笑嘻嘻走將進來。西門慶與他作了揖,讓他坐了。伯爵道:「哥,嫂子病體如何?」西門慶道:「多分有些不起解,不知怎的好。」因問:「你們前日多咱時分纔散?」伯爵道:「承吳道官再三苦留,散時也有二更多天氣。咱醉的要不的,倒是哥早早來家的便益些。」西門慶因問道:「你吃了飯不曾?」伯爵不好說不曾吃,因說道:「哥,你試猜。」西門慶道:「你敢是吃了?」伯爵掩口道:「這等猜不着。」{\pangpi{妙。}}西門慶笑道:「恠狗才,不吃便說不曾吃,有這等張致的!」一面叫小厮:「看飯來,咱與二叔吃。」伯爵笑道:「不然咱也吃了來了,咱聽得一件稀罕的事兒,來與哥說,要同哥去瞧瞧。」西門慶道:「甚麼稀罕的?」伯爵道:「就是前日吳道官所說的景陽岡上那隻大蟲,昨日被一箇人一頓拳頭打死了。」西門慶道:「你又來胡說了,咱不信。」伯爵道:「哥,說也不信,你聽着,等我細說。」

於是手舞足蹈說道:「這箇人有名有姓,姓武名松,排行第二。」先前怎的避難在柴大官人庄上,後來怎的害起病來,病好了又怎的要去尋他哥哥,過這景陽岡來,怎的遇了這虎,怎的怎的被他一頓拳脚打死了。一五一十說來,就像是親見的一般,又象這隻猛虎是他打的一般。說畢,西門慶搖着頭兒道:「既恁的,咱與你吃了飯同去看來。」伯爵道:「哥,不吃罷,怕誤過了。咱們倒不如大街上酒樓上去坐罷。」只見來興兒來放桌兒,西門慶道:「對你娘說,叫別要看飯了,拏衣服來我穿。」

須臾,換了衣服,與伯爵手拉着手兒同步出來。路上撞着謝希大,笑道:「哥們,敢是來看打虎的麼?」西門慶道:「正是。」謝希大道:「大街上好挨擠不開哩。」於是一同到臨街一箇大酒樓上坐下。不一時,只聽得鑼鳴鼓響,衆人都一齊瞧看。只見一對對纓槍的獵戶,擺將過來,後面便是那打死的老虎,好相錦布袋一般,四箇人還擡不動。末後一匹大白馬上,坐着一箇壯士,就是那打虎的這箇人。西門慶看了,咬着指頭道:「你說這等一箇人,若沒有千百斤水牛般氣力,怎能勾動他一動兒。」{\meipi{伏數語,便挑動酒樓之避,一針不漏。}}這裡三箇兒飲酒評品,按下不題。

單表迎來的這箇壯士怎生模樣?但見:

雄軀凜凜,七尺以上身材;闊面稜稜,二十四五年紀。雙目直豎,遠望處猶如兩點明星;兩手握來,近覷時好似一雙鐵碓。脚尖飛起,深山虎豹失精魂;拳手落時,窮谷熊羆皆䘮魄。頭戴着一頂萬字頭巾,上簪兩朵銀花;身穿着一領血腥衲襖,披着一方紅錦。

這人不是別人,就是應伯爵說所陽谷縣的武二郎。只為要來尋他哥子,不意中打死了這箇猛虎,被知縣迎請將來。衆人看着他迎入縣裡。卻說這時正值知縣陞堂,武松下馬進去,扛着大蟲在廳前。知縣看了武松這般模樣,心中自忖道:「不恁地,怎打得這箇猛虎!」便喚武松上廳。叅見畢,將打虎首尾訴說一遍。兩邊官吏都嚇呆了。知縣在廳上賜了三盃酒,將庫中衆土戶出納的賞錢五十兩,賜與武松。武松稟道:「小人托賴相公福廕,偶然僥倖打死了這箇大蟲,非小人之能,如何敢受這些賞賜!衆獵戶因這畜生,受了相公許多責罰,何不就把賞給散與衆人,也顯得相公恩典。」{\meipi{不貪財,不伐能,不吝□。}}知縣道:「既是如此,任從壯士處分。」武松就把這五十兩賞錢,在廳上散與衆獵戶去了。知縣見他仁德忠厚,又是一條好漢,有心要擡舉他,便道:「你雖是陽谷縣人氏,與我這清河縣只在咫尺。我今日就叅你在我縣裡做箇巡捕的都頭,專在河東水西擒拏賊盜,你意下如何?」武松跪謝道:「若蒙恩相擡舉,小人終身受賜。」

知縣隨即喚押司立了文案,當日便叅武松做了巡捕都頭。衆里長大戶都來與武松作賀慶喜,連連吃了數日酒。正要囘陽谷縣去抓尋哥哥,不料又在清河縣做了都頭,卻也歡喜。那時傳得東平一府兩縣,皆知武松之名。正是:

壯士英雄藝畧芳,挺身直上景陽岡。醉來打死山中虎,自此聲名播四方。

卻說武松一日在街上閑行,只聽背後一箇人叫道:「兄弟,知縣相公擡舉你做了巡捕都頭,怎不看顧我!」武松囘頭見了這人,不覺的:

欣從額角眉邊出,喜逐歡容笑口開。

這人不是別人,卻是武松日常間要去尋他的嫡親哥哥武大。卻說武大自從兄弟分別之後,因時遭饑饉,搬移在清河縣紫石街賃房居住。人見他為人懦弱,模樣猥蕤,起了他箇渾名,叫做『三寸丁谷樹皮』,俗語言其身上粗糙,頭臉窄狹故也。只因他這般軟弱樸實,多欺侮也。這也不在話下。且說武大無甚生意,終日挑担子出去街上賣炊餅度日,不幸把渾家故了,丟下箇女孩兒,年方十二歲,名喚迎兒,爺兒兩箇過活。那消半年光景,又消折了資本,移在大街坊張大戶家臨街房居住。張宅家下人見他本分,常看顧他,照顧他依舊賣些炊餅。閑時在鋪中坐地,武大無不奉承。因此張宅家下人箇箇都歡喜,在大戶面前一力與他說方便。因此大戶連房錢也不問武大要。卻說這張大戶有萬貫家財,百間房屋,年約六旬之上,身邊寸男尺女皆無。媽媽餘氏,主家嚴厲,房中並無清秀使女。只因大戶時常拍胸嘆氣道:「我許大年紀,又無兒女,雖有幾貫家財,終何大用。」媽媽道:「既然如此說,我叫媒人替你買兩箇使女,早晚習學彈唱,服侍你便了。」大戶聽了大喜,謝了媽媽。過了幾時,媽媽果然叫媒人來,與大戶買了兩箇使女,一箇叫做潘金蓮,一箇喚做白玉蓮。玉蓮年方二八,樂戶人家出身,生得白淨小巧。這潘金蓮卻是南門外潘裁的女兒,排行六姐。因他自幼生得有些姿色,纏得一雙好小脚兒,{\pangpi{是禍根。}}所以就叫金蓮。他父親死了,做娘的度日不過,從九歲賣在王招宣府裡,{\pangpi{伏脈。}}習學彈唱,閑常又教他讀書寫字。他本性機變伶俐,不過十二三,就會描眉畫眼,傅粉施朱,品竹彈絲,女工針指,知書識字,梳一箇纏髻兒,着一件扣身衫子,做張做致,喬模喬樣。{\pangpi{一生伎倆。}}到十五歲的時節,王招宣死了,潘媽媽爭將出來,三十兩銀子轉賣於張大戶家,與玉蓮同時進門。大戶教他習學彈唱,金蓮原自會的,甚是省力。金蓮學琵琶,玉蓮學箏,這兩箇同房歇臥。主家婆餘氏初時甚是擡舉二人,與他金銀首飾裝束身子。後日不料白玉蓮死了,止落下金蓮一人,長成一十八歲,出落的臉襯桃花,眉彎新月。張大戶每要收他,只礙主家婆厲害,不得到手。{\pangpi{倒好。}}一日主家婆隣家赴席不在,大戶暗把金蓮喚至房中,遂收用了。正是:

莫訝天台相見晚,劉郎還是老劉郎。{\pangpi{趣。}}

大戶自從收用金蓮之後,不覺身上添了四五件病症。{\pangpi{神效。}}端的悄五件?

第一腰便添疼,第二眼便添淚,第三耳便添聾,第四鼻便添涕,第五尿便添滴。

自有了這幾件病後,主家婆頗知其事,與大戶嚷罵了數日,將金蓮百般苦打。大戶知道不容,卻賭氣倒賠了房奩,要尋嫁得一箇相應的人家。大戶家下人都說武大忠厚,見無妻小,又住着宅內房兒,堪可與他。這大戶早晚還要看覷此女,{\pangpi{有理。}}因此不要武大一文錢,白白地嫁與他為妻。這武大自從娶了金蓮,大戶甚是看顧他。若武大沒本錢做炊餅,大戶私與他銀兩。武大若挑担兒出去,大戶候無人,便踅入房中與金蓮厮會。武大雖一時撞見,原是他的行貨,不敢聲言。朝來暮徃,也有多時。忽一日,大戶得患陰寒病症,嗚呼死了。主家婆察知其事,怒令家僮將金蓮、武大即時趕出。武大故此遂尋了紫石街西王皇親房子,賃內外兩間居住,依舊賣炊餅。原來這金蓮自嫁武大,見他一味老實,人物猥瑣,甚是憎嫌,{\pangpi{自然。}}常與他合氣。報怨大戶:「普天世界斷生了男子,何故將我嫁與這樣箇貨!每日牽着不走,打着倒退的,只是一味𠳹酒,着緊處卻是錐鈀也不動。奴端的那世裡悔氣,卻嫁了他!是好苦也!」常無人處,唱箇《山坡羊》為證:

想當初,姻緣錯配,奴把你當男兒漢看覷。不是奴自己誇獎,他烏鴉怎配鸞鳳對!奴真金子埋在土裡,他是塊高號銅,怎與俺金色比!他本是塊頑石,有甚福抱着我羊脂玉體!好似糞土上長出靈芝。奈何,隨他怎樣,到底奴心不美。聽知:奴是塊金磚,怎比泥土基!

看官聽說:但凡世上婦女,若自己有幾分顏色,所稟伶俐,配箇好男子便罷了,若是武大這般,雖好殺,也未免有幾分憎嫌。{\pangpi{況不好乎!}}自古佳人才子相配着的少,買金偏撞不着賣金的。武大每日自挑担兒出去賣炊餅,到晚方歸。那婦人每日打發武大出門,只在簾子下磕瓜子兒,{\pangpi{好消遣。}}一徑把那一對小金蓮故露出來,勾引浮浪子弟,日逐在門前彈胡博詞,撒謎語,叫唱:「一塊好羊肉,如何落在狗嘴裡?」油似滑的言語,無般不說出來。因此武大在紫石街又住不牢,要徃別處搬移,與老婆商議。婦人道:「賊餛飩不曉事的,你賃人家房住,淺房淺屋,可知有小人羅唣!不如添幾兩銀子,看相應的,典上他兩間住,卻也氣概些,免受人欺侮。」武大道:「我那裡有錢典房?」婦人道:「呸!濁才料,你是箇男子漢,倒擺佈不開,常交老娘受氣。沒有銀子,把我的釵梳湊辦了去,有何難處!過後有了再治不遲。」{\meipi{此處亦復能賢。}}武大聽老婆這般說,當下湊了十數兩銀子,典得縣門前樓上下兩層四間房屋居住。第二層是樓,兩箇小小院落,甚是乾淨。武大自從搬到縣西街上來,照舊賣炊餅過活。不想這日撞見自己嫡親兄弟。當日兄弟相見,心中大喜。一面邀請到家中,讓至樓上坐,房裡喚出金蓮來,與武松相見。因說道:「前日景陽岡上打死大蟲的,便是你的小叔。{\pangpi{好不氣概。}}今新充了都頭,是我一母同胞兄弟。」{\pangpi{值得賣弄。}}那婦人叉手向前,便道:「叔叔萬福。」武松施礼,倒身下拜。婦人扶住武松道:「叔叔請起,折殺奴家。」武松道:「嫂嫂受礼。」兩箇相讓了一囘,都平磕了頭起來。少頃,小女迎兒拏茶,二人吃了。武松見婦人十分妖嬈,只把頭來低着。{\pangpi{不老氣。}}不多時,武大安排酒飯,款待武松。說話中間,武大下樓買酒菜去了,丟下婦人,獨自在樓上陪武松坐地。看了武松身材凜凜,相貌堂堂,{\meipi{此想入神。}}又想他打死了那大蟲,畢竟有千百斤氣力。{\pangpi{慧想,慧想!}}口中不說,心下思量道:「一母所生的兄弟,怎生我家那身不滿尺的丁樹,三分似人,七分似鬼,奴那世裡遭瘟撞着他來!如今看起武松這般人壯健,何不叫他搬來我家住?想這段姻緣卻在這裡了。」{\pangpi{且看。}}於是一面堆下笑來,問道:「叔叔你如今在那裡居住?每日飯食誰人整理?」武松道:「武二新充了都頭,逐日答應上司,別處住不方便,胡亂在縣前尋了箇下處,每日撥兩箇土兵伏侍做飯。」婦人道:「叔叔何不搬來家裡住?省的在縣前土兵服侍做飯醃臢。一家裡住,早晚要些湯水吃時,也方便些。就是奴家親自安排與叔叔吃,也乾淨。」武松道:「深謝嫂嫂。」婦人又道:「莫不別處有嬸嬸?{\pangpi{細心。}}可請來厮會。」武松道:「武二並不曾婚娶。」婦人道:「叔叔青春多少?」武松道:「虛度二十八歲。」婦人道:「原來叔叔倒長奴三歲。叔叔今番從那裡來?」武松道:「在滄州住了一年有餘,只想哥哥在舊房居住,不道移在這裡。」婦人道:「一言難盡。自從嫁得你哥哥,吃他忒善了,被人欺負,纔到這裡來。若是叔叔這般雄壯,{\pangpi{二字得心應口。}}誰敢道箇不字!」武松道:「家兄從來本分,不似武松撒潑。」{\pangpi{和盤托出。}}婦人笑道:「怎的顛倒說!常言『人無剛強,安身不長』。奴家平生性快,看不上那三打不囘頭,四打和身轉的」武松道:「家兄不惹禍,免得嫂嫂憂心。」

二人在樓上一遞一句的說。有詩為證:

叔嫂萍蹤得偶逢,嬌嬈偏逞秀儀容。私心便欲成歡會,暗把邪言釣武松。

話說金蓮陪着武松正在樓上說話未了,只見武大買了些肉菜菓餅歸家。放在廚,走上樓來,叫道:「大嫂,你且下來則箇。」那婦人應道:「你看那不曉事的!叔叔在此無人陪侍,卻交我撇了下去。」{\pangpi{哥哥也陪得,不必定要嫂嫂。}}武松道:「嫂嫂請方便。」婦人道:「何不去間壁請王乾娘來安排?{\pangpi{伏脈。}}只是這般不見便。」武大便自去央了間壁王婆來。安排端正,都拏上樓來,擺在桌子上,無非是些魚肉菓菜點心之類。隨即燙酒上來。武大叫婦人坐了主位,武松對席,武大打橫。三人坐下,把酒來斟,武大篩酒在各人面前。那婦人拏起酒來道:「叔叔休恠,沒甚管待,請盃兒水酒。」武松道:「感謝嫂嫂,休這般說。」武大隻顧上下篩酒,那婦人笑容可掬,滿口兒叫:「叔叔,怎的肉菓兒也不揀一筯兒?」{\pangpi{還有肉卷兒哩。}}揀好的遞將過來。武松是箇直性的漢子,只把做親嫂嫂相待。誰知這婦人是箇使女出身,慣會小意兒。亦不想這婦人一片引人心。那婦人陪武松吃了幾盃酒,一雙眼只看着武松的身上。武松吃他看不過,只得倒低了頭。{\pangpi{二官太嫩。}}吃了一歇,酒闌了,便起身。武大道:「二哥沒事,再吃幾盃兒去。」武松道:「生受,我再來望哥哥嫂嫂罷。」都送下樓來。出的門外,婦人便道:「叔叔是必上心搬來家裡住,若是不搬來,俺兩口兒也吃別人笑話。親兄弟難比別人,與我們爭口氣,也是好處。」{\pangpi{大義激之。}}武松道:「既是嫂嫂厚意,今晚有行李便取來。」婦人道:「奴這裡等候哩!」正是:

滿前野意無人識,幾點碧桃春自開。

