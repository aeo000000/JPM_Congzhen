\part*{{\titlename}卷之九}
\addcontentsline{toc}{part}{{\titlename}卷之九}


\includepdf[pages={161,162},fitpaper=false]{tst.pdf}
\chapter*{第八十一囘 韓道國拐財遠遁 湯來保欺主背恩}
\addcontentsline{toc}{chapter}{第八十一囘 韓道國拐財遠遁 湯來保欺主背恩}
\markboth{{\titlename}卷之九}{第八十一囘 韓道國拐財遠遁 湯來保欺主背恩}


詩曰:

\begin{myquote}
燕入非傍舍,鷗歸只故池。\\斷橋無復板,臥柳自生枝。\\遂有山陽作,多慚鮑叔知。\\素交零落盡,白首淚雙垂。
\end{myquote}

話說韓道國與來保,自從拏着西門慶四千兩銀子,江南買貨物,到於揚州,抓尋苗青家內宿歇。苗青見了西門慶手劄,想他活命之恩,盡力趨奉。又討了一個女子,名喚楚雲,養在家裡,要送與西門慶,以報其恩。韓道國與來保兩箇且不置貨,成日尋花問柳,飲酒宿婦。{\pangpi{從來夥計皆如此。}}只到初冬天氣,景物蕭瑟,不勝旅思。方纔將銀徃各處買布疋,裝在揚州苗青家安下,待貨物買完起身。先是韓道國請個表子,是揚州舊院王玉枝兒,來保便請了林彩虹妹子小紅。一日,請揚州鹽客王海峰和苗青遊寶應湖,遊了一日,歸到院中。又値玉枝兒鴇子生日,{\pangpi{偏撞着。}}這韓道國又邀請衆人,擺酒與鴇子王一媽做生日。使後生胡秀,請客商汪東橋與錢晴川兩箇,白不見到。不一時,汪東橋與錢晴川就同王海峰來了。至日落時分,胡秀纔來,被韓道國帶酒罵了兩句,說:「這厮不知在那裡𠳹酒,𠳹到這咱纔來,口裡噴出來的酒氣。客人到先來了這半日,你不知那裡來,我到明日定和你算帳。」那胡秀把眼斜瞅着他,走到下邊,口裡喃喃吶吶,說:「你罵我,你家老婆在家裡仰𢵞着掙,你在這裡合蓬着丟!宅里老爹包着你家老婆,㒲的不値了,纔交你領本錢出來做買賣。你在這裡快活,你老婆不知怎麼受苦哩!得人不化白出你來,你落得為人就勾了。」{\meipi{當日西門慶圖小郎到南邊好看,誰知反弄的不好,看世事類然。}}對玉枝兒鴇子只顧說。鴇子便拉出他院子裡,說:「胡官人,你醉了,你徃房裡睡去罷。」那胡秀大喓大喝,白不肯進房。不料韓道國正陪衆客商在席上吃酒,聽見胡秀口內放屁辣臊,心中大怒,走出來踢了他兩脚,罵道:「賊野囚奴,我有了五分銀子,僱你一日,怕尋不出人來!」{\pangpi{罵不過。}}即時趕他去。那胡秀那裡肯出門,在院子內聲叫起來,說道:「你如何趕我?我沒壞了管帳事!你倒養老婆,倒趕我,看我到家說不說!」{\meipi{行止不端,便不能服人。}}被來保勸住韓道國,一手扯他過一邊,說道:「你這狗骨頭,原來這等酒硬!」那胡秀道:「保叔,你老人家休管他。我吃甚麼酒來,我和他做一做。」被來保推他徃屋裡挺覺去了。正是:

\begin{myquote}
酒不醉人人自醉,色不迷人人自迷。
\end{myquote}

來保打發胡秀房裡睡去不題。韓道國恐怕衆客商恥笑,和來保席上觥籌交錯,遞酒鬨笑。林彩虹、小紅姊妹二人並王玉枝兒三個唱的,彈唱歌舞,花攢錦簇,行令猜枚,吃至三更方散。次日,韓道國要打胡秀,胡秀說:「小的通不曉一字。」{\pangpi{收得妙。}}道國被苗青做好做歹勸住了。

話休饒舌。有日貨物置完,打包裝載上船。不想苗青討了送西門慶的那女子楚雲,忽生起病來,動身不得。{\pangpi{造化。}}苗青說:「等他病好了,我再差人送了來罷。」只打點了些人事禮物,抄寫書帳,打發二人並胡秀起身。王玉枝並林彩虹姊妹,少不的置酒馬頭,作別餞行。從正月初十日起身,一路無詞。

一日到臨清閘上,這韓道國正在船頭站立,忽見街坊嚴四郎,從上流坐船而來,徃臨清接官去。看見韓道國,舉手說:「韓西橋,你家老爹從正月間沒了。」說畢,船行得快,就過去了。這韓道國聽了此言,遂安心在懷,瞞着來保不說。不想那時河南、山東大旱,赤地千里,田蠶荒蕪不收,棉花布價一時踴貴,每疋布帛加三利息,各處鄉販都打着銀兩遠接,在臨清一帶馬頭迎着客貨而買。韓道國便與來保商議:「船上布貨約四千餘兩,見今加三利息,不如且賣一半,又便宜鈔關納稅,就到家發賣也不過如此。遇行市不賣,誠為可惜。」來保道:「夥計所言雖是,誠恐賣了,一時到家,惹當家的見恠,如之奈何?」韓道國便說:「老爹見恠,都在我身上。」來保強不過他,就在馬頭上,發賣了一千兩布貨。韓道國說:「雙橋,你和胡秀在船上等着納稅,我打旱路同小郎王漢,打着這一千兩銀子,先去報老爹知道。」來保道:「你到家,好歹討老爹一封書來,下與鈔關錢老爹,少納稅錢,先放船行。」韓道國應諾。同小郎王漢裝成馱垛,徃清河縣家中來。

有日進城,在甕城南門裡,日色漸落,忽撞遇看墳的張安,推着車輛酒米食鹽,正出南門。看見韓道國,便叫:「韓大叔,你來家了。」韓道國看見他帶着孝,問其故,張安說:「老爹死了,明日三月初九日斷七。大娘交我拏此酒米食盒徃墳上去,明日與老爹燒紙。」這韓道國聽了,說:「可傷,可傷!果然路上行人口似碑,話不虛傳。」打頭口徑進城中。到了十字街上,心中算計:「且住。有心要徃西門慶家去,況今他已死了,天色又晚,不如且歸家停宿一宵,和渾家商議了,明日再去不遲。」{\meipi{處義利之間,再算計不得,一算計便利重於義矣。}}於是和王漢打着頭口,徑到獅子街家中。二人下了頭口,打發趕脚人囘去,叫開門,王漢搬行李馱垛進入堂中。老婆一面迎接入門,拜了佛祖。王六兒替他脫衣坐下,丫頭點茶吃。韓道國先告訴徃囘一路之事,道:「我在路上撞遇嚴四哥與張安,纔知老爹死了。好好的,怎的就死了?」王六兒道:「天有不測風雲,人有暫時禍福。誰人保得無常!」韓道國一面把馱垛開啟,取出他江南置的許多衣裳細軟等物,並那一千兩銀子,一封一封都放在炕上。老婆開啟看,都是白光光雪花銀兩,便問:「這是那裡的?」韓道國說:「我在路上聞了信,就先賣了這一千兩銀子來了。」又取出兩包梯己銀子一百兩,因問老婆:「我去後,家中他也看顧你不曾?」王六兒道:「他在時倒也罷了,如今你這銀子還送與他家去?」韓道國道:「正是要和你商議,咱留下些,把一半與他如何?」老婆道:「呸,你這傻奴才料,這遭再休要傻了。如今他已是死了,這裡無人,咱和他有甚瓜葛?不急你送與他一半,交他招暗道兒,問你下落。到不如一狠二狠,把他這一千兩,咱顧了頭口,拐了上東京,投奔咱孩兒那裡。愁咱親家太師爺府中,安放不下你我!」{\meipi{僥西門慶之幸,而得竊太師之勢,轉欲倚太師之勢,而壓西門慶,小人心腸盡如此!}}韓道國道:「丟下這房子,急切打發不出去,怎了?」老婆道:「你看沒才料!何不叫將第二個來,{\pangpi{餘情不斷。}}留幾兩銀子與他,就叫他看守便了。等西門慶家人來尋你,保說東京咱孩兒叫了兩口去了。莫不他七個頭八個膽,敢徃太師府中尋咱們去?就尋去,你我也不怕他。」韓道國道:「爭奈我受大官人好處,怎好變心的?沒天理了!」{\pangpi{良心何嘗不在。}}老婆道:「自古有天理到沒飯吃哩。他佔用着老娘,使他這幾兩銀子,不差甚麼。{\pangpi{說來你自有理。}}想着他孝堂裡,我到好意備了一張插桌三牲,徃他家燒紙。他家大老婆那不賢良的淫婦,半日不出來,在屋裡罵的我好訕的。{\meipi{月娘從未罵人,只罵得王六兒幾句,便招怨失事,可見越是好人,越行惡事不得。}}我出又出不來,坐又坐不住,落後他第三個老婆出來陪我坐,我不去坐,就坐轎子來家了,想着他這個情兒,我也該使他這幾兩銀子。」一席話,說得韓道國不言語了。

夫妻二人,晚夕計議已定。到次日五更,叫將他兄弟韓二來,如此這般,叫他看守房子,又把與他一二十兩銀子盤纏。那二搗鬼千肯萬肯,說:「哥嫂只顧去,等我打發他。」這韓道國就把王漢小郎並兩箇丫頭,也跟他帶上東京去。僱了二十輛車,把箱籠細軟之物都裝在車上。投天明出西門,徑上東京去了。正是:

\begin{myquote}
撞碎玉籠飛彩鳳,頓開金鎖走蛟龍。
\end{myquote}

這裡韓道國夫婦東京去了不題。單表吳月娘次日帶孝哥兒,同孟玉樓、潘金蓮、西門大姐、奶子如意兒、女婿陳敬濟,徃墳上與西門慶燒紙。張安就告訴月娘,昨日撞見韓大叔來家一節,月娘道:「他來了,怎的不到我家來?只怕他今日來。」在墳上剛燒了紙,坐了沒多囘,老早就起身來家。使陳敬濟徃他家,「叫韓夥計去,問他船到那裡了?」初時叫着不聞人言,次則韓二出來,說:「俺姪女兒東京叫了哥嫂去了,船不知在那裡。」讓陳敬濟囘月娘。月娘不放心,使敬濟騎頭口徃河下尋船。去了一日,到臨清馬頭船上,尋着來保船隻。來保問:「韓夥計先打了一千兩銀子家去了。」敬濟道:「誰見他來?張安看見他進城,次日墳上來家,大娘使我問他去,他兩口子奪家連銀子都拐的上東京去了。如今爹死了,斷七過了,大娘不放心,使我來找尋船隻。」這來保口中不言,心內暗道:「這天殺,原來連我也瞞了,嗔道路上定要賣這一千兩銀子,乾淨要起毛心。正是人面咫尺,心隔千里。」這來保見西門慶已死,也安心要和他一路。把敬濟小夥兒引誘在馬頭上各唱店中、歌樓上飲酒,請表子頑耍。暗暗船上搬了八百兩貨物,卸在店家房內,封記了。一日鈔關上納了稅,放船過來,在新河口起脚裝車,徃清河縣城裡來,家中東廂房卸下。

自從西門慶死了,獅子街絲綿鋪已關了。對門段鋪,甘夥計、崔本賣了銀兩都交付明白,各辭歸房去了。房子也賣了,止有門首解當、生藥鋪,敬濟與傅夥計開着。原來這來保妻惠祥,有個五歲兒子,名僧寶兒。韓道國老婆王六兒有個姪女兒四歲,二人割衿做了親家。家中月娘通不知道。這來保交卸了貨物,就一口把事情都推在韓道國身上,說他先賣了二千兩銀子來家。{\pangpi{自然之理。}}那月娘再三使他上東京,問韓道國銀子下落。被他一頓話說:「咱早休去!一個太師老爺府中,誰人敢到?沒的招事惹非。得他不來尋你,咱家念佛。到沒的招惹蝨子頭上撓!」月娘道:「翟親家也虧咱家替他保親,莫不看些分上兒。」來保道:「他家女兒見在他家得時,他敢只護他娘老子,莫不護咱不成?{\pangpi{亦是正論。}}此話只好在家對我說罷了,外人知道,傳出去到不好了。只當丟這幾兩銀子罷,更休題了。」月娘聽了無法,也只得罷了。又交他會買頭,發賣布貨。他會了主兒來,月娘交陳敬濟兌銀講價錢,主兒都不服,拏銀出去了。來保硬說:「姐夫,你不知買賣甘苦。俺在江湖上走的多,曉得行情,寧可賣了悔,休要悔了賣。這貨來家得此價錢就勾了。你十分把弓兒拽滿,迸了主兒,顯的不會做生意。我不是托大說話,你年少不知事體。我莫不胳膊兒徃外撇?不如賣弔了,是一場事。」那敬濟聽了,使性兒不管了。他也不等月娘來分付,匹手奪過算盤,邀囘主兒來。把銀子兌了二千餘兩,一件件交付與敬濟經手,交進月娘收了,推貨出門。月娘與了他二三十兩銀子房中盤纏,他便故意兒昂昂大意不收,說道:「你老人家還收了。死了爹,你老人家死水兒,自家盤纏,又與俺們做甚?你收了去,我決不要。」一日晚夕,外邊吃的醉醉兒,走進月娘房中,搭伏着護炕,說念月娘:「你老人家青春少小,沒了爹,你自家守着這點孩子兒,不害孤另麼?」月娘一聲兒沒言語。{\meipi{來保之無禮不必論,使金蓮當此不知又作何狀,月娘亦可謂貞婦人矣。}}一日,東京翟管家寄書來,知道西門慶死了,聽見韓道國說,他家中有四個彈唱出色女子,該多少價錢,說了去,兌銀子來,要載到京中答應老太太。月娘見書,慌了手脚,叫將來保來計議,與他去好,不與他去好。來保進入房中,也不叫娘,只說:「你娘子人家不知事,不與他去,就惹下禍了。這個都是過世老頭兒惹的,恰似賣富一般,但擺酒請人,就叫家樂出去,有個不傳出去的?何況韓夥計女兒又在府中答應老太太,有個不說的?我前日怎麼說來,今果然有此勾當鑽出來。{\meipi{古人賞功,以不失人臣禮為上,深有感於此輩臣僕之可恨也。}}你不與他,他裁派府縣,差人坐名兒來要,不怕你不雙手兒奉與他,還是遲了。難說四個都與他,不如今日胡亂打發兩箇與他,還做面皮。」這月娘沉吟半晌。孟玉樓房中蘭香,與金蓮房中春梅,都不好打發。綉春又要看哥兒,不出門。因問他房中玉簫與迎春,情願要去。以此就差來保,顧車輛裝載兩箇女子,徃東京太師府中來。不料來保這厮,在路上把這兩箇女子都姦了。有日到東京,會見韓道國夫婦,把前後事都說了。韓道國謝來保道:「若不是親戚看顧我,在家阻住,我雖然不怕他,也未免多一番唇舌。」翟謙看見迎春、玉簫兩箇都生的好模樣兒,一個會箏,一個會弦子,都不上十七八歲,進入府中伏侍老太太,賞出兩錠元寶來。這來保還克了一錠,到家只拏出一錠元寶來與月娘,還將言語恐嚇月娘說:「若不是我去,還不得他這錠元寶拏家來。你還不知,韓夥計兩口兒在那府中好不受用富貴,獨自住着一所宅子,呼奴使婢,坐五行三。翟管家以老爹呼之,他家女兒韓愛姐,日逐上去答應老太太,寸步不離,要一奉十,揀口兒吃用,換套穿衣。如今又會寫,又會算,福至心靈,出落得好長大身材,姿容美貌。前日出來見我,打扮得如瓊林玉樹一般,百伶百俐,一口一聲叫我保叔。如今咱家這兩箇家樂到那裡,還在他手裡討針線哩。」說畢,月娘還甚是知感他不盡。{\meipi{月娘雖呆,終不失為好人。}}打發他酒饌吃了,與他銀子又不受,拏了一疋段子與他妻惠祥做衣服穿,不在話下。

這來保一日同他妻弟劉倉,徃臨清馬頭上,將封寄店內布貨,盡行賣了八百兩銀子,暗買下一所房子,就在劉倉右邊門首,就開雜貨鋪兒。他便日逐隨倚祀會茶。他老婆惠祥,要便對月娘說,假推徃娘家去。到房子裡,從新換了頭面衣服,珠子箍兒,插金戴銀,徃王六兒娘家王母豬家扳親家,行人情,坐轎看他家女兒去來。到房子裡,依舊換了慘澹衣裳,纔徃西門慶家中來,只瞞過月娘一人不知。來保這厮,常時吃醉了,來月娘房中,嘲話調戲,兩番三次。不是月娘為人正大,也被他說念的心邪,上了道兒。又有一般小厮媳婦,在月娘根前,說他媳婦子在外與王母豬作親家,插金戴銀,行三坐五。潘金蓮也對月娘說了幾次,月娘不信。

惠祥聽了此言,在廚房中罵大罵小。來保便裝胖學蠢,自己誇獎,說衆人:「你每只好在家裡說炕頭子上嘴罷了!相我水皮子上,顧瞻將家中這許多銀子貨物來家。若不是我,都吃韓夥計老年箝嘴,拐了徃東京去。{\meipi{只引最下者為比,以見己能,此人情世道所以日薄也。}}只『呀』的一聲,乾丟在水裡也不響。如今還不道俺每一個『是』,說俺轉了主子的錢了,架俺一篇是非。正是割股的也不知,燃香的也不知。自古信人調,丟了瓢。」媳婦子惠祥便罵:「賊嚼舌根的淫婦!說俺兩口子轉的錢大了,在外行三坐五扳親。老道出門,問我姊那裡借的幾件子首飾衣裳,就說是俺落的主子銀子治的!要擠撮俺兩口子出門,也不打緊。{\pangpi{自開端,妙甚。}}等俺每出去,料莫天也不着餓水鴉兒吃草。我洗淨着眼兒,看你這些淫婦奴才,在西門慶家裡住牢着!」月娘見他罵大罵小,尋繇頭兒和人嚷,鬧上弔;漢子又兩番三次,無人處在根前無禮,心裡也氣得沒入脚處,只得交他兩口子搬離了家門。這來保就大剌剌和他舅子開起個布鋪來,發賣各色細布,日逐會親友,行人情,不在話下。正是:

\begin{myquote}
勢敗奴欺主,時衰鬼弄人。
\end{myquote}

