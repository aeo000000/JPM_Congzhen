\includepdf[pages={181,182},fitpaper=false]{tst.pdf}
\chapter*{第九十一囘 孟玉樓愛嫁李衙內 李衙內怒打玉簪兒}
\addcontentsline{toc}{chapter}{第九十一囘 孟玉樓愛嫁李衙內 李衙內怒打玉簪兒}
\markboth{\titlename}{第九十一囘 孟玉樓愛嫁李衙內 李衙內怒打玉簪兒}


詩曰:

簟展湘紋浪欲生,幽懷自感夢難成。倚床剩覺添風味,開戶羞將待月明。擬倩蜂媒傳密意,難將螢火照離情。遙憐織女佳期近,時看銀河幾曲橫。

話說一日,陳敬濟聽見薛嫂兒說知孫雪娥之事。這陳敬濟乘着這箇根繇,就如此這般,使薛嫂兒徃西門慶家對月娘說。薛嫂只得見月娘,說:「陳姑夫在外聲言發話,說不要大姐,要寫狀子,巡撫、巡按處告示,說老爹在日,收着他父親寄放的許多金銀箱籠細軟之物。」{\meipi{孤兒寡婦之苦如此。}}這月娘一來因孫雪娥被來旺兒盜財拐去,二者又是來安兒小厮走了,三者家人來興媳婦惠秀又死了,剛打發出去,家中正七事八事,聽見薛嫂兒來說此話,唬的慌了手脚,連忙顧轎子,打發大姐家去。但是大姐床奩箱廚陪嫁之物,交玳安顧人,都擡送到陳敬濟家。敬濟說:「這是他隨身嫁我的床帳粧奩,還有我家寄放的細軟金銀箱籠,須索還我。」薛嫂道:「你大丈母說來,當初丈人在時,止收下這箇床奩嫁粧,並沒見你別的箱籠。」敬濟又要使女元宵兒。薛嫂兒和玳安兒來對月娘說。月娘不肯把元宵與他,說:「這丫頭是李嬌兒房中使的,如今留着晚早看哥兒哩。」把中秋兒打發將來,說:「原是買了伏侍大姐的。」這敬濟又不要中秋兒,兩頭來囘只教薛嫂兒走。他娘張氏向玳安說:「哥哥,你到家拜上你大娘,你家姐兒們多,也不稀罕這箇使女看守哥兒。既是與了大姐房裡好一向,你姐夫已是收用過了他,你大娘只顧留怎的?」玳安一面到家,把此話對月娘說了。月娘無言可對,只得把元宵兒打發將來。敬濟收下,滿心歡喜,說道:「可怎的也打我這條道兒來?」正是:

饒你奸似鬼,吃我洗脚水。

按下一頭。單說李知縣兒子李衙內,自從清明郊外看見吳月娘、孟玉樓兩人一般打扮,生的俱有姿色,知是西門慶妻小。衙內有心,愛孟玉樓生的長挑身材,瓜子面皮,模樣兒風流俏麗。原來衙內䘮偶,鰥居已久,一向着媒婦各處求親,都不遂意。及見玉樓,便覺動心,但無門可入,未知嫁與不嫁,從違如何。不期雪娥緣事在官,已知是西門慶家出來的,周旋委曲,在伊父案前,

將各犯用刑研審,追出賍物數目,望其來領。{\meipi{稍有影响,便欲下釣,寫出好色人一片痴心。}}月娘害怕,又不使人見官。衙內失望,因此纔將賍物入官,雪娥官賣。至是衙內謀之於廊吏何不韋,徑使官媒婆陶媽媽來西門慶家訪求親事,許說成此門親事,免縣中打卯,還賞銀五兩。

這陶媽媽聽了,喜歡的疾走如飛,一日到於西門慶門首。來昭正在門首立,只見陶媽媽向前道了萬福,說道:「動問管家哥一聲,此是西門老爹家?」來昭道:「你是那裡來的?老爹已下世了,有甚話說?」陶媽媽道:「累及管家進去稟聲,我是本縣官媒人,名喚陶媽媽,奉衙內小老爹鈞語,分付說咱宅內有位奶奶要嫁人,敬來說親。」那來昭喝道:「你這婆子,好不近理!我家老爹沒了一年有餘,止有兩位奶奶守寡,並不嫁人。常言疾風暴雨,不入寡婦之門。你這媒婆,有要沒緊,走來胡撞甚親事?還不走快着,惹的後邊奶奶知道,一頓好打。」那陶媽媽笑道:「管家哥,常言官差吏差,來人不差。小老爹不使我,我敢來?嫁不嫁,起動進去稟聲,我好囘話去。」來昭道:「也罷,與人方便,自己方便,你少待片時,等我進去。兩位奶奶,一位奶奶有哥兒,一位奶奶無哥兒,不知是那一位奶奶要嫁人?」陶媽媽道:「衙內小老爹說,清明那日郊外曾看見來,是面上有幾點白麻子的那位奶奶。」來昭聽了,走到後邊,如此這般告訴月娘說:「縣中使了箇官媒人在外面。」倒把月娘吃了一驚,說:「我家並沒半箇字兒迸出,外邊人怎得曉的?」來昭道:「曾在郊外,清明那日見來,說臉上有幾箇白麻子兒的。」月娘便道:「莫不孟三姐也『臘月裡羅卜——動箇心』?忽剌八要徃前進嫁人?」正是「世間海水知深淺,惟有人心難忖量」。一面走到玉樓房中坐下,便問:「孟三娘,奴有件事兒來問你,外面有箇保山媒人,說是縣中小衙內,清明那日曾見你一面,說你要徃前進。端的有此話麼?」{\meipi{衆人待我,衆人報之。玉樓雖賢,自無終守之理,月娘何見之晚。}}看官聽說,當時沒巧不成話,自古姻緣着線牽。那日郊外,孟玉樓看見衙內生的一表人物,風流博浪,兩家年甲多相彷彿,又會走馬拈弓弄箭,彼此兩情四目都有意,已在不言之表。但未知有妻子無妻子,口中不言,心內暗度:「男子漢已死,奴身邊又無所出。雖故大娘有孩兒,到明日長大了,各肉兒各疼。閃的我樹倒無陰,竹籃兒打水。」又見月娘自有了孝哥兒,心腸改變,不似徃時,「我不如徃前進一步,尋上箇葉落歸根之處,還只顧傻傻的守些甚麼?到沒的担閣了奴的青春年少。」正在思慕之間,不想月娘進來說此話,正是清明郊外看見的那箇人,心中又是歡喜,又是羞愧,口裡雖說:「大娘休聽人胡說,奴並沒此話。」不覺把臉來飛紅了,{\meipi{總是有良心人情景。}}正是:

含羞對衆休開口,理鬂無言只搵頭。

月娘說:「此是各人心裡事,奴也管不的許多。」一面叫來昭:「你請那保山進來。」來昭門首喚陶媽媽,進到後邊見月娘,行畢了禮數,坐下。小丫鬟倒茶吃了。月娘便問:「保山來,有甚事?」陶媽媽便道:「小媳婦無事不登三寶殿,奉本縣正宅衙內分付,說貴宅上有一位奶奶要嫁人,講說親事。」月娘道:「俺家這位娘子嫁人,又沒曾傳出去,你家衙內怎得知道?」陶媽媽道:「俺家衙內說來,清明那日,在郊外親見這位娘子,生的長挑身材,瓜子面皮,臉上有稀稀幾箇白麻子,便是這位奶奶。」月娘聽了,不消說就是孟三姐了。於是領陶媽媽到玉樓房中明間內坐下。

等勾多時,玉樓梳洗打扮出來。陶媽媽道了萬福,說道:「就是此位奶奶,果然話不虛傳,人材出衆,蓋世無雙,堪可與俺衙內老爹做箇正頭娘子。」玉樓笑道:「媽媽休得亂說。且說你衙內今年多大年紀?原娶過妻小沒有?房中有人也無?姓甚名誰?有官身無官身?從寔說來,休要搗謊。」陶媽媽道:「天麼,天麼!小媳婦是本縣官媒,不比外邊媒人快說謊。我有一句說一句,並無虛假。俺知縣老爹年五十多歲,止生了衙內老爹一人,今年屬馬的,三十一歲,正月二十三日辰時建生。見做國子監上舍,不久就是舉人、進士。{\meipi{妙贊。}}有滿腹文章,弓馬熟閑,諸子百家,無不通曉。沒有大娘子二年光景,房內止有一箇從嫁使女答應,又不出衆。要尋箇娘子當家,敬來宅上說此親事。若是咱府上做這門親事,老爹說來,門面差傜,墳塋地土錢糧,一例盡行蠲免,有人欺負,指名說來,拏到縣裡,任意拶打。」玉樓道:「你衙內有兒女沒有?原籍那裡人氏?誠恐一時任滿,千山萬水帶去,奴親都在此處,莫不也要同他去?」陶媽媽道:「俺衙內身邊,兒花女花沒有,好不單徑。原籍是咱北京真定府棗強縣人氏,過了黃河不上六、七百里。他家中田連阡陌,騾馬成群,人丁無數,走馬牌樓,都是撫按明文,聖旨在上,好不赫耀驚人。如今娶娘子到家,做了正房,過後他得了官,娘子便是五花官誥,坐七香車,為命婦夫人,有何不好?」{\meipi{說遠似近,說未見似目睹,說未來似現在,非有此嘴,如何做得媒人。}}這孟玉樓被陶媽媽一席話,說得千肯萬肯,一面喚蘭香放桌兒,看茶食點心與保山吃。因說:「保山,你休恠我叮嚀盤問。你這媒人們說謊的極多,奴也吃人哄怕了。」{\meipi{玉樓嫁西門慶,殊失其意,然度不可與爭,故厚薄親疏全不介意,所處似高,而其心實非坦然,觀「吃人哄怕」一語,底裡見矣。}}陶媽媽道:「好奶奶,只要一箇比一箇。清自清,渾自渾,好的帶累了歹的。小媳婦並不搗謊,只依本分做媒。奶奶若肯了,寫箇婚帖兒與我,好囘小老爹話去。」玉樓取了一條大紅段子,使玳安交鋪子裡傅夥計寫了生時八字。吳月娘便說:「你當初原是薛嫂兒說的媒,如今還使小厮叫將薛嫂兒來,兩箇同拏了貼兒去,說此親事,纔是禮。」不多時,使玳安兒叫了薛嫂兒來,見陶媽媽道了萬福。當行見當行,拏着貼兒出離西門慶家門,徃縣中囘衙內話去。一箇是這裡氷人,一箇是那頭保山,兩張口四十八箇牙,這一去管取說得月裡嫦娥尋配偶,巫山神女嫁襄王。陶媽媽在路上問薛嫂兒:「你就是這位娘子的原媒?」薛嫂道:「便是。」陶媽媽問他:「原先嫁這裡,根兒是何人家的女兒?嫁這裡是女兒,是再婚?」這薛嫂兒便一五一十,把西門慶當初從楊家娶來的話告訴一遍。因見婚貼兒上寫「女命三十七歲,十一月二十七日子時生」,說:「只怕衙內嫌年紀大些,怎了?他今纔三十一歲,倒大六歲。」薛嫂道:「咱拏了這婚貼兒,交箇過路的先生,算看年命妨礙不妨礙。若是不對,咱瞞他幾歲兒,也不算說謊。」{\pangpi{轉算湊趣。}}

二人走來,再不見路過响板的先生,只見路南遠遠的一箇卦肆,青布帳幔,掛着兩行大字:「子平推貴賤,鐵筆判榮枯;有人來算命,直言不容情。」帳子底下安放一張桌子,裡面坐着箇能寫快算靈先生。這兩箇媒人向前道了萬福,先生便讓坐下。薛嫂道:「有箇女命累先生算一算。」向袖中拏出三分命金來,說:「不當輕視,先生權且收了,路過不曾多帶錢來。」先生道:「請說八字。」陶媽媽遞與他婚帖看,上面有八字生日年紀,先生道:「此是合婚。」一百捏指尋紋,把運算元搖了一搖,開言說道:「這位女命今年三十七歲了,十一月廿七日子時生。甲子月,辛卯日,庚子時,理取印綬之格。女命逆行,見在丙申運中。丙合辛生,徃後大有威權,執掌正堂夫人之命。四柱中雖夫星多,然是財命,益夫發福,受夫寵愛,這兩年定見妨克,見過了不曾?」薛嫂道:「已克過兩位夫主了。」先生道:「若見過,後來好了。」薛嫂兒道:「他徃後有子沒有?」先生道:「子早哩。直到四十一歲纔有一子送老。一生好造化,富貴榮華無比。」{\meipi{玉樓一身,借算命口中說出,似然似不然,復不再見矣。妙法。}}取筆批下命詞四句道:

嬌姿不失江梅態,三揭紅羅兩畫眉。會看馬首陞騰日,脫卻寅皮任意移。

薛嫂問道:「先生,如何是『會看馬首陞騰日,脫卻寅皮任意移』?這兩句俺每不懂,起動先生講說講說。」先生道:「馬首者,這位娘子如今嫁箇屬馬的夫主,纔是貴星,享受榮華。寅皮是克過的夫主,是屬虎的,雖是寵愛,只是偏房。徃後一路功名,直到六十八歲,有一子,壽終,夫妻偕老。」兩箇媒人說道:「如今嫁的倒果是箇屬馬的,只怕大了好幾歲,配不來。求先生改少兩歲纔好。」先生道:「既要改,就改做丁卯三十四歲罷。」薛嫂道:「三十四歲,與屬馬的也合的着麼?」先生道:「丁火庚金,火逢金練,定成大器,正合得着。」當下改做三十四歲。

兩箇拜辭了先生,出離卦肆,徑到縣中。門子報入,衙內便喚進陶、薛二媒人,旋磕了頭。衙內便問:「那箇婦人是那裡的?」陶媽媽道:「是那邊媒人。」因把親事說成,告訴一遍,說:「娘子人才無比的好,只爭年紀大些。小媳婦不敢擅便,隨衙內老爹尊意,討了箇婚貼在此。」於是遞上去。李衙內看了,上寫着「三十四歲,十一月廿七日子時生」,說道:「就大三兩歲,也罷。」薛嫂兒插口道:「老爹見的是,自古道,妻大兩,黃金長;妻大三,黃金山。{\meipi{薛嫂此語說過兩遍,宛似今人一篇文章,到處皆用。}}這位娘子人材出衆,性格溫柔,諸子百家,當家理紀,自不必說。」衙內道:「我已見過,不必再相。只擇吉日良時,行茶禮過去就是了。」兩箇媒人稟說:「小媳婦幾時來伺候?」衙內道:「事不可稽遲,你兩箇明日來討話,徃他家說。」每箇賞了一兩銀子,做脚步錢。兩箇媒人歡喜出門,不在話下。這李衙內見親事已成,喜不自勝,即喚廊吏何不韋來商議,對父親李知縣說了。令陰陽生擇定四月初八日行禮,十五日準娶婦人過門。就兌出銀子來,委托何不韋、小張閑買辦茶紅酒禮,不必細說。

兩箇媒人次日討了日期,徃西門慶家囘月娘、玉樓話。正是:

姻緣本是前生定,曾向藍田種玉來。

四月初八日,縣中備辦十六盤羹菓茶餅,一副金絲冠兒,一副金頭面,一條瑪瑙帶,一副玎璫七事,金鐲銀釧之類,兩件大紅宮錦袍兒,四套粧花衣服,三十兩禮錢,其餘布絹綿花,共約二十餘擡。兩箇媒人跟隨,廊吏何不韋押担,到西門慶家下了茶。十五日,縣中撥了許多快手閑漢來,搬擡孟玉樓床帳嫁粧箱籠。月娘看着,但是他房中之物,盡數都交他帶去。原舊西門慶在日,把他一張八步彩漆床陪了大姐,月娘就把潘金蓮房中那張螺鈿床陪了他。{\meipi{前後脈絡照映,一毫不亂。}}玉樓交蘭香跟他過去,留下小鸞與月娘看哥兒。月娘不肯,說:「你房中丫頭,我怎好留下你的?左右哥兒有中秋兒、綉春和奶子,也勾了。」玉樓止留下一對銀囘囘壺與哥兒耍子,做一念兒,其餘都帶過去了。到晚夕,一頂四人大轎,四對紅紗燈籠,八箇皁隸跟隨來娶。玉樓戴着金梁冠兒,插着滿頭珠翠、胡珠子,身穿大紅通袖袍兒,先辭拜西門慶靈位,{\pangpi{辭靈不哭,情盡矣。}}然後拜月娘。月娘說道:「孟三姐,你好狠也!你去了,撇的奴孤另另獨自一箇,和誰做伴兒?」{\pangpi{傷心語。}}兩箇攜手哭了一囘。然後家中大小都送出大門。媒人替他帶上紅羅銷金蓋袱,抱着金寶瓶,月娘守寡出不的門,請大姨送親,送到知縣衙裡來。滿街上人看見說:「此是西門大官人第三娘子,嫁了知縣相公兒子衙內,今日吉日良時娶過門。」也有說好的,也有說歹的。說好者,當初西門大官人怎的為人做人,今日死了,止是他大娘子守寡正大,有兒子,房中攪不過這許多人來,都交各人前進,甚有張主。有那說歹的,街談巷議,指戳說道:「西門慶家小老婆,如今也嫁人了。當初這厮在日,專一違天害理,貪財好色,奸騙人家妻女。今日死了,老婆帶的東西,嫁人的嫁人,拐帶的拐帶,養漢的養漢,做賊的做賊,都野雞毛兒零撏了。常言三十年遠報,而今眼下就報了。」旁人紛紛議論不題。{\meipi{此一段是作書大意。}}

且說孟大姨送親到縣衙內,鋪陳床帳停當,留坐酒席來家。李衙內賞薛嫂兒、陶媽媽每人五兩銀子,一段花紅利市,打發出門。至晚,兩箇成親,極盡魚水之歡,于飛之樂。到次日,吳月娘送茶完飯。楊姑娘已死,孟大妗子、二妗子、孟大姨都送茶到縣中。衙內這邊下囘書,請衆親戚女眷做三日,扎彩山,吃筵席。都是三院樂人妓女,動鼓樂扮演戲文。吳月娘那日亦滿頭珠翠,身穿大紅通袖袍兒,百花裙,系蒙金帶,坐大轎來衙中,做三日赴席,在後廳吃酒。知縣奶奶出來陪待。月娘囘家,因見席上花攢錦簇,歸到家中,進入後邊院落,靜俏俏無箇人接應。想起當初,有西門慶在日,姊妹們那樣鬧熱,徃人家赴席來家,都來相見說話,一條板凳坐不了,如今並無一箇兒了。一面撲着西門慶靈床兒,不覺一陣傷心,放聲大哭。{\meipi{此時此景,真難為情。在鐵人也應下淚。}}哭了一囘,被丫鬟小玉勸止。正是:

平生心事無人識,只有穿窓皓月知。

這裡月娘憂悶不題。卻說李衙內和玉樓兩箇,女貌郎才,如魚如水,正合着油瓶蓋。每日燕爾新婚,在房中厮守,一步不離。{\pangpi{玉樓方遇知己。}}端詳玉樓容貌,越看越愛。又見帶了兩箇從嫁丫鬟,一箇蘭香,年十八歲,會彈唱;一箇小鸞,年十五歲,俱有顏色。心中歡喜沒入脚處。有詩為證:

堪誇女貌與郎才,天合姻緣礼所該。十二巫山雲雨會,兩情願保百年偕。

原來衙內房中,先頭娘子丟了一箇大丫頭,約三十年紀,名喚玉簪兒。專一搽胭抹粉,作恠成精。頭上打着盤頭揸髻,用手貼苫蓋,周圍勒銷金箍兒,假充作鬏髻,{\meipi{今人以氊帽捏巾者本此。}}身上穿一套恠綠喬紅的裙襖,脚上穿着雙撥船樣四箇眼的剪絨鞋,約長尺二。在人根前,輕身浪顙,做勢拏班。衙內未娶玉樓時,他便逐日頓羹頓飯,殷勤伏侍,不說強說,不笑強笑,何等精神。自從娶過玉樓來,見衙內和他如膠似漆,把他不去揪採,這丫頭就使性兒起來。

一日,衙內在書房中看書,這玉簪兒在廚下頓了一盞好菓仁泡茶,雙手用盤兒托來書房裡,笑嘻嘻掀開簾兒,送與衙內。不想衙內看了一囘書,搭伏定書桌就睡着了。{\meipi{此是衙內常態,非因夜作乏也。}}這玉簪兒叫道:「爹,誰似奴疼你,頓了這盞好茶兒與你吃。你家那新娶的娘子,還在被窩裡睡得好覺兒,怎不交他那小大姐送盞茶來與你吃?」因見衙內打盹,在眼前只顧叫不應,說道:「老花子,你黑夜做夜作使乏了也怎的?大白日裡盹磕睡,起來吃茶!」叫衙內醒了,看見是他,喝道:「恠硶奴才!把茶放下,與我過一邊去。」這玉簪兒滿臉羞紅,使性子把茶丟在桌上,出來說道:「好不識人敬重!奴好意用心,大清早辰送盞茶兒來你吃,倒喓喝我起來。常言『醜是家中寶,可喜惹煩惱』。我醜,你當初瞎了眼,誰交你要我來?」{\pangpi{此一語真囘他不得。}}被衙內聽見,趕上尺力踢了兩靴脚。這玉簪兒登時把那付奴臉膀的有房梁高,也不搽臉了,也不頓茶了。趕着玉樓,也不叫娘,只你也我也,無人處,一屁股就在玉樓床上坐下。玉樓亦不去理他。他背地又壓伏蘭香、小鸞說:「你休趕着我叫姐,只叫姨娘。我與你娘系大小之分。」{\pangpi{奇想。}}又說:「你只背地叫罷,休對着你爹叫。{\pangpi{此轉尤妙。}}你每日跟隨我行,用心做活,你若不聽我說,老娘拏煤鍬子請你。」{\meipi{寫恠奴恠態,不獨言語恠,衣裳恠,形貌舉止恠,並聲影、氣味、心思、胎骨之恠,俱為摹出,真爐錘造物之手。}}後來幾次見衙內不理他,他就撒懶起來,睡到日頭半天還不起來,飯兒也不做,地兒也不掃。玉樓分付蘭香、小鸞:「你休靠玉簪兒了,你二人自去廚下做飯,打發你爹吃罷。」這玉簪又氣不憤,使性謗氣,牽家打夥,在廚房內打小鸞,罵蘭香:「賊小奴才,小淫婦兒!碓磨也有箇先來後到,先有你娘來,先有我來?都是你娘兒們佔了罷,不獻這箇勤兒也罷了!當原先俺死的那箇娘也沒曾失口叫我聲玉簪兒,你進門幾日,就題名道姓叫我。我是你手裡使的人也怎的?你未來時,我和俺爹同床共枕,那一日不睡到齋時纔起來。和我兩箇如糖拌蜜,如蜜攪酥油一般打熱。房中事,那些兒不打我手裡過。自從你來了,把我蜜礶兒也打碎了,把我姻緣也拆散開了,一攆攆到我明間,冷清清支板凳打官鋪,再不得嘗着俺爹那件東西兒如今甚麼滋味了。{\pangpi{說得苦甚,趣甚,諧甚。}}我這氣苦也沒處聲訴。你當初在西門慶家,也曾做第三箇小老婆來,你小名兒叫玉樓,敢說老娘不知道?{\pangpi{愈轉愈妙。}}你來在俺家,你識我見,大家膿着些罷了。會那等喬張致,呼張喚李,誰是你買到的?屬你管轄?」不知玉樓在房聽見,氣的發昏,又不好聲言對衙內說。

一日熱天,也是合當有事。晚夕衙內分付他廚下熱水,拏浴盆來房中,要和玉樓洗澡。玉樓便說:「你交蘭香熱水罷,休要使他。」衙內不從,說道:「我偏使他,休要慣了這奴才。」玉簪兒見衙內要水,和婦人共浴蘭湯,效魚水之歡,心中正沒好氣,拏浴盆進房,徃地下只一墩,用大鍋澆上一鍋滾水,口中喃喃吶吶說道:{\meipi{不怕笑破人口。}}「也沒見這娘淫婦,刁鑽古恠,禁害老娘!無故也只是箇浪精𣭈,沒三日不拏水洗。像我與俺主子睡,成月也不見點水兒,也不見展汙了甚麼佛眼兒。偏這淫婦會,兩番三次刁蹬老娘。」直罵出房門來。玉樓聽見,也不言語。衙內聽了此言,心中大怒,澡也洗不成,精脊梁[]着鞋,向床頭取柺子,就要走出來。婦人攔阻住,說道:「隨他罵罷,你好惹氣。只怕熱身子出去,風試着你,倒值了多的。」衙內那裡按納得住,說道:「你休管。這奴才無禮!」向前一把手採住他頭髮,拖踏在地下,輪起柺子,雨點打將下來。饒玉樓在旁勸着,也打了二三十下在身。打的這丫頭急了,跪在地下告說:「爹,你休打我,我想爹也看不上我在家裡了,情願賣了我罷。」衙內聽了,亦發惱怒起來,又狠了幾下。玉樓勸道:「他既要出去,你不消打,倒沒得氣了你。」衙內隨令伴當即時叫將陶媽媽來,把玉簪兒領出去,便賣銀子來交,不在話下。正是:

蚊蟲遭扇打,只為嘴傷人。

有詩為證:

百禽啼後人皆喜,惟有鴉鳴事若何。見者多言聞者唾,只為人前口嘴多。

