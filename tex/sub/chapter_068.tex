\includepdf[pages={135,136},fitpaper=false]{tst.pdf}
\chapter*{第六十八囘 應伯爵戲啣玉臂 玳安兒密訪蜂媒}
\addcontentsline{toc}{chapter}{第六十八囘 應伯爵戲啣玉臂 玳安兒密訪蜂媒}
\markboth{{\titlename}卷之七}{第六十八囘 應伯爵戲啣玉臂 玳安兒密訪蜂媒}


詞曰:

\begin{myquote} 
鍾情太甚,到老也無休歇。月露烟雲都是態,況與玉人明說。軟語叮嚀,柔情婉戀,鎔盡肝腸鐵。岐亭把盞,水流花謝時節。

\raggedleft{——右調《翠雲吟半》\rightquadmargin}
\end{myquote} 

話說西門慶與李瓶兒燒紙畢,歸潘金蓮房中歇了一夜。到次日,先是應伯爵家送喜麵來。落後黃四領他小舅子孫文相,宰了一口豬、一罈酒、兩隻燒鵝、四隻燒雞、兩盒菓子來與西門慶磕頭。西門慶再三不受,黃四打旋磨兒跪着說:「蒙老爹活命之恩,舉家感激不淺。無甚孝順,些微薄禮,與老爹賞人,如何不受!」推阻了半日,西門慶止受豬酒:「留下送你錢老爹罷。」黃四道:「既是如此,難為小人一點窮心,無處所盡。」只得把羹菓擡囘去。又請問:「老爹幾時閑暇?小人問了應二叔,裡邊請老爹坐坐。」西門慶道:「你休聽他哄你哩!又費煩你,不如不央我了。」那黃四和他小舅子千恩萬謝出門去了。到十一月初一日,西門慶徃衙門中囘來,又徃李知縣衙內吃酒去,月娘獨自一人,素粧打扮,坐轎子徃喬大戶家與長姐做生日,都不在家。到後晌,有庵裡薛姑子,聽見月娘許下他初五日念經拜《血盆懺》,於是悄悄瞞着王姑子,買了兩盒禮物來見月娘。月娘不在家,李嬌兒、孟玉樓留他吃茶,說:「大姐姐徃喬親家做生日去了。你須等他來,他還和你說話哩。」那薛姑子就坐住了。潘金蓮思想着玉簫告他說,月娘吃了他的符水藥纔坐了胎氣,又見西門慶把奶子要了,恐怕一時奶子養出孩子來,攙奪了他寵愛。於是把薛姑子讓到前邊他房裡,悄悄央薛姑子,與他一兩銀子,替他配坐胎氣符藥,不在話下。到晚夕,等的月娘囘家,留他住了一夜。次日,問西門慶討了五兩銀子經錢寫法與他。這薛姑子就瞞着王姑子、大師父,到初五日早請了八衆女僧,在花園捲棚內建立道場,諷誦《華嚴》、《金剛》經咒,禮拜《血盆寶懺》。晚夕設放焰口施食。

那日請了吳大妗子、花大嫂並官客吳大舅、應伯爵、溫秀才吃齋。尼僧也不動响器,只敲木魚,擊手罄,念經而已。

那日伯爵領了黃四家人,具帖初七日在院中鄭愛月兒家置酒請西門慶。西門慶看了帖兒,笑道:「我初七日不得閑,張西村家吃生日酒。倒是明日空閑。」{\meipi{大老官口氣皆然。}}問還有誰,伯爵道:「再沒人。只請了我與李三相陪哥,又叫了四個女兒唱《西廂記》。」西門慶分付與黃四家人齋吃了,打發囘去,改了初六。伯爵便問:「黃四那日買了分甚麼禮來謝你?」西門慶如此這般:「我不受他的,再三磕頭禮拜,我只受了豬酒。添了兩疋白鷳紵絲、兩疋京段、五十兩銀子,謝了龍野錢公了。」伯爵道:「哥,你不接錢儘勾了,這個是他落得的。少說四疋尺頭値三十兩銀子,那二十兩,那裡尋這分上去?便益了他,救了他父子二人性命!」當日坐至晚夕方散。西門慶向伯爵說:「你明日還到這邊。」伯爵說:「我知道。」作別去了。八衆尼僧直亂到一更多,方纔道場圓滿,焚燒箱庫散了。至次日,西門慶早徃衙門中去了。

且說王姑子打聽得知,大清早晨走來,說薛姑子攬了經去,要經錢。月娘恠他道:「你怎的昨日不來?他說你徃王皇親家做生日去了。」王姑子道:「這個就是薛家老淫婦的鬼。他對着我說咱家挪了日子,到初六念經。難道經錢他都拏的去了,一些兒不留下?」月娘道:「還等到這咱哩?未曾念經,經錢寫法就都找與他了。早是我還與你留下一疋襯錢布在此。」教小玉連忙擺了些昨日剩下的齋食與他吃了,把與他一疋藍布。這王姑子口裡喃喃吶吶罵道:「這老淫婦,他印造經,撰了六娘許多銀子。原說這個經兒,咱兩個使,你又獨自掉攬的去了。」{\meipi{如此功德,能免罪過足矣。三姑六婆處心設慮,大抵如是。讀此可作有家氷鑑。}}月娘道:「老薛說你接了六娘《血盆經》五兩銀子,你怎的不替他念?」王姑子道:「他老人家五七時,我在家請了四位師父,念了半個月哩。」月娘道:「你念了,怎的掛口兒不對我題?你就對我說,我還送些襯施兒與你。」那王姑子便一聲兒不言語,訕訕的坐了一囘,{\meipi{道着心病,便開口不得,畢竟佛門弟子,良心不昧。}}徃薛姑子家嚷去了。正是:

\begin{myquote} 
佛會僧尼是一家,法輪常轉度龍華。\\此物只好圖生育,枉使金刀剪落花。
\end{myquote} 

卻說西門慶從衙門中囘來,吃了飯,應伯爵又早到了。盔的新段帽,沉香色𧜽褶,粉底皁靴,{\meipi{伯爵來徃太熟,從此忽又粧點一番,便見運筆不死。}}向西門慶聲喏,說:「這天也有晌午,好去了。他那裡使人邀了好幾遍了。」西門慶道:「咱今邀葵軒同走走去。」使王經:「徃對過請你溫師父來。」王經去不多時,囘說:「溫師父不在家,望朋友去了。」伯爵便說:「咱等不的他。秀才家有要沒緊望朋友,知多咱來?倒沒的誤了勾當。」西門慶分付琴童:「備黃馬與應二爹騎。」伯爵道:「我不騎。你依我:省的搖鈴打鼓,我先走一步兒,你坐轎子慢慢來就是了。」西門慶道:「你說的是,你先行罷。」那伯爵舉手先走了。

西門慶分付玳安、琴童、四個排軍,收拾下暖轎跟隨。纔待出門,忽平安兒慌慌張張從外拏着雙帖兒來報,說:「工部安老爹來拜。先差了個吏送帖兒,後邊轎子便來也。」慌的西門慶分付家中廚下備飯,使來興兒買攢盤點心伺候。良久,安郎中來到,西門慶冠冕出迎。安郎中穿着粧花雲鷺補子員領,起花萌金帶,進門拜畢,分賓主坐定,左右拏茶上來。茶罷,叙其間闊之情。西門慶道:「老先生榮擢,失賀,心甚缺然。前日蒙賜華紮厚儀,生正値䘮事,匆匆未及奉候起居為歉。」安郎中道:「學生有失弔問,罪罪!生到京也曾道達雲峰,未知可有禮到否?」西門慶道:「正是,又承翟親家遠勞致賻。」{\meipi{問答似閑,然情理鑿鑿,非俗筆可辦。}}安郎中道:「四泉已定今歲恭喜。」西門慶道,「在下纔微任小,豈敢非望。」又說:「老先生榮擢美差,足展雄纔。治河之功,天下所仰。」安郎中道:「蒙四泉過譽。一介寒儒,辱蔡老先生擡舉,謬典水利,修理河道,當此民窮財盡之時。前者皇船載運花石,毀閘折壩,所過倒懸,公私困弊之極。又兼賊盜梗阻,雖有神輸鬼役之才,亦無如之何矣。」西門慶道:「老先生大才展布,不日就緒,必大陞擢矣。」因問:「老先生勑書上有期限否?」安郎中道:「三年欽限。河工完畢,聖上還要差官來祭謝河神。」

說話中間,西門慶令放桌兒,安郎中道:「學生寔說,還要徃黃泰宇那裡拜拜去。」西門慶道:「既如此,少坐片時,教從者吃些點心。」不一時,就是春盛案酒,一色十六碗下飯,金鐘煖酒斟來,下人俱有攢盤點心酒肉。安郎中席間只吃了三鍾,就告辭起身,說:「學生容日再來請教。」西門慶款留不住,送至大門首,上轎而去。囘到廳上,解去冠帶,換了巾幘,止穿紫絨獅補直身。使人問:「溫師父來了不曾?」玳安囘說:「溫師父尚未囘哩。有鄭春和黃四叔家來定兒來邀,在這裡半日了。」

西門慶即出門上轎,左右跟隨,逕徃鄭愛月兒家來。比及進院門,架兒們都躱過一邊,只該日俳長兩邊站立,不敢跪接。{\meipi{未做官時,架兒討好,已做官時,架兒躱避。作者下筆直如此,分青理白。}}鄭春與來定兒先通報去了。應伯爵正和李三打雙陸,聽見西門慶來,連忙收拾不及。鄭愛月兒、愛香兒戴着海獺臥兔兒,一窩絲杭州攢,打扮的花仙也似,都出來門首迎接。西門慶下了轎,進入客位內。西門慶分付不消吹打,{\pangpi{好。}}止住鼓樂。先是李三、黃四見畢禮數,然後鄭家鴇子出來拜見了。纔是愛月兒姊妹兩個磕頭。正面安放兩張交椅,西門慶與應伯爵坐下,李智、黃四與鄭家姊妹打橫。玳安在旁稟問:「轎子在這裡,囘了家去?」西門慶令排軍和轎子都囘去,又分付琴童:「到家看你溫師父來了,拏黃馬接了來。」琴童應喏去了。伯爵因問:「哥怎的這半日纔來?」西門慶悉把安郎中來拜留飯之事說了一遍。

須臾,鄭春拏上茶來,愛香兒拏了一盞遞與伯爵。愛月兒便遞西門慶,那伯爵連忙用手去接,說:「我錯接,只說你遞與我來。」愛月兒道:「我遞與你?沒修這樣福來!」伯爵道:「你看這小淫婦兒,原來只認的他家漢子,倒把客人不着在意裡。」{\meipi{一到伯爵開口,諛則似莊,謔便帶韻,應是古今清客之祖。}}愛月兒笑道:「今日輪不着你做客人哩!」吃畢茶,須臾四個唱《西廂》妓女都出來與西門慶磕頭,一一問了姓名。西門慶對黃四說:「等住囘上來唱,只打鼓兒,不吹打罷。」黃四道:「小人知道。」鴇子怕西門慶冷,又教鄭春放下煖簾來,火盆內添上許多獸炭。只見幾個青衣圓社聽見西門慶在鄭家吃酒,走來門首伺候,探頭舒腦,不敢進去。有認得玳安的,向玳安打恭,央及作成作成。玳安悄俏進來替他稟問,被西門慶喝了一聲,{\pangpi{大威風。}}唬的衆人一溜烟走了。不一時,收拾菓品案酒上來,正面放兩張桌席:西門慶獨自一席,伯爵與溫秀才一席,留下溫秀才座位在左首。旁邊一席李三和黃四,右邊是他姊妹二人。端的餚堆異品,花插金瓶。鄭奉、鄭春在旁彈唱。

纔遞酒安席坐下,只見溫秀才到了。頭戴過橋巾,身穿綠雲襖,進門作揖。伯爵道:「老先生何來遲也?留席久矣。」溫秀才道:「學生有罪,不知老先生呼喚,適徃敝同窓處會書,來遲了一步。」{\pangpi{伏案。}}{\meipi{開口即腐,妙。}}慌的黃四一面安放鍾筯,與伯爵一處坐下。不一時,湯飯上來,兩個小優兒彈唱一囘下去。四個妓女纔上來唱了一折「遊藝中原」,只見玳安來說:「後邊銀姨那裡使了吳惠和蠟梅送茶來了。」原來吳銀兒就在鄭家後邊住,止隔一條巷。

聽見西門慶在這裡吃酒,故使送茶。西門慶喚入裡面,吳惠、蠟梅磕了頭,說:「銀姐使我送茶來爹吃。」揭開盒兒,斟茶上去,每人一盞瓜仁香茶。西門慶道:「銀姐在家做甚麼哩?」蠟梅道:「姐兒今日在家沒出門。」西門慶吃了茶,賞了他兩個三錢銀子,即令玳安同吳惠:「你快請銀姨去。」鄭愛月兒急俐,便就教鄭春:「你也跟了去,好歹纏了銀姨來。他若不來,你就說我到明日就不和他做夥計了。」應伯爵道:「我倒好笑,你兩個原來是販𣬼的夥計。」

{\pangpi{妙。}}溫秀才道:「南老好不近人情。自古同聲相應,同氣相求。{\pangpi{尤妙。}}本乎天者親上,本乎地者親下。{\pangpi{妙極,妙極。}}同他做夥計亦是理之當然。」愛月兒道:「應花子,你與鄭春他們都是夥計,當差供唱都在一處。」伯爵道:「傻孩子,我是老王八!那咱和你媽相交,你還在肚子裡!」說笑中間,妓女又上來唱了一套「半萬賊兵」。西門慶叫上唱鶯鶯的韓家女兒近前,問:「你是韓家誰的女兒?」愛香兒說:「爹,你不認的?他是韓金釧姪女兒,小名消愁兒,今年纔十三歲。」西門慶道:「這孩子到明日成個好婦人兒。舉止伶俐,又唱的好。」因令他上席遞酒。黃四下湯下飯,極盡殷勤。

不一時,吳銀兒來到。頭上戴着白縐紗鬏髻、珠子箍兒、翠雲鈿兒,周圍撇一溜小簪兒。上穿白綾對衿襖兒,粧花眉子,下着紗綠潞紬裙,羊皮金滾邊。脚上墨青素段鞋兒。{\meipi{描來素服倩粧,眉目生動。}}笑嘻嘻進門,向西門慶磕了頭,後與溫秀才等各位都道了萬福。伯爵道:「我倒好笑,來到就教我惹氣。俺每是後娘養的?只認的你爹,與他磕頭,望着俺每只一拜。原來你這麗春院小娘兒這等欺客!我若有五棍兒衙門,定不饒你。」{\meipi{句句自道,句句譽着大老官,的是老蔑之尤。}}愛月兒叫:「應花子,好沒羞的孩兒。你行頭不怎麼,光一味好撇。」一面安座兒,讓銀姐就在西門慶桌邊坐下。西門慶見他戴着白鬏髻,問:「你戴的誰人孝?」吳銀兒道:「爹故意又問個兒,與娘戴孝一向了。」西門慶一聞與李瓶兒戴孝,不覺滿心歡喜,與他側席而坐,兩個說話。

須臾,湯飯上來,愛月兒下來與他遞酒。吳銀兒下席說:「我還沒見鄭媽哩。」一面走到鴇子房內見了禮,出來,鴇子叫:「月姐,讓銀姐坐。只怕冷,教丫頭燒個火籠來,與銀姐烤手兒。」隨即添換熱菜上來,吳銀兒在旁只吃了半個點心,喝了兩口湯。放下筯兒,和西門慶攀話道:「娘前日斷七念經來?」西門慶道:「五七多謝你每茶。」吳銀兒道:「那日俺每送了些粗茶,倒教爹把人情囘了,又多謝重禮,教媽惶恐的要不的。昨日娘斷七,我會下月姐和桂姐,也要送茶來,又不知宅內念經不念。」西門慶道:「斷七那日,胡亂請了幾位女僧,在家拜了拜懺。親眷一個都沒請,恐怕費煩。」飲酒說話之間,吳銀兒又問:「家中大娘衆娘每都好?」西門慶道:「都好。」吳銀兒道:「爹乍沒了娘,到房裡孤孤兒的,心中也想麼?{\meipi{筆之所至,何所不至。}}」西門慶道:「想是不消說。前日在書房中,白日夢見他,哭的我要不的。」吳銀兒道:「熱突突沒了,可知想哩!」伯爵道:「你每說的知情話,把俺每只顧旱着,不說來遞鍾酒,也唱個兒與俺聽。俺每起身去罷!」慌的李三、黃四連忙攛掇他姐兒兩個上來遞酒。安下樂器,吳銀兒也上來。三個粉頭一般兒坐在席上,躧着火盆,合着聲兒唱了套《中呂•粉蝶兒》「三弄梅花」,端的有裂石流雲之響。

唱畢,西門慶向伯爵說:「你索落他姐兒三個唱,你也下來酬他一盃兒。」伯爵道:「不打緊,死不了人。等我打發他:仰靠着,直舒着,側臥着,金雞獨立,隨我受用;又一件,野馬踩場,野狐抽絲,猿猴獻菓,黃狗溺尿,仙人指路,哥,隨他揀着要。」{\meipi{熱處生情,冷處生韻,尖處生巧,調笑是恆情,措思不落俗調。}}愛香道:「我不好罵出來的,汗邪了你這賊花子,胡說亂道的。」應伯爵用酒碟安三個鍾兒,說:「我兒,你每在我手裡吃兩鍾。不吃,望身上只一潑。」愛香道:「我今日忌酒。」愛月兒道:「你跪着月姨,教我打個嘴巴兒,我纔吃。」伯爵道:「銀姐,你怎的說?」吳銀兒道:「二爹,我今日心裡不自在,吃半盞兒罷。」愛月兒道:「花子,你不跪,我一百年也不吃。」黃四道:「二叔,你不跪,顯的不是趣人。也罷,跪着不打罷。」愛月兒道:「跪了也不打多,只教我打兩個嘴巴兒罷。」伯爵道:「溫老先兒,你看着,恠小淫婦兒只顧趕盡殺絕。」於是奈何不過,眞個直撅兒跪在地下。那愛月兒輕揎彩袖,款露春纖,罵道:「賊花子,再可敢無禮傷犯月姨了?高聲兒答應。你不答應,我也不吃。」伯爵無法可處,只得應聲道:「再不敢傷犯月姨了。」這愛月兒方連打了兩個嘴巴,方纔吃那鍾酒。{\meipi{寫得活活現現,眞覺生旦淨醜一齊搬出,吾恐排場中有此做作,無此神情也。}}伯爵起來道:「好個沒仁義的小淫婦兒,你也剩一口兒我吃。把一鍾酒都吃的淨淨兒的。」愛月兒道:「你跪下,等我賞你一鍾吃。」於是滿滿斟上一盃,笑望伯爵口裡只一灌。伯爵道,「恠小淫婦兒,使促狹灌撒了我一身。我老寔說,只這件衣服,新穿了纔頭一日兒,就汙濁了我的。我問你家漢子要。」{\meipi{先描伯爵衣飾,卻從此處照出,作者針線綜脚一毫不漏。}}笑了一囘,各歸席上坐定。

看看天晚,掌燭上來。西門慶分付取個骰盆來。先讓溫秀才,秀才道:「豈有此理!還從老先生來。」於是西門慶與銀兒用十二個骰兒搶紅,下邊四個妓女拏着樂器彈唱。飲過一巡,吳銀兒卻轉過來與溫秀才、伯爵搶紅,愛香兒卻來西門慶席上遞酒猜枚。須臾過去,愛月兒近前與西門慶搶紅,吳銀兒卻徃下席遞李三、黃四酒。原來愛月幾旋徃房中新粧打扮出來,上着烟裡火廻紋錦對衿襖兒、鵝黃杭絹點翠縷金裙、粧花膝褲、大紅鳳嘴鞋兒,燈下海獺臥兔兒,越顯的粉濃濃雪白的臉兒。眞是:

\begin{myquote} 
芳姿麗質更妖嬈,秋水精神瑞雪標。\\白玉生香花解語,千金良夜實難消。
\end{myquote} 

西門慶見了,如何不愛。吃了幾鍾酒,半酣上來,因想着李瓶兒夢中之言:少貪在外夜飲。一面起身後邊淨手。慌的鴇子連忙叫丫鬟點燈,引到後邊。解手出來,愛月隨即跟來伺候。盆中淨手畢,拉着他手兒同到房中。房中又早月窓半啟,銀燭高燒,氣暖如春,蘭麝馥郁,於是脫了上蓋,止穿白綾道袍,兩個在床上腿壓腿兒做一處。先是愛月兒問:「爹今日不家去罷了。」西門慶道:「我還去。今日一者銀兒在這裡,不好意思;二者我居着官,今年考察在邇,恐惹是非,只是白日來和你坐坐罷了。」又說:「前日多謝你泡螺兒。你送了去,倒惹的我心酸了半日。當初止有過世六娘他會揀。他死了,家中再有誰會揀他!」{\meipi{情至語,楚人心鼻。}}愛月道:「揀他不難,只是要拏的着禁節兒便好。那瓜仁都是我口裡一個個兒嗑的,說應花子倒撾了好些吃了。」

西門慶道:「你問那訕臉花子,兩把撾去,喃了好些。只剩下沒多,我吃了。」愛月兒道:「倒便益了賊花子,恰好只孝順了他。」{\meipi{閑閑叙來,語語鬆,節節緊。}}又說:「多謝爹的衣梅。媽看見,吃了一個兒,歡喜的要不的。他要便痰火發了,晚夕咳嗽半夜,把人聒死了。常時口乾,得恁一個在口裡噙着他,倒生好些津液。我和俺姐姐吃了沒多幾個兒,連礶兒他老人家都收在房內早晚吃,誰敢動他!」西門慶道:「不打緊,我明日使小厮再送一礶來你吃。」愛月又問:「爹連日會桂姐沒有?」西門慶道:「自從孝堂內到如今,誰見他來?」愛月兒道:「六娘五七,他也送茶去來?」西門慶道:「他家使李銘送去來。」愛月道:「我有句話兒,只放在爹心裡。」西門慶問:「甚麼話?」那愛月又想了想說:「我不說罷。{\meipi{寫出靈心巧舌。}}若說了,顯的姐妹每恰似我背地說他一般,不好意思的。」西門慶一面摟着他脖子說道:「恠小油嘴兒,甚麼話?說與我,不顯出你來就是了。」

兩個正說得入港,猛然應伯爵入來大叫一聲:{\meipi{顯然便說有何情致?插入伯爵,文情文趣悠然不盡。}}「你兩個好人兒,撇了俺每走在這裡說梯己話兒!」愛月兒道:「噦,好個不得人意恠訕臉花子!猛可走來,唬了人恁一跳!」西門慶罵:「恠狗才,前邊去罷。丟的葵軒和銀姐在那裡,都徃後頭來了。」這伯爵一屁股坐在床上,說:「你拏胳膊來,我且咬口兒,我纔去。你兩個在這裡盡着㒲搗!」於是不繇分說,向愛月兒袖口邊勒出那賽鵝脂雪白的手腕兒來,{\pangpi{美哉!}}誇道:「我兒,你這兩隻手兒,天生下就是發𩫻䯲的行貨子。」{\pangpi{趣。}}{\meipi{恠花子,趣絕矣。}}愛月兒道:「恠攮刀子的,我不好罵出來!」被伯爵拉過來,咬了一口走了。咬得老婆恠叫,罵:「恠花子,平白進來鬼混人死了!」便叫桃花兒:「你看他出去了,把弄道子門關上。」愛月便把李桂姐如今又和王三官兒好一節說與西門慶:「怎的有孫寡嘴、祝麻子、小張閑,架兒於寬、聶鉞兒,踢行頭白囘子、向三,日逐標着在他家行走。如今丟開齊香兒,又和秦家玉芝兒打熱,兩下里使錢。使沒了,將皮襖當了三十兩銀子,拏着他娘子兒一副金鐲子放在李桂姐家,算了一個月歇錢。」西門慶聽了,口中罵道:「這小淫婦兒,我恁分付休和這小厮纏,他不聽,還對着我賭身發咒,恰好只哄着我。」

愛月兒道:「爹也沒要惱。我說與爹個門路兒,管情教王三官打了嘴,替爹出氣。」西門慶把他摟在懷裡說道:「我的兒,有甚門路兒,說與我知道。」愛月兒道:「我說與爹,休教一人知道。就是應花子也休對他題,只怕走了風。」西門慶道:「你告我說,我傻了,肯教人知道!」鄭愛月道:「王三官娘林太太,今年不上四十歲,生的好不喬樣!描眉畫眼,打扮的狐狸也似。{\meipi{是老淫像贊。}}他兒子鎭日在院裡,他專在家,只尋外遇。假托在姑姑庵裡打齋,但去,就在說媒的文嫂兒家落脚。文嫂兒單管與他做牽頭,只說好風月。我說與爹,到明日遇他遇兒也不難。又一個巧宗兒:王三官娘子兒今纔十九歲,是東京六黃太尉姪女兒,上畫般標緻,雙陸棋子都會。三官常不在家,他如同守寡一般,好不氣生氣死。為他也上了兩三遭弔,救下來了。爹難得先刮剌上了他娘,不愁媳婦兒不是你的。」{\meipi{此語大不可訓。甚矣,此輩之不可近也!}}當下被他一席話兒,說的西門慶心邪意亂,摟着粉頭說:「我的親親,你怎的曉的就裡?」愛月兒就不說常在他家唱,只說:「我一個熟人兒,如此這般和他娘在某處會過一面,也是文嫂兒說合。」西門慶問:「那人是誰?莫不是大街坊張大戶姪兒張二官兒?」{\pangpi{伏案。}}愛月兒道:「那張懋德兒,好㒲的貨,麻着個臉蛋子,密縫兩個眼,可不砢硶殺我罷了!{\meipi{細眼,麻子,大受削刮。}}只好蔣家百家奴兒接他。」西門慶道:「我猜不着,端的是誰?」愛月兒道:「教爹得知了罷:原是梳籠我的一個南人。他一年來此做買賣兩遭,正經他在裡邊歇不的一兩夜,倒只在外邊常和人家偷貓遞狗,幹此勾當。」西門慶聽了,見粉頭所事,合着他的板眼,亦發歡喜,說:「我兒,你既貼戀我心,我每月送三十兩銀子與你媽盤纏,也不消接人了。我遇閑就來。」愛月兒道:「爹,你若有我心時,甚麼三十兩二十兩,隨着掠幾兩銀子與媽,我自恁懶待留人,只是伺候爹罷了。」西門慶道:「甚麼話!我決然送三十兩銀子來。」說畢,兩個上床交歡。床上鋪的被褥約一尺高,愛月道:「爹脫衣裳不脫?」西門慶道:「咱連衣耍耍罷,只怕他們前邊等咱。」一面扯過枕頭來,粉頭解去下衣,仰臥枕畔,西門慶把他兩隻小小金蓮扛在肩上,解開藍綾褲子,那話使上托子。但見花心輕折,柳腰款擺。正是:

\begin{myquote} 
花嫩不禁柔,春風卒未休。\\花心猶未足,脈脈情無極。\\低低喚粉郎,春宵樂未央。{\meipi{六語道的中情,可勝千萬言。}}
\end{myquote} 

兩個交歡良久,至精欲泄之際,西門慶幹的氣喘吁吁,粉頭嬌聲不絕,鬢雲拖枕,滿口只教:「親達達,慢着些兒!」少頃,樂極情濃,一泄如注。雲收雨散,各整衣理容,淨了手,同攜手來到席上。

吳銀兒和愛香兒正與葵軒、伯爵擲色猜枚,觥籌交錯,耍在熱鬧處。衆人見西門慶進入,俱立起身來讓坐。伯爵道:「你也下般的,把俺每丟在這裡,你纔出來,拏酒兒且扶扶頭着。」西門慶道:「俺每說句話兒,有甚閑勾當!」伯爵道:「好話,你兩個原來說梯己話兒。」當下伯爵拏大鐘斟上煖酒,衆人陪西門慶吃。四個妓女拏樂器彈唱。玳安在旁說道:「轎子來了。」西門慶呶了個嘴兒與他,那玳安連忙分付排軍打起燈籠,外邊伺候。西門慶也不坐,陪衆人執盃立飲。分付四個妓女:「你再唱個『一見嬌羞』我聽。」那韓消愁兒拏起琵琶來,款放嬌聲,拏腔唱道:

\begin{myquote} 
一見嬌羞,雨意雲情兩意投。我見他千嬌百媚,萬種妖嬈,一撚溫柔。通書先把話兒勾,傳情暗裡秋波溜。記在心頭。心頭,未審何時成就。
\end{myquote} 

唱了一個,吳銀兒遞西門慶酒,鄭香兒便遞伯爵,愛月兒奉溫秀才,李智、黃四都斟上。四妓女又唱了一個。吃畢,衆人又彼此交換遞了兩轉,妓女又唱了兩個。

唱畢,都飲過,西門慶就起身。一面令玳安向書袋內取出大小十一包賞賜來:四個妓女每人三錢,廚役賞了五錢,吳惠、鄭春、鄭奉每人三錢,攛掇打茶的每人二錢,丫頭桃花兒也與了他三錢。俱磕頭謝了。黃四再三不肯放,道:「應二叔,你老人家說聲,天還早哩。老爹大坐坐,也盡小人之情,如何就要起身?我的月姨,你也留留兒。」愛月兒道:「我留他,他白不肯坐。」西門慶道:「你每不知,我明日還有事。」一面向黃四作揖道:「生受打攪!」黃四道:「惶恐!沒的請老爹來受餓,又不肯久坐,還是小人沒敬心。」說着,三個唱的都磕頭說道:「爹到家多頂上大娘和衆娘們,俺每閑了,會了銀姐徃宅內看看大娘去。」西門慶道:「你每閑了去坐上一日來。」一面掌起燈籠,西門慶下臺磯,鄭家鴇子迎着道萬福,說道:「老爹大坐囘兒,慌的就起身,嫌俺家東西不美口?還有一道米飯兒未曾上哩!」西門慶道:「勾了。我明日還要起早,衙門中有勾當。應二哥他沒事,教他大坐囘兒罷。」那伯爵就要跟着起來,被黃四使力攔住,說道:「我的二爺,你若去了,就沒趣死了。」伯爵道:「不是,你休攔我。你把溫老先生有本事留下,我就算你好漢。」那溫秀才奪門就走,被黃家小厮來定兒攔腰抱住。{\meipi{去的象個要去,留的象個要留,吃的象個要吃,寫生也。}}西門慶到了大門首,因問琴童兒:「溫師父有頭口在這裡沒有?」琴童道:「備了驢子在此,畫童兒看着哩。」西門慶向溫秀才道:「既有頭口,也罷,老先兒你再陪應二哥坐坐,我先去罷。」於是,都送出門來。那鄭月兒拉着西門慶手兒,悄悄捏了一把,{\meipi{一轉秋波。}}說道:「我說的話,爹你在心些,法不傳六耳。」西門慶道:「知道了。」愛月又叫鄭春:「你送老爹到家。」西門慶纔上轎去了。吳銀兒就在門首作辭了衆人並鄭家姐兒兩個,吳惠打着燈囘家去了。鄭月兒便叫:「銀姐,見了那個流人兒,{\pangpi{指桂姐。}}好歹休要說。」吳銀兒道:「我知道。」衆人囘至席上,重添獸炭,再泛流霞,歌舞吹彈,歡娛樂飲,直耍了三更方散。黃四擺了這席酒,也與了他十兩銀子,不在話下。

當日西門慶坐轎子,兩個排軍打着燈,逕出院門,打發鄭春囘家。一宿晚景題過。

到次日,夏提刑差答應的來,請西門慶早徃衙門中審問賊情等事,直問到晌午來家。吃了飯,早是沈姨夫差大官沈定,拏帖兒送了個後生來,在段子鋪煮飯做火頭,名喚劉包。西門慶留下了,正在書房中,拏帖兒與沈定囘家去了。

只見玳安在旁邊站立,西門慶便問道:「溫師父昨日多咱來的?」玳安道:「小的鋪子裡睡了好一囘,只聽見畫童兒打對過門,那咱有三更時分纔來了。今早問,溫師父倒沒酒;應二爹醉了,唾了一地,月姨恐怕夜深了,使鄭春送了他家去了。」西門慶聽了,哈哈笑了,因叫過玳安近前,說道:「舊時與你姐夫說媒的文嫂兒在那裡住?你尋了他來,對門房子裡見我。我和他說話。」玳安道:「小的不認的文嫂兒家,等我問了姐夫去。」西門慶道:「你問了他快去。」玳安走到鋪子裡問陳敬濟,敬濟道:「問他做甚麼?」玳安道:「誰知他做甚麼,猛可教我抓尋他去。」{\meipi{說得路數一些不差。}}敬濟道:「出了東大街一直徃南去,過了同仁橋牌坊轉過徃東,打王家巷進去,半中腰裡有個發放巡捕的廳兒,對門有個石橋兒,轉過石橋兒,緊靠着個姑姑庵兒,旁邊有個小衚衕兒,進小衚衕徃西走,第三家豆腐鋪隔壁上坡兒,有雙扇紅對門兒的就是他家。你只叫文媽,他就出來答應你。」玳安聽了說道:「再沒有?小爐匠跟着行香的走,瑣碎一浪蕩。你再說一遍我聽,{\pangpi{趣。}}只怕我忘了。」那陳敬濟又說了一遍,玳安道:「好近路兒!等我騎了馬去。」一面牽出大白馬來騎上,打了一鞭,那馬跑踍跳躍,一直去了。出了東大街逕徃南,過同仁橋牌坊,繇王家巷進去,果然中間有個巡捕廳兒,對門亦是座破石橋兒,裡首半截紅墻是大悲庵兒,徃西小衚衕上坡,挑着個豆腐牌兒,門首只見一個媽媽晒馬糞。玳安在馬上就問:「老媽媽,這裡有個說媒的文嫂兒?」那媽媽道:「這隔壁對門兒就是。」

玳安到他門首,果然是兩扇紅對門兒,連忙跳下馬來,拏鞭兒敲着門叫道:「文嫂在家不在?」只見他兒子文ら開了門,問道:「是那裡來的?」玳安道:「我是縣門前提刑西門老爹家來請,教文媽快去哩。」文ら聽見是提刑西門大官府裡來的,便讓家裡坐。那玳安把馬拴住,進入裡面。見上面供養着利市紙,有幾個人在那裡算進香帳哩。半日拏了鍾茶出來,說道:「俺媽不在了。來家說了,明日早去罷。」玳安道:「驢子見在家裡,如何推不在?」{\pangpi{賊。}}側身逕徃後走。不料文嫂和他媳婦兒,陪着幾個道媽媽子正吃茶,躱不及,被他看見了,說道:「這個不是文媽?就囘我不在家!」文嫂笑哈哈與玳安道了個萬福,說道:「累哥哥到家囘聲,我今日家裡會茶。不知老爹呼喚我做甚麼,我明日早去罷。」玳安道:「只分忖我來尋你,誰知他做甚麼。原來你在這咭溜搭剌兒裡住,教我抓尋了個小發昏。」文嫂兒道:「他老人家這幾年買使女,說媒,用花兒,自有老馮和薛嫂兒、王媽媽子走跳,稀罕俺每!今日忽剌八又冷鍋中豆兒爆,{\meipi{口角宛然。}}我猜着你六娘沒了,已定教我去替他打聽親事,要補你六娘的窩兒。」玳安道:「我不知道。你到那裡,俺爹自有話和你說。」文嫂兒道:「既如此,哥哥你畧坐坐兒,等我打發會茶人去了,同你去罷。」玳安道:「俺爹在家緊等的火裡火發,分付了又分付,教你快去哩。和你說了話,還要徃府裡羅同知老爹家吃酒去哩。」文嫂道:「也罷,等我拏點心你吃了,同你去。」玳安道:「不吃罷。」文嫂因問:「你大姐生了孩兒沒有?」玳安道:「還不曾見哩。」文嫂一面打發玳安吃了點心,穿上衣裳,說道:「你騎馬先行一步兒,我慢慢走。」玳安道:「你老人家放着驢子,怎不備上騎?」文嫂兒道:「我那討個驢子來?那驢子是隔壁豆腐鋪裡的,借俺院兒裡喂喂兒,你就當我的。」玳安道:「記的你老人家騎着匹驢兒來,徃那去了?」文嫂兒道:「這咱哩!那一年弔死人家丫頭,打官司把舊房兒也賣了,且說驢子哩!」玳安道:「房子到不打緊,且留着那驢子和你早晚做伴兒也罷了。別的罷了,我見他常時落下來好個大鞭子。」{\pangpi{妙謔。}}文嫂哈哈笑道:「恠猴子,短壽命,老娘還只當好話兒,側着耳朵聽。幾年不見,你也學的恁油嘴滑舌的。到明日,還教我尋親事哩!」玳安道:「我的馬走的快,你步行,赤道捱磨到多咱晚,不惹的爹說?你也上馬,咱兩個疊騎着罷。」文嫂兒道:「恠小短命兒,我又不是你影射的!街上人看着,恠剌剌的。」玳安道:「再不,你備豆腐鋪裡驢子騎了去,到那裡等我打發他錢就是了。」文嫂兒道:「這還是話。」一面教文ら將驢子備了,帶上眼紗,騎上,玳安與他同行,逕徃西門慶宅中來。正是:

\begin{myquote} 
欲向深閨求艷質,全憑紅葉是良媒。
\end{myquote} 

