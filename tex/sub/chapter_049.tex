\includepdf[pages={97,98},fitpaper=false]{tst.pdf}
\chapter*{第四十九囘 請巡按屈體求榮 遇胡僧現身施藥}
\addcontentsline{toc}{chapter}{第四十九囘 請巡按屈體求榮 遇胡僧現身施藥}
\markboth{{\titlename}卷之五}{第四十九囘 請巡按屈體求榮 遇胡僧現身施藥}


詩曰:

雅集無兼客,高情洽二難。一尊傾智海,八斗擅吟壇。話到如生旭,霜來恐不寒。為行王捨乞,玄屑帶雲餐。

話說夏壽到家囘覆了話,夏提刑隨即就來拜謝西門慶,說道:「長官活命之恩,不是托賴長官餘光,這等大力量,如何了得!」西門慶笑道:「長官放心。料着你我沒曾過為,隨他說去,老爺那里自有個明見。」一面在廳上放桌兒留飯,談笑至晚,方纔作辭囘家。到次日,依舊入衙門裡理事,不在話下。

卻表巡按曾公,見本上去不行,就知道二官打點了,心中忿怒。因蔡太師所陳七事,內多舛訛,皆損下益上之事,即赴京見朝覆命,上了一道表章。極言:「天下之財貴於通流,取民膏以聚京師,恐非太平之治。民間結糶俵糴之法不可行,當十大錢不可用,鹽鈔法不可屢更。臣聞民力殫矣,誰與守邦?」蔡京大怒,奏上徽宗天子,說他大肆倡言,阻撓國事。將曾公付吏部考察,黜為陝西慶州知州。陝西巡按御史宋盤,就是學士蔡攸之婦兄也。太師陰令盤就劾其私事,逮其家人,鍛練成獄,將孝序除名,竄於嶺表,以報其仇。此係後事,表過不題。

再說西門慶在家,一面使韓道國與喬大戶外甥崔本,拏倉鈔早徃高陽關戶部韓爺那裡趕着掛號。留下來保家中定下菓品,預備大桌面酒席,打聽蔡御史船到。一日,來保打聽得他與巡按宋御史船一同京中起身,都行至東昌府地方,使人來家通報。這裡西門慶就會夏提刑起身。來保從東昌府船上就先見了蔡御史,送了下程。然後,西門慶與夏提刑出郊五十里,迎接到新河口,地名百家村。先到蔡御史船上拜見了,備言邀請宋公之事。蔡御史道:「我知道,已定同他到府。」那時,東平胡知府,及合屬州縣方面有司軍衛官員、吏典生員、僧道陰陽,都具連名手本,伺候迎接。帥府周守備、荊都監、張團練,都領人馬披執跟隨,清蹕傳道,雞犬皆隱跡。鼓吹迎接宋巡按進東平府察院,各處官員都見畢,呈遞了文書,安歇一夜。

到次日,只見門吏來報:「巡鹽蔡爺來拜。」宋御史連忙出迎。叙畢禮數,分賓主坐下。獻茶已畢,宋御史便問:「年兄幾時方行?」蔡御史道:「學生還待一二日。」因告說:「清河縣有一相識西門千兵,乃本處巨族,為人清慎,富而好禮,亦是蔡老先生門下,與學生有一面之交。蒙他遠接,學生正要到他府上拜他拜。」宋御史問道:「是那個西門千兵?」蔡御史道:「他如今見是本處提刑千戶,昨日已叅見過年兄了。」宋御史令左右取手本來看,見西門慶與夏提刑名字,說道:「此莫非與翟雲峰有親者?」蔡御史道:「就是他。如今見在外面伺候,要央學生奉陪年兄到他家一飯。未審年兄尊意若何?」宋御史道:「學生初到此處,只怕不好去得。」蔡御史道:「年兄怕怎的?既是雲峰分上,你我走走何害?」{\pangpi{可笑。}}於是分付看轎,就一同起行,一面傳將出來。

西門慶知了此訊息,與來保、賁四騎快馬先奔來家,預備酒席。門首搭照山彩棚,兩院樂人奏樂,叫海鹽戲並雜耍承應。原來宋御史將各項伺候人馬都令散了,只用幾個藍旗清道官吏跟隨,與蔡御史坐兩頂大轎,打着雙簷傘,同徃西門慶家來。當時鬨動了東平府,大鬧了清河縣,都說:「巡按老爺也認的西門大官人,來他家吃酒來了。」慌的周守備、荊都監、張團練,各領本哨人馬把住左右街口伺候。西門慶青衣冠帶,遠遠迎接。兩邊鼓樂吹打,到大門首下了轎進去。宋御史與蔡御史都穿着大紅獬豸繡服,烏紗皁履,鶴頂紅帶,從人執着兩把大扇。只見五間廳上湘簾高捲,錦屏羅列。正面擺兩張吃看桌席,高頂方糖,定勝簇盤,十分齊整。二官揖讓進廳,與西門慶叙禮。蔡御史令家人具贄見之禮:兩端湖紬、一部文集、四袋芽茶、一方端溪硯。宋御史只投了個宛紅單拜帖,上書「侍生宋喬年拜」。向西門慶道:「久聞芳譽。學生初臨此地,尚未盡情,不當取擾。若不是蔡年兄邀來進拜,何以幸接尊顏?」慌的西門慶倒身下拜,說道:「僕乃一介武官,屬於按臨之下。今日幸蒙清顧,蓬蓽生光。」於是鞠恭展拜,禮容甚謙。宋御史亦答禮相還,叙了禮數。當下蔡御史讓宋御史居左,他自在右,西門慶垂首相陪。茶湯獻罷,堦下簫韶盈耳,鼓樂喧闐,動起樂來。西門慶遞酒安席已畢,下邊呈獻割道。說不盡餚列珍羞,湯陳桃浪,端的歌舞聲容,食前方丈。兩位轎上跟從人,每位五十瓶酒、五百點心、一百斤熟肉,都領下去。家人、吏書、門子人等,另在廂房中管待,不必細說。當日西門慶這席酒,也費勾千兩金銀。

那宋御史又系江西南昌人,為人浮躁,只坐了沒多大囘,聽了一折戲文就起來。慌的西門慶再三固留。蔡御史在旁便說:「年兄無事,再消坐一時,何遽囘之太速耶!」宋御史道:「年兄還坐坐,學生還欲到察院中處分些公事。」西門慶早令手下,把兩張桌席連金銀器,已都裝在食盒內,共有二十擡,叫下人夫伺候。宋御史的一張大桌席、兩罈酒、兩牽羊、兩封金絲花、兩疋段紅、一副金臺盤、兩把銀執壺、十個銀酒盃、兩個銀折盂、一雙牙筯。蔡御史的也是一般的。都遞上揭帖。宋御史再三辭道:「這個,我學生怎麼敢領?」因看着蔡御史。蔡御史道:「年兄貴治所臨,自然之道,我學生豈敢當之!」西門慶道:「些須微儀,不過侑觴而已,何為見外?」比及二官推讓之次,而桌席已擡送出門矣。宋御史不得已,方令左右收了揭帖,向西門慶致謝說道:「今日初來識荊,既擾盛席,又承厚貺,何以克當?餘容圖報不忘也。」因向蔡御史道:「年兄還坐坐,學生告別。」於是作辭起身。西門慶還要遠送,宋御史不肯,急令請囘,舉手上轎而去。

西門慶囘來,陪侍蔡御史,解去冠帶,請去捲棚內後坐。因分付把樂人都打發散去,只留下戲子。西門慶令左右重新安放桌席,擺設珍羞菓品上來,二人飲酒。蔡御史道:「今日陪我這宋年兄坐便僭了,又叨盛筵並許多酒器,何以克當?」西門慶笑道:「微物惶恐,表意而已!」因問道:「宋公祖尊號?」蔡御史道:「號松原。松樹之松,原泉之原。」又說起:「頭裡他再三不來,被學生因稱道四泉盛德,與老先生那邊相熟,他纔來了。他也知府上與雲峰有親。」西門慶道:「想必翟親家有一言於彼。我觀宋公為人有些蹺蹊。」蔡御史道:「他雖故是江西人,倒也沒甚蹊蹺處。只是今日初會,怎不做些模樣!」說畢笑了。西門慶便道:「今日晚了,老先生不囘船上去罷了。」蔡御史道:「我明早就要開船長行。「西門慶道:「請不棄在舍留宿一宵,明日學生長亭送餞。」蔡御史道:「過蒙愛厚。」因分付手下人:「都囘門外去罷,明早來接。」衆人都應諾去了,只留下兩個家人伺候。

西門慶見手下人都去了,走下席來,叫玳安兒附耳低言,如此這般:「即去院裡坐名叫了董嬌兒、韓金釧兒兩個,打後門裏用轎子擡了來,休交一人知道。」那玳安一面應諾去了。西門慶復上席,陪蔡御史吃酒。海鹽子弟在旁歌唱。西門慶因問:「老先生到家多少時就來了?令堂老夫人起居康健麼?」蔡御史道:「老母到也安。學生在家,不覺荏苒半載,囘來見朝,不想被曹禾論劾,將學生敝同年一十四人之在史館者,一時皆黜授外職。{\meipi{做官的此等處要自反。}}學生便選在西臺,新點兩淮巡鹽。宋年兄便在貴處巡按,也是蔡老先生門下。」西門慶問道:「如今安老先生在那裡?」蔡御史道:「安鳳山他已陞了工部主事,徃荊州催攢皇木去了。也待好來也。」說畢,西門慶教海鹽子弟上來遞酒。蔡御史分付:「你唱個《漁家傲》我聽。」子弟排手在旁正唱着,只見玳安走來請西門慶下邊說話。玳安道:「叫了董嬌兒、韓金釧打後門來了,在娘房裡坐着哩。」西門慶道:「你分付把轎子擡過一邊纔好。」玳安道:「擡過一邊了。」這西門慶走至上房,兩個唱的向前磕頭。西門慶道:「今日請你兩個來,晚夕在山子下扶侍你蔡老爹。他如今見做巡按御史,你不可怠慢,用心扶侍他,我另酬答你。」韓金釧兒笑道:「爹不消分付,俺每知道。」西門慶因戲道:「他南人的營生,好的是南風,你每休要扭手扭脚的。」董嬌兒道:「娘在這裡聽着,爹你老人家『羊角蔥靠南墻——越發老辣了』。王府門首磕了頭,俺們不吃這井裡水了?」西門慶笑的徃前邊來。走到儀門首,只見來保和陳敬濟拏着揭帖走來,與西門慶看,說道:「剛纔喬親家爹說,趁着蔡老爹這囘閑,爹倒把這件事對蔡老爹說了罷,只怕明日起身忙了。教姐夫寫了俺兩個名字在此。」西門慶道:「你跟了來。」來保跟到捲棚槅子外邊站着。西門慶飲酒中間因題起:「有一事在此,不敢干瀆。」蔡御史道:「四泉,有甚事只顧分付,學生無不領命。」西門慶道:「去歲因舍親在邊上納過些糧草,坐派了些鹽引,正派在貴治揚州支鹽。望乞到那裡青目青目,早些支放就是愛厚。」因把揭帖遞上去,蔡御史看了。上面寫着:「商人來保、崔本,舊派淮鹽三萬引,乞到日早掣。」蔡御史看了笑道:「這個甚麼打緊。」一面把來保叫至跟前跪下,分付:「與你蔡爺磕頭。」蔡御史道:「我到揚州,你等徑來察院見我。我比別的商人早掣一個月。」西門慶道:「老先生下顧,早放十日就勾了。」蔡御史把原帖就袖在袖內。一面書童旁邊斟上酒,子弟又唱。唱畢,已有掌燈時分,蔡御史便說:「深擾一日,酒告止了罷。」因起身出席,左右便欲掌燈,西門慶道:「且休掌燭,請老先生後邊更衣。」於是從花園裡遊玩了一囘,讓至翡翠軒,那裡又早湘簾低簇,銀燭熒煌,設下酒席。海鹽戲子,西門慶已命打發去了。書童把捲棚內家活收了,關上角門,只見兩個唱的盛粧打扮,立於堦下,向前插燭也似磕了四個頭。但見:

綽約容顏金縷衣,香塵不動下堦墀。時來水濺羅裙濕,好似巫山行雨歸。

蔡御史看見,欲進不能,欲退不捨。便說道:「四泉,你如何這等愛厚?恐使不得。」西門慶笑道:「與昔日東山之遊,又何異乎?」蔡御史道:「恐我不如安石之才,而君有王右軍之高致矣。」於是月下與二妓攜手,恍若劉阮之入天台。因進入軒內,見文物依然,因索紙筆,就欲留題相贈。西門慶即令書童連忙將端溪硯,研的墨濃濃的,拂下錦箋。這蔡御史終是狀元之才,拈筆在手,文不加點,字走龍蛇,燈下一揮而就,作詩一首。詩曰:

不到君家半載餘,軒中文物尚依稀。雨過書童開藥圃,風囘仙子步花臺。飲將醉處鍾何急,詩到成時漏更催。此去又添新悵望,不知何日是重來。

寫畢,教書童粘於壁上,以為後日之遺焉。因問二妓:「你們叫甚名字?」一個道:「小的姓董,名喚嬌兒。他叫韓金釧兒。」蔡御史又道:「你二人有號沒有?」董嬌兒道:「小的無名娼妓,那討號來?」蔡御史道:「你等休要太謙。」問至再三,韓金釧方說:「小的號玉卿。」董嬌兒道:「小的賤號薇仙。」{\meipi{此字原佳。}}蔡御史一聞「薇仙」二字,心中甚喜,遂留意在懷。令書童取棋桌來,擺下棋子,蔡御史與董嬌兒兩個着棋。西門慶陪侍,韓金釧兒把金樽在旁邊遞酒,書童歌唱。蔡御史贏了一盤棋,董嬌兒吃過,又囘奉蔡御史一盃。韓金釧這裡也遞與西門慶一盃陪飲。飲了酒,兩人又下。董嬌兒贏了,連忙遞酒一盃與蔡御史,西門慶在旁又陪飲一盃。飲畢,蔡御史道:「四泉,夜深了,不勝酒力,」於是走出外邊來,站立在花下。

那時正是四月半頭,月色纔上。西門慶道:「老先生,天色還早哩。還有韓金釧,不曾賞他一盃酒。」蔡御史道:「正是。你喚他來,我就此花下立飲一盃。」於是韓金釧拏大金桃盃,滿斟一盃,用纖手捧遞上去。董嬌兒在旁捧果,蔡御史吃過,又斟了一盃,賞與韓金釧兒。因告辭道:「四泉,今日酒太多了,令盛价收過去罷。」於是與西門慶握手相語,說道:「賢公盛情盛德,此心懸懸。非斯文骨肉,何以至此?向日所貸,學生耿耿在心,在京已與雲峰表過。倘我後日有一步寸進,斷不敢有辜盛德。」西門慶道:「老先生何出此言?到不消介意。」韓金釧見他一手拉着董嬌兒,知局就徃後邊去了。到了上房裡,月娘問道:「你怎的不陪他睡,來了?」韓金釧笑道:「他留下董嬌兒了,我不來,只管在那裡做甚麼?」

良久,西門慶亦告了安置進來,叫了來興兒分付:「明日早五更,打發食盒酒米點心下飯,叫了廚役,跟了徃門外永福寺去,與你蔡老爹送行。叫兩個小優兒答應。休要誤了。」來興兒道:「家裡二娘上壽,沒有人看。」西門慶道:「留下棋童兒買東西,叫廚子後邊大竈上做罷。」

不一時,書童、玳安收下家活來,又討了一壺好茶,徃花園裡去與蔡老爹漱口。翡翠軒書房床上,鋪陳衾枕俱各完備。蔡御史見董嬌兒手中拏着一把湘妃竹泥金面扇兒,上面水墨畫着一種湘蘭平溪流水。董嬌兒道:「敢煩老爹賞我一首詩在上面。」蔡御史道:「無可為題,就指着你這薇仙號。」於是燈下拈起筆來,寫了四句在上:

小院閑庭寂不譁,一池月上浸窓紗。邂逅相逢天未晚,紫薇郎對紫薇花。

寫畢,那董嬌兒連忙拜謝了。兩個收拾上床就寢。書童、玳安與他家人在明間裡睡。一宿晚景不題。

次日早晨,蔡御史與了董嬌兒一兩銀子,用紅紙大包封着,到於後邊,拏與西門慶瞧。西門慶笑說道:「文職的營生,他那裡有大錢與你!這個就是上上籤了。」因交月娘每人又與了他五錢銀子,從後門打發去了。書童舀洗面水,打發他梳洗穿衣。西門慶出來,在廳上陪他吃了粥。手下又早伺候轎馬來接,與西門慶作辭,謝了又謝。西門慶又道:「學生日昨所言之事,老先生到彼處,學生這裡書去,千萬留神一二,足仞不淺。」蔡御史道:「休說賢公華紮下臨,只盛价有片紙到,學生無不奉行。」說畢,二人同上馬,左右跟隨。出城外,到於永福寺,借長老方丈擺酒餞行。來興兒與廚役早已安排桌席停當。李銘、吳惠兩個小優彈唱。數盃之後,坐不移時,蔡御史起身,夫馬、坐轎在於三門外伺候。臨行,西門慶說起苗青之事:「乃學生相知,因詿誤在舊大巡曾公案下,行牌徃揚州案候捉他。此事情已問結了。倘見宋公,望乞借重一言,彼此感激。」蔡御史道:「這個不妨,我見宋年兄說,設使就提來,放了他去就是了。」西門慶又作揖謝了。看官聽說:後來宋御史徃濟南去,河道中又與蔡御史會在那船上。公人揚州提了苗青來,蔡御史說道:「此係曾公手裡案外的,你管他怎的?」遂放囘去了。倒下詳去東平府,還只把兩個船家決不待時,安童便放了。正是:

公道人情兩是非,人情公道最難為。若依公道人情失,順了人情公道虧。

當日西門慶要送至船上,蔡御史不肯,說道:「賢公不消遠送,只此告別。」西門慶道:「萬惟保重,容差小价問安。」說畢,蔡御史上轎而去。西門慶囘到方丈坐下,長老走來合掌問訊,遞茶,西門慶答禮相還。見他雪眉交白,便問:「長老多大年紀?」長老道:「小僧七十有四。」西門慶道:「到還這等康健。」因問法號,長老道:「小僧法名道堅。」又問:「有幾位徒弟?」長老道:「止有兩個小徒。本寺也有三十餘僧行。」西門慶道:「這寺院也寬大,只是欠修整。」長老道:「不瞞老爹說,這座寺原是周秀老爹蓋造,長住裡沒錢糧修理,丟得壞了。」西門慶道:「原來就是你守備府周爺的香火院。我見他家庄子不遠。不打緊處,你稟了你周爺,寫個緣簿,別處也再化些,我也資助你些布施。」道堅連忙又合掌問訊謝了。西門慶分付玳安兒:「取一兩銀子謝長老。今日打攪。」道堅道:「小僧不知老爹來,不曾預備齋供。」西門慶道:「我要徃後邊更更衣去。」道堅連忙叫小沙彌開門。西門慶更了衣,因見方丈後面五間大禪堂,有許多雲遊和尚在那裡敲着木魚看經。西門慶不因不繇,信步走入裡面觀看。見一個和尚形骨古恠,相貌搊搜,生的豹頭凹眼,色若紫肝,戴了雞蠟箍兒,穿一領肉紅直裰。頦下髭鬚亂拃,頭上有一溜光簷,就是個形容古恠真羅漢,未除火性獨眼龍。{\meipi{細看此僧,卻像何物。}}在禪床上旋定過去了,垂着頭,把脖子縮到腔子裡,鼻孔中流下玉筯來。西門慶口中不言,心中暗道:「此僧必然是個有手段的高僧。不然,如何因此異相?等我叫醒他,問他個端的。」於是高聲叫:「那位僧人,你是那裡人氏,何處高僧?」叫了頭一聲不答應;第二聲也不言語;第三聲,只見這個僧人在禪床上把身子打了個挺,伸了伸腰,睜開一隻眼,跳將起來,向西門慶點了點頭兒,{\meipi{和尚舉止,與陽物原差不遠。}}粗聲應道:「你問我怎的?貧僧行不更名,坐不改姓,乃西域天竺國密松林、齊腰峰、寒庭寺下來的胡僧,雲遊至此,施藥濟人。官人,你叫我有甚話說?」西門慶道:「你既是施藥濟人,我問你求些滋補的藥兒,你有也沒有?」胡僧道:「我有,我有。」又道:「我如今請你到家,你去不去?」胡僧道:「我去,我去。」西門慶道:「你說去,即此就行。」那胡僧直豎起身來,向床頭取過他的鐵柱杖來拄着,背上他的皮褡褳。褡褳內盛了兩個藥葫蘆兒。下的禪堂,就徃外走。西門慶分付玳安:「叫了兩個驢子,同師父先徃家去等着,我就來。」胡僧道:「官人不消如此,你騎馬只顧先行。貧僧也不騎頭口,管情比你先到。」西門慶道:「已定是個有手段的高僧。不然如何開這等朗言。」恐怕他走了,分付玳安:「好歹跟着他同行。」於是作辭長老上馬,僕從跟隨,逕直進城來家。

那日四月十七日,不想是王六兒生日,家中又是李嬌兒上壽,有堂客吃酒。後晌時分,只見王六兒家沒人使,使了他兄弟王經來請西門慶。分付他宅門首隻尋玳安兒說話,不見玳安在門首,只顧立。立了約一個時辰,正值月娘與李嬌兒送院裡李媽媽出來上轎,看見一個十五六歲扎包髻兒小厮,問是那裡的。那小厮三不知走到跟前,與月娘磕了個頭,說道:「我是韓家,尋安哥說話。」月娘問:「那安哥?」平安在旁邊,恐怕他知道是王六兒那裡來的,恐怕他說岔了話,向前把他拉過一邊,對月娘說:「他是韓夥計家使了來尋玳安兒,問韓夥計幾時來。」以此哄過。月娘不言語,囘後邊去了。

不一時,玳安與胡僧先到門首,走的兩腿皆酸,渾身是汗,抱怨的要不的。那胡僧體貌從容,氣也不喘。平安把王六兒那邊使了王經來請爹,尋他說話一節,對玳安兒說了一遍,道:「不想大娘看見,早是我在旁邊替他摭拾過了。不然就要露出馬脚來了。等住囘娘若問,你也是這般說。」那玳安走的睜睜的,只顧𢵞扇子:「今日造化低也怎的?平白爹交我領了這賊禿囚來。好近路兒!從門外寺裡直走到家,路上通沒歇脚兒,走的我上氣兒接不着下氣兒。爹交顧驢子與他騎,他又不騎。他便走着沒事,難為我這兩條腿了!把鞋底子也磨透了,脚也踏破了。攘氣的營生!」平安道:「爹請他來家做甚麼?」玳安道:「誰知道!他說問他討甚麼藥哩。」

正說着,只聞喝道之聲。西門慶到家,看見胡僧在門首,說道:「吾師真乃人中神也。果然先到。」一面讓至裡面大廳上坐。西門慶叫書童接了衣裳,換了小帽,陪他坐的。吃了茶,那胡僧睜眼觀見廳堂高遠,院宇深沉,門上掛的是龜背紋蝦鬚織抹綠珠簾,地下鋪獅子滾繡球絨毛線毯。正當中放一張蜻蜓腿、螳螂肚、肥皂色起楞的桌子,桌子上安着縧環樣須彌座大理石屏風。周圍擺的都是泥鰍頭、楠木靶腫觔的交倚,兩壁掛的畫都是紫竹桿兒綾邊、瑪瑙軸頭。{\meipi{讀此書者於器用食物,皆病其贅,誠潛心細讀數遍,方知其非贅也。}}正是:

鼉皮畫鼓振庭堂,烏木春臺盛酒器。

胡僧看畢,西門慶問道:「吾師用酒不用?」胡僧道:「貧僧酒肉齊行。」西門慶一面分付小厮:「後邊不消看素饌,拏酒飯來。」

那時正是李嬌兒生日,廚下餚饌下飯都有。安放桌兒,只顧拏上來。先綽邊兒放了四碟菓子、四碟小菜,又是四碟案酒:一碟頭魚、一碟糟鴨、一碟烏皮雞、一碟舞鱸公。又拏上四樣下飯來:一碟羊角蔥𤆑炒的核桃肉、一碟細切的[][]樣子肉、一碟肥肥的羊貫腸、一碟光溜溜的滑鰍。次又拏了一道湯飯出來:一個碗內兩個肉圓子,夾着一條花腸滾子肉,名喚一龍戲二珠湯;一大盤裂破頭高裝肉包子。西門慶讓胡僧吃了,教琴童拏過團靶鉤頭雞脖壺來,打開腰州精製的紅泥頭,一股一股邈出滋陰摔白酒來,傾在那倒垂蓮蓬高脚鍾內,遞與胡僧。那胡僧接放口內,一吸而飲之。隨即又是兩樣添換上來:一碟寸扎的騎馬腸兒、一碟子醃臘鵝脖子。又是兩樣艷物與胡僧下酒:一碟子癩葡萄、一碟子流心紅李子。落後又是一大碗鱔魚麵與菜卷兒,一齊拏上來與胡僧打散。登時把胡僧吃的楞子眼兒,便道:「貧僧酒醉飯飽,足以勾了。」

西門慶叫左右拏過酒桌去,因問他求房術的藥兒。胡僧道:「我有一枝藥,乃老君練就,王母傳方。非人不度,非人不傳,專度有緣。既是官人厚待於我,我與你幾丸罷。」於是向褡褳內取出葫蘆來,傾出百十丸,分付:「每次只一粒,不可多了,用燒酒送下。」又將那一個葫兒捏了,取二錢一塊粉紅膏兒,分付:「每次只許用二厘,不可多用。若是脹的慌,用手捏着,兩邊腿上只顧摔打,百十下方得通。你可樽節用之,不可輕泄於人。」西門慶雙手接了,說道:「我且問你,這藥有何功效?」胡僧說:「形如雞卵,色似鵝黃。三次老君炮練,王母親手傳方。外視輕如糞土,內覷貴乎玕琅。比金金豈換,比玉玉何償!任你腰金衣紫,任你大廈高堂,任你輕裘肥馬,任你才俊棟梁,此藥用托掌內,飄然身人洞房。洞中春不老,物外景長芳;玉山無頹敗,丹田夜有光。一戰精神爽,再戰氣血剛。不拘嬌艷寵,十二美紅粧,交接從吾好,徹夜硬如槍。服久寬脾胃,滋腎又扶陽。百日鬚髮黑,千朝體自強。固齒能明目,陽生姤始藏。恐君如不信,拌飯與貓嘗:三日淫無度,四日熱難當;白貓變為黑,尿糞俱停亡;夏月當風臥,冬天水裡藏。若還不解泄,毛脫盡精光。每服一厘半,陽興愈健強。一夜歇十女,其精永不傷。老婦顰眉蹙,淫娼不可當。有時心倦怠,收兵罷戰場。冷水吞一口,陽囘精不傷。快美終宵樂,春色滿蘭房。贈與知音客,永作保身方。」

西門慶聽了,要問他求方兒,說道:「請醫須請良,傳藥須傳方。吾師不傳於我方兒,倘或我久後用沒了,那裡尋師父去?隨師父要多少東西,我與師父。」因令玳安:「後邊快取二十兩白金來。」遞與胡僧,要問他求這一枝藥方。那胡僧笑道:「貧僧乃出家之人,雲遊四方,要這資財何用?{\meipi{果然用不着。高僧,高僧。}}官人趁早收拾囘去。」一面就要起身。西門慶見他不肯傳方,便道:「師父,你不受資財,我有一疋五丈長大布,與師父做件衣服罷。」即令左右取來,雙手遞與胡僧。胡僧方纔打問訊謝了。臨出門又分付:「不可多用,戒之!戒之!」言畢,背上褡褳,拴定柺杖,出門揚長而去。正是:

柱杖挑擎雙日月,芒鞋踏遍九軍州。

