\includepdf[pages={35,36},fitpaper=false]{tst.pdf}
\chapter*{第十八囘 賂相府西門脫禍 見嬌娘敬濟銷魂}
\addcontentsline{toc}{chapter}{第十八囘 賂相府西門脫禍 見嬌娘敬濟銷魂}
\markboth{\titlename}{第十八囘 賂相府西門脫禍 見嬌娘敬濟銷魂}


詞曰:

有個人人,海棠標韻,飛燕輕盈。酒暈潮紅,羞蛾一笑生春。為伊無限傷心,更說甚巫山楚雲!斗帳香銷,紗窓月冷,着意溫存。

——右調《桺梢青》

話分兩頭。不說蔣竹山在李瓶兒家招贅,單表來保、來旺二人上東京打點,朝登紫陌,暮踐紅塵,一日到東京,進了萬壽門,投旅店安歇。到次日,街前打聽,只聽見街談巷議,都說兵部王尚書昨日會問明白,聖旨下來,秋後處決。止有楊提督名下親族人等,未曾拏完,尚未定奪。來保等二人把禮物打在身邊,急來到蔡府門首。舊時幹事來了兩遍,道路久熟,立在龍德街牌樓底下,探聽府中訊息。少頃,只見一個青衣人,慌慌打府中出來,徃東去了。來保認得是楊提督府裡親隨楊幹辦,待要叫住問他一聲事情如何,因家主不曾分付,以此不言語,放過他去了。遲了半日,兩個走到府門前,望着守門官深深唱個喏:「動問一聲,太師老爺在家不在?」那守門官道:「老爺朝中議事未囘。你問怎的?」來保又問道:「管家翟爺請出來,小人見見,有事稟白。」那官吏道:「管家翟叔也不在了。」來保見他不肯寔說,曉得是要些東西,就袖中取出一兩銀子遞與他。那官吏接了便問:「你要見老爺,要見學士大爺?老爺便是大管家翟謙稟,大爺的事便是小管家高安稟,各有所掌。況老爺朝中未囘,{\meipi{蔡太師明明迴避,只說朝中未散,口角隱隱約約,寫得逼真。}}止有學士大爺在家。你有甚事,我替你請出高管家來,稟見大爺也是一般。」這來保就借情道:「我是提督楊爺府中,有事稟見。」官吏聽了,不敢怠慢,進入府中。良久,只見高安出來。來保慌忙施礼,遞上十兩銀子,說道:「小人是楊爺的親,同楊幹辦一路來見老爺討信。因後邊吃飯,來遲了一步,不想他先來了。所以不曾趕上。」高安接了禮物,說道:「楊幹辦只剛纔去了,老爺還未散朝。你且待待,我引你再見見大爺罷。」一面把來保領到第二層大廳傍邊,另一座儀門進去。坐北朝南三間敞廳,綠油欄杆,朱紅牌額,石青鎮地,金字大書天子御筆欽賜「學士琴堂」四字。

原來蔡京兒子蔡攸,也是寵臣,見為祥和殿學士兼礼部尚書、提點太乙宮使。來保在門外伺候,高安先入說了,出來,然後喚來保入見,當廳跪下。蔡攸深衣軟巾,坐於堂上,問道:「你是那裡來的?」來保稟道:「小人是楊爺的親家陳洪的家人,同府中楊幹辦來稟見老爺討信。不想楊幹辦先來見了,小人趕來後見。」因向袖中取出揭帖遞上。蔡攸見上面寫着「白米五百石」,叫來保近前說道:「蔡老爺亦因言官論列,連日迴避。閣中之事,並昨日三法司會問,都是右相李爺秉筆。楊老爺的事,昨日內裡有訊息出來,聖上寬恩,另有處分了。其手下用事有名人犯,待查明問罪。你還到李爺那裡去說。」來保只顧磕頭道:「小的不認的李爺府中,望爺憐憫,看家楊老爺分上。」蔡攸道:「你去到天漢橋邊北高坡大門樓處,問聲當朝右相、資政殿大學士兼礼部尚書諱邦彥的。{\meipi{指路中叙出官御,妙。}}你李爺,誰是不知道!也罷,我這裡還差個人同你去。」即令祗候官呈過一緘,使了圖書,就差管家高安同去見李爺,如此替他說。那高安承應下了,同來保去了府門,叫了來旺,帶着禮物,轉過龍德街,逕到天漢橋李邦彥門首。

正值邦彥朝散纔來家,穿大紅縐紗袍,腰繫玉帶,送出一位公卿上轎而去,囘到廳上,門吏稟報說:「學士蔡大爺差管家來見。」先叫高安進去說了囘話,然後喚來保、來旺進見,跪在廳臺下。高安就在傍邊遞了蔡攸封緘,並禮物揭帖,來保下邊就把禮物呈上。邦彥看了說道:「你蔡大爺分上,又是你楊老爺親,我怎麼好受此禮物?況你楊爺,昨日聖心囘動,已沒事。但隻手下之人,科道叅語甚重,已定問發幾個。」即令堂候官取過昨日科中送的那幾個名字與他瞧。上面寫着:「王黼名下書辦官董陞,家人王廉,班頭黃玉,楊戩名下壞事書辦官盧虎,幹辦楊盛,府掾韓宗仁、趙弘道,班頭劉成,親黨陳洪、西門慶、胡四等,皆鷹犬之徒,狐假虎威之輩。乞勑下法司,將一干人犯,或投之荒裔以御魍魎,或置之典刑,以正國法。」來保見了,慌的只顧磕頭,告道:「小人就是西門慶家人,望老爺開天地之心,超生性命則個!」高安又替他跪稟一次。邦彥見五百兩金銀,只買一個名字,如何不做分上?即令左右擡書案過來,取筆將文捲上西門慶名字改作賈廉,{\meipi{改名巧甚。此等舞文之才,文官偏有。}}一面收上禮物去。邦彥打發來保等出來,就拏囘帖囘學士,賞了高安、來保、來旺一封五兩銀子。來保路上作辭高管家,囘到客店,收拾行李,還了房錢,星夜囘清河縣。

來家見西門慶,把東京所幹的事,從頭說了一遍。西門慶聽了,如提在冷水盆內,對月娘說:「早時使人去打點,不然怎了!」正是,這囘西門慶性命,有如:

落日已沉西嶺外,卻被扶桑喚出來。

於是一塊石頭方纔落地。過了兩日,門也不關了,花園照舊還蓋,漸漸出來街上走動。{\meipi{經此一番,便當收斂。西門慶事過即已,所謂小人而無忌憚也。}}一日,玳安騎馬打獅子街過,看見李瓶兒門首開個大生藥鋪,裡邊堆着許多生熟藥材。朱紅小櫃,油漆牌匾,弔着幌子,甚是熱鬧。歸來告與西門慶說——還不知招贅蔣竹山一節,只說:「二娘搭了個新夥計,開了個生藥鋪。」{\meipi{何不使人一候。}}西門慶聽了,半信不信。一日,七月中旬,金風淅淅,玉露泠泠。西門慶正騎馬街上走着,撞見應伯爵、謝希大。兩人叫住,下馬唱喏,問道:「哥,一向怎的不見?兄弟到府上幾遍,見大門關着,又不敢叫,整悶了這些時。端的哥在家做甚事?嫂子娶進來不曾?也不請兄弟們吃酒。」西門慶道:「不好告訴的。因舍親陳宅那邊為些閒事,替他亂了幾日。親事另改了日期了。」伯爵道:「兄弟們不知哥吃驚。今日既撞遇哥,兄弟二人肯空放了?如今請哥同到裡邊吳銀姐那裡吃三盃,權當解悶。」不繇分說,把西門慶拉進院中來。正是:

高榭樽開歌妓迎,漫誇解語一含情。纖手傳盃分竹葉,一簾秋水浸桃笙。

當日西門慶被二人拉到吳銀兒家,吃了一日酒。到日暮時分,已帶半酣,纔放出來。打馬正走到東街口上,撞見馮媽媽從南來,走得甚慌。西門慶勒住馬,問道:「你那裡去?」馮媽媽道:「二娘使我徃門外寺裡魚籃會,替過世二爺燒箱庫去來。」西門慶醉中道:「你二娘在家好麼?我明日和他說話去。」{\meipi{瓶兒向等沾戀,事完即當徃。何至此時撞着方問,西門慶太托大。太做身分,故有此失也。}}馮媽媽道:「還問甚麼好?把個見見成成做熟了飯的親事,吃人掇了鍋兒去了。」西門慶聽了失聲驚問道:「莫不他嫁人去了?」馮媽媽道:「二娘那等使老身送過頭面,徃你家去了幾遍不見你,大門關着。對大官兒說進去,教你早動身,你不理。今教別人成了,你還說甚的?」西門慶問:「是誰?」馮媽媽悉把半夜三更婦人被狐狸纏着,染病看看至死,怎的請了蔣竹山來看,吃了他的藥怎的好了,某日怎的倒踏門招進來,成其夫婦,見今二娘拏出三百兩銀子與他開了生藥鋪,從頭至尾說了一遍。這西門慶不聽便罷,聽了氣的在馬上只是跌脚,叫道:「苦哉!你嫁別人,我也不惱,如何嫁那矮王八!他有甚麼起解?」於是一直打馬來家。剛下馬進儀門,只見吳月娘、孟玉樓、潘金蓮並西門大姐四個,在前廳天井內月下跳馬索兒耍子。見西門慶來家,月娘、玉樓、大姐三個都徃後走了。只有金蓮不去,且扶着庭柱兜鞋,{\pangpi{偏作態。}}被西門慶帶酒罵道:「淫婦們閑的聲喚,{\pangpi{一「們」字原有心罵月娘。}}平白跳甚麼百索兒?」趕上金蓮踢了兩脚。走到後邊,也不徃月娘房中去脫衣裳,走在西廂一間書房內,要了鋪蓋,那裡宿歇。打丫頭,罵小厮,只是沒好氣。衆婦人同站在一處,都甚是着恐,不知是那緣故。吳月娘埋怨金蓮:「你見他進門有酒了,兩三步叉開一邊便了。還只顧在跟前笑成一塊,且提鞋兒,卻教他蝗蟲螞蚱一例都罵着。」玉樓道:「罵我們也罷,如何連大姐姐也罵起淫婦來了?沒槽道的行貨子!」金蓮接過來道:「這一家子只是我好欺負的!一般三個人在這裡,只踢我一個兒。那個偏受用着甚麼也怎的?」月娘就惱了,說道:「你頭裡何不叫他連我踢不是?你沒偏受用,誰偏受用?恁的賊不識高低貨!我到不言語,你只顧嘴頭子嗶哩薄喇的!」{\meipi{金蓮乖人,開口亦惹人惱;月娘賢婦,觸着也要恠人。可見家庭老婆舌頭,有所不免。}}金蓮見月娘惱了,便把話兒來摭,說道:「姐姐,不是這等說。他不知那裡因着甚麼頭繇兒,只拏我煞氣。要便睜着眼望着俺叫,千也要打個臭死,萬也要打個臭死!」月娘道:「誰教你只要嘲他來?他不打你,卻打狗不成!」玉樓道:「大姐姐,且叫小厮來問他聲,今日在誰家吃酒來?早晨好好出去,如何來家恁個腔兒!」不一時,把玳安叫到跟前,月娘罵道:「賊囚根子!你不寔說,教大小厮來拷打你和平安兒,每人都是十板。」玳安道:「娘休打,待小的寔說了罷。爹今日和應二叔們都在院裡吳家吃酒,散了來在東街口上,撞遇馮媽媽,說花二娘等爹不去,嫁了大街住的蔣太醫了。爹一路上惱的要不的。」月娘道:「信那沒廉恥的歪淫婦,浪着嫁了漢子,來家拏人煞氣。」玳安道:「二娘沒嫁蔣太醫,把他倒踏門招進去了。如今二娘與他本錢,開了好不興的生藥鋪。我來家告爹說,爹還不信。」孟玉樓道:「論起來,男子漢死了多少時兒?服也還未滿,就嫁人,使不得的!」月娘道:「如今年程,論的甚麼使的使不的。漢子孝服未滿,浪着嫁人的,纔一個兒?淫婦成日和漢子酒裡眠酒裡臥的人,他原守的甚麼貞節!」看官聽說:月娘這一句話,一棒打着兩個人:孟玉樓與潘金蓮都是孝服不曾滿再醮人的,聽了此言,未免各人懷着慚愧歸房,不在話下。正是:

不如意事常八九,可與人言無二三。

卻說西門慶當晚在前邊廂房睡了一夜。到次日早,把女婿陳敬濟安在他花園中,同賁四管工記帳,換下來招教他看守大門。西門大姐白日裡便在後邊和月娘衆人一處吃酒,晚夕歸到前邊廂房中歇。陳敬濟每日只在花園中管工,非呼喚不敢進入中堂,飲食都是內裡小厮拏出來吃。所以西門慶手下這幾房婦人都不曾見面。一日,西門慶不在家,與提刑所賀千戶送行去了。月娘因陳敬濟一向管工辛苦,不曾安排一頓飯兒酬勞他,向孟玉樓、李嬌兒說:「待要管,又說我多攬事;我待欲不管,又看不上。{\pangpi{口角妙甚。}}人家的孩兒在你家,每日早起睡晚,辛辛苦苦,替你家打勤勞兒,那個與心知慰他一知慰兒也怎的?」玉樓道:「姐姐,你是個當家的人,你不上心誰上心!」月娘於是分付廚下,安排了一桌酒餚點心,午間請陳敬濟進來吃一頓飯。這陳敬濟撇了工程教賁四看管,逕到後邊叅見月娘,作揖畢,旁邊坐下。小玉拏茶來吃了,安放桌兒,拏蔬菜按酒上來。月娘道:「姐夫每日管工辛苦,要請姐夫進來坐坐,白不得個閑。今日你爹不在家,無事,治了一盃水酒,權與姐夫酬勞。」敬濟道:「兒子蒙爹娘擡舉,有甚勞苦,這等費心!」月娘陪着他吃了一囘酒。月娘使小玉:「請大姑娘來這裡坐。」小玉道:「大姑娘使着手,就來。」少頃,只聽房中抹得牌響。敬濟便問:「誰人抹牌?」月娘道:「是大姐與玉簫丫頭弄牌。」敬濟道:「你看沒分曉,娘這裡呼喚不來,且在房中抹牌。」一不時,大姐掀簾子出來,與他女婿對面坐下,一同飲酒。月娘便問大姐:「陳姐夫也會看牌不會?」大姐道:「他也知道些香臭兒。」{\pangpi{妙語。}}月娘只知敬濟是志誠的女婿,卻不道這小夥子兒詩詞歌賦,{\pangpi{未必。}}雙陸象棋,拆牌道字,無所不通,無所不曉。正是:

自幼乖滑伶俐,風流博浪牢成。愛穿鴨綠出爐銀,雙陸象棋幫襯。琵琶笙箏簫管,彈丸走馬員情。只有一件不堪聞:見了佳人是命。

月娘便道:「既是姐夫會看牌,何不進去,咱同看一看?」{\meipi{月娘自引狼入室,卻又誰尤?}}敬濟道:「娘和大姐看罷,兒子卻不當。」{\pangpi{假志誠。}}月娘道:「姐夫至親間,怕怎的?」一面進入房中,只見孟玉樓正在床上鋪茜紅毡看牌,見敬濟進來,抽身就要走。月娘道:「姐夫又不是別人,見個礼兒罷。」{\pangpi{壞事徃徃在人。}}向敬濟道:「這是你三娘哩。」那敬濟慌忙躬身作揖,玉樓還了萬福。當下玉樓、大姐三人同抹,敬濟在傍邊觀看。抹了一囘,大姐輸了下來,敬濟上來又抹。玉樓出了個天地分;敬濟出了個恨點不到;吳月娘出了個四紅沉八不就,雙三不搭兩麼兒,和兒不出,左來右去配不着色頭。只見潘金蓮掀簾子進來,銀絲鬏髻上戴着一頭鮮花兒,{\pangpi{媚甚。}}笑嘻嘻道:「我說是誰,原來是陳姐夫在這裡。」{\pangpi{似老成,卻有心。}}慌的陳敬濟扭頸囘頭,猛然一見,不覺心蕩目搖,精魂已失。正是:五百年冤家相遇,三十年恩愛一旦遭逢。月娘道:「此是五娘,姐夫也只見個長礼兒罷。」敬濟忙向前深深作揖,金蓮一面還了萬福。月娘便道:「五姐你來看,小雛兒倒把老鴉子來贏了。」這金蓮近前一手扶着床護炕兒,一隻手拈着白紗團扇兒,在傍替月娘指點道:「大姐姐,這牌不是這等出了,把雙三搭過來,卻不是天不同和牌?還贏了陳姐夫和三姐姐。」衆人正抹牌在熱鬧處,只見玳安抱進毡包來,說:「爹來家了。」月娘連忙攛掇小玉送姐夫打角門出去了。{\meipi{既至親不妨,何又慌避如此?情竇皆月娘自開。}}西門慶下馬進門,先到前邊工上觀看了一遍,然後踅到潘金蓮房中來。金蓮慌忙接着,與他脫了衣裳,說道:「你今日送行去來的早。」西門慶道:「提刑所賀千戶新陞新平寨知寨,合衛所相知都郊外送他來,拏帖兒知會我,不好不去的。」金蓮道:「你沒酒,教丫鬟看酒來你吃。」不一時,放了桌兒飲酒,菜蔬都擺在面前。飲酒中間,因說起後日花園捲棚上梁,約有許多親朋都要來遞菓盒酒,掛紅,少不得叫廚子置酒管待。說了一囘,天色已晚。春梅掌燈歸房,二人上床宿歇。西門慶因起早送行,着了辛苦,吃了幾盃酒就醉了。倒下頭鼾睡如雷,齁齁不醒。那時正值七月二十頭天氣,夜間有些餘熱,這潘金蓮怎生睡得着?忽聽碧紗帳內一派蚊雷,不免赤着身子起來,執燭滿帳照蚊。照一個,燒一個。囘首見西門慶仰臥枕上,睡得正濃,搖之不醒。其腰間那話,帶着托子,累垂偉長,不覺淫心輒起,放下燭臺,用纖手捫弄。{\meipi{閨閫之私,何所不有?但不堪說破耳。}}弄了一囘,蹲下身去,用口吮之。吮來吮去,西門慶醒了,罵道:「恠小淫婦兒,你達達睡睡,就摑掍死了。」一面起來,坐在枕上,亦發叫他在下盡着吮咂;又垂首玩之,以暢其美。正是:恠底佳人風性重,夜深偷弄紫簫吹。又有蚊子雙關《踏莎行》詞為證:

我愛他身體輕盈,楚腰膩細。行行一派笙歌沸。黃昏人未掩朱扉,潛身撞入紗廚內。款傍香肌,輕憐玉體。嘴到處,胭脂記。耳邊廂造就百般聲,夜深不肯教人睡。

婦人頑了有一頓飯時,西門慶忽然想起一件事來,叫春梅篩酒過來,在床前執壺而立。將燭移在床背板上,教婦人馬爬在他面前,那話隔山取火,托入牡中,令其自動,在上飲酒取樂。婦人罵道:「好個刁鑽的強盜!從幾時新興出來的例兒,恠剌剌教丫頭看答着,甚麼張致!」西門慶道:「我對你說了罷,當初你瓶姨和我常如此幹,叫他家迎春在傍執壺斟酒,到好耍子。」婦人道:「我不好罵出來的,甚麼瓶姨鳥姨,題那淫婦做甚,奴好心不得好報。那淫婦等不的,浪着嫁漢子去了。你前日吃了酒來家,一般的三個人在院子裡跳百索兒,只拏我煞氣,只踢我一個兒,倒惹的人和我辨了囘子嘴。想起來,奴是好欺負的!」西門慶問道:「你與誰辨嘴來?」婦人道:「那日你便進來了,上房的好不和我合氣,說我在他跟前頂嘴來,罵我不識高低的貨。我想起來為甚麼?『養蛤蟆得水蠱兒病』,如今倒教人惱我!」西門慶道:「不是我也不惱,那日應二哥他們拉我到吳銀兒家,吃了酒出來,路上撞見馮媽媽子,這般告訴我,把我氣了個立睜。若嫁了別人,我到罷了。那蔣太醫賊矮忘八,那花大怎不咬下他下截來?{\pangpi{映入心病,又恨又悔。}}他有甚麼起解?招他進去,與他本錢,教他在我眼面前開鋪子,大剌剌的做買賣!」婦人道:「虧你臉嘴還說哩!{\pangpi{此時自然有得說。}}奴當初怎麼說來?先下米兒先吃飯。你不聽,只顧來問大姐姐。常言:信人調,丟了瓢。你做差了,你埋怨那個?」西門慶被婦人幾句話,冲得心頭一點火起,雲山半壁通紅,便道:「你繇他,教那不賢良的淫婦說去。到明日休想我理他!」看官聽說:自古讒言罔行,君臣、父子、夫婦、昆弟之間,皆不能免。饒吳月娘恁般賢淑,西門慶聽金蓮衽席睥睨之間言,卒致於反目,其他可不慎哉!

自是以後,西門慶與月娘尚氣,彼此覿面,都不說話。月娘隨他徃那房裡去,也不管他;來遲去早,也不問他;或是他進房中取東取西,只教丫頭上前答應,也不理他。兩個都把心冷淡了。正是:

前車倒了千千輛,後車到了亦如然。分明指與平川路,卻把忠言當惡言。

且說潘金蓮自西門慶與月娘尚氣之後,見漢子偏聽,以為得志。每日抖擻着精神,粧飾打扮,希寵市愛。因為那日後邊會着陳敬濟一遍,見小夥兒生的乖猾伶俐,有心也要勾搭他。但只畏懼西門慶,不敢下手。只等西門慶徃那裡去,便使了丫鬟叫進房中,與他茶水吃,常時兩個下棋做一處。一日西門慶新蓋捲棚上梁,親友掛紅慶賀,遞菓盒。許多匠作,都有犒勞賞賜。大廳上管待客官,吃到午晌,人纔散了。西門慶因起得早,就歸後邊睡去了。陳敬濟走來金蓮房中討茶吃。金蓮正在床上彈弄琵琶,道:「前邊上梁,吃了這半日酒,你就不曾吃些甚麼,還來我屋裡要茶吃?」敬濟道:「兒子不瞞你老人家說,從半夜起來,亂了這一五更,誰吃甚麼來!」婦人問道:「你爹在那裡?」{\pangpi{寫出私心。}}敬濟道:「爹後邊睡去了。」婦人道:「你既沒吃甚麼,」叫春梅:「揀籹裡拏我吃的那蒸酥菓餡餅兒來,與你姐夫吃。」這小夥兒就在他炕桌兒上擺着四碟小菜,吃着點心。因見婦人彈琵琶,戲問道:「五娘,你彈的甚曲兒?怎不唱個兒我聽。」婦人笑道:「好陳姐夫,奴又不是你影射的,{\pangpi{自開門路。}}如何唱曲兒你聽?我等你爹起來,看我對你爹說不說!」那敬濟笑嘻嘻,慌忙跪着央及道:「望乞五娘可憐見,兒子再不敢了!」{\meipi{又是一種勾挑,妙甚。}}那婦人笑起來了。自此這小夥兒和這婦人日近日親,或吃茶吃飯,穿房入屋,打牙犯嘴,挨肩擦背,通不忌憚。月娘托以兒輩,放這樣不老實的女婿在家,自家的事卻看不見。正是:

只曉採花成釀蜜,不知辛苦為誰甜。

