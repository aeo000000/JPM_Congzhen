\includepdf[pages={199,200},fitpaper=false]{tst.pdf}
\chapter*{第一百囘 韓愛姐路遇二搗鬼 普靜師幻度孝哥兒}
\addcontentsline{toc}{chapter}{第一百囘 韓愛姐路遇二搗鬼 普靜師幻度孝哥兒}
\markboth{{\titlename}卷之十}{第一百囘 韓愛姐路遇二搗鬼 普靜師幻度孝哥兒}


詩曰:

\begin{myquote}
舊日豪華事已空,銀屏金屋夢魂中。\\黃蘆晚日空殘壘,碧草寒烟鎖故宮。\\隧道魚燈油欲盡,粧臺鸞鏡匣長封。\\憑誰話盡興亡事,一衲閒雲兩袖風。
\end{myquote}

話說韓道國與王六兒,歸到謝家酒店內,無女兒,道不得箇坐吃山崩,使陳三兒去,又把那何官人勾來續上。那何官人見地方中沒了劉二,除了一害,依舊又來王六兒家行走,和韓道國商議:「你女兒愛姐,只是在府中守孝,不出來了,等我賣盡貨物,討了賒帳,你兩口跟我徃湖州家去罷,省得在此做這般道路。」韓道國說:「官人下顧,可知好哩。」{\meipi{道國此時翟氏之勢何在?西門之財何在?可嘆。}}一日賣盡了貨物,討上賒帳,顧了船,同王六兒跟徃湖州去了,不題。

卻表愛姐在府中,與葛翠屏兩箇持貞守節,姊妹稱呼,甚是合當。白日裡與春梅做伴兒在一處。那時金哥兒大了,年方六歲。孫二娘所生玉姐年長十歲,相伴兩箇孩兒,便沒甚事做。誰知自從陳敬濟死後,守備又出征去了。這春梅每日珍饈百味,綾錦衣衫,頭上黃的金,白的銀,圓的珠,光照的無般不有。只是晚夕難禁獨眠孤枕,慾火燒心。{\meipi{飽飯思淫,有家宜鑑。}}因見李安一條好漢,只因打殺張勝,巡風早晚十分小心。

一日,冬月天氣,李安正在監獄內上宿,忽聽有人敲後門,忙問道:「是誰?」只聞叫道:「你開門則箇。」李安連忙開了房門,卻見一箇人搶入來,閃身在燈光背後。李安看時,卻認得是養娘金匱。李安道:「養娘,你這咱晚來有甚事?」金匱道:「不是我私來,裡邊奶奶差出我來的。」李安道:「奶奶叫你來怎麼?」金匱笑道:「你好不理會得。看你睡了不曾,教我把一件物事來與你。」向背上取下一包衣服,「把與你,包內又有幾件婦女衣服與你娘。前日多累你押解老爺行李車輛,又救得奶奶一命,不然也吃張勝那厮殺了。」

說畢,留下衣服,出門走了兩步,又囘身道:「還有一件要緊的。」又取出一錠五十兩大元寶來,撇與李安自去了。

當夜躊躇不決。次早起來,徑拏衣服到家與他母親。做娘的問道:「這東西是那裡的?」李安把夜來事說了一遍。做母親的聽言叫苦:「當初張勝幹壞事,一百棍打死,他今日把東西與你,卻是甚麼意思?我今六十已上年紀,自從沒了你爹爹,滿眼只看着你,若是做出事來,老身靠誰?明早便不要去了。」{\meipi{此母當與王陵、徐庶之母異出同歸。明以保國,智以保身,是一流人物。}}李安道:「我不去,他使人來叫,如何答應?」婆婆說:「我只說你感冒風寒病了。」李安道:「終不成不去,惹老爺不見恠麼?」做娘的便說:「你且投到你叔叔,山東夜叉李貴那裡住上幾箇月,再來看事故何如。」這李安終是箇孝順的男子,就依着娘的話,收拾行李,徃青州府投他叔叔李貴去了。春梅以後見李安不來,三四五次使小伴當來叫。婆婆初時答應家中染病,次後見人來驗看,纔說徃原籍家中,討盤纏去了。這春梅終是惱恨在心,不題。

時光迅速,日月如梭,又早臘盡陽囘,正月初旬天氣。統制領兵一萬三千,在東昌府屯住已久,使家人周忠,稍書來家。教搬取春梅、孫二娘,並金哥、玉姐家小上車。止留下周忠:「東庄上請你二爺看守宅子。」原來統制還有箇族弟周宣,在庄上住。周忠在府中,與周宣、葛翠屏、韓愛姐看守宅子。周仁與衆軍牢保定車輛,徃東昌府來。此一去,不為身名離故土,爭知此去少囘程。有詞一篇,單道周統制果然是一員好將材。當此之時,中原蕩掃,志欲吞胡。但見:

\begin{myquote}
四方盜起如屯峰,狼烟烈焰薰天紅。\\將軍一怒天下安,腥羶掃盡夷從風。\\公事忘私願已久,此身許國不知有。{\meipi{此等人材無難。}}\\金戈抑日酬戰征,麒麟圖畫功為首。\\雁門關外秋風烈,鐵衣披張臥寒月。\\汗馬卒勤二十年,贏得斑斑鬂如雪。\\天子明見萬里餘,幾番勞積來旌書。\\肘懸金印大如斗,無負堂堂七尺軀。
\end{myquote}

有日,周仁押家眷車輛到於東昌。統制見了春梅、孫二娘、金哥、玉姐,衆丫鬟家小都到了,一路平安,心中大喜。就在統制府衙後廳居住。周仁悉把東庄上請了二爺來宅內,同小的老子周忠看守宅舍,說了一遍。周統制又問:「怎的李安不見?」春梅道:「又題甚李安?那厮我因他捉獲了張勝,好意賞了他兩件衣服,與他娘穿。他到晚夕巡風,進入後廳,把他二爺東庄上收的子粒銀一包五十兩,放在明間卓上,偷的去了。幾番使伴當叫他,只是推病不來。落後又使叫去,他躲的上青州原籍家去了。」統制便道:「這厮我倒看他,原來這等無恩!等我慢慢差人拏他去。」這春梅也不題起韓愛姐之事。

過了幾日,春梅見統制日逐理論軍情,幹朝廷國務,焦心勞思,日中尚未暇食,至於房幃色慾之事,久不沾身。因見老家人周忠次子周義,年十九歲,生的眉清目秀,眉來眼去,兩箇暗地私通,就勾搭了。{\meipi{以統制之忠赤而受春梅淫穢之報,謂有天理歟?然而此等事,世間正少?}}朝朝暮暮,兩箇在房中下棋飲酒,只瞞過統制一人不知。

一日,不想北國大金皇帝滅了遼國。又見東京欽宗皇帝登基,集大勢番兵,分兩路寇亂中原。大元帥粘沒喝,領十萬人馬,出山西太原府井陘道,來搶東京;副帥幹離不繇檀州來搶高陽關。邊兵抵擋不住,慌了兵部尚書李綱、大將种師道,星夜火牌羽書,分調山東、山西、河南、河北、關東、陝西分六路統制人馬,各依要地,防守截殺。那時陝西劉延慶領延綏之兵,關東王稟領汾絳之兵,河北王煥領魏博之兵,河南辛興宗領彰德之兵,山西楊惟忠領澤潞之兵,山東周秀領青兗之兵。

卻說周統制,見大勢番兵來搶邊界,兵部羽書火牌星火來,連忙整率人馬,全裝披掛,兼道進兵。比及哨馬到高陽關上,金國幹離不的人馬,已搶進關來,殺死人馬無數。正值五月初旬,黃沙四起,大風迷目。統制提兵進趕,不防被幹離不兜馬反攻,沒鞦一箭,正射中咽喉,隨馬而死。衆番將就用鉤索搭去,被這邊將士向前僅搶屍首,馬戴而還,所傷軍兵無數。可憐周統制一旦陣亡,亡年四十七歲。正是:于家為國忠良將,不辯賢愚血染沙。古人意不盡,作詩一首,以嘆之曰:

\begin{myquote}
勝敗兵家不可期,安危端自命為之。\\出師未捷身先䘮,落日江流不勝悲。
\end{myquote}

巡撫張叔夜,見統制沒於陣上,連忙鳴金收軍,查點折傷士卒,退守東昌。星夜奏朝廷,不在話下。部下士卒,載屍首還到東昌府。春梅合家大小,号哭動天,合棺木盛殮,交割了兵符印信。一日,春梅與家人周仁,發䘮載靈柩歸清河縣。不題。

話分兩頭。單表葛翠屏與韓愛姐,自從春梅去後,兩箇在家清茶淡飯,守節持貞,過其日月。正值春盡夏初天氣,景物鮮明,日長針指睏倦。姊妹二人閑中徐步,到西書院花亭上。見百花盛開,鶯啼燕語,觸景傷情。葛翠屏心還坦然,這韓愛姐,一心只想念陳敬濟,凡事無情無緒,睹物傷悲,不覺潸然淚下。{\meipi{聖人云:或安而行之,或勉強而行之,及其成功則一,翠屏、愛姐之謂也。然傳中於愛姐收拾獨詳,豈亦有取於其勉強而之於自然歟?所謂放下屠刀,立地證佛,信然,信然。}}姊妹二人正在悲悽之際,只見二爺周宣,走來勸道:「你姊妹兩箇少要煩惱,須索解嘆。我連日做得夢,有些不吉。夢見一張弓掛在旗竿上,旗竿折了,不知是兇是吉?」韓愛姐道:「倒只怕老爺邊上,有些說話。」正在猶疑之間,忽見家人周仁,掛着一身孝,慌慌張張走來,報道:「禍事,老爺如此這般,五月初七日,在邊關上陣亡了!大奶奶、二奶奶家眷,載着靈車都來了。」慌了二爺周宣,收拾打掃前廳乾淨,停放靈柩,擺下祭祀,合家大小,哀號起來。一面做齋累七,僧道念經。金哥、玉姐披麻帶孝,弔客徃來,擇日出殯,安葬於祖塋。俱不必細說。

卻說二爺周宣,引着六歲金哥兒,行文書申奏朝廷,討祭葬,襲替祖職。朝廷明降,兵部覆題引奏:已故統制周秀,奮身報國,沒於王事,忠勇可嘉。遣官諭祭一罈,墓頂追封都督之職。伊子照例優養,出幼襲替祖職。這春梅在內頤養之餘,淫情愈盛。常留周義在香閣中,鎮日不出。朝來暮徃,淫慾無度,生出骨蒸癆病症。逐日吃藥,減了飲食,消了精神,體瘦如柴,而貪淫不已。一日,過了他生辰,到六月伏暑天氣,早晨晏起,不料他摟着周義在床上,一泄之後,{\pangpi{極樂世界。}}鼻口皆出涼氣,淫津流下一窪口,就鳴呼哀哉,死在周義身上。亡年二十九歲。{\meipi{所謂牡丹花下死,做鬼也風流。死得快活,死得快活。}}這周義見沒了氣兒,就慌了手脚,向箱內抵盜了些金銀細軟,帶在身邊,逃走出外。丫鬟養娘不敢隱匿,報與二爺周宣得知。把老家人周忠鎖了,押着抓尋周義。可霎作恠,正走在城外他姑娘家投住,一條索子拴將來。已知其情,恐揚出醜去,金哥久後不可襲職,拏到前廳,不繇分說,打了四十大棍,即時打死。{\meipi{此是調停善法,亦是苦心。}}把金哥與孫二娘看着。一面發䘮於祖塋,與統制合葬畢。房中兩箇養娘並海棠、月桂,都打發各尋投向嫁人去了。止有葛翠屏與韓愛姐,再三勸他,不肯前去。

一日,不想大金人馬搶了東京汴梁,太上皇帝與靖康皇帝,都被虜上北地去了。中原無主,四下荒亂。兵戈匝地,人民逃竄。黎庶有塗炭之哭,百姓有倒懸之苦。大勢番兵已殺到山東地界,民間夫逃妻散,鬼哭神號,父子不相顧。葛翠屏已被他娘家領去,各逃生命。止丟下韓愛姐,無處依倚,不免收拾行裝,穿着隨身慘澹衣衫,出離了清河縣,前徃臨清找尋他父母。到臨清謝家店,店也關閉,主人也走了。不想撞見陳三兒,三兒說:「你父母去年就跟了何官人,徃江南湖州去了。」這韓愛姐一路上懷抱月琴,唱小詞曲,徃前抓尋父母。隨路飢餐渴飲,夜住曉行,忙忙如䘮家之犬,急急如漏網之魚。弓鞋又小,千辛萬苦。

行了數日,來到徐州地方,天色晚了,投在孤村裡面。一箇婆婆,年紀七旬之上,正在竈上杵米造飯。這韓愛姐便向前道了萬福,告道:「奴家是清河縣人氏,因為荒亂,前徃江南投親,不期天晚,權借婆婆這裡投宿一宵,明早就行,房金不少。」那婆婆看這女子,不是貧難人家婢女,生得舉止典雅,容貌非俗。因說道:「既是投宿,娘子請炕上坐,等老身造飯,有幾箇挑河夫子來吃。」那老婆婆炕上柴竈,登時做出一大鍋稗稻插豆子乾飯,又切了兩大盤生菜,撮上一包鹽,只見幾箇漢子,都蓬頭精腿,裩褲兜襠,脚上黃泥,進來放下鍬钁,便問道:「老娘有飯也未?」婆婆道:「你每自去盛吃。」當下各取飯菜,四散正吃。只見內一人,約四十四五年紀,紫面黃發,便問婆婆:「這炕上坐的是甚麼人?」婆婆道:「此位娘子,是清河縣人氏,前徃江南尋父母去,天晚在此投宿。」那人便問:「娘子,你姓甚麼?」愛姐道:「奴家姓韓,我父親名韓道國。」那人向前扯住問道:「姐姐,你不是我姪女韓愛姐麼?」那愛姐道:「你倒好似我叔叔韓二。」兩箇抱頭相哭做一處。因問:「你爹娘在那裡?你在東京,如何至此?」這韓愛姐一五一十,從頭說了一遍,「因我嫁在守備府裡,丈夫沒了,我守寡到如今。我爹娘跟了何官人,徃湖州去了。我要找尋去,荒亂中又沒人帶去,胡亂單身唱詞,覓些衣食前去,不想在這裡撞見叔叔。」那韓二道:「自從你爹娘上東京,我沒營生過日,把房兒賣了,在這裡挑河做夫子,每日覓碗飯吃。既然如此,我和你徃湖州,尋你爹娘去。」愛姐道:「若是叔叔同去,可知好哩。」當下也盛了一碗飯,與愛姐吃。愛姐呷了一口,見粗飯,不能嚥,只呷了半碗,就不吃了。

一宿晚景題過。到次日到明,衆夫子都去了,韓二交納了婆婆房錢,領愛姐作辭出門,望前途所進。那韓愛姐本來嬌嫩,弓鞋又小,身邊帶着些細軟釵梳,都在路上零碎盤纏。將到淮安上船,迤逶望江南湖州來,非止一日,抓尋到湖州何官人家,尋着父母,相見會了。不想何官人已死,家中又沒妻小,止是王六兒一人,丟下六歲女兒,有幾頃水稻田地。不上一年,韓道國也死了。王六兒原與韓二舊有揸兒,就配了小叔,種田過日。{\meipi{韓二至此,反得其所。}}那湖州有富家子弟,見韓愛姐生的聰明標致,都來求親。韓二再三教他嫁人,愛姐割髮毀目,出家為尼,誓不再配他人。{\meipi{難得難得。對此不自愧者,世有幾人?}}後來至三十一歲,無疾而終。正是:

\begin{myquote}
貞骨未歸三尺土,怨魂先徹九重天。
\end{myquote}

後韓二與王六兒成其夫婦,請受何官人家業田地,不在話下。

卻說大金人馬,搶過東昌府來,看看到清河縣地界。只見官吏逃亡,城門晝閉,人民逃竄,父子流亡。但見:

\begin{myquote}
烟生四野,日蔽黃沙。封豕長蛇,互相吞噬。龍爭虎鬪,各自爭強。皁幟紅旗,布滿郊野。男啼女哭,萬戶驚惶。番軍虜將,一似蟻聚蜂屯;短劍長槍,好似森森密竹。一處處死屍朽骨,橫三豎四;一攢攢折刀斷劍,七斷八截。箇箇攜男抱女,家家閉門關戶。十室九空,不顯鄉村城郭;獐奔鼠竄,那存禮樂衣冠。
\end{myquote}

正是:得多少

\begin{myquote}
宮人紅袖哭,王子白衣行。
\end{myquote}

那時,吳月娘見番兵到了,家家都關鎖門戶,亂竄逃去,不免也打點了些金珠寶玩,帶在身邊。那時吳大舅已死,止同吳二舅、玳安、小玉,領着十五歲孝哥兒,把家中前後都倒鎖了,要徃濟南府投奔雲理守。一來避兵,二者與孝哥完就親事。一路上只見人人荒亂,箇箇驚駭。可憐這吳月娘,穿着隨身衣服,和吳二舅男女五口,雜在人隊裡挨出城門,到於郊外,徃前奔行。到於空野十字路口,只見一箇和尚,身披紫褐袈裟,手執九環錫杖,脚[]芒鞋,肩上背着條布袋,袋內裹着經典,大移步迎將來,與月娘打了箇問訊,高聲大叫道:「吳氏娘子,你到那裡去?還與我徒弟來!」唬的月娘大驚失色,說道:「師父,你問我討甚麼徒弟?」那和尚又道:「娘子,你休推睡裡夢裡,你曾記的十年前,在岱嶽東峰,被殷天錫趕到我山洞中投宿。我就是那雪洞老和尚,法號普靜。你許下我徒弟,如何不與我?」吳二舅便道:「師父出家人,如何不近道?此等荒亂年程,亂竄逃生,他有此孩兒,久後還要接代香火,他肯捨與你出家去?」和尚道:「你真箇不與我去?」吳二舅道:「師父,你休閑說,誤了人的去路。後面只怕番兵來到,朝不保暮。」和尚道:「你既不與我徒弟,如今天色已晚,也走不出路去。番人就來,也不到此處,你且跟我到這寺中歇一夜,明早去罷。」吳月娘問:「師父,是那寺中?」那和尚用手只一指,道:「那路旁便是。」和尚引着來到永福寺。吳月娘認的是永福寺,曾走過一遭。比及來到寺中,長老僧衆都走去大半,止有幾箇禪和尚在後邊打座。佛前點着一大盞硫璃海燈,燒看一爐香。

已是日色啣山時分,當晚吳月娘與吳二舅、玳安、小玉、孝哥兒,男女五口兒,投宿在寺中方丈內。小和尚有認的,安排了些飯食,與月娘等吃了。那普靜老師,跏趺在禪堂床上敲木魚,口中念經。月娘與孝哥兒、小玉在床上睡,吳二舅和玳安做一處,着了荒亂辛苦底人,都睡着了。止有小玉不曾睡熟,起來在方丈內,打門縫內看那普靜老師父念經。看看念至三更時,只見金風悽悽,斜月朦朦,人烟寂靜,萬籟無聲。佛前海燈,半明不暗。這普靜老師見天下荒亂,人民遭劫,陣亡橫死者極多,發慈悲心,施廣惠力,禮白佛言,薦拔幽魂,解釋宿冤,絕去掛礙,各去超生。於是誦念了百十遍解冤經咒。少頃,陰風悽悽,冷氣颼颼。有數十輩焦頭爛額,蓬頭泥面者,或斷手摺臂者,或有刳腹剜心者,或有無頭跛足者,或有弔頸枷鎖者,都來悟領禪師經咒,列於兩旁。禪師便道:「你等衆生,冤冤相報,不肯解脫,何日是了?汝當諦聽吾言,隨方托化去罷。偈曰:

\begin{myquote}
勸爾莫結冤,冤深難解結。\\一日結成冤,千日解不徹。\\若將冤解冤,如湯去潑雪。\\若將冤報冤,如狼重見蝎。\\我見結冤人,盡被冤磨折。\\我今此懺悔,各把性悟徹。\\照見本來心,冤愆自然雪。\\仗此經力深,薦拔諸惡業。\\汝當各托生,再勿將冤結。{\meipi{楞嚴耶?法華耶?大悲耶?亦復如此觀。讀此書而以為淫者、穢者,無目者也。}}
\end{myquote}

當下衆魂都拜謝而去。小玉竊看,都不認得。{\meipi{試看全傳收此一段中,清清皎皎,如琉璃光明,映徹永珍,所謂芥子納須彌,亦作如是觀。}}少頃,又一大漢進來,身長七尺,形容魁偉,全裝貫甲,胸前關着一矢箭,自稱「統制周秀,因與番將對敵,折於陣上,今蒙師薦拔,今徃東京,托生於沈鏡為次子,名為沈守善去也。」言未已,又一人,素體榮身,口稱是清河縣富戶西門慶,「不幸溺血而死,今蒙師薦拔,今徃東京城內,托生富戶沈通為次子沈越去也。」小玉認的是他爹,唬的不敢言語。已而又有一人,提着頭,渾身皆血,自言是陳敬濟,「因被張勝所殺,蒙師經功薦拔,今徃東京城內,與王家為子去也。」已而又見一婦人,也提着頭,胸前皆血。自言:「奴是武大妻、西門慶之妾潘氏是也。不幸被仇人武松所殺。蒙師薦拔,今徃東京城內黎家為女托生去也。」已而又有一人,身軀矮小,面背青色,自言是武植,「因被王婆唆潘氏下藥吃毒而死,蒙師薦拔,今徃徐州鄉民范家為男,托生去也。」{\meipi{諸鬼俱來,而王婆老狗不至,想墮阿鼻地獄矣。}}已而又有一婦人,面色黃瘦,血水淋漓,自言:「妾身李氏,乃花子虛之妻,西門慶之妾,因害血山崩而死。蒙師薦拔,今徃東京城內,袁指揮家托生為女去也。」已而又一男,自言花子虛,「不幸被妻氣死,蒙師薦拔,今徃東京鄭千戶家托生為男。」已而又見一女人,頸纏脚帶,自言西門慶家人來旺妻宋氏,「自縊身死,蒙師薦拔,今徃東京朱家為女去也。」已而又一婦人,面黃肌瘦,自言周統制妻龐氏春梅,「因色癆而死,蒙師薦拔,今徃東京與孔家為女,托生去也。」已而又一男子,裸形披髮,渾身杖痕,自言是打死的張勝,「蒙師薦拔,今徃東京大興衛貧人高家為男去也。」已而又有一女人,項上纏着索子,自言是西門慶妾孫雪娥,不幸自縊身死,「蒙師薦拔,今徃東京城外貧民姚家為女去也。」已而又一女人,年小,項纏脚帶,自言「西門慶之女,陳敬濟之妻,西門大姐是也,不幸自縊身死,蒙師薦拔,今徃東京城外,與番役鍾貴為女,托生去也。」已而又見一小男子,自言周義,「亦被打死,蒙師薦拔,今徃東京城外高家為男,名高留住兒,托生去也。」言畢,各恍然不見。小玉唬的戰慄不已。

原來這和尚,只是和這些鬼說話。正欲向床前告訴吳月娘,不料月娘睡得正熟,一靈真性,同吳二舅衆男女,身帶着一百顆胡珠,一柄寶石縧環,前徃濟南府,投奔親家雲理守。一路到於濟南府,尋問到雲叅將寨門,通報進去。雲叅將聽見月娘送親來了,一見如故。叙畢禮數。原來新近沒了娘子,央浼隣舍王婆來陪待月娘,在後堂酒飯,甚是豐盛。吳二舅、玳安另在一處管待。因說起避兵就親之事,因把那百顆胡珠、寶石、縧環教與雲理守,權為茶禮。雲理守收了,並不言其就親之事。到晚,又教王婆陪月娘一處歇臥。將言說念月娘,以挑探其意,說:「雲理守雖武官,乃讀書君子,從割衫襟之時,就留心娘子。{\meipi{假詞可思。}}不期夫人沒了,鰥居至今。今據此山城,雖是任小,上馬管軍,下馬管民,生殺在於掌握。娘子若不棄,願成伉儷之歡,一雙兩好,令郎亦得諧秦晉之配。等待太平之日,再囘家去不遲。」月娘聽言,大驚失色,半晌無言。這王婆囘報雲理寺。次日晚夕,置酒後堂,請月娘吃酒。月娘只知他與孝哥兒完親,連忙來到席前叙坐。雲理守乃道:「嫂嫂不知,下官在此雖是山城,管着許多人馬,有的是財帛衣服,金銀寶物,缺少一箇主家娘子。下官一向思想娘子,如渴思漿,如熱思涼。不想今日娘子到我這裡與令郎完親,天賜姻緣,一雙兩好,成其夫婦,在此快活一世,有何不可?」月娘聽了,心中大怒,罵道:「雲理守,誰知你人皮包着狗骨!我過世丈夫不曾把你輕待,{\meipi{此雖夢語,非節氣人罵不出。}}如何一旦出此犬馬之言?」雲理守笑嘻嘻向前,把月娘摟住,求告說:「娘子,你自家中,如何走來我這裡做甚?自古上門買賣好做,不知怎的,一見你,魂靈都被你攝在身上。沒奈何,好歹完成了罷。」一面拏過酒來和月娘吃。月娘道:「你前邊叫我兄弟來,等我與他說句話。」雲理守笑道:「你兄弟和玳安兒小厮,已被我殺了。」即令左右:「取那件物事,與娘子看。」不一時,燈光下,血瀝瀝提了吳二舅、玳安兩顆頭來。唬的月娘面如土色,一面哭倒在地。被雲理守向前抱起:「娘子不須煩惱,你兄弟已死,你就與我為妻。我一箇總兵官,也不玷辱了你。」月娘自思道:「這賊漢將我兄弟家人害了命,我若不從,連我命也䘮了。」乃囘嗔作喜,說道:「你須依我,奴方與你做夫妻。」雲理守道:「不拘甚事,我都依。」月娘道:「你先與我孩兒完了房,我卻與你成婚。」雲理守道:「不打緊。」一面叫出雲小姐來,和孝哥兒推在一處,飲合卺盃,綰同心結,成其夫婦。然後扯月娘和他雲雨。這月娘卻拒阻不肯,被雲理守忿然大怒,罵道:「賤婦!你哄的我與你兒子成了婚姻,敢笑我殺不得你的孩兒?」向床頭提劍,隨手而落,血濺數步之遠。正是:三尺利刀着項上,滿腔鮮血濕模糊。月娘見砍死孝哥兒,不覺大叫一聲。不想撒手驚覺,卻是南柯一夢。唬的渾身是汗,遍體生津。連道:「恠哉,恠哉。」小玉在旁,便問:「奶奶怎的哭?」月娘道:「適間做得一夢,不祥。」不免告訴小玉一遍。小玉道:「我倒剛纔不曾睡着,悄悄打門縫見那和尚原來和鬼說了一夜話。剛纔過世俺爹、五娘、六娘和陳姐夫、周守備、孫雪娥、來旺兒媳婦子、大姐都來說話,各四散去了。」月娘道:「這寺後見埋着他每,夜靜時分,屈死淹魂如何不來!」

娘兒們說了囘話,不覺五更,雞叫天明。吳月娘梳洗面貌,走到禪堂中,禮佛燒香。只見普靜老師在禪床上高叫:「那吳氏娘子,你如今可省悟得了麼?」這月娘便跪下叅拜:「上告尊師,弟子吳氏,肉眼凡胎,不知師父是一尊古佛。適間一夢中都已省悟了。」{\meipi{月娘大有根器。}}老師道:「既已省悟,也不消前去,你就去,也無過只是如此。倒沒的䘮了五口兒性命。你這兒子,有分有緣遇着我,都是你平日一點善根所種。不然,定然難免骨肉分離。當初,你去世夫主西門慶造惡非善,此子轉身托化你家,本要蕩散其財本,傾覆其產業,臨死還當身首異處。{\meipi{可畏,可思。}}今我度脫了他去,做了徒弟,常言『一子出家,九祖昇天』,你那夫主冤愆解釋,亦得超生去了。你不信,跟我來,與你看一看。」於是叉步來到方丈內,只見孝哥兒還睡在床上。老師將手中禪杖,向他頭上只一點,教月娘衆人看。忽然翻過身來,卻是西門慶,項帶沉枷,腰繫鐵索。{\meipi{徃沈通家為次子者,又是誰?}}復用禪杖只一點,依舊是孝哥兒睡在床上。月娘見了,不覺放聲大哭,原來孝哥兒即是西門慶托生。{\meipi{西門慶反落好處。}}

良久,孝哥兒醒了。月娘問他:「如今你跟了師父出家。」在佛前與他剃頭,摩頂受記。可憐月娘扯住慟哭了一場,乾生受養了他一場。到十五歲,指望承家嗣業,不想被這老師幻化去了。{\meipi{讀至此,使人哭不得,笑不得。吾為月娘孤苦伶仃,則肝腸欲斷;為西門慶度脫苦海,則眉眼欲舒,閱者着眼。}}吳二舅、小玉、玳安亦悲不勝。當下這普靜老師,領定孝哥兒,起了他一箇法名,喚做明悟。{\pangpi{一部本旨。}}作辭月娘而去。臨行,分付月娘:「你們不消徃前途去了。如今不久番兵退去,南北分為兩朝,中原已有箇皇帝,多不上十日,兵戈退散,地方寧靜了,你每還囘家去安心度日。」月娘便道:「師父,你度托了孩兒去了,甚年何日我母子再得見面?」不覺扯住,放聲大哭起來。老師便道:「娘子休哭!那邊又有一位老師來了。」哄的衆人扭頸囘頭,當下化陣清風不見了。正是:

\begin{myquote}
三降塵寰人不識,倏然飛過岱東峰。
\end{myquote}

不說普靜老師幻化孝哥兒去了,且說吳月娘與吳二舅衆人,在永福寺住了十日光景,果然大金國立了張邦昌在東京稱帝,置文武百官。徽宗、欽宗兩君北去,康王泥馬渡江,在建康即位,是為高宗皇帝。拜宗澤為大將,復取山東、河北。分為兩朝,天下太平,人民復業。後月娘歸家,開了門戶,家產器物都不曾疎失。後就把玳安改名做西門安,承受家業,人稱呼為「西門小員外」。{\meipi{此子原不俗。}}養活月娘到老,壽年七十歲,善終而亡。此皆平日好善看經之報。有詩為證:

\begin{myquote}
閥閱遺書思惘然,誰知天道有循環。\\西門豪橫難存嗣,敬濟顛狂定被殲。\\樓月善良終有壽,瓶梅淫佚早歸泉。\\可恠金蓮遭惡報,遺臭千年作話傳。
\end{myquote}

\bookmarksetup{startatroot}
