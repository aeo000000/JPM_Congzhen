\includepdf[pages={115,116},fitpaper=false]{tst.pdf}
\chapter*{第五十八囘 潘金蓮打狗傷人 孟玉樓周貧磨鏡}
\addcontentsline{toc}{chapter}{第五十八囘 潘金蓮打狗傷人 孟玉樓周貧磨鏡}
\markboth{{\titlename}卷之六}{第五十八囘 潘金蓮打狗傷人 孟玉樓周貧磨鏡}


詞曰:

\begin{myquote} 
愁旋釋,還似織;淚暗拭,又偷滴。嗔怒着丫頭,強開懷,也只是恨懷千疊。拚則而今已拚了,忘只怎生便忘得!又還倚欄杆,試重聽訊息。

\raggedleft{——右調《帝臺春後》\rightquadmargin}
\end{myquote} 

話說當日西門慶陪親朋飲酒,吃的酩酊大醉,走入後邊孫雪娥房裡來。雪娥正顧竈上,看收拾家伙,聽見西門慶徃房裡去,慌的兩步做一步走。{\meipi{喜極時光景。}}先是郁大姐在他炕上坐的,一面攛掇他徃月娘房裡和玉簫、小玉一處睡去了。原來孫雪娥也住着一明兩暗三間房,一間床房,一間炕房。西門慶也有一年多沒進他房中來。聽見今日進來,連忙向前替西門慶接衣服,安頓中間椅子上坐的。一面揩抹涼蓆,收拾鋪床,薰香澡牝,走來遞茶與西門慶吃了,攙扶上床,脫靴解帶,打發安歇。一宿無話。

到次日廿八,乃西門慶正生日。剛燒畢紙,只見韓道國後生胡秀,到了門首,下頭口。左右稟知西門慶,就叫胡秀到廳上,磕頭見了,問他貨船在那裡。胡秀遞上書帳,說道:「韓大叔在杭州置了一萬兩銀子段絹貨物,見今直抵臨清鈔關,缺少稅鈔銀兩,未曾裝載進城。」西門慶看了書帳,心內大喜,分付棋童看飯與胡秀吃了,教他徃喬親家爹那裡見見去。就進來對吳月娘說:「韓夥計貨船到了臨清,使後生胡秀送書帳上來,如今少不的把對門房子打掃,卸到那裡,尋夥計收拾,開鋪子發賣。」月娘聽了,就說:「你上緊尋着,也不早了。」西門慶道:「如今等應二哥來,我就對他說。」

不一時,應伯爵來了。西門慶陪着他在廳上坐,就對他說:「韓夥計杭州貨船到了,缺少個夥計發賣。」伯爵就說:「哥,恭喜!今日華誕的日子,貨船到,決增十倍之利,喜上加喜。{\meipi{貨到與生日何關?然自是諛者投機語。}}哥若尋賣手,不打緊,我有一相識,卻是父交子徃的朋友,原是段子行賣手,連年運拙,閑在家中,今年纔四十多歲,眼力看銀水是不消說,寫算皆精,又會做買賣。此人姓甘,名潤,字出身,現在石橋兒巷住,倒是自己房兒。」西門慶道:「若好,你明日叫他見我。」

正說着,只見李銘、吳惠、鄭奉三個先來磕頭。不一時,雜耍樂工都到了。廂房中打發吃飯。只見答應的節級拏票來囘話說:「小的叫唱的,止有鄭愛月兒不到。他家鴇子說,收拾了纔待來,被王皇親家人攔徃宅裡唱去了。小的只叫了齊香兒、董嬌兒、洪四兒三個,收拾了便來也。」西門慶聽見他不來,便道:「胡說!怎的不來?」便叫過鄭奉問:「怎的你妹子我這裡叫他不來?果系是被王皇親家攔了去?」那鄭奉跪下便道:「小的另住,不知道。」西門慶道:「他說徃王皇親家唱就罷了?敢量我拏不得來!」便叫玳安兒近前分付:「你多帶兩個排軍,就拏我個侍生帖兒,到王皇親家宅內見你王二老爹,就說我這裡請幾位客吃酒,鄭愛月兒答應下兩三日了,好歹放了他來。倘若推辭,連那鴇子都與我鎖了,墩在門房兒裡。這等可惡!」一面叫鄭奉:「你也跟了去。」那鄭奉又不敢不去,走出外邊來,央及玳安兒說道:「安哥,你進去,我在外邊等着罷。已定是王二老爹府裡叫,怕不還沒去哩。有累安哥,若是沒動身,看怎的將就叫他好好的來罷。」玳安道:「若果然徃王家去了,等我拏帖兒討去;若是在家藏着,你進去對他媽說,教他快收拾一答兒來,俺就替他迴護兩句言語兒,爹就罷了。你每不知道他性格,他從夏老爹宅裡定下,你不來,他可知惱了哩。」這鄭奉一面先徃家中說去,玳安同兩個排軍、一名節級也隨後走來。

且說西門慶打發玳安去了,因向伯爵道:「這個小淫婦兒,這等可惡!在別人家唱,我這裡叫他不來。」伯爵道:「小行貨子,他曉的甚麼?他還不知你的手段哩!」西門慶道:「我倒見他酒席上說話兒伶俐,叫他來唱兩日試他,倒這等可惡!」伯爵道:「哥今日揀這四個粉頭,都是出類拔萃的尖兒了。」李銘道:「二爹,你還沒見愛月兒哩!」伯爵道:「我同你爹在他家吃酒,他還小哩,這幾年倒沒曾見,不知出落的怎樣的了。」李銘道:「這小粉頭子,雖故好個身段兒,光是一味粧飾,唱曲也會,怎生趕的上桂姐一半兒。爹這裡是那裡?叫着敢不來!就是來了,虧了你?還是不知輕重。」正說着,只見胡秀來囘話道:「小的到喬爹那邊見了來了,伺候老爹示下。」西門慶教陳敬濟:「後邊討五十兩銀子,令書童寫一封書,使了印色,差一名節級,明日早起身,一同下去,與你鈔關上錢老爹,教他過稅之時青目一二。」須臾,陳敬濟取了一封銀子來交與胡秀,胡秀領了文書並稅帖,次日早同起身,不在話下。

忽聽喝的道子响,平安來報:「劉公公與薛公公來了。」西門慶忙冠帶迎接至大廳,見畢禮數,請至捲棚內,寬去上蓋蟒衣,上面設兩張交椅坐下。應伯爵在下,與西門慶關席陪坐。薛內相便問:「此位是何人?」西門慶道:「去年老太監會過來,乃是學生故友應二哥。」薛內相道:「卻是那快耍笑的應先兒麼?」應伯爵欠身道:「老公公還記的,就是在下。」須臾,拏茶上來吃了。

只見平安走來稟道:「府裡周爺差人拏帖兒來說,今日還有一席,來遲些,叫老爹這裡先坐,不須等罷。」西門慶看了帖兒,便說:「我知道了。」薛內相因問:「西門大人,今日誰來遲?」西門慶道:「周南軒那邊還有一席,使人來說休要等他,只怕來遲些。」薛內相道:「既來說,咱虛着他席面就是。」正說話間,王經拏了兩個帖兒進來:「兩位秀才來了。」西門慶見帖兒上,一個是倪鵬,一個是溫必古,就知倪秀才舉薦了同窓朋友來了,連忙出來迎接。見都穿着衣巾進來,且不看倪秀才,只見那溫必古,年紀不上四旬,生的端莊質樸,落腮鬍,儀容謙仰,舉止溫恭。未知行藏如何,先觀動靜若是。有幾句單道他好:

\begin{myquote} 
雖抱不羈之才,慣遊非禮之地。功名蹭蹬,豪傑之志已灰;家業凋零,浩然之氣先䘮。把文章道學,一併送還了孔夫子;將致君澤民的事業及榮身顯親的心念,都撇在東洋大海。和光混俗,惟其利慾是前;隨方逐圓,不以廉恥為重。峨其冠,博其帶,而眼底旁若無人;闊其論,高其談,而胸中實無一物。三年叫案,而小考尚難,豈望月桂之高攀;廣坐啣盃,遁世無悶,且作巖穴之隱相。
\end{myquote} 

西門慶讓至廳上叙禮,每人遞書帕二事與西門慶祝壽。交拜畢,分賓主而坐。西門慶道:「久仰溫老先生大才,敢問尊號?」溫秀才道:「學生賤字日新,號葵軒。」西門慶道:「葵軒老先生。」又問:「貴庠?何經?」溫秀才道:「學生不才,府學備數。初學《易經》。{\pangpi{口角妙甚。}}一向久仰大名,未敢進拜。昨因我這敝同窓倪桂巖,道及老先生盛德,敢來登堂恭謁。」西門慶道:「承老先生先施,學生容日奉拜。只因學生一個武官,粗俗不知文理,徃來書柬無人代筆。前者因在敝同僚府上,會遇桂巖老先生,甚是稱道老先生大才盛德。正欲趨拜請教,不意老先生下降,兼承厚貺,感激不盡。」溫秀才道:「學生匪才薄德,謬承過譽。」茶罷,西門慶讓至捲棚內,有薛、劉二老太監在座。薛內相道:「請二位老先生寬衣進來。」西門慶一面請寬了青衣,請進裡面,各遜讓再四,方纔一邊一位,垂首坐下。

正敍談間,吳大舅、范千戶到了,叙禮坐定。不一時,玳安與同答應的和鄭奉都來囘話道:「四個唱的都叫來了。」西門慶問:「可是王皇親那裡?」玳安道:「是王皇親宅內叫,還沒起身,小的要拏他鴇子墩鎖,他慌了,纔上轎,都一答兒來了。」西門慶即出到廳臺基上站立。只見四個唱的一齊進來,向西門慶磕下頭去。那鄭愛月兒穿着紫紗衫兒,白紗挑線裙子。腰肢嬝娜,猶如楊柳輕盈;花貌娉婷,好似芙蓉艷麗。正是:

\begin{myquote} 
萬種風流無處買,千金良夜實難消。
\end{myquote} 

西門慶便向鄭愛月兒道:「我叫你,如何不來?這等可惡!敢量我拏不得你來!」那鄭愛月兒磕了頭起來,一聲兒也不言語,笑着同衆人一直徃後邊去了。{\meipi{媚極。若出一聲,便費分解,使俗筆為之,不知如何絮絮矣。}}到後邊,與月娘衆人都磕了頭。看見李桂姐、吳銀兒都在跟前,各道了萬福,說道:「你二位來的早。」李桂姐道:「我每兩日沒家去了。」因說:「你四個怎的這咱纔來?」董嬌兒道:「都是月姐帶累的俺們來遲了。收拾下,只顧等着他,白不起身。」鄭愛月兒用扇兒遮着臉,只是笑,不做聲。月娘便問:「這位大姐是誰家的?」董嬌兒道:「娘不知道,他是鄭愛香兒的妹子鄭愛月兒。纔成人,還不上半年光景。」月娘道:「可倒好個身段兒。」說畢,看茶吃了,一面放桌兒,擺茶與衆人吃。

潘金蓮且揭起他裙子,撮弄他的脚看,說道:「你每這裡邊的樣子,只是恁直尖了,不象俺外邊的樣子趫。俺外邊尖底停勻,你裡邊的後跟子大。」{\meipi{一到金蓮,遂多此一番絜長較短,然不如此,不足以為金蓮也。}}月娘向大妗子道:「偏他恁好勝,問他怎的!」一囘又取下他頭上金魚撇杖兒來瞧,因問:「你這樣兒是那裡打的?」鄭愛月兒道:「是俺裡邊銀匠打的。」須臾,擺下茶,月娘便叫:「桂姐、銀姐,你陪他四個吃茶。」不一時,六個唱的做一處同吃了茶。李桂姐、吳銀兒便向董嬌兒四個說:「你每來花園裡走走。」董嬌兒道:「等我每到後邊走走就來。」{\pangpi{伏案。}}李桂姐和吳銀兒就跟着潘金蓮、孟玉樓,出儀門徃花園中來。因有人在大捲棚內,就不曾過那邊去。只在這邊看了囘花草,就徃李瓶兒房裡看官哥兒。官兒心中又有些不自在,睡夢中驚哭,吃不下奶去。李瓶兒在屋裡守着不出來。看見李桂姐、吳銀兒和孟玉樓、潘金蓮進來,連忙讓坐。桂姐問道:「哥兒睡哩?」李瓶兒道:「他哭了這一日,纔睡下了。」玉樓道:「大娘說,請劉婆子來看他看,你怎的不使小厮請去?」李瓶兒道:「今日他爹好日子,明日請他去罷。」

正說話中間,只見四個唱的和西門大姐、小玉走來。大姐道:「原來你每都在這裡,卻教俺花園內尋你。」玉樓道:「花園內有人,咱們不好去的,瞧了瞧兒就來了。」李桂姐問洪四兒:「你每四個在後邊做甚麼,這半日纔來?」洪四兒道:「俺每在後邊四娘房裡吃茶來。」潘金蓮聽了,望着玉樓、李瓶兒笑,問洪四兒:「誰對你說是四娘來?」董嬌兒道:「他留俺每在房裡吃茶,他每問來:『還不曾與你老人家磕頭,不知娘是幾娘?』他便說:『我是你四娘哩。』」金蓮道:「沒廉恥的小婦奴才,別人稱你便好,誰家自己稱是四娘來。這一家大小,誰興你、誰數你、誰叫你是四娘?漢子在屋裡睡了一夜兒,得了些顏色兒,就開起染房來了。若不是大娘房裡有他大妗子,他二娘房裡有桂姐,你房裡有楊姑奶奶,李大姐有銀姐在這裡,我那屋裡有他潘姥姥,且輪不到徃你那屋裡去哩!」{\jiapi{敍述處好不扯淡,在金蓮又是絕正經事。}}{\meipi{一開口便非一二語可了,吾恨不得犁其舌。}}玉樓道:「你還沒曾見哩,今日早晨起來,打發他爹徃前邊去了,在院子裡呼張喚李的,便那等花哨起來。」金蓮道:「常言道:奴才不可逞,小孩兒不宜哄。」{\meipi{六宮生妒,士亦悲焉,況妒嫉如金蓮者乎!}}又問小玉:「我聽見你爹對你奶奶說,要替他尋丫頭。說你爹昨日在他屋裡,見他只顧收拾不了,因問他。那小淫婦就趁勢兒對你爹說:『我終日不得個閑收拾屋裡,只好晚夕來這屋裡睡罷了。』你爹說:『不打緊,到明日對你娘說,尋一個丫頭與你使便了。』真個有此話?」{\meipi{一入有心者之眼,便面目都是疑團,入世之難如是,可嘆,可嘆。}}小玉道:「我不曉的,敢是玉簫聽見來?」金蓮向桂姐道:「你爹不是俺各房裡有人,等閑不徃他後邊去。莫不俺每背地說他,本等他嘴頭子不達時務,慣傷犯人,俺每急切不和他說話。」{\meipi{本為一宵之忿,忽又纏入其生平,小人故入人罪,徃徃皆然。}}

正說着,綉春拏了茶上來。正吃間,忽聽前邊鼓樂响動,荊都監衆人都到齊了,遞酒上座,玳安兒來叫四個唱的,就徃前邊去了。那日,喬大戶沒來。先是雜耍百戲,吹打彈唱。隊舞纔罷,做了個笑樂院本。割切上來,獻頭一道湯飯。只見任醫官到了,冠帶着進來。西門慶迎接至廳上叙禮。任醫官令左右,毡包內取出一方壽帕、二星白金來,與西門慶拜壽。說道:「昨日韓明川說,纔知老先生華誕。恕學生來遲!」西門慶道:「豈敢動勞車駕,又兼謝盛儀。外日多謝妙藥。」彼此拜畢,任醫官還要把盞,西門慶辭道:「不消了。」一面脫了大衣,與衆人見過,就安在左首第四席,與吳大舅相近而坐。獻上湯飯並手下攢盒,任醫官謝了,令僕從領下去。四個唱的彈着樂器,在旁唱了一套壽詞。西門慶令上席分頭遞酒。下邊樂工呈上揭帖,劉、薛二內相揀了「韓湘子度陳半街」《陞仙會》雜劇。纔唱得一折,只見喝道之聲漸近,平安進來稟道:「守備府周爺來了。」西門慶慌忙迎接。未曾相見,就先請寬盛服。周守備道:「我來要與四泉把一盞。」薛內相說道:「周大人不消把盞,只見禮兒罷。」于是二人交拜畢,纔與衆人作揖,左首第三席安下鍾筯。下邊就是湯飯割切上來,又是馬上人兩盤點心、兩盤熟肉、兩瓶酒。

周守備謝了,令左右領下去,然後坐下。一面觥籌交錯,歌舞吹彈,花攢錦簇飲酒。正是:

\begin{myquote} 
舞低楊柳樓頭月,歌罷桃花扇底風。
\end{myquote} 

吃至日暮,先是任醫官隔門去的早。西門慶送出來,任醫官因問:「老夫人貴恙覺好了?」西門慶道:「拙室服了良劑,已覺好些。這兩日不知怎的,又有些不自在。明日還望老先生過來看看。」說畢,任醫官作辭上馬而去。落後又是倪秀才、溫秀才起身。西門慶再三款留不住,送出大門,說道:「容日奉拜請教。寒家就在對門收拾一所書院,與老先生居住。連寶眷都搬來,一處方便。學生每月奉上束脩,以備菽水之需。」溫秀才道:「多承厚愛,感激不盡。」倪秀才道:「此是老先生崇尚斯文之雅意矣。」打發二秀才去了。西門慶陪客飲酒,吃至更闌方散。

四個唱的都歸在月娘房內,唱與月娘、大妗子、楊姑娘衆人聽。西門慶還在前邊留下吳大舅、應伯爵,復坐飲酒。看着打發樂工酒飯吃了,先去了。其餘席上家伙都收了,又分付從新後邊拏果碟兒上來,教李銘、吳惠、鄭奉上來彈唱,拏大盃賞酒與他吃。應伯爵道:「哥今日華誕設席,列位都是喜歡。」李銘道:「今日薛爺和劉爺也費了許多賞賜,落後見桂姐、銀姐又出來,每人又遞了一包與他。只是薛爺比劉爺年小,快頑些。」不一時,畫童兒拏上菓碟兒來,應伯爵看見酥油蚫螺,就先揀了一個放在口內,如甘露灑心,入口而化。說道:「倒好吃。」西門慶道:「我的兒,你倒會吃!此是你六娘親手揀的。」伯爵笑道:「也是我女兒孝順之心。」說道:「老舅,你也請個兒。」于是揀了一個,放在吳大舅口內。又叫李銘、吳惠、鄭奉近前,每人揀了一個賞他。

正飲酒間,伯爵向玳安道:「你去後邊,叫那四個小淫婦出來。我便罷了,也叫他唱個兒與老舅聽,再遲一囘兒,便好去。今日連遞酒,他只唱了兩套,休要便宜了他。」那玳安不動身,說道:「小的叫了他了,在後邊唱與妗子和娘每聽哩,便來也。」伯爵道:「賊小油嘴,你幾時去來?還哄我。」因叫王經:「你去。」那王經又不動。伯爵道:「我使着你每都不去,等我自去罷。」

正說着,只聞一陣香風過,覺有笑聲,四個粉頭都用汗巾兒答着頭出來。伯爵看見道:「我的兒,誰養的你恁乖!搭上頭兒,心裡要去的情,好自在性兒。不唱個曲兒與俺每聽,就指望去?好容易!連轎子錢就是四錢銀子,買紅梭兒米買一石七八斗,勾你家鴇子和你一家大小吃一個月。」董嬌兒道:「哥兒,恁便宜衣飯兒,你也入了籍罷了。」洪四兒道:「這咱晚,七八有二更,放了俺每去罷了。」齊香兒道:「俺每明日還要起早,徃門外送殯去哩。」伯爵道:「誰家?」齊香兒道:「是房簷底下開門的那家子。」伯爵道:「莫不又是王三官兒家?前日被他連累你那場事,多虧你大爹這裡人情,替李桂兒說,連你也饒了。這一遭,雀兒不在那窠兒罷了。」{\meipi{語語靈穎,俗筆有一字否?補出時線索生動,的是針工匠斧。}}齊香兒笑罵道:「恠老油嘴,汗邪了你,恁胡說。」伯爵道:「你笑話我老?我半邊俏!把你這四個小淫婦兒還不勾擺佈哩。」洪四兒笑道:「哥兒,我看你行頭不怎麼好,光一味好撇。」伯爵道:「我那兒,到跟前看手段還錢。」又道:「鄭家那賊小淫婦兒,吃了糖五老座子兒,白不言語,有些出神的模樣,敢記掛着那孤老兒在家裡?」董嬌兒道:「他剛纔聽見你說,在這裡有些怯床。」伯爵道:「怯床不怯床,拏樂器來,每人唱一套,你每去罷,我也不留你了。」西門慶道:「也罷,你們兩個遞酒,兩個唱一套與他聽罷。」齊香兒道:「等我和月姐唱。」當下,鄭月兒琵琶,齊香兒彈箏,坐在交床上,歌美韻,放嬌聲,唱了一套《越調•鬪鵪鶉》「夜去明來」。董嬌兒遞吳大舅酒,洪四兒遞應伯爵酒,在席上交盃換盞,倚翠偎紅。正是:

\begin{myquote} 
舞囘明月墜秦樓,歌遏行雲迷楚館。
\end{myquote} 

當下酒進數巡,歌吟兩套,打發四個唱的去了。西門慶還留吳大舅坐,又叫春鴻上來唱了一套南曲,纔分付棋童備馬,拏燈籠送大舅。大舅道:「姐夫不消備馬,我同應二哥一路走罷。」西門慶道:「既如此,教棋童打燈籠送到家。」吳大舅與伯爵起身作別。西門慶送至大門首,因和伯爵說:「你明日好歹上心,約會了那甘夥計來見我,批合同。我會了喬親家,好收拾那邊房子卸貨。」伯爵道:「哥不消分付,我知道。」一面作辭,與吳大舅同行,棋童打着燈籠。吳大舅便問:「剛纔姐夫說收拾那裡房子?」伯爵道:「韓夥計貨船到,他新開個段子鋪,收拾對門房子,叫我替他尋個夥計。」大舅道:「幾時開張?咱每親朋少不的作賀作賀。」{\meipi{此段似閑,然得此便覺餘波縈迥,文勢不窘。}}須臾,出大街,到了伯爵小衚衕口上,吳大舅要棋童:「打燈籠送你應二爹到家。」伯爵不肯,說道:「棋童,你送大舅,我不消燈籠,進巷內就是了。」一面作辭,分路囘家。棋童便送大舅去了。

西門慶打發李銘等唱錢去了,囘後邊月娘房中歇了一夜。到次日,果然伯爵領了甘出身,穿青衣走來拜見,講說買賣之事。西門慶叫將崔本來會喬大戶,那邊收拾房子,開張舉事。喬大戶對崔本說:「將來凡一應大小事,隨你親家爹這邊只顧處,不消計較。」當下就和甘夥計批了合同。就立伯爵作保,得利十分為率:西門慶五分,喬大戶三分,其餘韓道國、甘出身與崔本三分均分。一面修蓋土庫,裝畫牌面,待貨車到日,堆卸開張。後邊又獨自收拾一所書院,請將溫秀才來作西賓,專修書柬,囘答徃來士夫。每月三兩束脩,四時禮物不缺,又撥了畫童兒小厮伏侍他。{\pangpi{伏。}}西門慶家中宴客,常請過來陪侍飲酒,俱不必細說。

不覺過了西門慶生辰。第二日早晨,就請了任醫官來看李瓶兒,又在對門看着收拾。楊姑娘先家去了,李桂姐、吳銀兒還沒家去。吳月娘買了三錢銀子螃蠏,午間煮了,請大妗子、李桂姐、吳銀兒衆人圍着吃了一囘。只見月娘請的劉婆子來看官哥兒,吃了茶,李瓶兒就陪他徃前邊房裡去了。劉婆子說:「哥兒驚了,要住了奶。」又留下幾服藥。月娘與了他三錢銀子,打發去了。孟玉樓、潘金蓮和李桂姐、吳銀兒、大姐都在花架底下,放小桌兒,鋪毡條,同抹骨牌賭酒頑耍。孫雪娥吃衆人贏了七八鍾酒,不敢久坐,就去了。衆人就拏李瓶兒頂缺。金蓮又教吳銀兒、桂姐唱了一套。當日衆姊妹飲酒至晚,月娘裝了盒子,相送李桂姐、吳銀兒家去了。

潘金蓮吃的大醉歸房,因見西門慶夜間在李瓶兒房裡歇了一夜,早晨又請任醫官來看他,惱在心裡。知道他孩子不好,進門不想天假其便,黑影中躧了一脚狗屎,到房中叫春梅點燈來看,一雙大紅段子鞋,滿幫子都展汙了。登時柳眉剔豎,星眼圓睜,叫春梅打着燈把角門關了,拏大棍把那狗沒高低只顧打,打的恠叫起來。李瓶兒使過迎春來說:「俺娘說,哥兒纔吃了老劉的藥,睡着了,教五娘這邊休打狗罷。」潘金蓮坐着,半日不言語。一面把那狗打了一囘,開了門放出去,又尋起秋菊的不是來。看着那鞋,左也惱,右也惱,因把秋菊喚至跟前說:「這咱晚,這狗也該打發去了,只顧還放在這屋裡做甚麼?是你這奴才的野漢子?你不發他出去,教他恁遍地撒屎,{\meipi{「教他」二字來得奇特。}}把我恁雙新鞋兒,連今日纔三四日兒,躧了恁一鞋幫子屎。知道我來,你也該點個燈兒出來,你如何恁推聾粧啞裝憨兒的?」春梅道:「我頭裡就對他說,你趁娘不來,早喂他些飯,關到後邊院子裡去罷。他佯打耳睜的不理我,還拏眼兒瞅着我。」婦人道:「可又來,賊膽大萬殺的奴才,我知道你在這屋裡成了把頭,把這打來不作準。」因叫他到跟前:「瞧,躧的我這鞋上的齷齪!」哄得他低頭瞧,提着鞋拽巴,兜臉就是幾鞋底子。打的秋菊嘴唇都破了,只顧搵着抹血,忙走開一邊。婦人罵道:「好賊奴才,你走了!」教春梅:「與我採過來跪着,取馬鞭子來,把他身上衣服與我扯去。好好教我打三十馬鞭子便罷,但扭一扭兒,我亂打了不算。」春梅於是扯了他衣裳,婦人教春梅把他手扯住,雨點般鞭子打下來,打的這丫頭殺豬也似叫。

那邊官哥纔合上眼兒,又驚醒了。{\meipi{可恨。}}又使了綉春來說:「俺娘上覆五娘,饒了秋菊罷,只怕唬醒了哥哥。」那潘姥姥正𢱉在裡間炕上,聽見打的秋菊叫,一骨碌子爬起來,在旁邊勸解。見金蓮不依,落後又見李瓶兒使過綉春來說,又走向前奪他女兒手中鞭子,說道:「姐姐少打他兩下兒罷,惹得他那邊姐姐說,只怕唬了哥哥。為驢扭棍不打緊,倒沒的傷了紫荊樹。」金蓮緊自心裡惱,又聽見他娘說了這一句,越發心中攛上把火一般。須臾,紫漒了面皮,把手只一推,險些兒不把潘姥姥推了一交。便道:「恠老貨,你與我過一邊坐着去!不干你事,來勸甚麼?甚麼紫荊樹、驢扭棍,單管外合裡應。」潘姥姥道:「賊作死的短壽命,{\pangpi{罵得痛快。}}我怎的外合裡應?我來你家討冷飯吃,教你恁頓摔我?」金蓮道:「你明日夾着那老𣭈走,怕他家拏長鍋煮吃了我!」{\meipi{一念情慾之起,忿怒之發,不難滅倫敗紀,不獨一金蓮也。}}潘姥姥聽見女兒這等擦他,走到裡邊屋裡嗚嗚咽咽哭去了,隨着婦人打秋菊。打勾二三十馬鞭子,然後又蓋了十欄杆,打的皮開肉綻,纔放出來。又把他臉和腮頰都用尖指甲掐的稀爛。{\meipi{可恨。}}李瓶兒在那邊,只是雙手握着孩子耳朵,腮邊墮淚,敢怒而不敢言。西門慶在對門房子裡,與伯爵、崔本、甘夥計吃了一日酒散了,逕徃玉樓房中歇息。

到次日,周守備家請吃補生日酒,不在家。李瓶兒見官哥兒吃了劉婆子藥不見動靜,夜間又着驚唬,一雙眼只是徃上吊吊的。因那日薛姑子、王姑子家去,走來對月娘說:「我向房中拏出他壓被的一對銀獅子來,要教薛姑子印造《佛頂心陀羅經》,趕八月十五日岳廟裡去捨。」那薛姑子就要拏着走,被孟玉樓在旁說道:「師父你且住,大娘,你還使小厮叫將賁四來,替他兌兌多少分兩,就同他徃經鋪裡講定個數兒來,每一部經多少銀子,到幾時有,纔好。你教薛師父去,他獨自一個,怎弄的來?」{\meipi{老到。}}月娘道:「你也說的是。」一面使來安兒叫了賁四來,向月娘衆人作了揖,把那一對銀獅子上天平兌了,重四十一兩五錢。月娘分付,同薛師父徃經鋪印造經數去了。潘金蓮隨即叫孟玉樓:「咱送送兩位師父去,就前邊看看大姐,他在屋裡做鞋哩。」兩個攜着手兒徃前邊來。賁四同薛姑子、王姑子去了。

金蓮與玉樓走出大廳東廂房門首,見大姐正在簷下納鞋,金蓮拏起來看,卻是沙綠潞紬鞋面。玉樓道:「大姐,你不要這紅鎖線子,爽利着藍頭線兒,好不老作些!你明日還要大紅提跟子?」大姐道:「我有一雙是大紅提跟子的。這個,我心裡要藍提跟子,所以使大紅線鎖口。」金蓮瞧了一囘,三個都在廳臺基上坐的。玉樓問大姐:「你女婿在屋裡不在?」大姐道:「他不知那裡吃了兩盅酒,在屋裡睡哩。」孟玉樓便向金蓮道:「剛纔若不是我在旁邊說着,李大姐恁哈帳行貨,就要把銀子交姑子拏了印經去。經也印不成,沒脚蠏行貨子藏在那大人家,你那裡尋他去?早是我說,叫將賁四來,同他去了。」金蓮道:「恁有錢的姐姐,不撰他些兒是傻子,只相牛身上拔一根毛兒。你孩兒若沒命,休說捨經,隨你把萬里江山捨了也成不的。如今這屋裡,只許人放火,不許俺每點燈。大姐聽着,也不是別人。偏染的白兒不上色,偏他會那等輕狂使勢,大清早晨,刁蹬着漢子請太醫看。他亂他的,俺每又不管。每常在人前會那等撇清兒說話:『我心裡不耐煩,他爹要便進我屋裡推看孩子,雌着和我睡,誰耐煩!教我就攛掇徃別人屋裡去了。俺每自恁好罷了,{\pangpi{好得有數。}}{\meipi{說得鑿鑿,即使瓶兒百吻,亦無可辨。}}背地還嚼說俺們。』那大姐姐偏聽他一面詞兒。不是俺每爭這個事,怎麼昨日漢子不進你屋裡去,你使丫頭在角門子首叫進屋裡?推看孩子,你便吃藥,一徑把漢子作成,和吳銀兒睡了一夜,{\meipi{說作成銀兒,隱然見不作成我為可怨,把自家長技冤人,固是小人度君子之腹。}}一徑顯你那乖覺,叫漢子喜歡你,那大姐姐就沒的話說了。昨日晚夕,人進屋裡躧了一脚狗屎,打丫頭趕狗,也嗔起來,使丫頭過來說,唬了他孩子了。俺娘那老貨,又不知道,走來勸甚麼的驢扭棍傷了紫荊樹。我惱他那等輕聲浪氣,叫我墩了他兩句,他今日使性子家去了。去了罷!教我說,他家有你這樣窮親戚也不多,沒你也不少。」玉樓笑道:「你這個沒訓教的子孫,你一個親娘母兒,你這等訌他!」金蓮道:「不是這等說,惱人的腸子,單管黃貓黑尾,外合裡應,只替人說話。吃人家碗半,被人家使喚。得不的人家一個甜頭兒,千也說好,萬也說好。想着迎頭兒養了這個孩子,把漢子調唆的生根也似的,把他便扶的正正兒的,把人恨不的躧到泥裡頭還躧。今日恁的天也有眼,你的孩兒也生出病來了。」

正說着,只見賁四徃經鋪裡交囘銀子,來囘月娘話,看見玉樓、金蓮和大姐都在廳臺基上坐的,只顧在儀門外立着,不敢進來。來安走來說道:「娘每閃閃兒,賁四來了。」金蓮道:「恠囚根子,你叫他進去,不是纔乍見他來?」來安兒說了,賁四低着頭,一直後邊見月娘、李瓶兒,說道:「銀子四十一兩五錢,眼同兩個師父交付與翟經兒家收了。講定印造綾殼《陀羅》五百部,每部五分;絹殼經一千部,每部三分。共該五十五兩銀子。除收過四十一兩五錢,還找與他十三兩五錢。準在十四日早擡經來。」李瓶兒連忙向房裡取出一個銀香球來,叫賁四上天平兌了,十五兩。李瓶兒道:「你拏了去,除找與他,別的你收着,換下些錢,到十五日廟上捨經,與你們做盤纏就是了,省的又來問我要。」賁四於是拏了香球出來,李瓶兒道:「四哥,多累你。」賁四躬着身說道:「小人不敢。」走到前邊,金蓮、玉樓又叫住問他:「銀子交付與經鋪了?」賁四道:「已交付明白。共一千五百部經,共該五十五兩銀子,除收過四十一兩五錢,剛纔六娘又與了這件銀香球。」玉樓、金蓮瞧了瞧,沒言語,賁四便囘家去了。玉樓向金蓮說道:「李大姐象這等都枉費了錢。他若是你的兒女,就是榔頭也摏不死;他若不是你兒女,莫說捨經造像,隨你怎的也留不住他。信着姑子,甚麼繭兒幹不出來!」

兩個說了一囘,都立起來。金蓮道:「咱每徃前邊大門首走走去。」因問大姐:「你去不去?」大姐道:「我不去。」潘金蓮便拉着玉樓手兒,兩個同來到大門裡首站立。因問平安兒:「對門房子都收拾了?」平安道:「這咱哩?昨日爹看着就都打掃乾淨了。後邊樓上堆貨,昨日教陰陽來破土,樓底下還要裝廂房三間,土庫擱段子,門面開啟,一溜三間,都教漆匠裝新油漆,在出月開張。」玉樓又問:「那寫書的溫秀才,家小搬過來了不曾?」平安道,「從昨日就過來了。今早爹分付,把後邊那一張涼床拆了與他,又搬了兩張桌子、四張椅子與他坐。」金蓮道:「你沒見他老婆怎的模樣兒?」平安道:「黑影子坐着轎子來,誰看見他來!」

正說着,只見遠遠一個老頭兒,斯琅琅搖着驚閨葉過來。潘金蓮便道:「磨鏡子的過來了。」教平安兒:「你叫住他,與俺每磨磨鏡子。我的鏡子這兩日都使的昏了,分付你這囚根子,看着過來再不叫!俺每出來站了多大囘,怎的就有磨鏡子的過來了?」那平安一面叫住磨鏡老兒,放下担兒,金蓮便問玉樓道:「你要磨,都教小厮帶出來,一答兒裡磨了罷。」於是使來安兒:「你去我屋裡,問你春梅姐討我的照臉大鏡子、兩面小鏡子兒,就把那大四方穿衣鏡也帶出來,教他好生磨磨。」玉樓分付來安:「你到我屋裡,教蘭香也把我的鏡子拏出來。」那來安兒去不多時,兩隻手提着大小八面鏡於,懷裡又抱着四方穿衣鏡出來。金蓮道:「臭小囚兒,你拏不了,做兩遭兒拏,如何恁拏出來?一時叮噹了我這鏡子怎了?」玉樓道:「我沒見你這面大鏡子,是那裡的?」金蓮道:「是人家當的,我愛他且是亮,安在屋裡,早晚照照。」因問:「我的鏡子只三面?」玉樓道:「我大小只兩面。」金蓮道:「這兩面是誰的?」來安道:「這兩面是春梅姐的,稍出來也叫磨磨。」金蓮道:「賊小肉兒,他放着他的鏡子不使,成日只撾着我的鏡子照,弄的恁昏昏的。」共大小八面鏡子,交付與磨鏡老叟,教他磨。當下絆在坐架上,使了水銀,那消頓飯之間,都淨磨的耀眼爭光。婦人拏在手內,對照花容,猶如一汪秋水相似。有詩為證:

\begin{myquote} 
蓮萼菱花共照臨,風吹影動碧沉沉。\\一池秋水芙蓉現,好似姮娥傍月陰。
\end{myquote} 

婦人看了,就付與來安兒收進去。玉樓便令平安,問鋪子裡傅夥計櫃上要五十文錢與磨鏡的。那老子一手接了錢,只顧立着不去。玉樓教平安問那老子:「你怎的不去?敢嫌錢少?」那老子不覺眼中撲簌簌流下淚來,哭了。平安道:「俺當家的奶奶問你怎的煩惱。」老子道:「不瞞哥哥說,老漢今年痴長六十一歲,在前丟下個兒子,二十二歲尚未娶妻,專一浪遊,不幹生理。老漢日逐出來掙錢養活他。他又不守本分,常與街上搗子耍錢。昨日惹了禍,同拴到守備府中,當土賊打囘二十大棍。歸來把媽媽的裙襖都去當了。媽媽便氣了一場病,打了寒,睡在炕上半個月。老漢說他兩句,他便走出來不徃家去,教老漢逐日抓尋他不着個下落。待要賭氣不尋他,老漢恁大年紀,止生他一個兒子,徃後無人送老;有他在家,見他不成人,又要惹氣。似這等,乃老漢的業障。有這等負屈啣冤,各處告訴,所以淚出痛腸。」玉樓叫平安兒:「你問他,你這後娶婆兒今年多大年紀了?」老子道:「他今年五十五歲了,男女花兒沒有,如今打了寒纔好些,只是沒將養的,心中想塊臘肉兒吃。老漢在街上恁問了兩三日,白討不出塊臘肉兒來。甚可嗟嘆人子。」玉樓道:「不打緊處,我屋裡抽屜內有塊臘肉兒哩。」即令來安兒:「你去對蘭香說,還有兩個餅錠,教他拏與你來。」金蓮叫:「那老頭子,問你家媽媽兒吃小米兒粥不吃?」老漢子道:「怎的不吃!那裡有?可知好哩。」金蓮也叫過來安兒來:「你對春梅說,把昨日你姥姥稍來的新小米兒量二升,就拏兩根醬瓜兒出來,與他媽媽兒吃。」那來安去不多時,拏出半腿臘肉、兩個餅錠、二升小米、兩個醬瓜兒,叫道:「老頭子過來,造化了你!你家媽媽子不是害病想吃,只怕害孩子坐月子,想定心湯吃。」那老子連忙雙手接了,安放在担內,望着玉樓、金蓮唱了個喏,揚長挑着担兒,搖着驚閨葉去了。平安道:「二位娘不該與他這許多東西,被這老油嘴設智誆的去了。他媽媽子是個媒人,昨日打這街上走過去不是,幾時在家不好來?」金蓮道:「賊囚,你早不說做甚麼來?」平安道:「罷了,也是他造化。可哥二位娘出來看見叫住他,照顧了他這些東西去了。」正是:

\begin{myquote} 
閑來無事倚門楣,恰見驚閨一老來。\\不獨纖微能濟物,無緣滴水也難為。
\end{myquote} 

