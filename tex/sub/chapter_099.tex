\includepdf[pages={197,198},fitpaper=false]{tst.pdf}
\chapter*{第九十九囘 劉二醉罵王六兒 張勝竊聽陳敬濟}
\addcontentsline{toc}{chapter}{第九十九囘 劉二醉罵王六兒 張勝竊聽陳敬濟}
\markboth{{\titlename}卷之十}{第九十九囘 劉二醉罵王六兒 張勝竊聽陳敬濟}


詞曰:

\begin{myquote}
白雲山,紅葉樹,閱盡興亡,一似朝還暮。多少夕陽芳草渡,潮落潮生,還送人來去。阮公途,楊子路,九折羊腸,曾把車輪誤。記得寒蕪嘶馬處,翠管銀箏,夜夜歌樓曙。

\raggedleft{——右調《蘇幕遮》\rightquadmargin}
\end{myquote}

話說陳敬濟,過了兩日,到第三日,卻是五月二十日他的生日,後廳整置酒餚,與他上壽,合家歡樂了一日。次日早晨,敬濟說:「我一向不曾徃河下去,今日沒事,去走一遭,一者和主管算帳,二來就避炎暑,走走便囘。」春梅分付:「你去坐一乘轎子,少要勞碌。」交兩箇軍牢擡着轎子,小姜兒跟隨,徑徃河下大酒樓店中來。

一路無詞,午後時分到了,下轎進入裡面。兩箇主管齊來叅見,說:「官人貴體好些?」敬濟道:「生受二位夥計掛心。」他一心只在韓愛姐身上,坐了一囘便起身,分付主管:「查下帳目,等我來算。」就轉身到後邊。八老又早迎見,報與王六兒夫婦。韓愛姐正在樓上,憑欄盼望,揮毫作詩遣懷。忽報陳敬濟來了,連忙輕移蓮步,款蹙湘裙,走下樓來。母子面上堆下笑來迎接,說道:「官人,貴人難見面,那陣風兒吹你到俺這裡?」敬濟與他母子作了揖,同進閣兒內坐定。少頃,王六兒點茶上來。吃畢茶,愛姐道:「請官人到樓上奴房內坐。」敬濟上的樓來,兩箇如魚得水,似膝投膠,無非說些深情密意的話兒。愛姐硯臺底下,露出一幅花箋,敬濟取來觀看。愛姐便說:「此是奴家盼你不來,作得一首詩,以消遣悶懷,恐汙官人貴目。」敬濟念了一遍,上寫着:

\begin{myquote}[\markfont]
倦倚綉床愁懶動,閒垂錦帳鬢鬟低。玉郎一去無訊息,一日相思十二時。
\end{myquote}

敬濟看了,極口稱羨不已。不一時,王六兒安排酒餚上樓,撥過鏡架,就擺在梳粧卓上。兩箇並坐,愛姐篩酒一盃,雙手遞與敬濟,深深道箇萬福,說:「官人一向不來,妾心無時不念。前八老來,又多謝盤纏,舉家感之不盡。」敬濟接酒在手,還了喏,說:「賤疾不安,有失期約,姐姐休恠。」酒盡,也篩一盃敬奉,愛姐吃過,兩箇坐定,把酒來斟。王六兒、韓道國上來,也陪吃了幾盃,各取方便下樓去了,{\meipi{極大法門。}}教他二人自在吃幾盃,叙些闊別話兒。

良久,吃得酒濃時,情興如火,免不得再把舊情一叙。交歡之際,無限恩情。穿衣起來,洗手更酌,又飲數盃。醉眼朦朧,餘興未盡。這小郎君,一向在家中不快,又心在愛姐,一向未與渾家行事。今日一旦見了情人,未肯一次即休。

正是生死冤家,五百年前撞在一處,敬濟魂靈都被他引亂。少頃,情竇復起,又幹一度。自覺身體睏倦,打熬不過,午飯也沒吃,倒在床上就睡着了。也是合當禍起,不想下邊販絲綿何官人來了,王六兒陪他在樓下吃酒。韓道國出去街上買菜蔬、餚品、菓子來配酒。兩箇在下邊行房。落後韓道國買將菓菜來,三人又吃了幾盃。約日西時分,只見灑家店坐地虎劉二,吃的酩酊大醉,軃開衣衫,露着一身紫肉,提着拳頭走來酒樓下,大叫:「採出何蠻子來!」唬的兩箇主管見敬濟在樓上睡,恐他聽見,慌忙走出櫃來,向前聲諾,說道:「劉二哥,何官人並不曾來。」這劉二那裡依聽。大拔步撞入後邊韓道國屋裡,一手把門簾扯去半邊,看見何官人正和王六兒並肩飲酒,心中大怒,便罵何官人:「賊狗男女,我㒲你娘!那裡沒尋你,卻在這裡。你在我店中,佔着兩箇粉頭,幾遭歇錢不與,又塌下我兩箇月房錢,卻來這裡養老婆!」{\meipi{掛炭部的看樣。}}那何官人忙出來道:「老二你休恠,我去罷。」那劉二罵道:「去你這狗㒲的!」不防颼的一拳來,正打在何官人面上,登時就青腫起來。那何官人也不顧,徑奪門跑了。劉二將王六兒酒卓,一脚登翻,家活都打了。王六兒便罵道:「是那裡少死的賊殺才!無事來老娘屋裡放屁。老娘不是耐驚耐怕兒的人!」{\meipi{罵搗鬼的英風猶在。}}被劉二向前一脚,跥了箇仰八叉,罵道:「我㒲你淫婦娘!你是那裡來的無名少姓私窠子?不來老爺手裡報過,許你在這酒店內趁熟?還與我搬去!若搬遲,須吃我一頓好拳頭。」那王六兒道:「你是那裡來的光棍搗子?老娘就沒了親戚兒?許你便來欺負老娘,要老娘這命做甚麼?」一頭撞倒哭起來。劉二罵道:「我把淫婦腸子也踢斷了,你還不知老爺是誰哩!」這裡喧亂,兩邊隣舍並街上過徃人,登時圍看約有許多。有知道的旁邊人說:「王六兒,你新來不知,他是守備老爺府中管事張虞候的小舅子,有名坐地虎劉二。在灑家店住,專一是打粉頭的班頭,降酒店的領袖。你讓他些兒罷,休要不知利害。這地方人,誰敢惹他!」王六兒道:「還有大似他的,采這殺才做甚麼?」陸秉義見劉二打得兇,和謝胖子做好做歹,把他勸的去了。陳敬濟正睡在床上,聽見樓下攘亂,便起來看時,天已日西時分,問:「那裡攘亂?」那韓道國不知走的徃那裡去了,只見王六兒披髮垢面上樓,如此這般告訴說:「那裡走來一箇殺才搗子,諢名喚坐地虎劉二,在灑家店住,說是咱府裡管事張虞候小舅子。因尋酒店,無事把我踢打,罵了恁一頓去了。又把家活酒器都打得粉碎。」一面放聲大哭起來。敬濟就叫上兩箇主管去問。兩箇主管隱瞞不住,只得說:「是府中張虞候小舅子劉二,來這裡尋何官人討房錢,見他在屋裡吃酒,不繇分說,把簾子扯下半邊來,打了何官人一拳,唬的何官人跑了。又和老韓娘子兩箇相罵,踢了一交,烘的滿街人看。」敬濟聽了,便曉得是前番做道士,被他打的劉二了。欲要聲張,又恐劉二潑皮行兇,一時鬬他不過。又見天色晚了,因問:「劉二那厮如今在那裡?」主管道:「被小人勸他囘去了。」敬濟安撫王六兒道:「你母子放心,有我哩,不妨事。你母子只情住着,我家去自有處置。」主管算了利錢銀兩遞與他,打發起身上轎,伴當跟隨。剛趕進城來,天已昏黑,心中甚惱。到家見了春梅,交了利息銀兩,歸入房中。

一宿無話。到次日,心心念念要告春梅說,輾轉尋思:「且住,等我慢慢尋張勝那厮幾件破綻,亦發教我姐姐對老爺說了,斷送了他性命。{\meipi{此念大惡,故受其害。}}叵耐這厮,幾次在我身上欺心,敢說我是他尋得來,知我根本出身,量視我禁不得他。」正是:

\begin{myquote}
冤仇還報當如此,機會遭逢莫遠圖。\\踏破鐵鞋無覓處,得來全不費工夫。
\end{myquote}

一日,敬濟來到河下酒店內,見了愛姐母子,說:「外日吃驚。」又問陸主管道:「劉二那厮可曾走動?」陸主管道:「自從那日去了,再不曾來。」又問韓愛姐:「那何官人也沒來行走?」愛姐道:「也沒曾來。」這敬濟吃了飯,算畢帳目,不免又到愛姐樓上。兩箇叙了囘衷腸之話,幹訖一度出來,因閑中叫過量酒陳三兒近前,如此這般,打聽府中張勝和劉二幾樁破綻。這陳三兒千不合,萬不合,說出張勝包占着府中出來的雪娥,在灑家店做表子。劉二又怎的各處巢窩,加三討利,舉放私債,逞着老爺名壞事。{\meipi{如此人極其該處。讀者須知,不可以敬濟之成敗論也。}}這敬濟聽記在心,又與了愛姐二三兩盤纏,和主管算了帳目,包了利息銀兩,作別騎頭口來家。

閑話休題。一向懷意在心,一者也是冤家相湊,二來合當禍起。不料東京朝中徽宗天子,見大金人馬犯邊,搶至腹內地方,聲息十分緊急。天子慌了,與大臣計議,差官徃北國講和,情願每年輸納歲幣,金銀彩帛數百萬。一面傳位與太子登基,改宣和七年為靖康元年,宣帝號為欽宗。皇帝在位,徽宗自稱太上道君皇帝,退居龍德宮。朝中陞了李綱為兵部尚書,分部諸路人馬。种師道為大將,總督內外軍務。

一日,降了一道勑書來濟南府,陞周守備為山東都統制,提調人馬一萬,前徃東昌府駐紮,會同巡撫都御史張叔夜,防守地方,阻擋金兵。守備領了勑書,不敢怠慢,一面叫過張勝、李安兩箇虞候近前分付,先押兩車箱馱行李細軟器物家去。原來在濟南做了一年官,也撰得巨萬金銀。都裝在行李馱箱內,委托二人押到家中:「交割明白,晝夜巡風仔細。我不日會同你巡撫張爺,調領四路兵馬,打清河縣起身。」二人當日領了鈞旨,打點車輛,起身先行。一路無詞。有日到了府中,交割明白,二人晝夜內外巡風,不在話下。

卻說陳敬濟見張勝押車輛來家,守備陞了山東統制,不久將到,正欲把心腹中事要告訴春梅,等守備來家,發露張勝之事。不想一日因渾家葛翠屏徃娘家囘門住去了,他獨自箇在西書房寢歇,春梅驀進房中看他。見無丫鬟跟隨,兩箇就解衣在房內雲雨做一處。不防張勝搖着鈴,巡風過來,{\meipi{冤家。}}到書院角門外,聽見書房內彷彿有婦人笑語之聲,就把鈴聲按住,慢慢走來窓下竊聽。

原來春梅在裡面與敬濟交媾。聽得敬濟告訴春梅說:「叵耐張勝那厮,好生欺壓於我,說我當初虧他尋得來,幾次在下人前敗壞我。昨日見我在河下開酒店,一徑使小舅子坐地虎劉二,來打我的酒店,把酒客都打散了。專一倚逞他在姐夫麾下,在那裡開巢窩,放私債,又把雪娥隱佔在外姦宿,只瞞了姐姐一人眼目。我幾次含忍,不敢告姐姐說,趁姐夫來家,若不早說知,徃後我定然不敢徃河下做買賣去了。」春梅聽了,說道:「這厮恁般無禮。雪娥那賤人,我賣了他,如何又留住在外?」敬濟道:「他非是欺壓我,就是欺壓姐姐一般。」春梅道:「等他爺來家,交他定結果了這厮。」{\meipi{罪不至此,太毒。}}常言道:「隔墻須有耳,窓外豈無人。」兩箇只管在內說,卻不知張勝窓外聽得明明白白,口中不言,心內暗道:「此時教他算計我,不如我先算計了他罷。」{\meipi{張勝此舉,似有鬼物憑之。}}一面撇下鈴,走到前邊監獄內,取了把解腕鋼刀,說時遲,那時快,在石上磨了兩磨,走入書院中來。不想天假其便,還是春梅不該死於他手。忽被後邊小丫鬟蘭花兒,慌慌走來叫春梅,報說:「小衙內金哥兒忽然風搖倒了,快請奶奶看去。」唬的春梅兩步做一步走,奔了後房中看孩兒去了。剛進去了,那張勝提着刀子,徑奔到書房內,不見春梅,只見敬濟睡在被窩內。見他進來,叫道:「阿呀,你來做甚麼?」張勝怒道:「我來殺你!{\pangpi{爽利。}}你如何對淫婦說,倒要害我?我尋得你來不是了?反恩將仇報!常言『黑頭蟲兒不可救,救之就要吃人肉』,休走,吃我一刀子!明年今日是你死忌!」{\meipi{伶伶俐俐,斬斬截截,張勝作事,大類武松。}}那敬濟光赤條身子,沒處躱,只摟着被,吃他拉過一邊,向他身就紮了一刀子來。紮着軟肋,鮮血就邈出來。這張勝見他掙扎,復又一刀去,攘着胸膛上,動彈不得了。一面採着頭髮,把頭割下來,{\meipi{試觀張勝前後始終之局,西門氏之豫讓也。}}正是:

\begin{myquote}
三寸氣在千般用,一日無常萬事休。
\end{myquote}

可憐敬濟青春不上三九,死於非命。張勝提刀,遶屋裡床背後,尋春梅不見,大拔步徑望後廳走。走到儀門首,只見李安背着牌鈴,在那裡巡風。一見張勝兇神也似提着刀跑進來,便問:「那裡去?」張勝不答,只顧走,被李安攔住。張勝就向李安戳一刀來。李安冷笑,說道:「我叔叔有名山東夜叉李貴,我的本事不用借。」早飛起右脚,只聽忒楞的一聲,把手中刀子踢落一邊。{\meipi{李安臨事從容。}}張勝急了,兩箇就揪採在一處,被李安一箇潑脚,跌番在地,解下腰間纏帶登時綁了。嚷的後廳春梅知道,說:「張勝持刀入內,小的拏住了。」

那春梅方救得金哥甦醒,聽言大驚失色。走到書院內,見敬濟已被殺死在房中,一地鮮血橫流,不覺放聲大哭。一面使人報知渾家。葛翠屏慌奔家來,看見敬濟殺死,哭倒在地,不省人事。被春梅扶救甦醒過來。拖過屍首,買棺材裝殯。把張勝墩鎖在監內,單等統制來家處治這件事。

那消數日,只見軍情事務緊急,兵牌來催促。周統制調完各路兵馬,張巡撫又早先徃東昌府那裡等候取齊。統制到家,春梅把殺死敬濟一節說了。李安將兇器放在面前,跪稟前事。統制大怒,坐在廳上,提出張勝,也不問長短,喝令軍牢,五棍一換,打一百棍,登時打死。隨馬上差旗牌快手,徃河下捉拏坐地虎劉二,鎖解前來。孫雪娥見拏了劉二,恐怕拏他,走到房中,自縊身死。旗牌拏劉二到府中,統制也分付打一百棍,當日打死。{\meipi{此囘一舉而除數害,可喜,可喜。}}烘動了清河縣,大鬧了臨清州。正是:平生作惡欺天,今日上蒼報應。有詩為證:

\begin{myquote}
為人切莫用欺心,舉頭三尺有神明。\\若還作惡無報應,天下兇徒人食人。
\end{myquote}

當時統制打死二人,除了地方之害。分付李安將馬頭大酒店還歸本主,把本錢收算來家。分付春梅在家,與敬濟修齋做七,打發城外永福寺葬埋。{\meipi{雖不得金蓮同穴,而相去咫尺,敬濟雖死,花星猶照。}}留李安、周義看家,把周忠、周仁帶去軍門答應。春梅晚夕與孫二娘,置酒送餞,不覺簇地兩行淚下,說:「相公此去,未知幾時囘還,出戰之間,須要仔細。番兵猖獗,不可輕敵。」統制道:「你每自在家清心寡慾,好生看守孩兒,不必憂念。我既受朝廷爵祿,盡忠報國。至於吉兇存亡,付之天也。」{\meipi{丈夫語,忠臣語。}}囑咐畢,過了一宿。次日,軍馬都在城外屯集,等候統制起程。一路無詞。有日到了東昌府下,統制差一面令字藍旗,打報進城。巡撫張叔夜,聽見周統制人馬來到,與東昌府知府達天道出衙迎接。至公廳叙禮坐下,商議軍情,打聽聲息緊慢。駐馬一夜,次日人馬早行,徃關上防守去了。不在話下。

卻表韓愛姐母子,在謝家樓店中聽見陳敬濟已死,愛姐晝夜只是哭泣,茶飯都不吃,一心只要徃城內統制府中,見敬濟屍首一見,死也甘心。{\meipi{難得。}}父母、旁人百般勸解不從。韓道國無法可處,使八老徃統制府中打聽,敬濟靈柩已出了殯,埋在城外永福寺內。這八老走來,囘了話。愛姐一心要到他墳上燒紙,哭一場,也是和他相交一場。做父母的只得依他。顧了一乘轎子,到永福寺中,問長老葬於何處。長老令沙彌引到寺後,{\pangpi{亦是寺後。}}新墳堆便是。這韓愛姐下了轎子,到墳前點着紙錢,道了萬福,叫聲:「親郎我的哥哥!奴實指望和你同諧到老,誰想今日死了!」放聲大哭,哭的昏暈倒了,頭撞於地下,就死過去了。{\meipi{的的貞心,千古無兩。}}慌了韓道國和王六兒,向前扶救,叫姐姐,叫不應,越發慌了。

不想那日,正是葬的三日,春梅與渾家葛翠屏坐着兩乘轎子,伴當跟隨,擡三牲祭物,來與他煖墓燒紙。看見一箇年小的婦人,穿着縞素,頭戴孝髻,哭倒在地。一箇男子漢和一中年婦人,摟抱他扶起來,又倒了,不省人事,吃了一驚。因問那男子漢是那裡的,這韓道國夫婦向前施禮,把從前已徃話,告訴了一遍:「這箇是我的女孩兒韓愛姐。」春梅一聞愛姐之名,就想起昔日曾在西門慶家中會過,又認得王六兒。韓道國悉把東京蔡府中出來一節,說了一遍:「女孩兒曾與陳官人有一面之交,不料死了。他只要來墳前見他一見,燒紙錢,不想到這裡,又哭倒了。」當下兩箇救了半日,這愛姐吐了口粘痰,方纔甦醒,尚哽咽哭不出聲來。痛哭了一場起來,與春梅、翠屏插燭也似磕了四箇頭,{\pangpi{可憐。}}說道:「奴與他雖是露水夫妻,他與奴說山盟,言海誓,情深意厚,實指望和他同諧到老,誰知天不從人願,一旦他先死了,撇得奴四脯着地。他在日曾與奴一方吳綾帕兒,上有四句情詩。知道宅中有姐姐,奴願做小,倘不信——」向袖中取出吳綾帕兒來,上面寫詩四句,春梅同葛翠屏看了。詩云:

\begin{myquote}
吳綾帕兒織廻紋,灑翰揮毫墨蹟新。\\寄與多情韓五姐,永諧鸞鳳百年情。{\meipi{此詩不及愛姐多多。}}
\end{myquote}

愛姐道:「奴也有箇小小鴛鴦錦囊,與他佩載在身邊。兩面都扣綉着並頭蓮,每朶蓮花瓣兒一箇字兒:寄與情郎陳君膝下。」春梅便問翠屏:「怎的不見這箇香囊?」翠屏道:「在底褲子上拴着,奴替他裝殮在棺槨內了。」當下祭畢,讓他母子到寺中擺茶飯,勸他吃了些。王六兒見天色將晚,催促他起身,他只顧不思動身。一面跪着春梅、葛翠屏哭說:「奴情願不歸父母,同姐姐守孝寡居。明日死,傍他魂靈,也是奴和他恩情一場,說是他妻小。」說着那淚如泉湧。{\meipi{益發難得。}}翠屏只顧不言語。春梅便說:「我的姐姐,只怕年小青春,守不住,卻不誤了你好時光。」愛姐便道:「奶奶說那裡話?奴既為他,雖刳目斷鼻也當守節,誓不再配他人。」囑付他父母:「你老公婆囘去罷,我跟奶奶和姐姐府中去也。」那王六兒眼中垂淚,哭道:「我承望你養活俺兩口兒到老,纔從虎穴龍潭中奪得你來。今日倒閃賺了我。」那愛姐口裡只說:「我不去了。你就留下我,到家也尋了無常。」{\meipi{敬濟生平狂悖薄劣,死未罄辜,而有愛姐、翠屏為之誓死靡慝。涼德而受美報,天下事盡多不可解者如此。}}那韓道國因見女兒堅意不去,和王六兒大哭一場,灑淚而別,囘上臨清店中去了。這韓愛姐同春梅、翠屏,坐轎子徃府裡來。那王六兒一路上悲悲切切,只是捨不的他女兒,哭了一場又一場。{\meipi{錢樹子去矣,安得不哭。}}那韓道國又怕天色晚了,顧上兩匹頭口,望前趕路。正是:

\begin{myquote}
馬遲心急路途窮,身似浮萍類轉蓬。\\只有都門樓上月,照人離恨各西東。
\end{myquote}

