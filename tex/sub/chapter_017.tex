\includepdf[pages={33,34},fitpaper=false]{tst.pdf}
\chapter*{第十七囘 宇給事劾倒楊提督 李瓶兒許嫁蔣竹山}
\addcontentsline{toc}{chapter}{第十七囘 宇給事劾倒楊提督 李瓶兒許嫁蔣竹山}
\markboth{{\titlename}卷之二}{第十七囘 宇給事劾倒楊提督 李瓶兒許嫁蔣竹山}


詩曰:

\begin{myquote}
早知君愛歇,本自無容妬;\\誰使恩情深,今來反相誤。\\愁眠羅帳曉,泣坐金閨暮;\\獨有夢中魂,猶言意如故。
\end{myquote}

話說五月二十日,帥府周守備生日。西門慶封五星分資、兩方手帕,打選衣帽齊整,騎匹大白馬,四個小厮跟隨,徃他家拜壽。席間也有夏提刑、張團練、荊千戶、賀千戶一班武官兒飲酒,鼓樂迎接,搬演戲文。玳安接了衣裳,囘馬來家。到日西時分,又騎馬去接,走到西街口上,撞見馮媽媽,問道:「馮媽媽那裡去?」馮媽媽道:「你二娘使我來請你爹。顧銀匠整理頭面完備,今日送來,請你爹那裡瞧去。你二娘還和你爹說話哩!」玳安道:「俺爹今日在守備府周老爺處吃酒,我如今接去。你老人家囘罷。等我到那裡,對爹說就是了。」馮媽媽道:「累你好歹說聲,你二娘等着哩!」這玳安打馬逕到守備府。衆官員正飲酒間,玳安走到西門慶席前,說道:「小的囘馬家來時,在街口撞遇馮媽媽,二娘使了來說,顧銀匠送了頭面來了,請爹瞧去,還要和爹說話哩。」西門慶聽了,就要起身,那周守備那裡肯放,攔門拏巨盃相勸。西門慶道:「蒙大人見賜,寧可飲一盃,還有些小事,不能盡情,恕罪,恕罪!」於是一飲而盡,辭周守備上馬,逕到李瓶兒家。婦人接着,茶湯畢,西門慶分付玳安囘馬家去,明日來接。玳安去了。李瓶兒叫迎春盒兒內取出頭面來,與西門慶過目。黃烘烘火焰般一付好頭面,收過去,單等二十四日行禮,出月初四日準娶。婦人滿心歡喜,連忙安排酒來,和西門慶暢飲開懷。吃了一囘,使丫鬟房中搽抹涼蓆乾淨。兩個在紗帳之中,香焚蘭麝,衾展鮫綃,脫去衣裳,並肩疊股,飲酒調笑。良久,春色橫眉,淫心蕩漾。西門慶先和婦人雲雨一囘,然後乘着酒興,坐於床上,令婦人橫躺於衽蓆之上,與他品簫。但見:

\begin{myquote}
不竹不絲不石,肉音別自唔咿。流蘇瑟瑟碧紗垂,辨不出宮商角徵。一點櫻桃欲綻,纖纖十指頻移。深吞添吐兩情癡,不覺靈犀味美。
\end{myquote}

西門慶醉中戲問婦人:「當初花子虛在時,也和他幹此事不幹?」婦人道:「他逐日睡生夢死,奴那裡耐煩和他幹這營生!他每日只在外邊胡撞,就來家,奴等閑也不和他沾身。況且老公公在時,和他另在一間房睡着,我還把他罵的狗血噴了頭。好不好,對老公公說了,{\meipi{瓶兒與老公公頗相好,開口不忘。}}要打倘棍兒。奴與他這般頑耍,可不砢硶殺奴罷了!誰似冤家這般可奴之意,就是醫奴的藥一般。白日黑夜,教奴只是想你。」兩個耍一囘,又幹了一囘。傍邊迎春伺候下一個小方盒,都是各樣細巧菓品,小金壺內滿泛瓊漿。從黃昏掌上燈燭,且幹且歇,直耍到一更時分。只聽外邊一片聲,打的大門響,使馮媽媽開門瞧去,原來是玳安來了。西門慶道:「我分付明日來接,這咱晚又來做甚麼?」因叫進來問他。那小厮慌慌張張走到房門首,因西門慶與婦人睡着,又不敢進來,只在簾外說道:「姐姐、姐夫都搬來了,許多箱籠在家中。大娘使我來請爹,快去計較話哩。」這西門慶聽了,只顧猶豫:「這咱晚,端的有甚緣故?須得到家瞧瞧。」連忙起來。婦人打發穿上衣服,做了一盞暖酒與他吃。

打馬一直到家,只見後堂中秉着燈燭,女兒女婿都來了,堆着許多箱籠床帳家伙,先吃了一驚,因問:「怎的這咱來家?」女婿陳敬濟磕了頭,哭說:「近日朝中,俺楊老爺被科道官叅論倒了。聖旨下來,拏送南牢問罪。門下親族用事人等,都問擬枷充軍。昨日府中楊幹辦連夜奔來,透報與父親知道。父親慌了,教兒子同大姐和些家伙箱籠,且暫在爹家中寄放,躲避些時。他便起身徃東京我姑娘那裡,打聽訊息去了。待事寧之日,恩有重報,不敢有忘。」西門慶問:「你爹有書沒有?」陳敬濟道:「有書在此。」向袖中取出,遞與西門慶。折開觀看,上面寫道:

\begin{myquote}[\markfont]
眷生陳洪頓首,書奉大德西門親家臺覽:餘情不叙。茲因北虜犯邊,搶過雄州地界,兵部王尚書不發救兵,失誤軍機,連累朝中楊老爺,{\pangpi{猶護局。}}俱被科道官叅劾太重。聖旨惱怒,拏下南牢監禁,會同三法司審問。其門下親族用事人等,俱照例發邊衛充軍。生一聞訊息,舉家驚惶,無處可投,先打發小兒、令愛,隨身箱籠家活,暫借親家府上寄寓。生即上京,投在姐夫張世廉處,打聽示下。待事務寧帖之日,囘家恩有重報,不敢有忘。誠恐縣中有甚聲色,{\pangpi{周密。}}生令小兒外具銀五百兩,相煩親家費心處料,容當叩報,沒齒不忘。燈下草書,不宣。

\raggedleft{仲夏二十日{\quad\ }洪再拜。\rightquadmargin}
\end{myquote}

西門慶看了,慌了手脚,教吳月娘安排酒飯,管待女兒、女婿。就令家下人等,打掃廳前東廂房三間,與他兩口兒居住。把箱籠細軟都收拾月娘上房來。{\pangpi{伏。}}陳敬濟取出他那五百兩銀子,交與西門慶打點使用。西門慶叫了吳主管來,與他五兩銀子,教他連夜徃縣中承行房裡,抄錄一張東京行下來的文書邸報來看。上面端的寫的是甚言語?

\begin{myquote}[\markfont]
兵科給事中宇文虛中等一本,懇乞宸斷,亟誅誤國權奸,以振本兵,以消虜患事:臣聞夷狄之禍,自古有之。周之獫狁,漢之匈奴,唐之突厥,迨及五代而契丹浸強,至我皇宋建國,大遼縱橫中原者已非一日。然未聞內無夷狄而外萌夷狄之患者。語云:霜降而堂鐘鳴,雨下而柱礎潤。以類感類,必然之理。譬若病夫,腹心之疾已久,元氣內消,風邪外入,四肢百骸,無非受病,雖盧扁莫之能救,焉能久乎?{\meipi{絕妙議論,當選入名臣奏疏中。}}今天下之勢,正猶病夫兀羸之極矣。君猶元首也,輔臣猶腹心也,百官猶四肢也。陛下端拱於九重之上,百官庶政各盡職於下。元氣內充,榮衛外扞,則虜患何繇而至哉?今招夷虜之患者,莫如崇政殿大學士蔡京者:本以倹邪奸險之資,濟以寡廉鮮恥之行,讒諂面諛,上不能輔君當道,贊元理化;下不能宣德布政,保愛元元。徒以利祿自資,希寵固位,樹黨懷奸,矇蔽欺君,中傷善類。忠士為之解體,四海為之寒心。聯翩朱紫,萃聚一門。邇者河湟失議,主議伐遼,內割三郡,郭藥師之叛,卒使金虜背盟,憑陵中原。此皆誤國之大者,皆繇京之不職也。王黼貪庸無賴,行比俳優。蒙京汲引,荐居政府,未幾謬掌本兵。惟事慕位苟安,終無一籌可展。乃者張達殘於太原,為之張惶失散。今虜犯內地,則又挈妻子南下,為自全之計。其誤國之罪,可勝誅戮?楊戩本以紈絝膏粱,叨承祖蔭,憑籍寵靈,典司兵柄,濫膺閫外,大奸似忠,怯懦無比。此三臣者,皆朋黨固結,內外蒙蔽,為陛下腹心之蠱者也。數年以來,招災致異,䘮本傷元,役重賦煩,生民離散,盜賊猖獗,夷虜犯順,天下之膏腴已盡,國家之綱紀廢弛,雖擢髮不足以數京等之罪也。臣等待罪該科,備員諫職,徒以目擊奸臣誤國,而不為皇上陳之,則上辜君父之恩,下負平生所學。伏乞宸斷,將京等一干黨惡人犯,或下廷尉,以示薄罰;或致極典,以彰顯戮;或照例枷號,或投之荒裔,以御魑魅。庶天意可囘,人心暢快,國法以正,虜患自消。天下幸甚!臣民幸甚!

奉聖旨:「蔡京姑留輔政。王黼、楊戩着拏送三法司,會問明白來說。欽此欽遵。」續該三法司會問過,並黨惡人犯王黼、楊戩,本兵不職,縱虜深入,荼毒生民,損兵折將,失陷內地,律應處斬。手下壞事家人、書辦、官掾、親黨:董陞、盧虎、楊盛、龐宣、韓宗仁、陳洪、黃玉、劉盛、趙弘道等,查出有名人犯,俱問擬枷號一個月,滿日發邊衛充軍。
\end{myquote}

西門慶不看萬事皆休;看了耳邊廂只聽「颼」的一聲,魂魄不知徃那裡去了。就是:

\begin{myquote}
驚傷六葉連肝肺,嚇壞三毛七孔心。
\end{myquote}

當下即忙打點金銀寶玩,馱裝停當,把家人來保、來旺叫到臥房中,悄悄分付,如此這般:「顧頭口星夜上東京打聽訊息。不消到你陳親家老爹下處。但有不好聲色,取巧打點停當,速來囘報。」又與了他二人二十兩銀子。絕早五更顧脚伕起程,上東京去了,不在話下。

西門慶通一夜不曾睡着,到次日早,分付來昭、賁四,把花園工程止住,各項匠人都且囘去,不做了。每日將大門緊閉,家下人無事亦不許徃外去。西門慶只在房裡走來走去,憂上加憂,悶上加悶,如熱地蜒蚰一般,把娶李瓶兒的勾當,丟在九霄雲外去了。吳月娘見他愁眉不展,面帶憂容,只得寬慰他,說道:「他陳親家那邊為事,各人冤有頭債有主,你也不需焦愁如此。」西門慶道:「你婦人都知道些甚麼?陳親家是我的親家,女兒、女婿兩個孽障搬來咱家住着,平昔街坊隣舍惱咱的極多,常言:『機兒不快梭兒快,打着羊駒驢戰。』倘有小人指搠,拔樹尋根,你我身家不保。」{\meipi{強梁人結怨,何當不自知。}}正是:關門家裡坐,禍從天上來。這裡西門慶在家納悶,不題。

且說李瓶兒等了一日兩日,不見動靜,一連使馮媽媽來了兩遍,大門關得鐵桶相似。等了半日,沒一個人牙兒出來,竟不知怎的。看看到二十四日,李瓶兒又使馮媽媽送頭面來,就請西門慶過去說話。叫門不開,立在對過房簷下等。少頃,只見玳安出來飲馬,看見便問:「馮媽媽,你來做甚麼?」馮媽媽說:「你二娘使我送頭面來,怎的不見動靜?請你爹過去說話哩。」玳安道:「俺爹連日有些事兒,不得閑。你老人家還拏頭面去,等我飲馬囘來,對俺爹說就是了。」馮媽媽道:「好哥哥,我這在裡等着,你拏進頭面去和你爹說去。你二娘那裡好不惱我哩!」這玳安一面把馬拴下,走到裡邊,半日出來道:「對爹說了,頭面爹收下了,教你上覆二娘,再待幾日兒,我爹出來徃二娘那裡說話。」這馮媽媽一直走來,囘了婦人話。婦人又等了幾日,看看五月將盡,六月初旬,朝思暮盼,音信全無,夢攘魂勞,佳期間阻。正是:

\begin{myquote}
懶把蛾眉掃,羞將粉臉勻。\\滿懷幽恨積,憔悴玉精神。
\end{myquote}

婦人盼不見西門慶來,每日茶飯頓減,精神恍惚。到晚夕,孤眠枕上輾轉躊躕。忽聽外邊打門,彷彿見西門慶來到。婦人迎門笑接,攜手進房,問其爽約之情,各訴衷腸之話。綢繆繾綣,徹夜歡娛。雞鳴天曉,便抽身囘去。婦人恍然驚覺,大呼一聲,精魂已失。馮媽媽聽見,慌忙進房來看。婦人說道:「西門他爹剛纔出去,你關上門不曾?」馮媽媽道:「娘子想得心迷了,那裡得大官人來?影兒也沒有!」婦人自此夢境隨邪,夜夜有狐狸假名抵姓,攝其精髓。漸漸形容黃瘦,飲食不進,臥床不起。馮媽媽向婦人說,請了大街口蔣竹山來看。其人年不上三十,生的五短身材,人物飄逸,極是輕浮狂詐。請入臥室,婦人則霧髩雲鬟,{\meipi{「則」字下得妙,已有更端之意。}}擁衾而臥,{\pangpi{病態嫣甚。}}似不勝憂愁之狀。茶湯已罷,丫鬟安放褥墊。竹山就床診視脈息畢,因見婦人生有姿色,{\pangpi{醫者常情。}}便開口說道:「學生適診病源,娘子肝脈弦出寸口而洪大,厥陰脈出寸口久上魚際,主六慾七情所致。陰陽交爭,乍寒乍熱,似有鬱結於中而不遂之意也。似瘧非瘧,似寒非寒,白日則倦怠嗜臥,精神短少;夜晚神不守舍,夢與鬼交。若不早治,久而變為骨蒸之疾,必有屬纊之憂矣。可惜,可惜!」婦人道:「有累先生,俯賜良劑。奴好了,重加酬謝。」竹山道:「學生無不用心,娘子若服了我的藥,必然貴體全安。」說畢起身。這裡送藥金五星,使馮媽媽討將藥來。婦人晚間吃了藥下去,夜裡得睡,便不驚恐。漸漸飲食加添,起來梳頭走動。那消數日,精神復舊。{\meipi{一醫便好,情淺可知。}}

一日,安排了一席酒餚,備下三兩銀子,使馮媽媽請過竹山來相謝。蔣竹山自從與婦人看病,懷覬覦之心已非一日。一聞其請,即具服而徃。延之中堂,婦人盛粧出見,道了萬福,茶湯兩換,請入房中。酒餚已陳,麝蘭香藹。小丫鬟綉春在傍,描金盤內托出三兩白金。婦人高擎玉盞,向前施禮,說道:「前日,奴家心中不好,蒙賜良劑,服之見效。今粗治了一盃水酒,請過先生來知謝知謝。」竹山道:「此是學生分內之事,理當措置,何必計較!」因見三兩謝禮,說道:「這個學生怎麼敢領?」婦人道:「些須微意,不成禮數,萬望先生笑納。」辭讓了半日,竹山方纔收了。婦人遞酒,安下坐次。飲過三巡,竹山偷眼睃視婦人,粉粧玉琢,嬌艷驚人,先用言以挑之,因道:「學生不敢動問,娘子青春幾何?」婦人道:「奴虛度二十四歲。」竹山道:「似娘子這等妙年,生長深閨,處於富足,何事不遂,而前日有此鬱結不足之病?」{\pangpi{勾挑亦微。}}婦人聽了,微笑道:「不瞞先生,奴因拙夫棄世,家事蕭條,獨自一身,{\meipi{病根在此,故徃徃謂西門慶為醫奴之藥。}}憂愁思慮,何得無病!」竹山道:「原來娘子夫主歿了,多少時了?」婦人道:「拙夫從去歲十一月得傷寒病死了,今已八個月。」竹山道:「曾吃誰的藥來?」{\pangpi{人死問病,妙。}}婦人道:「大街上胡先生。」竹山道:「是那東街上劉太監房子住的胡鬼嘴兒?他又不是我太醫院出身,知道甚麼脈,娘子怎的請他?」婦人道:「也是因街坊上人薦舉請他來看。還是拙夫沒命,不干他事。」竹山又道:「娘子也還有子女沒有?」婦人道:「兒女俱無。」竹山道:「可惜娘子這般青春妙齡之際,獨自孀居,又無所出,何不尋其別進之路?甘為幽悶,豈不生病!」{\meipi{忙忙中又着一段諧語,令人失笑,一味弄筆。}}婦人道:「奴近日也講着親事,早晚過門。」竹山便道:「動問娘子與何人作親?」婦人道:「是縣前開生藥鋪西門大官人。」竹山聽了道:「苦哉,苦哉!娘子因何嫁他?學生常在他家看病,最知詳細。此人專在縣中包攬說事,廣放私債,販賣人口,家中丫頭不算,大小五六個老婆,着緊打倘棍兒,稍不中意,就令媒人領出賣了。就是打老婆的班頭,坑婦女的領袖。娘子早是對我說,不然進入他家,如飛蛾投火一般,坑你上不上,下不下,那時悔之晚矣。況近日他親家那邊為事幹連,在家躲避不出,房子蓋的半落不合的,都丟下了。東京關下文書,坐落府縣拏人。到明日他蓋這房子,多是入官抄沒的數兒。娘子沒來由嫁他做甚?」一篇話把婦人說的閉口無言。況且許多東西丟在他家,尋思半晌,暗中跌脚,{\meipi{瓶兒與西門慶徃還不淺,何至聞言而尋思?二語寫出瓶兒之愚。}}嗔恠道:「一替兩替請着他不來,他家中為事哩!」又見竹山語言活動,一團謙恭:「奴明日若嫁得恁樣個人也罷了,{\pangpi{寫出瓶兒之淺。}}不知他有妻室沒有?」因說道:「既蒙先生指教,奴家感戴不淺,倘有甚相知人家,舉保來說,奴無有個不依之理。」竹山乘機請問:「不知要何等樣人家?學生打聽的實,好來這裡說。」婦人道:「人家到也不論大小,只要象先生這般人物的。」這蔣竹山不聽便罷,聽了此言,歡喜的滿心癢,不知搔處,慌忙走下席來,雙膝跪下告道:「不瞞娘子說,學生內幃失助,中餽乏人,鰥居已久,子息全無。倘蒙娘子垂憐,肯結秦晉之緣,足稱平生之願。學生雖啣環結草,不敢有忘。」{\meipi{卑辭屈禮,隱隱為竹山畫一花面,作者玩弄極矣。}}婦人笑笑,以手攜之,說道:「且請起,未審先生鰥居幾時?貴庚多少?既要做親,須得要個保山來說,方成禮數。」竹山又跪下哀告道:「學生行年二十九歲,正月二十七日卯時建生,不幸去年荊妻已故,家緣貧乏,實出寒微。今既蒙金諾之言,何用氷人之講。」婦人笑道:「你既無錢,我這裡有個媽媽姓馮,拉他做個媒證。也不消你行聘,擇個吉日良時,招你進來,入門為贅。你意下若何?」這蔣竹山連忙倒身下拜:「娘子就如同學生重生父母,再長爹娘。夙世有緣,三生大幸矣!」一面兩個在房中各遞了一盃交歡酒,已成其親事。竹山飲至天晚囘家。婦人這裡與馮媽媽商議說:「西門慶如此這般為事,吉兇難保。{\pangpi{薄情語。}}況且奴家這邊沒人,不好了一場,險不䘮了性命。為今之計,不如把這位先生招他進來,有何不可?」到次日,就使馮媽媽遞信過去,擇六月十八日大好日子,把蔣竹山倒踏門招進來,成其夫妻。過了三日,婦人湊了三百兩銀子,與竹山開啟兩間門面,店內煥然一新。初時徃人家看病只是走,後來買了一匹驢兒騎着,在街上徃來,不在話下。正是:

\begin{myquote}
一窪死水全無浪,也有春風擺動時。
\end{myquote}

