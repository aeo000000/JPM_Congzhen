\includepdf[pages={175,176},fitpaper=false]{tst.pdf}
\chapter*{第八十八囘 陳敬濟感舊祭金蓮 龐大姐埋屍托張勝}
\addcontentsline{toc}{chapter}{第八十八囘 陳敬濟感舊祭金蓮 龐大姐埋屍托張勝}
\markboth{{\titlename}卷之九}{第八十八囘 陳敬濟感舊祭金蓮 龐大姐埋屍托張勝}


詩曰:

\begin{myquote}
夢中雖暫見,及覺始知非。\\輾轉不成寐,徒倚獨披衣。\\悽悽曉風急,醃醃月光微。\\空床常達旦,所思終不歸。
\end{myquote}

話說武松殺了婦人、王婆,劫去財物,逃上梁山去了,不題。且說王潮兒街上叫了保甲來,見武松家前後門都不開,又王婆家被劫去財物,房中衣服丟的橫三豎四,就知是武松殺人劫財而去。未免開啟前後門,見血瀝瀝兩個死屍倒在地下,婦人心肝五臟用刀插在後樓房簷下。迎兒倒扣在房中。問其故,只是哭泣。{\pangpi{一毫不假。}}次日早衙,呈報到本縣,殺人兇刃都拏放在面前。本縣新任知縣也姓李,雙名昌期,乃河北真定府棗強縣人氏。聽見殺人公事,即委差當該吏典,拘集兩隣保甲,並兩家苦主王潮、迎兒。眼同當街,如法檢驗。生前委被武松因忿帶酒,殺潘氏、王婆二命,疊成文案,就委地方保甲瘞埋看守。掛出榜文,四廂差人跟尋,訪拏正犯武松,有人首告者,官給賞銀五十兩。守備府中張勝、李安打着一百兩銀子到王婆家,看見王婆、婦人俱已被武松殺死,縣中差人檢屍,捉拏兇犯。二人囘報到府中。春梅聽見婦人死了,整哭了兩三日,茶飯都不吃。慌了守備,使人門前叫調百戲的貨郎兒進去,耍與他觀看,只是不喜歡。日逐使張勝、李安打聽,拏住武松正犯,告報府中知道,不在話下。

按下一頭。且表陳敬濟前徃東京取銀子,一心要贖金蓮,成其夫婦。不想走到半路,撞見家人陳定從東京來,告說家爺病重之事:「奶奶使我來請大叔徃家去,囑托後事。」這敬濟一聞其言,兩程做一程,路上趲行。有日到東京他姑夫張世廉家。張世廉已死,止有姑娘見在。他父親陳洪已是沒了三日,滿家帶孝。敬濟叅見他父親靈座。與他母親張氏並姑娘磕頭。張氏見他成人,母子哭做一處,通同商議:「如今一則以喜,一則以憂。」{\meipi{父死而有子長成,喜可知也。然而殊不然,可為疼心。}}敬濟便道:「如何是喜,如何是憂?」張氏道:「喜者,如今朝廷冊立東宮,郊天大赦;憂則不想你爹爹病死在這裡,你姑夫又沒了,姑娘守寡,這裡住着不是常法,如今只得和你打發你爹爹靈柩囘去,葬埋鄉井,也是好處。」敬濟聽了,心內暗道:「這一囘傳送,裝載靈柩家小粗重上車,少說也得許多日期耽閣,卻不誤了六姐?{\meipi{知好色則慕少女,此其一驗。}}不如先誆了兩車細軟箱籠家去,待娶了六姐,再來搬取靈柩不遲。」一面對張氏說道:「如今隨路盜賊,十分難走。假如靈柩家小箱籠一同起身,未免起眼,倘遇小人怎了?寧可耽遲不耽錯。我先押兩車細軟箱籠家去,收拾房屋。母親隨後和陳定、家眷並父親靈柩,過年正月同起身囘家,寄在城外寺院,然後做齋念經、築墳安葬,也是不遲。」張氏終是婦人家,不合一時聽信敬濟巧言,就先打點細軟箱籠,裝載兩大車,上插旗號,扮做香車。從臘月初一日東京起身,不上數日,到了山東清河縣家門首,對他母舅張團練說:「父親已死,母親押靈車,不久就到。我押了兩車行李,先來收拾打掃房屋。」他母舅聽說:「既然如此,我仍搬囘家去便了。」一面就令家人搬家活,騰出房子來。敬濟見母舅搬去,滿心歡喜,說:「且得冤家離眼前,落得我娶六姐來家,自在受用。我父親已死,我娘又疼我。先休了那個淫婦,然後一紙狀子,把俺丈母告到官,追要我寄放東西,誰敢道個不字?{\meipi{即此一想,折盡平生之福矣。後之流落不得其死,何恠!}}又挾制俺家充軍人數不成!」正是:

\begin{myquote}
人便如此如此,天理不然不然。
\end{myquote}

這敬濟就打了一百兩銀子在腰裡,另外又袖着十兩謝王婆,來到紫石街王婆門首。可霎作恠,只見門前街旁埋着兩個屍首,上面兩杆槍交叉挑着個燈籠,門前掛着一張手榜,上書:

\begin{myquote}[\markfont]
本縣為人命事:兇犯武松,殺死潘氏、王婆二命,有人捕獲首告官司者,官給賞銀五十兩。
\end{myquote}

這敬濟仰頭看見,便立睜了。只見窩鋪中站出兩個人來,喝聲道:「甚麼人?看此榜文做甚?見今正身兇犯捉拏不着,你是何人?」大叉步便來捉獲。敬濟慌的奔走不迭,恰走到石橋下酒樓邊,只見一個人,頭戴萬字巾,身穿青衲襖,隨後趕到橋下,說道:「哥哥,你好大膽,平白在此看他怎的?」這敬濟扭囘頭看時,卻是一個識熟朋友——鐵指甲楊二郎。二人聲喏。楊二道:「哥哥一向不見,那裡去來?」敬濟便把東京父死徃囘之事,告說一遍:「恰纔這殺死婦人,是我丈人的小,潘氏。不知他被人殺了。適纔見了榜文,方知其故。」楊二郎告道:「他是小叔武松,充配在外,遇赦囘還,不知因甚殺了婦人,連王婆子也不饒。他家還有個女孩兒,在我姑夫姚二郎家養活了三四年。昨日他叔叔殺了人,走的不知下落。我姑夫將此女縣中領出,嫁與人為妻小去了。{\meipi{迎兒愚蠢極矣,所遭窮苦至矣,而究竟不失嫁為人妻。作者拈完此案,不無微意。}}見今這兩個屍首,日久只顧埋着,只是苦了地方保甲看守,更不知何年月日纔拏住兇犯武松。」說畢,楊二郎招了敬濟,上酒樓飲酒:「與哥拂塵。」敬濟見婦人已死,心中痛苦不了,那裡吃得下酒。約莫飲勾三盃,就起身下樓,作別來家。

到晚夕,買了一陌錢紙,在紫石街離王婆門首遠遠的石橋邊,叫着婦人:「潘六姐,我小兄弟陳敬濟,今日替你燒陌錢紙。皆因我來遲了一步,誤了你性命。你活時為人,死後為神,早保佑捉獲住仇人武松,替你報仇雪恨。我在法場上看着剮他,方趁我平生之志。」說畢哭泣,燒化了錢紙。敬濟囘家,閉了門戶。走歸房中,恰纔睡着,似睡不睡,夢見金蓮身穿素服,一身帶血,向敬濟哭道:「我的哥哥,我死的好苦也!實指望與你相處在一處,不期等你不來,被武松那厮害了性命。如今陰司不收,我白日遊遊蕩蕩,夜歸各處尋討漿水,適間蒙你送了一陌錢紙與我。但只是仇人未獲,我的屍首埋在當街,你可念舊日之情,買具棺材盛了葬埋,免得日久暴露。」敬濟哭道:「我的姐姐,我可知要葬埋你。但恐我丈母那無仁義的淫婦知道。他只恁賴我,倒趁了他機會。姐姐,你須徃守備府中,對春梅說知,教他葬埋你身屍便了。」婦人道:「剛纔奴到守備府中,又被那門神戶尉攔擋不放,奴須慢慢再哀告他則個。」敬濟哭着,還要拉着他說話,被他身上一陣血腥氣,撇手掙脫,卻是南柯一夢。枕上聽那更鼓時,正打三更三點,說道:「恠哉!我剛纔分明夢見六姐向我訴告衷腸,教我葬埋之意,又不知甚年何日拏着武松,是好傷感人也!」正是:

\begin{myquote}
夢中無限傷心事,獨坐空房哭到明。
\end{myquote}

按下一頭。卻表縣中訪拏武松,約兩個月有餘,捕獲不着,已知逃遁梁山為盜。地方保甲隣佑呈報到官,所有兩個屍首,相應責令家屬領埋。王婆屍首,便有他兒子王潮領的埋葬。止有婦人身屍,無人來領。卻說府中春梅,兩三日一遍,使張勝、李安來縣中打聽。囘去只說兇犯還未拏住,屍首照舊埋瘞,地方看守,無人敢動。直捱過年,正月初旬時節,忽一日晚間,春梅作一夢。恍恍惚惚,夢見金蓮雲髻蓬鬆,渾身是血,叫道:「龐大姐,我的好姐姐,奴死的好苦也!好容易來見你一面,又被門神把住嗔喝,不敢進來。今仇人武松已是逃走脫了。所有奴的屍首,在街暴露日久,風吹雨灑,雞犬作踐,無人領埋。奴舉眼無親,你若念舊日母子之情,買具棺木,把奴埋在一個去處,奴在陰司口眼皆閉。」{\meipi{金蓮一身,生時任人狼籍,即路倒路埋,所不惜也。及死後轉戀戀此屍,亦大可笑。}}說畢大哭不止。春梅扯住他,還要再問他別的話,被他掙開,撇手驚覺,卻是南柯一夢。從睡夢中直哭醒來,心內猶疑不定。

次日,叫進張勝、李安分付:「你二人去縣中打聽,那埋的婦人、婆子屍首還有也沒有。」張勝、李安應諾去了。不多時,來囘報:「正犯兇身已自逃走脫了。所有殺死身屍,地方看守,日久不便,相應責令各人家屬領埋。那婆子屍首,他兒子招領的去了。那婦人無人來領,還埋在街心。」春梅道:「既然如此,我這樁事兒,累你二人替我幹得來,我還重賞你。」二人跪下道:「小夫人說那裡話,若肯在老爺前擡舉小人一二,便消受不了。雖赴湯跳火,敢說不去?」春梅走到房中,拏出十兩銀子,兩疋大布,委付二人道:「這死的婦人,是我一個嫡親姐姐,嫁在西門慶家,今日出來,被人殺死。你二人休教你老爺知道,拏這銀子替我買一具棺材,把他裝殮了,擡出城外,擇方便地方埋葬停當,我還重賞你。」二人道「這個不打緊,小人就去。」李安說:「只怕縣中不教你我領屍怎了?須拏老爺個貼兒,下與縣官纔好。」張勝道:「只說小夫人是他妹子,嫁在府中,那縣官不敢不依,何消貼子。」於是領了銀子,來到監獄內。張勝便向李安說:「想必這死的婦人,與小夫人曾在西門慶家做一處,相結的好,今日方這等為他費心。想着死了時,整哭了三四日,不吃飯,直教老爺門前叫了調百戲貨郎兒,調與他觀看,還不喜歡。今日他無親人領去,小夫人豈肯不葬埋他?咱每若替他幹得此事停當,早晚他在老爺跟前,只方便你我,就是一點福星。{\meipi{春梅一女奴也,忽變而為福星,斯亦奇矣。及認為福星,則又變而為惡煞矣,造化不測如此。}}見今老爺百依百隨,聽他說話,正經大奶奶、二奶奶且打靠後。」

說畢,二人拏銀子到縣前遞了領狀,就說他妹子在老爺府中,來領屍首。使了六兩銀子,合了一具棺材,把婦人屍首掘出,把心肝填在肚內,用線縫上,用布裝殮停當,裝入材內。{\pangpi{二人幹事殊不苟。}}張勝說:「就埋在老爺香火院永福寺裡罷,那裡有空閑地。」就叫了兩名伴當,擡到永福寺,對長老說:「這是宅內小夫人的姐姐,要一塊地兒葬埋。」長老不敢怠慢,就在寺後揀一塊空心白楊樹下那裡葬埋。已畢,走來宅內囘春梅話,說:「除買棺材裝殮,還剩四兩銀子。」交割明白。春梅分付:「多有起動,你二人將這四兩銀子,拏二兩與長老道堅,教他早晚替他念些經懺,超度他昇天。」又拏出一大罈酒,一腿豬肉,一腿羊肉:「這二兩銀子,你每人將一兩家中盤纏。」二人跪下,那裡敢接?只說:「小夫人若肯在老爺面前擡舉小人,消受不了。這些小勞,豈敢接受銀兩。」春梅道:「我賞你,不收,我就惱了。」二人只得磕頭領了出來。兩個監獄吃酒,甚是稱念小夫人好處。{\pangpi{寫盡淺人眼孔。}}次日,張勝送銀子與長老念經,春梅又與五錢銀子買紙,與金蓮燒,俱不在話下。

卻說陳定從東京載靈柩家眷到清河縣城外,把靈柩寄在永福寺,等念經傳送,歸葬墳內。敬濟在家聽見母親張氏家小車輛到了,父親靈柩寄停在城外永福寺,收卸行李已畢,與張氏磕了頭。張氏恠他:「就不去接我一接。」{\pangpi{微露一斑。}}敬濟只說:「心中不好,家裡無人看守。」張氏便問:「你舅舅怎的不見?」敬濟道:「他見母親到,連忙搬囘家去了。」張氏道:「且教你舅舅住着,慌搬去怎的?」一面他母舅張團練來看姐姐。姊妹抱頭而哭,置酒敍說,不必細說。

次日,張氏早使敬濟拏五兩銀子、幾陌金銀錢紙,徃門外與長老,替他父親念經。正騎頭口街上走,忽撞遇他兩個朋友陸大郎、楊大郎,下頭口聲喏。二人問道:「哥哥那裡去?」敬濟悉言:「先父靈柩寄在門外寺裡,明日二十日是終七,家母使我送銀子與長老,做齋念經。」二人道:「兄弟不知老伯靈柩到了,有失弔問。」因問:「幾時發引安葬?」敬濟道:「也只在一二日之間,念經畢,入墳安葬。」說罷,二人舉手作別。這敬濟又叫住,因問楊大郎:「縣前我丈人的小,那潘氏屍首怎不見?被甚人領的去了?」楊大郎便道:「半月前,地方因捉不着武松,稟了本縣相公,令各家領去葬埋。王婆是他兒子領去。這婦人屍首,丟了三四日,被守備府中買了一口棺材,差人擡出城外永福寺去葬了。」敬濟聽了,就知是春梅在府中收葬了他屍首。因問二郎:「城外有幾個永福寺?」二郎道:「南門外只有一個永福寺,是周秀老爺香火院,那裡有幾個永福寺來?」敬濟聽了,暗喜:「就是這個永福寺,也是緣法湊巧,喜得六姐亦葬在此處。」一面作別二人,打頭口出城,徑到永福寺中。見了長老,且不說念經之事,就先問長老道堅:「此處有守備府中新近葬的一個婦人,在那裡?」長老道:「就在寺後白楊樹下。說是宅內小夫人的姐姐。」這陳敬濟且不叅見他父親靈柩,{\meipi{寫敬濟不孝處刺骨,然此等不孝中人,上下皆有之,讀者不可徒笑敬濟,而不自省也。}}先拏錢紙祭物,至於金蓮墳上,與他祭了,燒化錢紙,哭道:「我的六姐,你兄弟陳敬濟來與你燒一陌紙錢,你好處安身,苦處用錢。」祭畢,然後纔到方丈內他父親靈柩跟前燒紙祭祀。遞與長老經錢,教他二十日請八衆禪僧,念斷七經。長老接了經襯,備辦齋供。敬濟到家,囘了張氏話。二十日都去寺中拈香,擇吉發引,把父親靈柩歸到祖塋。安葬已畢,來家母子過日不題。

卻表吳月娘,一日二月初旬,天氣融和,孟玉樓、孫雪娥、西門大姐、小玉,出來大門首站立,觀看來徃車馬,人烟熱鬧。忽見一簇男女,跟着個和尚,生的十分胖大,頭頂三尊銅佛,身上構着數枝燈樹,杏黃袈裟風兜袖,赤脚行來泥沒踝。當時古人有幾句,赞的這行脚僧好處:

\begin{myquote}
打坐叅禪,講經說法。鋪眉苦眼,習成佛祖家風;賴教求食,立起法門規矩。白日裡賣杖搖鈴,黑夜間舞槍弄棒。有時門首磕光頭,餓了街前打响嘴。空色色空,誰見衆生離下土?去來來去,何曾接引到西方。
\end{myquote}

那和尚見月娘衆婦人在門首,便向前道了個問訊,說道:「在家老菩薩施主,既生在深宅大院,都是龍華一會上人。貧僧是五臺山下來的,結化善緣,蓋造十王功德,三寶佛殿。仰賴十方施主菩薩,廣種福田,捨資才共成勝事,種來生功果。貧僧只是挑脚漢。」月娘聽了他這般言語,便喚小玉徃房中以一頂僧帽,一雙僧鞋,一弔銅錢,一斗白米。原來月娘平昔好齋僧布施,常時發心做下僧帽、僧鞋,預備來施。這小玉取出來,月娘分付:「你叫那師父近前來,布施與他。」這小玉故做嬌態,高聲叫道:「那變驢的和尚,過不過來!{\meipi{和尚能變驢,當更有人愛。}}俺奶奶布施與你這許多東西,還不磕頭哩。」月娘便罵道:「恠墮業的小臭肉兒,一個僧家,是佛家弟子,你有要沒緊,恁謗他怎的?不當家化化的,你這小淫婦兒,到明日不知墮多少罪業!」小玉笑道:「奶奶,這賊和尚,我叫他,他怎的把一雙賊眼,眼上眼下打量我?」那和尚雙手接了鞋帽錢來,打問訊說道:「多謝施主老菩薩布施。」小玉道:「這禿厮好無礼。這些人站着,只打兩個問訊兒,就不與我打一個兒?」月娘道:「小肉兒,還恁說白道黑道。他一個佛家之子,你也消受不的他這個問訊。」小玉道:「奶奶,他是佛爺兒子,誰是佛爺女兒?」{\meipi{戲謔得有韻有趣,可作《世說》補。}}月娘道:「相這比丘尼姑僧,是佛的女兒。」小玉道:「譬若說,相薛姑子、王姑子、大師父,都是佛爺女兒,誰是佛爺女婿?」月娘忍不住笑,罵道:「這賊小淫婦兒,也學的油嘴滑舌,見見就說下道兒去了。」小玉道:「奶奶只罵我,本等這禿和尚賊眉豎眼的只看我。」{\meipi{小玉情竇之開,而耳目多事,不必言說,和尚看他,卻未必盡謗也。}}孟玉樓道:「他看你,想必認得你,要度脫你去。」小玉道:「他若度我,我就去。」說着,衆婦女笑了一囘。月娘喝道:「你這小淫婦兒,專一毀僧謗佛。」那和尚得了布施,頂着三尊佛揚長而去了。小玉道:「奶奶還嗔我罵他,你看這賊禿,臨去還看了我一眼纔去了。」有詩單道月娘修善施僧好處:

\begin{myquote}
守寡看經歲月深,私邪空色久違心。\\奴身好似天邊月,不許浮雲半點侵。
\end{myquote}

月娘衆人正在門首說話,忽見薛嫂兒提着花箱兒,從街上過來。見月娘衆人道了萬福。月娘問:「你徃那裡去來?怎的影跡兒也不來我這裡走走?」薛嫂兒道:「不知我終日窮忙的是些甚麼。這兩日,大街上掌刑張二老爹家,與他兒子和北邊徐公公家做親,娶了他姪女兒,也是我和文嫂兒說的親事。昨日三朝,擺大酒席,忙的連守備府裡咱家小大姐那裡叫我,也沒去,不知怎麼惱我哩。」月娘問道:「你如今徃那裡去?」薛嫂道:「我有樁事,敬來和你老人家說來。」月娘道:「你有話進來說。」一面讓薛嫂兒到後邊上房裡坐下,吃了茶。薛嫂道:「你老人家還不知道,你陳親家從去年在東京得病沒了,親家母叫了姐夫去,搬取老小靈柩。從正月來家,已是念經傳送,墳上安葬畢。我只說你老人家這邊知道,怎不去燒張紙兒,探望探望。」月娘道:「你不來說,俺怎得曉的,又無人打聽。{\meipi{絕不露來蹤去跡。}}倒只知道潘家的吃他小叔兒殺了,和王婆子都埋在一處,卻不知如今怎樣了。」薛嫂兒道:「自古生有地兒死有處。五娘他老人家,不因那些事出去了,卻不好來。平日不守本分,幹出醜事來,出去了,若在咱家裡,他小叔兒怎得殺了他?還是冤有頭,債有主。倒還虧了咱家小大姐春梅,越不過娘兒們情場,差人買了口棺材,領了他屍首,葬埋了。不然只顧暴露着,又拏不着小叔子,誰去管他?」孫雪娥在旁說:「春梅在守備府中多少時兒,就這等大了?手裡拏出銀子,替他買棺材埋葬,那守備也不嗔,當他甚麼人?」{\meipi{不知天下士,猶作布衣看。}}薛嫂道:「耶嚛,你還不知,守備好不喜他,每日只在他房裡歇臥,說一句依十句,一娶了他,見他生的好模樣兒,乖覺伶俐,就與他西廂房三間房住,撥了個使女伏侍他。老爺一連在他房裡歇了三夜,替他裁四季衣服,上頭。三日吃酒,賞了我一兩銀子,一疋段子。他大奶奶五十歲,雙目不明,吃長齋,不管事。東廂孫二娘生了小姐,雖故當家,撾着個孩子。如今大小庫房鑰匙,倒都是他拏着,守備好不聽他說話哩。且說銀子,手裡拏不出來?」{\meipi{聞此語月娘猶可,雪娥將氣死矣。}}幾句說的月娘、雪娥都不言語。坐了一囘,薛嫂起身。月娘分付:「你明日來,我這裡備一張祭桌,一疋尺頭,一分冥紙,你來送大姐與他公公燒紙去。」薛嫂兒道:「你老人家不去?」月娘道:「你只說我心中不好,改日望親家去罷。」{\meipi{月娘為德不卒,到此未免有慚色。}}那薛嫂約定:「你教大姐收拾下等着我。飯罷時候我來。」月娘道:「你如今到那裡去?守備府中不去也罷。」薛嫂道:「不去,就惹他恠死了。他使小伴當叫了我好幾遍了。」月娘道:「他叫你做甚麼?」薛嫂道:「奶奶,你不知。他如今有了四五個月身孕了,老爺好不喜歡,叫了我去,已定賞我。」提着花箱,作辭去了。雪娥便說:「老淫婦說的沒個行款也!他賣與守備多少時,就有了半肚孩子,那守備身邊少說也有幾房頭,莫就興起他來,這等大道?」{\meipi{世上一種輕薄人,只是眼淺。}}月娘道:「他還有正景大奶奶,房裡還有一個生小姐的娘子兒哩。」雪娥道:「可又來!到底還是媒人嘴,一尺水十丈波的。」不因今日雪娥說話,正是:從天降下鉤和線,就地引來是非來。有詩為證:

\begin{myquote}
曾記當年侍主旁,誰知今日變風光。\\世間萬事皆前定,莫笑浮生空自忙。
\end{myquote}

