\includepdf[pages={105,106},fitpaper=false]{tst.pdf}
\chapter*{第五十三囘 潘金蓮驚散幽歡 吳月娘拜求子息}
\addcontentsline{toc}{chapter}{第五十三囘 潘金蓮驚散幽歡 吳月娘拜求子息}
\markboth{\titlename}{第五十三囘 潘金蓮驚散幽歡 吳月娘拜求子息}


詞曰:

小院閑堦玉砌,墻隈半簇蘭芽。一庭萱草石榴花,多子宜男愛插。休使風吹雨打,老天好為藏遮。莫教變作杜鵑花,粉褪紅銷香罷。

——右調《應天長》

話說陳敬濟與金蓮不曾得手,悵怏不題。單表西門慶赴黃、安二主事之席。乘着馬,跟隨着書童、玳安四五人,來到劉太監庄上。早有承局報知,黃、安二主事忙整衣冠,出來迎接。那劉太監是地主,也同來相迎。西門慶下了馬,劉太監一手挽了西門慶,笑道:「咱三箇等候的好半日了,老丈卻纔到來。」西門慶答道:「蒙兩位老先生見招,本該早來,實為家下有些小事,反勞老公公久待,望乞恕罪。」三箇大打恭,進儀門來。讓到廳上,西門慶先與黃主事作揖,次與安主事、劉太監都作了揖,四人分賓主而坐。第一位讓西門慶坐了,第二就該劉太監坐。劉太監再四不肯,道:「咱忝是房主,還該兩位老先生,是遠客。」安主事道:「定是老先兒。」西門慶道:「若是序齒,還該劉公公。」劉大監推卻不過,向黃、安兩主事道:「斗膽占了。」便坐了第二位。黃、安二主事坐了主席。一班小優兒上來磕了頭,左右獻過茶,當值的就遞上酒來。黃、安二主事起身安席坐下。小優兒拏檀板、琵琶、絃索、簫管上來,合定腔調,細細唱了一套《宜春令》「青陽候烟雨淋」。唱畢,劉太監舉盃勸衆官飲酒。安主事道:「這一套曲兒,做的清麗無比,定是一箇絕代才子。況唱的聲音嘹亮,響遏行雲,卻不是箇雙絕了麼!」西門慶道:「那箇也不當奇,今日有黃、安二位做了賢主,劉公公做了地主,這纔是難得哩!」黃主事笑道:「也不為奇。劉公公是出入紫禁,日覲龍顏,可不是貴臣?西門老丈,堆金積玉,彷彿陶朱,可不是富人?富貴雙美,這纔是奇哩!」{\meipi{贊處妙在帶三分笑罵意。}}四箇人哈哈大笑。當值的斟上酒來,又飲了一囘。小優兒又拏碧玉洞簫,吹得悠悠咽咽,和着板眼,唱一套《沽美酒》「桃花溪,楊桺腰」的時曲。唱畢,衆客又赞了一番,歡樂飲酒不題。

且說陳敬濟因與金蓮不曾得手,耐不住滿身慾火。見西門慶吃酒到晚還未來家,依舊閃入捲棚後面,探頭探腦張看。原來金蓮被敬濟鬼混了一場,也十分難熬,正在無人處手托香腮,沉吟思想。不料敬濟三不知走來,黑影子裡看見了,恨不的一碗水咽將下去。就大着膽,悄悄走到背後,將金蓮雙手抱住,便親了箇嘴,說道:「我前世的娘!起先吃孟三兒那冤家開啟了,幾乎把我急殺了。」金蓮不提防,吃了一嚇。囘頭看見是敬濟,心中又驚又喜,便罵道:{\meipi{驚喜便罵,因知婦人罵人,必定驚而喜矣。}}「賊短命,閃了我一閃,快放手,有人來撞見怎了!」敬濟那裡肯放,便用手去解他褲帶。金蓮猶半推半就,早被敬濟一扯扯斷了。金蓮故意失驚道:「恠賊囚,好大膽!就這等容容易易要奈何小丈母!」{\pangpi{猶立名分,妙。}}敬濟再三央求道:「我那前世的親娘,要敬濟的心肝煮湯吃,我也肯割出來。沒奈何,只要今番成就成就。」敬濟口裡說着,腰下那話已是硬帮帮的露出來,朝着金蓮單裙只顧亂插。金蓮桃頰紅潮,情動久了。初還假做不肯,{\meipi{寫佯推故就,字字銷魂。}}及被敬濟累垂敖曹觸着,就禁不的把手去摸。{\pangpi{真情露矣。}}敬濟便趁勢一手掀開金蓮裙子,盡力徃內一插,不覺沒頭露腦。原來金蓮被纏了一囘,臊水濕漉漉的,因此不費力送進了。兩箇緊傍在紅欄干上,任意抽送,敬濟還嫌不得到根,教金蓮倒在地下:「待我奉承你一箇不亦樂乎!」金蓮恐散了頭髮,又怕人來,{\meipi{敬濟一味急,金蓮雖急又急不得,更苦。}}推道:「今番且將就些,後次再得相聚,憑你便了。」{\pangpi{自開後約。}}一箇「達達」連聲,一箇「親親」不住,厮並了半箇時辰。只聽得隔墻外籟籟的響,又有人說話,兩箇一鬨而散。

敬濟雲情未已,金蓮雨意方濃。卻是書童、玳安拏着冠帶拜匣,都醉醺醺的嚷進門來。月娘聽見,知道是西門慶來家,忙差小玉出來看。書童、玳安道:「爹隨後就到了。我兩人怕晚了,先來了。」不多時,西門慶下馬進門,已醉了,直奔到月娘房裡來。摟住月娘就待上床。月娘因要他明日進房,應二十三壬子日服藥行事,便不留他,道:「今日我身子不好,你徃別房裡去罷。」西門慶笑道:「我知道你嫌我醉了,不留我。也罷,別要惹你嫌。我去了,明晚來罷。」{\pangpi{連絡得妙。}}月娘笑道:「我真有些不好,月經還未淨。誰嫌你?明晚來罷。」西門慶就徃潘金蓮房裡去了。金蓮正與敬濟不盡興囘房,眠在炕上,一見西門慶進來,忙起來笑迎道:「今日吃酒,這咱時纔來家。」西門慶也不答應,一手摟將過來,連親了幾箇嘴,一手就下邊一摸,摸着他牝戶,道:「恠小淫婦兒,你想着誰來?兀那話濕搭搭的。」{\meipi{暖昧處偏識破,卻又當面瞞過,為得奇險驚人。}}金蓮自覺心虛,也不做聲。{\pangpi{做一聲便不妙。}}只笑推開了西門慶,向後邊澡牝去了。當晚與西門慶雲情雨意,不消說得。

且表吳月娘次日起身,正是二十三壬子日,梳洗畢,就教小玉擺着香桌,上邊放着寶爐,燒起名香,又放上《白衣觀音經》一卷。月娘向西皈依礼拜,拈香畢,將經展開,念一遍,拜一拜,念了二十四遍,拜了二十四拜,圓滿。然後箱內取出丸藥放在桌上,又拜了四拜,禱告道:「我吳氏上靠皇天,下賴薛師父、王師父這藥,{\meipi{以二尼並大祝赞,妙刺。}}仰祈保佑,早生子嗣。」告畢,小玉燙的熱酒,傾在盞內。月娘接過酒盞,一手取藥調勻,西向跪倒,先將丸藥嚥下,又取末藥也服了,喉嚨內微覺有些腥氣。{\pangpi{映出胎衣。}}月娘迸着氣一口呷下,{\pangpi{愚人之苦。}}又拜了四拜。當日不出房,只在房裡坐的。西門慶在潘金蓮房中起身,就叫書童寫謝宴貼,徃黃、安二主事家謝宴。書童去了,就是應伯爵來到。西門慶出來,應伯爵作了揖,說道:「哥,昨在劉太監家吃酒,幾時來家?」西門慶道:「承兩公十分相愛,灌了好幾盃酒,歸路又遠,更餘來家。已是醉了,這咱纔起身。」玳安捧出早飯,西門慶正和伯爵同吃,又報黃主事、安主事來拜。西門慶整衣冠,教收過家活出迎。應伯爵忙迴避了。黃、安二主事一齊下轎。進門厮見畢,三人坐下,一面捧出茶來吃了。黃、安二主事道:「夜來有褻,」西門慶道:「多感厚情,正要叩謝兩位老先生,如何反勞臺駕先施!」安主事道:「昨晚老先生還未盡興,為何就別了?」西門慶道:「晚生已大醉了。臨起身,又被劉公公灌上十數盃葡萄酒,在馬上就要嘔,耐得到家,睡到今日還有些不醒哩。」笑了一番,又吃過三盃茶,說些閑話,作別去了。應伯爵也推事故家去。西門慶囘進後邊吃了飯,就坐轎答拜黃、安二主事去。又寫兩箇紅礼帖,分付玳安備辦兩副下程,趕到他家面送。當日無話。

西門慶來家,吳月娘打點床帳,等候進房。西門慶進了房,月娘就教小玉整設餚饌,燙酒上來,兩人促膝而坐。西門慶道:「我昨夜有了盃酒,你便不肯留我,又假推甚麼身子不好,這咱搗鬼!」月娘道,「這不是搗鬼,果然有些不好。難道夫妻之間恁地疑心?」西門慶吃了十數盃酒,又吃了些鮮魚鴨臘,便不吃了,月娘交收過了。小玉薰的被窩香噴噴的,兩箇洗澡已畢,脫衣上床。枕上綢繆,被中繾綣,言不可盡。這也是吳月娘該有喜事,恰遇月經轉,兩下似水如魚,便得了子了。{\meipi{月娘得子,寫得與藥毫不相干,春秋妙筆。}}正是:

花有並頭蓮並蒂,帶宜同挽結同心。

次日,西門慶起身梳洗,月娘備有羊羔美酒、雞子腰子補腎之物,與他吃了,打發進衙門去。西門慶衙門散了囘來,就進李瓶兒房看哥兒。李瓶兒抱着孩子向西門慶道:「前日我有些心願未曾了。這兩日身子有些不好,坐淨桶時,常有些血水淋得慌。早晚要酬酬心願,你又忙碌碌的,不得箇閑空。」西門慶道:「你既要了願時,我叫玳安去接王姑子來,{\meipi{西門慶平時最鄙薄姑子,今日忽曰「接來」,所謂愚人易惑也。}}與他商量,做些好事就是了。」便叫玳安,分付接王姑子。玳安應諾去了。

書童又報:「常二叔和應二爹來到。」西門慶便出迎厮見。應伯爵道:「前日謝子純在這裡吃酒,我說的黃四、李三的那事,哥應付了他罷。」西門慶道:「我那裡有銀子?」應伯爵道:「哥前日已是許下了,如何又變了卦?哥不要瞞我,等地財主,說箇無銀出來?隨分湊些與他罷。」西門慶不答應他,只顧呆了臉看常峙節。{\meipi{財主只一不答應,便令求者無所施其喙。}}常峙節道:「連日不曾來,哥,小哥兒長養麼?」西門慶道:「生受注念,卻纔你李家嫂子要酬心願,只得去請王姑子來家做些好事。」應伯爵道:「但凡人家富貴,專待子孫掌管。養得來時,須要十分保護。譬如種五穀的,初長時也得時時灌溉,纔望箇秋收。小哥兒萬金之軀,是箇掌中珠,又比別的不同。小兒郎三歲有關,六歲有厄,九歲有煞,又有出痧出痘等症。哥,不是我口直,論起哥兒,自然該與他做些好事,廣種福田。若是嫂子有甚願心,正宜及早了當,管情交哥兒無災無害好養。」說話間,只見玳安來囘話道:「王姑子不在庵裡,到王尚書府中去了。小的又到王尚書府中找尋他,半日纔得出來。與他說了,便來了。」西門慶聽罷,依舊和伯爵、常峙節說話兒,一處坐地,書童拏些茶來吃了。伯爵因開言道:「小弟蒙哥哥厚愛,一向因寒家房子窄隘,不敢簡褻,多有疎失。今日稟明了哥,若明後日得空,望哥同常二哥出門外花園裡頑耍一日,少盡兄弟孝順之心。」常峙節從旁赞道:「應二哥一片獻芹之心,哥自然鑑納,決沒有見卻的理。」西門慶道:「若論明日,到沒事,只不該生受。」伯爵道:「小弟在宅裡,筷子也不知吃了多少下去,{\pangpi{似非謙詞。}}今日一盃水酒,當的甚麼。」西門慶道:「既如此,我便不徃別處去了。」伯爵道:「只是還有一件,小優兒,小弟便叫了。但郊外去,必須得兩箇唱的去,方有興趣。」西門慶道:「這不打緊,我叫人去叫了吳銀兒與韓金釧兒就是了。」伯爵道:「如此可知好哩。只是又要哥費心,不當。」西門慶一面就叫琴童,分付去叫吳銀兒、韓金釧兒,明日早徃門外花園內唱。琴童應諾去了。不多時,王姑子來到廳上,見西門慶道箇問訊:「動問施主,今日見召,不知有何分付?老身因王尚書府中有些小事去了,不得便來,方纔得脫身。」西門慶道:「因前日養官哥許下些願心,一向忙碌碌,未曾完得。托賴皇天保護,日漸長大。我第一來要酬報佛恩,第二來要消災延壽,因此請師父來商議。」王姑子道:「小哥兒萬金之軀,全憑佛力保護。老爹不知道,我們佛經上說,人中生有夜叉羅剎,常喜啖人,令人無子,傷胎奪命,皆是諸惡鬼所為。{\meipi{僧尼專拏神鬼嚇人,故易使人怕,怕則信,信則從矣。}}如今小哥兒要做好事,定是看經念佛,其餘都不是路了。」西門慶便問做甚功德好,王姑子道:「先拜卷《藥師經》,待迴向後,再印造兩部《陀羅經》,極有功德。」西門慶問道:「不知幾時起經?」王姑子道:「明日到是好日,就我庵中完願罷。」西門慶點着頭道:「依你,依你。」王姑子說畢,就徃後邊,見吳月娘和六房姊妹都在李瓶兒房裡。王姑子各打了問訊。月娘便道:「今日央你做好事保護官哥,你幾時起經頭?」王姑子道:「來日黃道吉日,就我庵裡起經。」小玉拏茶來吃了。李瓶兒因對王姑子道:「師父,我還有句話,一發央及你。」王姑子道:「你老人家有甚話,但說不妨。」李瓶兒道:「自從有了孩子,身子便有些不好。明日疏意裡邊,帶通一句何如?行的去,我另謝你。」王姑子道:「這也何難。且待寫疏的時節,一發寫上就是了。」正是:

禍因惡積非無種,福自天來定有根。

