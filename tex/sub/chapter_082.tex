\includepdf[pages={163,164},fitpaper=false]{tst.pdf}
\chapter*{第八十二囘 陳敬濟弄一得雙 潘金蓮熱心冷面}
\addcontentsline{toc}{chapter}{第八十二囘 陳敬濟弄一得雙 潘金蓮熱心冷面}
\markboth{{\titlename}卷之九}{第八十二囘 陳敬濟弄一得雙 潘金蓮熱心冷面}


詞曰:

聞道雙啣鳳帶,不妨單着鮫綃。夜香知為阿誰燒?悵望水沉烟梟。雲𩬆風前綠捲,玉顏想處紅潮,莫交空負可憐宵,月下雙灣步俏。

——右調《西江月》

話說潘金蓮與陳敬濟,自從在廂房裡得手之後,兩箇人嘗着甜頭兒,日逐白日偷寒,黃昏送暖。或倚肩嘲笑,或並坐調情,掐打揪撏,通無忌憚。或有人跟前不得說話,將心事寫了,搓成紙條兒,丟在地下,你有話傳與我,我有話傳與你。一日,四月天氣,潘金蓮將自己袖的一方銀絲汗貼兒,裹着一箇紗香袋兒,裡面裝一縷頭髮並些松柏兒,封的停當,要與敬濟。不想敬濟不在廂房內,遂打窓眼內投進去。後敬濟進房,看見彌封甚厚,開啟卻是汗巾香袋兒,紙上寫一詞,名《寄生草》:

將奴這銀絲帕,並香囊寄與他。當初結下青絲髮。松柏兒要你常牽掛,淚珠兒滴寫相思話。夜深燈照的奴影兒孤,休負了夜深潛等荼縻架。

敬濟見詞上約他在荼縻架下等候,私會佳期。隨即封了一柄湘妃筆金扇兒,亦寫了一詞在上囘答他,袖入花園內。不想月娘正在金蓮房中坐着,這敬濟三不知,走進角門就叫:「可意人在家不在?」這金蓮聽見是他語音,恐怕月娘聽見決撒了,連忙掀簾子走出來。看着他擺手兒,佯說:「我道是誰,原來是陳姐夫來尋大姐。大姐剛纔在這裡,和他每徃花園亭子上摘花兒去了。」{\meipi{凡入此境,更有許多剛巧剛不巧情景,使人遮遮掩掩,驚驚喜喜。}}這敬濟見有月娘在房裡,就把物事暗暗遞與婦人袖了,他就出去了。月娘便問:「陳姐夫來做甚麼?」金蓮道:「他來尋大姐,我囘他徃花園中去了。」以此瞞過月娘。少頃,月娘起身囘後邊去了。金蓮向袖中取出拆開,卻是湘妃竹金扇兒一柄,上面一種青蒲,半溪流水,有《水仙子》一首詞兒:

紫竹白紗甚逍遙,綠青蒲巧製成,金鉸銀錢十分妙。美人兒堪用着,遮炎天少把風招。有人處常常袖着,無人處慢慢輕搖,休教那俗人見偷了。{\meipi{此詞疑是敬濟的筆。}}

婦人看見其詞,到於晚夕月上時,早把春梅、秋菊兩箇丫頭打發些酒與他吃,關在那邊炕屋睡。然後自在房中,綠窓半啟,絳燭高燒,收拾床鋪衾枕,薰香澡牝,獨立木香棚下,專等敬濟來赴佳期。西門大姐那夜恰好被月娘請去後邊,聽王姑子宣卷去了,只有元宵兒在屋裡。敬濟梯己與了他一方手帕,分付他:「看守房中,我徃你五娘那邊下棋去。等大姑娘進來,你快來叫我。」元宵兒應諾了。敬濟得手,走來花園中,只見花篩月影,叅差掩映。走到荼縻架下,遠望見婦人摘去冠兒,亂挽烏雲,悄悄在木香棚下獨立。這敬濟猛然從荼縻架下突出,雙手把婦人抱住。把婦人唬了一跳,說:「呸,小短命!猛然鑽出來,唬了我一跳。早是我,你摟便將就罷了,若是別人,你也恁膽大摟起來?」敬濟吃得半酣兒,笑道:「早是摟了你,就錯摟了紅娘,也是沒奈何。」{\pangpi{趁勢就插入春梅,妙甚。}}兩箇於是相摟相抱,攜手進入房中。房中熒煌煌掌着燈燭,桌上設着酒餚,一面頂了角門,並肩而坐飲酒。婦人便問:「你來,大姐在那裡?」敬濟道:「大姐後邊聽宣卷去了,我分付下元宵兒,有事來這裡叫,我只說在這裡下棋。」說畢,兩箇歡笑做一處。飲酒多時,常言「風流茶說合,酒是色媒人」,不覺竹葉穿心,桃花上臉,一箇嘴兒相親,一箇腮兒厮搵,罩了燈,上床交接。有《六娘子》小詞為證:

入門來,奴摟抱在懷。奴把錦被兒伸開,俏冤家頑的十分恠。嗏,將奴脚兒擡。脚兒擡,揉亂了烏雲,鬏髻兒歪。

兩人雲雨纔畢,只聽得元宵叫門說:「大姑娘進房中來了。」這敬濟慌的穿衣去了。正是:

狂蜂浪蝶有時見,飛入梨花無處尋。

原來潘金蓮那邊三間樓上,中間供養佛像,兩邊稍間堆放生藥香料。兩箇自此以後,情沾肺腑,意密如漆,無日不相會做一處。一日也是合當有事,潘金蓮早晨梳粧打扮,走來樓上觀音菩薩前燒香。不想陳敬濟正拏鑰匙上樓,開庫房門拏藥材香料,撞遇在一處。這婦人且不燒香,{\pangpi{金蓮也燒香,大奇。}}見樓上無人,兩箇摟抱着親嘴咂舌,一箇叫「親親五娘」,一箇呼「心肝短命」,{\pangpi{此纔是金蓮燒的香。}}因說:「趁無人,咱在這裡幹了罷。」一面解褪衣褲,就在一張春凳上雙鳧飛肩,靈根半入,不勝綢繆。當初沒巧不成話,兩箇正幹得好,不防春梅正上樓來,拏盒子取茶葉看見。兩箇湊手脚不迭,都吃了一驚。春梅恐怕羞了他,連忙倒退囘身子,走下胡梯。慌的敬濟兜小衣不迭,婦人穿上裙子,忙叫春梅:「我的好姐姐,你上來,我和你說話。」那春梅於是走上樓來。金蓮道:「我的好姐姐,你姐夫不是別人,我今叫你知道了罷。俺兩箇情孚意合,拆散不開。你千萬休對人說,只放在你心裡。」春梅便說:「好娘,說那裡話。奴伏侍娘這幾年,豈不知娘心腹,肯對人說!」婦人道:「你若肯遮蓋俺們,趁你姐夫在這裡,你也過來和你姐夫睡一睡,我方信你。你若不肯,只是不可憐見俺每了。」那春梅把臉羞的一紅一白,只得依他。卸下湘裙,解開褲帶,仰在凳上,盡着這小夥兒受用。{\meipi{金蓮分惠耶,拖人落水耶?春梅屈從耶,歡喜領受耶?再四思之不得。}}有這等事!正是:

明珠兩顆皆無價,可奈檀郎盡得鑽。

有《紅繡鞋》為證:

假認做女婿親厚,徃來和丈母歪偷。人情裡包藏鬼胡油。明講做兒女禮,暗結下燕鶯儔,他兩箇見今有。

當下盡着敬濟與春梅耍完,大家方纔走散。自此以後,潘金蓮便與春梅打成一家,與這小夥兒暗約偷期,非只一日,只背着秋菊。六月初一日,潘姥姥老病沒了,有人來說。吳月娘買一張插桌,三牲冥紙,教金蓮坐轎子徃門外探䘮祭祀,去了一遭囘來。到次日,六月初三日,金蓮起來得早,在月娘房裡坐着,說了半日話出來,走在大廳院子裡墻根下,急了溺尿。正撩起裙子,蹲踞溺尿。原來西門慶死了,沒人客來徃,等閑大廳儀門只是關閉不開。敬濟在東廂房住,纔起來,忽聽見有人在墻根溺的尿刷刷的響,悄悄向窓眼裡張看,卻不想是他,便道:「是那箇撒野,在這裡溺尿?撩起衣服,看濺濕了裙子?」這婦人連忙繫上裙子,走到窓下問道:「原來你在屋裡,這咱纔起來,好自在。大姐沒在房裡麼?」敬濟道:「在後邊,幾時出來!昨夜三更纔睡,大娘後邊拉着我聽宣《紅羅寶卷》,坐到那咱晚,險些兒沒把腰累[]瘑了,今日白扒不起來。」{\meipi{月娘強人聽宣卷,亦大是苦事。}}金蓮道:「賊牢成的,就休搗謊哄我!昨日我不在家,你幾時在上房內聽宣卷來?丫鬟說你昨日在孟三兒房裡吃飯來。」{\pangpi{又生枝葉,妙。}}敬濟道:「早是大姐看着,俺每都在上房內,幾時在他屋裡去來!」說着,這小夥兒站在炕上,把那話弄得硬硬的,直豎的一條棍,隔窓眼裡舒過來。{\meipi{奇想,發千古所未發。}}婦人一見,笑的要不得,{\pangpi{喜甚。}}罵道:「恠賊牢拉的短命,猛可舒出你老子頭來,唬了我一跳。你趁早好好抽進去,我好不好拏針刺與你一下子,教你忍痛哩!」敬濟笑道:「你老人家這囘兒又不待見他起來,你好歹打發他箇好處去,也是你一點陰騭。」{\pangpi{語語趣而諧。}}婦人罵道:「好箇恠牢成久慣的囚根子!」一面向腰裡摸出面青銅小鏡來,放在窓櫺上,假做勻臉照鏡,一面用朱唇吞裹吮咂他那話,{\meipi{此想更奇。情真意切,便有許多急智。}}吮咂的這小郎君一點靈犀灌頂,滿腔春意融心。正咂在熱鬧處,忽聽得有人走的脚步兒響,這婦人連忙摘下鏡子,走過一邊。敬濟便把那話抽囘去。卻不想是來安兒小厮走來,說:「傅大郎前邊請姐夫吃飯哩。」敬濟道:「教你傅大郎且吃着,我梳頭哩,就來。」來安兒囘去了。婦人便悄悄向敬濟說:「晚夕你休徃那裡去了,在屋裡,我使春梅叫你。好歹等我,有話和你說。」敬濟道:「謹依來命。」婦人說畢,囘房去了。敬濟梳洗畢,徃鋪中自做買賣。不題。

不一時,天色晚來。那日,月黑星密,天氣十分炎熱。婦人令春梅燒湯熱水,要在房中洗澡,修剪足甲。床上收拾衾枕,趕了蚊子,放下紗帳子,小篆內炷了香。春梅便叫:「娘不知,今日是頭伏,你不要些鳳仙花染指甲?我替你尋些來。」{\pangpi{春梅頗有情興。}}婦人道:「你那裡尋去?」春梅道:「我直徃那邊大院子裡纔有,我去拔幾根來。娘教秋菊尋下杵臼,搗下蒜。」婦人附耳低言,悄悄分付春梅:「你就廂房中請你姐夫晚夕來,我和他說話。」春梅去了,這婦人在房中,比及洗了香肌,修了足甲,也有好一囘。只見春梅拔了幾顆鳳仙花來,整叫秋菊搗了半日。婦人又與了他幾鍾酒吃,打發他廚下先睡了。婦人燈光下染了十指春蔥,令春梅拏櫈子放在天井內,鋪着涼簟衾枕納涼。約有更闌時分,但見朱戶無聲,玉繩低轉,牽牛、織女二星隔在天河兩岸。又忽聞一陣花香,幾點螢火。婦人手拈紈扇,伏枕而待。春梅把角門虛掩。正是:

待月西廂下,迎風戶半開。隔墻花影動,疑是玉人來。

原來敬濟約定搖木瑾花樹為號,就知他來了。婦人見花枝搖影,知是他來,便在院內咳嗽接應。他推開門進來,兩箇並肩而坐。婦人便問:「你來,房中有誰?」敬濟道:「大姐今日沒出來,我已分付元宵兒在房裡,有事先來叫我。」因問:「秋菊睡了?」婦人道:「已睡熟了。」說畢,相摟相抱,二人就在院內櫈上,赤身露體,席上交歡。不勝繾綣。但見:

情興兩和諧,摟定香肩臉搵腮。手撚香乳綿似軟,實奇哉!掀起脚兒脫繡鞋,玉体着郎懷。舌送丁香口便開,倒鳳顛鸞雲雨罷,囑多才:明朝千萬早些來。

兩箇雲雨畢,婦人拏出五兩碎銀子來,遞與敬濟說:「門外你潘姥姥死了,棺材已是你爹在日與了他。三日入殮時,你大娘教我去探䘮燒紙來了。明日出殯,你大娘不放我去,說你爹熱孝在身,只見出門。這五兩銀子交與你,明早央你蚤去門外傳送傳送你潘姥姥,打發擡錢,看着下入土內,你來家。就同我去一般。」{\pangpi{親親之詞。}}這敬濟一手接了銀子,說:「這箇不打緊。我明日絕早就出門,幹畢事,來囘你老人家。」說畢,恐大姐進房,老早歸廂房中去了。一宿晚景休題。到次日,到飯時就來家。金蓮纔起來,在房中梳頭。敬濟走來囘話,就門外昭化寺裡,拏了兩枝茉莉花兒來婦人戴。婦人問:「棺材下了葬了?」敬濟道:「我管何事,不打發他老人家黃金入了櫃,我敢來囘話!還剩了二兩六七錢銀子,交付與你妹子收了,盤纏度日。千恩萬謝,多多上覆你。」婦人聽見他娘入土,落下淚來。{\meipi{至性終在。}}便叫春梅:「把花兒浸在盞內,看茶來與你姐夫吃。」不一時,兩盒兒蒸酥,四碟小菜,打發敬濟吃了茶,徃前邊去了。繇是越發與這小夥兒日親日近。

一日,七月天氣,婦人早晨約下他:「你今日休徃那裡去,在房中等着,我徃你房裡,和你頑耍。」這敬濟答應了,不料那日被崔本邀了他,和幾箇朋友徃門外耍子。去了一日,吃的大醉來家,倒在床上就睡着了,不知天高地下。黃昏時分,金蓮驀地到他房中,見他挺在床上,推他推不醒,就知他在那裡吃了酒來。可霎作恠,不想婦人摸到他袖子裡,弔下一根金頭蓮瓣簪兒來,上面趿着兩溜字兒:「金勒馬嘶芳草地,玉樓人醉杏花天。」{\meipi{八回中便有此簪,只以為點綴之妙,孰知其伏冷脈至此,始知高文絕無穿鑿之跡。}}迎亮一看,認的是孟玉樓簪子:「怎生落在他袖中?想必他也和玉樓有些首尾。不然,他的簪子如何他袖着?恠道這短命,幾次在我面上無情無緒。我若不留幾箇字兒與他,只說我沒來。等我寫四句詩在壁上,使他知道。待我見了,慢慢追問他下落。」於是取筆在壁上寫了四句。詩曰:

獨步書齋睡未醒,空勞神女下巫雲。襄王自是無情緒,辜負朝朝暮暮情。

寫畢,婦人囘房去了。

卻說敬濟一覺酒醒起來,房中掌上燈,因想起今日婦人來相會,我卻醉了。囘頭見壁上寫了四句詩在壁上,墨蹟猶新,念了一遍,就知他來到,空囘去了。心中懊悔不已。「這咱已是起更時分,大姐、元宵兒都在後邊未出來,我若徃他那邊去,角門又關了。」走來木槿花下,搖花枝為號,不聽見裡面動靜,不免踩着太湖石扒過粉墻去。那婦人見他有酒,醉了挺覺,大恨歸房,悶悶在心,就渾衣上床𢱉睡。不料半夜他扒過墻來,見院內無人,想丫鬟都睡了,悄悄躡足潛蹤走到房門首,見門虛掩,就挨身進來。窓間月色照見床上婦人獨自朝裡𢱉着,低聲叫「可意人」,數聲不應,說道:「你休恠我,今日崔大哥衆朋友,邀了我徃門外五里原庄上射箭耍子了一日,來家就醉了。不知你到,有負你之約,恕罪恕罪。」那婦人也不理他。敬濟見他不理,慌了,一面跪在地下,說了一遍又重復一遍。{\meipi{金蓮從未受此軟款溫存,敬濟似為西門慶補遺。}}被婦人反手望臉上撾了一下,罵道:「賊牢拉負心短命,還不悄悄的,丫頭聽見!我知道你有了人,把我不放到心上。你今日端的那去來?」敬濟道:「我本被崔大哥拉了門外射箭去,灌醉了來,就睡着了,失誤你約,你休惱。我看見你留詩在壁上,就知惱了你。」婦人道:「恠搗鬼牢拉的,別要說嘴,與我禁聲!你搗的鬼如泥彈兒圓,我手內放不過。你今日便是崔本叫了你吃酒,醉了來家,你袖子裡這根簪子,卻是那裡的?」敬濟道:「是那日花園中拾的,今兩三日了。」婦人道:「你還㒲神搗鬼,是那花園裡拾的?你再拾一根來,我纔信你。{\meipi{歡會多矣,又疑惱酸醋一番,文情變幻炫人。}}這簪子是孟三兒那麻淫婦的頭上簪子,{\pangpi{便罵,妙。}}我認的千真萬真,上面還趿着他名字,你還哄我。嗔道前日我不在,他叫你房裡吃飯,原來你和他七箇八箇。我問你,還不肯認。你不和他兩箇有首尾,他的簪子緣何到你手裡?原來把我的事都透露與他,恠道他前日見了我笑,{\pangpi{寫疑心,令人絕倒。}}原來有你的話在裡頭。自今以後,你是你,我是我,『綠豆皮兒——請退了』。」敬濟聽了,急的賭神發咒,繼之以哭,{\pangpi{妙。}}道:「我敬濟若與他有一字絲麻皁線,靈的是東嶽城隍,活不到三十歲,生來碗大疔瘡,害三五年黃病,要湯不湯,要水不水。」{\meipi{好狠咒。}}那婦人終是不信,說道:「你這賊才料,說來的牙疼誓,虧你口內不害硶!」兩箇絮聒了一囘,見夜深了,不免解卸衣衫,挨身上床躺下。那婦人把身子扭過,倒背着他,使箇性兒不理他,由着他姐姐長、姐姐短,只是反手望臉上撾過去。{\meipi{當此情景,似苦而實樂,然不可為淺人道。}}唬的敬濟氣也不敢出一口兒來,乾霍亂了一夜。將天明,敬濟恐怕丫頭起身,依舊越墻而過,徃前邊廂房中去了。正是:

三光有影遣誰系?萬事無根只自生。

