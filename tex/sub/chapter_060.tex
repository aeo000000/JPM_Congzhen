\includepdf[pages={119,120},fitpaper=false]{tst.pdf}
\chapter*{第六十囘 李瓶兒病纏死孽 西門慶官作生涯}
\addcontentsline{toc}{chapter}{第六十囘 李瓶兒病纏死孽 西門慶官作生涯}
\markboth{{\titlename}卷之六}{第六十囘 李瓶兒病纏死孽 西門慶官作生涯}


詞曰:

\begin{myquote}
倦睡懨懨生怕起,如癡如醉如慵。半垂半捲舊簾櫳。眼穿芳草綠,淚襯落花紅。追憶當年魂夢斷,為雲為雨為風。悽悽樓上數歸鴻。悲淚三兩陣,哀緒萬千重。

\raggedleft{——右調《臨江仙》\rightquadmargin}
\end{myquote}

話說潘金蓮見孩子沒了,每日抖擻精神,百般稱快,指着丫頭罵道:「賊淫婦!我只說你日頭常晌午,卻怎的今日也有錯了的時節?你『斑鳩跌了蛋——也嘴答谷了』!『春櫈折了靠背兒——沒的椅了』!『王婆子賣了磨——推不的了』!『老鴇子死了粉頭——沒指望了』!卻怎的也和我一般!」{\meipi{官哥既死,怨妒俱可相忘,而猶喋喋不已,何哉?豈花子虛附之而逼其命耶!}}李瓶兒這邊屋裡分明聽見,不敢聲言,背地裡只是掉淚。着了這暗氣暗惱,又加之煩惱憂戚,漸漸精神恍亂,夢魂顛倒,每日茶飯都減少了。自從葬了官哥兒第二日,吳銀兒就家去了。老馮領了個十三歲的丫頭來,五兩銀子賣與孫雪娥房中使喚,改名翠兒,不在話下。

這李瓶兒一者思念孩兒,二者着了重氣,把舊病又發起來,照舊下邊經水淋漓不止。西門慶請任醫官來看,討將藥來吃下去,如水澆石一般,越吃越旺。那消半月之間,漸漸容顏頓減,肌膚消瘦,而精彩丰標無復昔時之態矣。正是:肌骨大都無一把,如何禁架許多愁!一日,九月初旬,天氣淒涼,金風漸漸。李瓶兒夜間獨宿房中,銀床枕冷,紗窓月浸,不覺思想孩兒,欷歔長嘆,恍恍然恰似有人彈的窓櫺響。李瓶兒呼喚丫鬢,都睡熟了不答,乃自下床來,倒靸弓鞋,翻披繡襖,開了房門。出戶視之,彷彿見花子虛抱着官哥兒叫他,新尋了房兒,同去居住。李瓶兒還捨不的西門慶,不肯去,雙手就抱那孩兒,被花子虛只一推,跌倒在地。撒手驚覺,卻是南柯一夢。嚇了一身冷汗,嗚嗚咽咽,只哭到天明。{\meipi{明知為子虛之報,而猶憐惜,不忍讀甚矣,情色之奪理也。}}正是:有情豈不等,着相自家迷。有詩為證:

\begin{myquote}
纖纖新月照銀屏,人在幽閨欲斷魂。\\益悔風流多不足,須知恩愛是愁根。
\end{myquote}

那時,來保南京貨船又到了,使了後生王顯上來取車稅銀兩。西門慶這裡寫書,差榮海拏一百兩銀子,又具羊酒金段禮物謝主事:「就說此貨過稅,還望青目一二。」家中收拾鋪面完備,又擇九月初四日開張,就是那日卸貨,連行李共裝二十大車。那日,親朋遞菓盒掛紅者約有三十多人,夏提刑也差人送禮花紅來。喬大戶叫了十二名吹打的樂工、雜耍撮弄。西門慶這裡,李銘、吳惠、鄭春三個小優兒彈唱。甘夥計與韓夥計都在櫃上發賣,一個看銀子,一個講說價錢,崔本專管收生活。西門慶穿大紅冠帶着,燒罷紙,{\meipi{市井氣,可笑。}}各親友遞菓盒把盞畢,後邊廳上安放十五張桌席,五菓五菜、三湯五割,從新遞酒上坐,鼓樂喧天。在坐者有喬大戶、吳大舅、吳二舅、花大舅、沈姨夫、韓姨夫、吳道官、倪秀才、溫葵軒、應伯爵、謝希大、常峙節,還有李智、黃四、傅自新等衆夥計主管並街坊隣舍,都坐滿了席面。三個小優兒在席前唱了一套《南呂•紅衲襖》「混元初生太極」。{\meipi{此題蓋指富貴功名,俱從財出。}}須臾,酒過五巡,食割三道,下邊樂工吹打彈唱,雜耍百戲過去,席上觥籌交錯。應伯爵、謝希大飛起大鐘來,盃來盞去。飲至日落時分,把衆人打發散了,西門慶只留下吳大舅、沈姨夫、韓姨夫、溫葵軒、應伯爵、謝希大,從新擺上桌席,留後坐。那日新開張,夥計攢帳,就賣了五百餘兩銀子。西門慶滿心歡喜,晚夕收了鋪面,把甘夥計、韓夥計、傅夥計、崔本、賁四連陳敬濟都邀來,到席上飲酒。吹打良久,把吹打樂工也打發去了,止留下三個小優兒在席前唱。應伯爵吃的已醉上來,走出前邊解手,叫過李銘問道:「那個紮包髻兒清俊的小優兒,是誰家的?」李銘道:「二爹原來不知道?」因說道:「他是鄭奉的兄弟鄭春。前日爹在他家吃酒,請了他姐姐愛月兒了。」伯爵道:「眞個?恠道前日上紙送殯都有他。」於是歸到酒席上,向西門慶道:「哥,你又恭喜,又擡了小舅子了。」西門慶笑道:「恠狗才,休要胡說。」一面叫過王經來:「斟與你應二爹一大盃酒。」伯爵向吳大舅說道:「老舅,你怎麼說?這鐘罰的我沒名。」西門慶道:「我罰你這狗才一個出位妄言。」伯爵低頭想了想兒,呵呵笑了,道:「不打緊處,等我吃,我吃死不了人。」又道:「我從來吃不得啞酒,你叫鄭春上來唱個兒我聽,我纔罷了。」當下,三個小優一齊上來彈唱。伯爵令李銘、吳惠下去:「不要你兩個。我只要鄭春單彈着箏兒,只唱個小小曲兒我下酒罷。」{\meipi{寫笑則有聲,寫想則有形,寫舉止語默則俱有心,何得文人刻畫至此。}}謝希大叫道:「鄭春你過來,依着你應二爹唱個罷。」西門慶道:「和花子講過:有一個曲兒吃一鍾酒。」叫玳安取了兩個大銀鍾放在應二面前。那鄭春款按銀箏,低低唱《清江引》道:{\meipi{即打起黃鶯兒之意。}}

\begin{myquote}
「一個姐兒十六七,見一對蝴蝶戲。香肩靠粉墻,春筍彈珠淚。喚梅香趕他去別處飛。」
\end{myquote}

鄭春唱了請酒,伯爵纔飲訖,玳安又連忙斟上。鄭春又唱:

\begin{myquote}
「轉過雕欄正見他,斜倚定荼䕷架;佯羞整鳳衩,不說昨宵話,笑吟吟掐將花片兒打。」{\meipi{寫私會,幽冷之極。}}
\end{myquote}

伯爵吃過,連忙推與謝希大,說道:「罷,我是成不的,成不的!這兩大鐘把我就打發了。」謝希大道:「傻花子,你吃不得推與我來,我是你家有𣬼的蠻子?」伯爵道:「傻花子,我明日就做了堂上官兒,少不的是你替。」西門慶道:「你這狗才,到明日只好做個韶武。」伯爵笑道:「傻孩兒,我做了韶武,把堂上讓與你就是了。」西門慶笑令玳安兒:「拏磕瓜來打這賊花子!」謝希大悄悄向他頭上打了一個響瓜兒,說道:「你這花子,溫老先生在這裡,你口裡只恁胡說。」伯爵道:「溫老先兒他斯文人,不管這閒事。」溫秀才道:「二公與我這東君老先生,原來這等厚。酒席中間,誠然不如此也不樂。悅在心,樂主發散在外,自不覺手之舞之,足之蹈之如此。」{\meipi{滿堂醉人荒言穢語中,忽點出一段酸腐之談,錯織如錦。語云:嬉笑怒罵皆文章。閱此方知其言之妙。}}沈姨夫向西門慶說:「姨夫,不是這等。請大舅上席,還行個令兒,或擲骰,或猜枚,或看牌,不拘詩詞歌賦、頂眞續麻、急口令,說不過來吃酒。這個庶幾均勻,彼此不亂。」西門慶道:「姨夫說的是。」先斟了一盃,與吳大舅起令。吳大舅拏起骰盆兒來說道:「列位,我行一令:順着數去,遇點要個花名,花名下要頂眞,不拘詩詞歌賦說一句。說不來,罰一大盃。我就是一起:

\begin{myquote}
一擲一點紅,紅梅花對白梅花。」
\end{myquote}

吳大舅擲了個二,多一盃。飲過酒,該沈姨夫接擲。沈姨夫說道:

\begin{myquote}
「二擲並頭蓮,蓮漪戲彩鴛。」
\end{myquote}

沈姨夫也擲了個二,飲過兩盃,就過盆與韓姨夫行令。韓姨夫說道:

\begin{myquote}
「三擲三春李,李下不整冠。」
\end{myquote}

韓姨夫擲完,吃了酒,送與溫秀才。秀才道:「我學生奉令了:

\begin{myquote}
四擲狀元紅,紅紫不以為褻服。」{\pangpi{到底帶酸。}}
\end{myquote}

溫秀才只遇了一盃酒,吃過,該應伯爵行令。伯爵道:「我在下一個字也不識,不會頂眞,只說個急口令兒罷:

\begin{myquote}
一個急急脚脚的老小,左手拏着一個黃豆巴斗,右手拏着一條綿花叉口,望前只管跑走。一個黃白花狗,咬着那綿花叉口,那急急脚脚的老小,放下那左手提的那黃豆巴斗,走向前去打那黃白花狗。不知手鬪過那狗,狗鬪過那手。」
\end{myquote}

西門慶笑罵道:「你這賊謅斷腸子的天殺的,誰家一個手去逗狗來?一口不被那狗咬了?」伯爵道:「誰叫他不拏個棍兒來!我如今抄化子不見了柺棒兒,受狗的氣了。」{\pangpi{又自露破膁,妙。}}謝希大道:「大官人,你看花子自家倒了架,說他是花子。」西門慶道:「該罰他一鍾,不成個令。謝子純,你行罷!」謝希大道:「我也說一個,比他更妙:

\begin{myquote}
墻上一片破瓦,墻下一匹騾馬。落下破瓦,打着騾馬。不知是那破瓦打傷騾馬,不知是那騾馬踏碎了破瓦。」
\end{myquote}

伯爵道:「你笑話我的令不好,你這破瓦倒好?你家娘子兒劉大姐就是個騾馬,我就是個破瓦,俺兩個破磨對瘸驢。」謝希大道:「你家那杜蠻婆老淫婦,撒把黑豆只好喂豬哄狗,也不要他。」兩個人鬪了囘嘴,每人斟了一鍾,該韓夥計擲。韓道國道:「老爹在上,小人怎敢佔先?」西門慶道:「順着來,不要遜了。」於是韓道國說道:

\begin{myquote}
「五擲臘梅花,花裡遇神仙。」
\end{myquote}

擲畢,該西門慶擲。西門慶道:「我要擲個六:

\begin{myquote}
六擲滿天星,星辰冷落碧潭水。」
\end{myquote}

果然擲出個六來。應伯爵看見,說道:「哥今年上冬,管情加官進祿,主有慶事。」{\meipi{歸到奉承上,方不失旨。}}於是斟了一大盃酒與西門慶。一面李銘等三個上來彈唱,頑耍至更闌方散。西門慶打發小優兒出門,看收了家伙,派定韓道國、甘夥計、崔本、來保四人輪流上宿,分付仔細門戶,就過那邊去了。一宿晚景不題。

次日,應伯爵領了李智、黃四來交銀子,說:「此遭只關了一千四百五六十兩銀子,不勾還人,只挪了三百五十兩銀子與老爹。等下遭關出來再找完,不敢遲了。」伯爵在旁又替他說了兩句美言。西門慶教陳敬濟來,把銀子兌收明白,打發去了。銀子還擺在桌上,西門慶因問伯爵道:「常二哥說他房子尋下了,前後四間,只要三十五兩銀子。他來對我說,正値小兒病重,我心裡亂,就打發他去了。不知他對你說來不曾?」伯爵道:「他對我說來,我說,你去的不是了,他乃郎不好,他自亂亂的,有甚麼心緒和你說話?你且休囘那房主兒,等我見哥,替你題就是了。」西門慶道:「也罷,你吃了飯,拏一封五十兩銀子,今日是個好日子,替他把房子成了來罷。剩下的,叫常二哥門面開個小鋪兒,月間撰幾錢銀子兒,就勾他兩口兒盤攪了。」{\meipi{西門慶一段脫手相贈,全無吝色處,亦古今所難。}}伯爵道:「此是哥下顧他了。」不一時,放桌兒擺上飯來,西門慶陪他吃了飯,道:「我不留你。你拏了這銀子去,替他幹幹這勾當去罷。」伯爵道:「你這裡還教個大官和我去。」西門慶道:「沒的扯淡,你袖了去就是了。」伯爵道:「不是這等說,今日我還有小事。實和哥說,家表弟杜三哥生日,早晨我送了些禮兒去,他使小厮來請我後晌坐坐。我不得來囘你話,教個大官兒跟了去,成了房子,好教他來囘你話的。」西門慶道:「若是恁說,叫王經跟你去罷。」一面叫王經跟伯爵來到了常家。

常峙節正在家,見伯爵至,讓進裡面坐。伯爵拏出銀子來與常峙節看,說:「大官人如此如此,教我同你今日成房子去,我又不得閑,杜三哥請我吃酒。我如今了畢你的事,我方纔得去。」常峙節連忙叫渾家快看茶來,說道:「哥的盛情,誰肯!」一面吃茶畢,叫了房中人來,同到新市街,兌與賣主銀子,寫立房契。伯爵分付與王經,歸家囘西門慶話。剩的銀子,叫與常峙節收了。他便與常峙節作別,徃杜家吃酒去了。西門慶看了文契,還使王經送與常二收了,不在話下。正是:

\begin{myquote}
求人須求大丈夫,濟人須濟急時無。\\一切萬般皆下品,誰知恩德是良圖。
\end{myquote}

