\includepdf[pages={177,178},fitpaper=false]{tst.pdf}
\chapter*{第八十九囘 清明節寡婦上新墳 永福寺夫人逢故主}
\addcontentsline{toc}{chapter}{第八十九囘 清明節寡婦上新墳 永福寺夫人逢故主}
\markboth{\titlename}{第八十九囘 清明節寡婦上新墳 永福寺夫人逢故主}


詞曰:

佳人命薄,嘆絕代紅粉,幾多黃土。豈是老天渾不管,好惡隨人自取?既賦嬌容,又全慧性,卻遣輕歸去。不平如此,問天天更不語。可惜國色天香,隨時飛謝,埋沒今如許。借問繁華何處在?多少樓臺歌舞,紫陌春遊,綠窓晚綉,姊妹嬌眉嫵。人生失意,從來無問今古。

——右調《翠樓吟》

話說月娘次日備了一張桌,並冥紙尺頭之類,大姐身穿孝服,坐轎子,先叫薛嫂押祭禮,到陳宅來。只見陳敬濟正在門首站立,便問:「是那裡的?」薛嫂道了萬福,說:「姐夫,你休推不知。你丈母家來與你爹燒紙,送大姐來了。」{\meipi{月娘禮短,即薛嫂說來亦覺口澁。}}敬濟便道:「我𩫻𩫵㒲的纔是丈母!{\pangpi{妙語。}}『正月十六貼門神——來遲了半個月』。人也入了土,纔來上祭。」薛嫂道:「好姐夫,你丈母說,寡婦家沒脚蠏,不知親家靈柩來家,遲了一步,休恠。」正說着,只見大姐轎子落在門首。敬濟問:「是誰?」薛嫂道:「再有誰?你丈母心內不好,一者送大姐來家,二者敬與你爹燒紙。」敬濟罵道:「趁早把淫婦擡囘去!好的死了萬萬千千,我要他做甚麼?」{\pangpi{忽插入金蓮,妙不容言。}}薛嫂道:「常言道:嫁夫着主。怎的說這個話?」敬濟道:「我不要這淫婦了,還不與我走?」那擡轎的只顧站立不動,被敬濟向前踢了兩脚,罵道:「還不與我擡了去,我把你花子脚砸折了,把淫婦髩毛都蒿淨了!」那擡轎子的見他踢起來,只得擡轎子徃家中走不迭。比及薛嫂叫出他娘張氏來,轎子已擡去了。

薛嫂兒沒奈何,教張氏收下祭禮,走來囘覆吳月娘。把吳月娘氣的一個發昏,說道:「恁個沒天理的短命囚根子!當初你家為了官事,搬來丈人家居住,養活了這幾年,今日反恩將仇報起來了。只恨死鬼當初攬的好貨在家裡,弄出事來,到今日教我做臭老鼠,教他這等放屁辣臊。」{\meipi{養活女婿幾年,便以為恩;收女婿許多東西,便不題。這燒香好佛人大都如此。}}對着大姐說:「孩兒,你是眼見的,丈人、丈母那些兒虧了他來?你活是他家人,死是他家鬼,我家裡也難以留你。你明日還去,休要怕他,料他挾你不到井裡。他好膽子,恆是殺不了人,{\meipi{月娘數語,而拚送大姐與敬濟打罵矣。}}難道世間沒王法管他也怎的!」當晚不題。

到次日,一頂轎子,教玳安兒跟隨着,把大姐又送到陳敬濟家來。不想陳敬濟不在家,徃墳上替他父親添土疊山子去了。張氏知禮,把大姐留下,對着玳安說:「大官到家多多上覆親家,多謝祭禮,休要和他一般見識。他昨日已有酒了,故此這般。等我慢慢說他。」一面管待玳安兒,安撫來家。

至晚,陳敬濟墳上囘來,看見了大姐,就行踢打,罵道:「淫婦,你又來做甚麼?還說我在你家雌飯吃,你家收着俺許多箱籠,因起這大產業,不道的白養活了女婿!好的死了萬千,我要你這淫婦做甚?」大姐亦罵:「沒廉恥的囚根子!沒天理的囚根子!淫婦出去吃人殺了,沒的禁拏我煞氣。」被敬濟扯過頭髮,盡力打了幾拳頭。他娘走來解勸,把他娘推了一交。他娘叫罵哭喊,說:「好囚根子,紅了眼,把我也不認的了!」到晚上,一頂轎子,把大姐又送將來,分付道:「不討將寄放粧奩箱籠來家,我把你這淫婦活殺了。」{\meipi{既送大姐來,則粧奩箱籠應該還他,為何留下?自是月娘理短。}}這大姐害怕,躲在家中居住,再不敢去了。這正是:

誰知好事多更變,一念翻成怨恨媒。

這裡不去。不題。

且說一日,三月清明佳節。吳月娘備辦香燭、金錢冥紙、三牲祭物,擡了兩大食盒,要徃城外墳上與西門慶上新墳祭掃。留下孫雪娥和大姐、衆丫頭看家。帶了孟玉樓和小玉,並奶子如意兒抱着孝哥兒,都坐轎子徃墳上去。又請了吳大舅和大妗子二人同去。出了城門,只見那郊原野曠,景物芳菲,花紅桺綠,仕女遊人不斷。一年四季,無過春天,最好景致。{\meipi{一篇絕少遊春賦。}}日謂之麗日,風謂之和風,吹桺眼,綻花心,拂香塵。天色暖,謂之暄。天色寒,謂之料峭。騎的馬,謂之寶馬。坐的轎,謂之香車。行的路,謂之芳徑。地下飛的塵,謂之香塵。千花發蕊,萬草生芽,謂之春信。韶光淡蕩,淑景融和。小桃深粧臉妖嬈,嫩桺嬝宮腰細膩。百轉黃鸝,驚囘午夢;數聲紫燕,說破春愁。日舒長暖澡鵝黃,水渺茫浮香鴨綠。隔水不知誰院落,鞦韆高掛綠楊烟。端的春景果然是好。有詩為證:

清明何處不生烟,郊外微風掛紙錢。人笑人歌芳草地,乍晴乍雨杏花天。海棠枝上綿鶯語,楊桺堤邊醉客眠。紅粉佳人爭畫板,彩繩搖拽學飛仙。

吳月娘等轎子到五里原墳上,玳安押着食盒,先到廚下生起火來,{\pangpi{冷落。}}廚役落作整理不題。月娘與玉樓、小玉、奶子如意兒抱着孝哥兒,到於庄院客坐內坐下吃茶,等着吳大妗子,不見到。玳安向西門慶墳上祭臺上,擺設桌面三牲,羹飯祭物,列下紙錢,只等吳大妗子。原來大妗子顧不出轎子來,約已牌時分,纔同吳大舅顧了兩個驢兒騎將來。{\meipi{雖寫蕭條情景,而誤入寺摧起身之脈,俱淡淡結此,何等幽細。}}月娘便說:「大妗子顧不出轎子來,這驢兒怎的騎?」一面吃了茶,換了衣服,同來西門慶墳上祭掃。那月娘手拈着五根香,自拏一根,遞一根與玉樓,又遞一根與奶子如意兒替孝哥上,那兩根遞與吳大舅、大妗子。月娘插在香爐內,深深拜下去,說道:「我的哥哥,你活時為人,死後為神。今日三月清明佳節,你的孝妻吳氏三姐、孟三姐和你週歲孩童孝哥兒,敬來與你墳前燒一陌錢紙。你保佑他長命百歲,替你做墳前拜掃之人。我的哥哥,我和你做夫妻一場,想起你那模樣兒並說的話來,是好傷感人也。」拜畢,掩面痛哭。玉樓向前插上香,也深深拜下,同月娘大哭了一場。玉樓上了香,奶子如意兒抱着哥兒也跪下上香,磕了頭。吳大舅、大妗子都炷了香。行畢禮數,玳安把錢紙燒了。讓到庄上捲棚內,放桌席擺飯,收拾飲酒。月娘讓吳大舅、大妗子上坐。月娘與玉樓下陪。小玉和奶子如意兒,同大妗子家使的老姐蘭花,也在兩邊打橫列坐,把酒來斟。按下這裡吃酒不題。

卻表那日周守備府裡也上墳。先是春梅隔夜和守備睡,假推做夢,睡夢中哭醒了。{\meipi{前真哭,此則假哭矣。世情之假徃徃從真來,故難測識。}}守備慌的問:「你怎的哭?」春梅便說:「我夢見我娘向我哭泣,說養我一場,怎地不與他清明寒食燒紙,因此哭醒了。」守備道:「這個也是養女一場,你的一點孝心。不知你娘墳在何處?」春梅道:「在南門外永福寺後面便是。」守備說:「不打緊,永福寺是我家香火院,明日咱家上墳,你叫伴當擡些祭物,徃那裡與你娘燒分紙錢,也是好處。」至次日,守備令家人收拾食盒酒菓祭品,徑徃城南祖墳上。那裡有大庄院、廳堂、花園、享堂、祭臺。大奶奶、孫二娘並春梅,都坐四人轎,排軍喝路,上墳耍子去了。

卻說吳月娘和大舅、大妗子吃了囘酒,恐怕晚來,分付玳安、來安兒收拾了食盒酒菓,先徃杏花村酒樓下,揀高阜去處,人烟熱鬧,那裡設放桌席等候。又見大妗子沒轎子,都把轎子擡着,後面跟隨不坐,領定一簇男女,吳大舅牽着驢兒,壓後同行,踏青遊玩。三月桃花店,五里杏花村,只見那隨路上墳遊玩的王孫士女,花紅桺綠,鬧鬧喧喧,不知有多少。正走之間,也是合當有事,遠遠望見綠槐影裡,一座庵院,蓋造得十分齊整。但見:

山門高聳,梵宇清幽。當頭勑額字分明,兩下金剛形勢猛。五間大殿,龍鱗瓦砌碧成行;兩下僧房,龜背磨磚花嵌縫。前殿塑風調雨順,後殿供過去未來。鐘鼓樓森立,藏經閣巍峨。旗竿高峻接青雲,寶塔依稀侵碧漢。木魚橫掛,雲板高懸。佛前燈燭瑩煌,爐內香烟繚遶。幢旗不斷,觀音殿接祖師堂;寶蓋相連,鬼母位通羅漢殿。時時護法諸天降,歲歲降魔尊者來。

吳月娘便問:「這座寺叫做甚麼寺?」吳大舅便說:「此是周秀老爺香火院,名喚永福禪林。前日姐夫在日,曾捨幾拾兩銀子在這寺中,重修佛殿,方是這般新鮮。」月娘向大妗子說:「咱也到這寺裡看一看。」於是領着一簇男女,進入寺中來。不一時,小沙彌看見,報與長老知道:「見有許多男女……」便出方丈來迎請,見了吳大舅、吳月娘,向前合掌道了問訊,連忙喚小和尚開了佛殿:「請施主菩薩隨喜遊玩,小僧看茶。」那小沙彌開了殿門,領月娘一簇男女,前後兩廊叅拜觀看了一囘,然後到長老方丈。長老連忙點上茶來,吳大舅請問長老道號,那和尚答說:「小僧法名道堅。這寺是恩主帥府周爺香火院,小僧忝在本寺長老,廊下管百十衆僧行,後邊禪堂中還有許多雲遊僧行,常時坐禪,{\pangpi{映前胡僧。}}與四方檀越答報功德。」一面方丈中擺齋,讓月娘:「衆菩薩請坐。」月娘道:「不當打攪長老寶剎。」一面拏出五錢銀子,教大舅遞與長老,佛前請香燒。那和尚打問訊謝了,說道:「小僧無甚管待,施主菩薩稍坐,略備一茶而已,何勞費心賜與布施。」不一時,小和尚放下桌兒,拏上素菜齋食餅饊上來。那和尚在旁陪坐,纔舉筯兒讓衆人吃時,忽見兩個青衣漢子,走的氣喘吁吁,暴雷也一般報與長老,說道:「長老還不快出來迎接,府中小奶奶來祭祀來了!」慌的長老披袈裟,戴僧帽不迭,分付小沙彌連忙收了家活,「請列位菩薩且在小房避避,打發小夫人燒了紙,祭畢去了,再款坐一會不遲。」吳大舅告辭,和尚死活留住,又不肯放。

那和尚慌的鳴起鐘鼓來,出山門迎接,遠遠在馬道口上等候。只見一族青衣人,圍着一乘大轎,從東雲飛般來,轎伕走的個個汗流滿面,衣衫皆濕。那長老躬身合掌說道:「小僧不知小奶奶前來,理合遠接,接待遲了,萬勿見罪。」這春梅在轎內答道:「起動長老。」那手下伴當,又早向寺後金蓮墳上,忙將祭桌紙錢來擺設下。春梅轎子來到,也不到寺,徑入寺後白楊樹下金蓮墳前下轎。兩邊青衣人伺候。這春梅不慌不忙,來到墳前,擺了香,拜了四拜,說道:「我的娘,今日龐大姐特來與你燒陌紙錢,你好處昇天,苦處用錢。早知你死在仇人之手,奴隨問怎的也娶來府中,和奴做一處。還是奴耽誤了你,悔已是遲了。」說畢,令左右把錢紙燒了。這春梅向前放聲大哭不已。吳月娘在僧房內,只知有宅內小夫人來到,長老出山門迎接,又不見進來。問小和尚,小和尚說:「這寺後有小奶奶的一個姐姐,新近葬下,今日清明節,特來祭掃燒紙。」孟玉樓便道:「怕不就是春梅來了?也不見的。」月娘道:「他那得個姐來死了葬在此處?」又問小和尚:「這府裡小夫人姓甚麼?」小和尚道:「姓龐,前日與了長老四五兩經錢,教替他姐姐念經,薦拔生天。」玉樓道:「我聽見他爹說春梅娘家姓龐,叫龐大姐,莫不是他?」正說話,只見長老先來,分付小沙彌:「好看好茶。」不一時,轎子擡進方丈二門裡纔下。月娘和玉樓衆人打僧房簾內望外張看,怎樣的小夫人。定睛仔細看時,卻是春梅。但比昔時出落得長大身材,面如滿月,打扮的粉粧玉琢,頭上戴着冠兒,珠翠堆滿,鳳釵半卸,上穿大紅粧花襖,下着翠蘭縷金寬斕裙子,帶着玎璫禁步,比昔不同許多。但見:

寶髻巍峨,鳳釵半卸。胡珠環耳邊低掛,金挑鳳𩬆後雙拖。紅繡襖偏襯玉香肌,翠紋裙下映金蓮小。行動處,胸前搖響玉玎璫;坐下時,一陣麝蘭香噴鼻。膩粉粧成脖頸,花鈿巧帖眉尖。舉止驚人,貌比幽花殊麗;姿容閑雅,性如蘭蕙溫柔。若非綺閣生成,定是蘭房長就。{\pangpi{二語微帶春秋。}}儼若紫府瓊姬離碧漢,宛如蕊宮仙子下塵寰。

那長老上面獨獨安放一張公座椅兒,讓春梅坐下。長老叅見已畢,小沙彌拏上茶來。長老遞茶上去,說道:「今日小僧不知小奶奶來這裡祭祀,有失迎接,萬望恕罪。」春梅道:「外日多有起動長老誦經追薦。」那和尚說:「小僧豈敢。有甚殷勤補報恩主?多蒙小奶奶賜了許多經錢襯施。小僧請了八衆禪僧,整做道場,看經禮懺一日。晚夕,又與他老人家裝些廂庫焚化。道場圓滿,纔打發兩位管家進城,宅裡囘小奶奶話。」春梅吃了茶,小和尚接下鍾盞來。長老只顧在旁一遞一句與春梅說話,把吳月娘衆人攔阻在內,又不好出來的。月娘恐怕天晚,使小和尚請下長老來,要起身。那長老又不肯放,走來方丈稟春梅說:「小僧有件事稟知小奶奶。」春梅道:「長老有話,但說無妨。」長老道:「適間有幾位遊玩娘子,在寺中隨喜,不知小奶奶來。如今他要囘去,未知小奶奶尊意如何。」春梅道:「長老何不請來相見。」那長老慌的來請。吳月娘又不肯出來,只說:「長老不見罷。天色晚了,俺們告辭去了。」{\meipi{月娘為相輕薄春梅,為申二姐罵春梅,臨賣又不與一件衣物,今日自無顏見春梅。}}長老見收了他布施,又沒管待,又意不過,只顧再三催促。吳月娘與孟玉樓、吳大妗子推阻不過,只得出來,春梅一見便道:「原來是二位娘與大妗子。」於是先讓大妗子轉上,花枝招展磕下頭去。慌的大妗子還禮不迭,說道:「姐姐,今非昔比,折殺老身。」春梅道:「好大妗子,如何說這話,奴不是那樣人。{\meipi{春梅曰「奴不是那樣人」,則月娘是那樣人可知矣。}}尊卑上下,自然之禮。」拜了大妗子,然後向月娘、孟玉樓插燭也似磕頭。月娘、玉樓亦欲還禮,春梅那裡肯,扶起,磕下四個頭,說:「不知是娘們在這裡,早知也請出來相見。」月娘道:「姐姐,你自從出了家門在府中,一向奴多缺禮,沒曾看你,你休恠。」{\pangpi{懷慚之語。}}春梅道:「好奶奶,奴那裡出身,豈敢說恠。」{\meipi{此時人刮目春梅矣,而春梅毫不改常作態,大是可兒。}}因見奶子如意兒抱着孝哥兒,說道:「哥哥也長的恁大了。」月娘說:「你和小玉過來,與姐姐磕過頭兒。」那如意兒和小玉二人笑嘻嘻過來,亦與春梅都平磕了頭。月娘道:「姐姐,你受他兩個一禮兒。」春梅向頭上拔下一對金頭銀簪兒來,插在孝哥兒帽兒上。月娘說:「多謝姐姐簪兒,還不與姐姐唱個喏兒。」如意兒抱着哥兒,真個與春梅唱個喏,把月娘喜歡的要不得。玉樓道:「姐姐,你今日不到寺中,咱娘兒們怎得遇在一處相見。」春梅道:「便是因俺娘他老人家新埋葬在這寺後,奴在他手裡一場,他又無親無故,奴不記掛着替他燒張紙兒,怎生過得去。」月娘道:「我記的你娘沒了好幾年,不知葬在這裡。」{\pangpi{月娘亦太老實。}}孟玉樓道:「大娘還不知龐大姐說話,說的是潘六姐死了。多虧姐姐,如今把他埋在這裡。」月娘聽了,就不言語了。吳大妗子道:「誰似姐姐這等有恩,不肯忘舊,還葬埋了。{\pangpi{大妗子轉乖。}}你逢節令題念他,來替他燒錢化紙。」春梅道:「好奶奶,想着他怎生擡舉我來!今日他死的苦,這般拋露丟下,怎不埋葬他?」{\meipi{語語知恩報恩,自令結怨人內愧。}}說畢,長老教小和尚放桌兒,擺齋上來。兩張大八仙桌子,蒸酥點心,各樣素饌菜蔬,堆滿春臺,絕細春芽雀舌甜水好茶。衆人吃了,收下家活去。吳大舅自有僧房管待,不在話下。

孟玉樓起身,心裡要徃金蓮墳上看看,替他燒張紙,也是姊妹一場。見月娘不動身,{\meipi{金蓮自坐淫耳,未嘗傷及月娘也,月娘何絕之深。}}拏出五分銀子,教小沙彌買紙去。長老道:「娘子不消買去,我這裡有金銀紙,拏幾分燒去。」玉樓把銀子遞與長老,使小沙彌領到後邊白楊樹下金蓮墳上,見三尺墳堆,一堆黃土,數桺青蒿。{\meipi{到此方寫景,隱冷之極。}}上了根香,把紙錢點着,拜了一拜,說道:「六姐,不知你埋在這裡。今日孟三姐誤到寺中,與你燒陌錢紙,你好處昇天,苦處用錢。」一面放聲大哭。那奶子如意兒見玉樓徃後邊,也抱了孝哥兒來看一看。月娘在方丈內和春梅說話,教奶子休抱了孩子去,只怕唬了他。如意兒道:「奶奶,不妨事,我知道。」徑抱到墳上,看玉樓燒紙哭罷囘來。春梅和月娘勻了臉,換了衣裳,分付小伴當將食盒開啟,將各樣細菓甜食,餚品點心攢盒,擺下兩桌子,布甑內篩上酒來,銀鍾牙筯,請大妗子、月娘、玉樓上坐,他便主位相陪。奶子、小玉,都在兩邊打橫。吳大舅另放一張桌子在僧房內。正飲酒中間,忽見兩個青衣伴當走來,跪下稟道:「老爺在新庄,差小的來請小奶奶看雜耍調百戲的。大奶奶、二奶奶都去了,請奶奶快去哩。」

這春梅不慌不忙,{\meipi{連用不慌不忙,轉似宜慌忙者,春梅婢作夫人,到底不饒。}}說:「你囘去,知道了。」那二人應諾下來,又不敢去,在下邊等候。大妗子、月娘便要起身,說:「姐姐,不可打攪。天色晚了,你也有事,俺們去罷。」那春梅那裡肯放,只顧令左右將大鐘來勸道:「咱娘兒們會少離多,彼此都見長着,休要斷了這門親路。奴也沒親沒故,到明日娘的好日子,奴徃家裡走走去。」月娘道:「我的姐姐,說一聲兒就勾了,怎敢起動你?容一日,奴去看姐姐去。」{\meipi{月娘前何倨而後何恭?人情乎?勢利乎?君子乎?小人乎?思之可涕。}}飲過一盃,月娘說:「我酒勾了,你大妗子沒轎子,十分晚了,不好行的。」春梅道:「大妗子沒轎子,我這裡有跟隨小馬兒,撥一匹與妗子騎,關了家去。」大妗子再三不肯,辭了,方一面收拾起身。春梅叫過長老來,令小伴當拏出一疋大布、五錢銀子與長老。長老拜謝了,送出山門。春梅與月娘拜別,看着月娘、玉樓衆人上了轎子,他也坐轎子,兩下分路,一簇人跟隨喝道,徃新庄上去了。正是:

樹葉還有相逢處,豈可人無得運時。

