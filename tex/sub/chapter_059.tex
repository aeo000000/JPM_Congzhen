\includepdf[pages={117,118},fitpaper=false]{tst.pdf}
\chapter*{第五十九囘 西門慶露陽驚愛月 李瓶兒睹物哭官哥}
\addcontentsline{toc}{chapter}{第五十九囘 西門慶露陽驚愛月 李瓶兒睹物哭官哥}
\markboth{{\titlename}卷之六}{第五十九囘 西門慶露陽驚愛月 李瓶兒睹物哭官哥}


詩曰:

\begin{myquote} 
楓葉初丹槲葉黃,河陽愁髩恰新霜。\\鬼門徒憶空囘首,泉路憑誰說斷腸?\\路杳雲迷愁漠漠,珠沉玉殞事茫茫。\\惟有淚珠能結雨,盡傾東海恨無疆。
\end{myquote} 

話說孟玉樓和潘金蓮,在門首打發磨鏡叟去了。忽見從東一人,帶着大帽、眼紗,騎着騾子,走得甚急,逕到門首下來,慌的兩個婦人徃後走不迭。落後揭開眼紗,卻是韓夥計來家了。平安忙問道:「貨車到了不曾?」韓道國道:「貨車進城了,稟問老爹卸在那裡?」平安道:「爹不在家,徃周爺府裡吃酒去了,教卸在對門樓上哩。你老人家請進裡邊去。」不一時,陳敬濟出來,陪韓道國入後邊見了月娘,出來廳上,拂去塵土,把行李搭褳教王經送到家去。月娘一面打發出飯來與他吃了。不一時,貨車纔到。敬濟拏鑰匙開了那邊樓上門,就有卸車的小脚子領籌搬運,一箱箱都堆卸在樓上。十大車段貨,直卸到掌燈時分。崔本也來幫扶。完畢,查數鎖門,貼上封皮,打發小脚錢出門。早有玳安徃守備府報西門慶去了。西門慶聽見家中卸貨,吃了幾盃酒,約掌燈以後就來家。韓夥計等着見了,在廳上坐的,悉把前後徃囘事說了一遍。西門慶因問:「錢老爹書下了,也見些分上不曾?」韓道國道:「全是錢老爹這封書,十車貨少使了許多稅錢。小人把段箱,兩箱並一箱,三停只報了兩停,都當茶葉、馬牙香櫃上稅過來了。通共十大車貨,只納了三十兩五錢鈔銀子。老爹接了報單,也沒差巡攔下來查點,就把車喝過來了。」西門慶聽言,滿心歡喜,{\meipi{此一喜,要知其不獨為銀子便益。}}因說:「到明日,少不的重重買一分禮謝他。」於是分付陳敬濟陪韓夥計、崔大哥坐,後邊拏菜出來,留吃了一囘酒,方纔各散囘家。

王六兒聽見韓道國來了,分付丫頭春香、錦兒,伺候下好茶好飯。等的晚上,韓道國到家,拜了家堂,脫了衣裳,淨了面目,夫妻二人各訴離情一遍。韓道國悉把買賣得意一節告訴老婆,老婆又見搭褳內沉沉重重許多銀兩,因問他,替己又帶了一二百兩貨物酒米,卸在門外店裡,慢慢發賣了銀子來家。老婆滿心歡喜道:「我聽見王經說,又尋了個甘夥計做賣手,咱每和崔大哥與他同分利錢使,這個又好了。到出月開鋪了。」韓道國道:「這裡使着了人做賣手,南邊還少個人立庄置貨,老爹已定還裁派我去。」老婆道:「你看貨才料,自古能者多勞。你不會做買賣那老爹托你麼!常言『不將辛苦意,難得世間財』。你外邊走上三年,你若懶得去,等我對老爹說了,教姓甘的和保官兒打外,你便在家賣貨就是了。」韓道國道:「外邊走熟了,也罷了。」老婆道:「可又來,你先生迷了路,在家也是閑!」說畢,擺上酒來,夫婦二人飲了幾盃闊別之酒,收拾就寢。是夜歡娛無度,不必細說。次日卻是八月初一日,韓道國早到房子內,同崔本、甘夥計看着收拾裝修土庫,不在話下。

卻說西門慶見貨物卸了,家中無事,忽然心中想起,要徃鄭愛月兒家去。暗暗使玳安兒送了三兩銀子、一套紗衣服與他。鄭家鴇子聽見西門老爹來請他家姐兒,如天上落下來的一般,連忙收下禮物,沒口子向玳安道:「你多頂上老爹,就說他姐兒兩個都在家裡伺候老爹,請老爹早些兒下降。」玳安走來家中書房內,囘了西門慶話。西門慶約午後時分,分付玳安收拾着涼轎,頭上戴着披巾,身上穿青緯羅暗補子直身,粉底皁靴,先走在房子看了一囘裝修土庫,然後起身,坐上涼轎,放下斑竹簾來,琴童、玳安跟隨,留王經在家,止叫春鴻背着直袋,逕徃院中鄭愛月兒家。正是:

\begin{myquote} 
天仙機上整香羅,入手先拖雪一窩。\\不獨桃源能問渡,卻來月窟伴嫦娥。
\end{myquote} 

卻說鄭愛香兒打扮的粉面油頭,見西門慶到,笑吟吟在半門裡首迎接進去。到於明間客位,道了萬福。西門慶坐下,就分付小厮琴童:「把轎囘了家去,晚夕騎馬來接。」琴童跟轎家去,止留玳安和春鴻兩個伺候。少頃,鴇子出來拜見,說道「外日姐兒在宅內多有打攪,老爹來這裡,自恁走走罷了,如何又賜將禮來?又多謝與姐兒的衣服。」西門慶道:「我那日叫他,怎的不去?只認王皇親家了!」鴇子道:「俺每如今還恠董嬌兒和李桂兒。{\meipi{開口只恠別人,是鴇兒口角。}}不知是老爹生日叫唱,他每都有了禮,只俺們姐兒沒有。若早知時,決不答應王皇親家唱,先徃老爹宅裡去了。落後,老爹那裡又差了人來,慌的老身背着王家人,連忙攛掇姐兒打後門上轎去了。」西門慶道:「先日我在他夏老爹家酒席上,就定下他了。他若那日不去,我不消說的就惱了。怎的他那日不言不語,不做喜歡,{\meipi{此鄭月兒深處,西門慶淺人,所以不知。}}端的是怎麼說?」鴇子道:「小行貨子家,自從梳弄了,那裡好生出去供唱去!到老爹宅內,見人多,不知唬的怎樣的。他從小是恁不出語,嬌養慣了。{\meipi{語語洗髮鄭月兒嬌癡之性。}}你看,甚時候纔起來!老身該催促了幾遍,說老爹今日來,你早些起來收拾了罷。他不依,還睡到這咱晚。」

不一時,丫鬟拏茶上來,鄭愛香兒向前遞了茶吃了。鴇子道:「請老爹到後邊坐罷。」鄭愛香兒就讓西門慶進入鄭愛月兒的房外明間內坐下,西門慶看見上面楷書「愛月軒」三字。坐了半日,忽聽簾櫳響處,鄭愛月兒出來,不戴鬏髻,頭上挽着一窩絲杭州纘,梳的黑鬖鬖光油油的烏雲,雲𩬆堆鴉,猶若輕烟密霧。上着白藕絲對衿仙裳,下穿紫綃翠紋裙,脚下露紅鴛鳳嘴鞋,前搖寶玉玲瓏,越顯那芙蓉粉面。正是:

\begin{myquote} 
若非道子觀音畫,定然延壽美人圖。
\end{myquote} 

愛月兒走到下面,望上不端不正與西門慶道了萬福,就用灑金扇兒掩着粉臉坐在旁邊。西門慶注目停視,比初見時節越發齊整,不覺心搖目蕩,不能禁止。

不一時,丫鬟又拏一道茶來。這粉頭輕搖羅袖,微露春纖,取一鍾,雙手遞與西門慶,然後與愛香各取一鐘相陪。吃畢,收下盞托去,請寬衣服房裡坐。西門慶叫玳安上來,把上蓋青紗衣寬了,搭在椅子上。{\meipi{鄭月兒深情人,不肯便滿面春風,西門慶又恐失官體,所以乍見時疎疎落落。}}進入粉頭房中,但見瑤窓繡幕,錦褥華裀,異香襲人,極其清雅,眞所謂神仙洞府,人跡不可到者也。彼此攀話調笑之際,只見丫鬟進來安放桌兒,擺下許多精製菜蔬。先請吃荷花細餅,鄭愛月兒親手揀攢肉絲捲就,安放小泥金碟兒內,遞與西門慶吃。須臾,吃了餅,收了家伙去,就鋪茜紅氊條,取出牙牌三十二扇,與西門慶抹牌。抹了一囘,收過去,擺上酒來。但見盤堆異果,酒泛金波,十分齊整。姊妹二人遞了酒,在旁箏排雁柱,款跨絞綃,愛香兒彈箏,愛月兒琵琶,唱了一套「兜的上心來」。{\meipi{曲題妙絕,不獨當西門慶之心,而來情去脈,隱隱接上。}}端的詞出佳人口,有裂石遶梁之聲。唱畢,促席而坐,拏骰盆兒與西門慶搶紅猜枚。

飲勾多時,鄭愛香兒推更衣出去了,獨有愛月兒陪着西門慶吃酒。先是西門慶向袖中取出白綾汗巾兒,上頭束着個金穿心盒兒。鄭愛月兒只道是香茶,便要開啟。西門慶道:「不是香茶,是我逐日吃的補藥。{\meipi{分明損藥,到說是補藥,妙,妙。}}我的香茶不放在這裡面,只用紙包着。」於是袖中取出一包香茶桂花餅兒遞與他。那愛月兒不信,還伸手徃他袖子裡掏,又掏出個紫縐紗汗巾兒,上拴着一副揀金挑牙兒,拏在手中觀看,甚是可愛。說道:「我見桂姐和吳銀姐都拏着這樣汗巾兒,原來是你與他的。」西門慶道:「是我揚州船上帶來的。不是我與他,誰與他的?你若愛,與了你罷。到明日,再送一副與你姐姐。」說畢,西門慶就着鍾兒裡酒,把穿心盒兒內藥吃了一服,把粉頭摟在懷中,兩個一遞一口兒飲酒咂舌,無所不至。西門慶又舒手摸弄他香乳,緊緊就就賽麻圓滑膩。一面扯開衫兒觀看,白馥馥猶如瑩玉一般。揣摩良久,淫心輒起,腰間那話突然而興。解開褲帶,令他纖手籠揝。粉頭見其粗大,唬的吐舌害怕,雙手摟定西門慶脖項說道:「我的親親,你今日初會,將就我,只放半截兒罷!若都放進去,我就死了。你敢吃藥養的這等大,不然,如何天生恁恠剌剌兒的?紅赤赤,紫漒漒,好砢硶人子!」西門慶笑道:「我的兒!你下去替我品品。」愛月兒道:「慌怎的,徃後日子多如樹葉兒。今日初會,人生面不熟,再來,等我替你品。」說畢,西門慶欲與他交歡,愛月兒道:「你不吃酒了?」西門慶道:「我不吃了,咱睡罷。」愛月兒便叫丫鬟把酒桌擡過一邊,與西門慶脫靴,他便徃後邊更衣澡牝去了。西門慶脫靴時,還賞了丫頭一塊銀子,打發先上床睡,炷了香,放在薰籠內。良久,婦人進房,問西門慶:「你吃茶不吃?」西門慶道:「我不吃。」{\meipi{兩不吃,情急甚矣。}}一面掩上房門,放下綾綃來,將絹兒安放在褥下,解衣上床。兩個枕上鴛鴦,被中鸂鶒。西門慶見粉頭肌膚纖細,牝淨無毛,猶如白麵蒸餅一般,柔嫩可愛。抱了抱腰肢,未盈一掬。誠為軟玉溫香,千金難買。於是把他兩隻白生生銀條般嫩腿兒,夾在兩邊腰眼間,那話上使了托子,向花心裡頂入。龜頭昂大,濡攪半晌,方纔沒稜。那愛月兒把眉頭縐在一處,兩手攀擱在枕上,隱忍難捱。朦朧着星眼,低聲說道:「今日你饒了鄭月兒罷!」西門慶聽了,愈覺銷魂,肆行抽送,不勝歡娛。正是:得多少

\begin{myquote} 
春點桃花紅綻蕊,風欺楊柳綠翻腰。
\end{myquote} 

西門慶與鄭月兒留戀至三更,方纔囘家。到次日,吳月娘打發他徃衙門中去了,和玉樓、金蓮、李嬌兒都在上房坐的。只見玳安進來上房取尺頭匣兒,徃夏提刑送生日禮去。月娘因問玳安:「你爹昨日坐轎於徃誰家吃酒,吃到那咱晚纔囘家?想必又在韓道國家,望他那老婆去來。原來賊囚根子成日只瞞着我,背地替他幹這等繭兒!」{\meipi{月娘不開口則已,開口亦不饒人。}}玳安道:「不是。他漢子來家,爹怎好去的!」月娘道:「不是那裡,卻是誰家?」那玳安又不說,只是笑。取了段匣,送禮去了。潘金蓮道:「大姐姐,你問這賊囚根子,他怎肯寔說?我聽見說蠻小厮昨日也跟了去來,只叫蠻小厮來問就是了。」一面把春鴻叫到跟前。金蓮問:「你昨日跟了你爹轎子去,在誰家吃酒來?你寔說便罷,不寔說,如今你大娘就要打你。」那春鴻跪下便道:「娘休打小的,待小的說就是了。小的和玳安、琴童哥三個,跟俺爹從一座大門樓進去,轉了幾條街巷,到個人家,只半截門兒,都用鋸齒兒鑲了。門裡立着個娘娘,打扮的花花黎黎的。」金蓮聽見笑了,說道:「囚根子,一個院裡半門子也不認的?趕着粉頭叫娘娘起來。」又問道:「那個娘娘怎麼模樣?你認的他不認的?」春鴻道:「我不認的他,也象娘每頭上戴着這個假殼。進入裡面,一個白頭的阿婆出來,望俺爹拜了一拜。落後請到後邊,又是一位年小娘娘出來,不戴假殼,生的瓜子面,搽的嘴唇紅紅的,陪着俺爹吃酒。」金蓮道:「你們都在那裡坐來?」春鴻道:「我和玳安、琴童哥便在阿婆房裡,陪着俺每吃酒並肉兜子來。」把月娘、玉樓笑的了不得。因問道:「你認的他不認的?」春鴻道:「那一個好似在咱家唱的。」{\meipi{若玳安開口說破,有何趣味!妙在令春鴻隱隱約約畫個影子,似是而實非,涵養文情,眞如生龍活虎。}}玉樓笑道:「就是李桂姐了。」月娘道:「原來摸到他家去來。」{\meipi{說來想來,自是桂姐無疑,雖百口亦難置辨,而孰知其不然。天下事不可意度如此。}}李嬌兒道:「俺家沒半門子。」金蓮道:「只怕你家新安了半門子是的。」問了一囘。西門慶來家,就徃夏提刑家拜壽去了。

卻說潘金蓮房中養的一隻白獅子貓兒,渾身純白,只額兒上帶龜背一道黑,名喚雪裡送炭,又名雪獅子。又善會口啣汗巾子,拾扇兒。西門慶不在房中,婦人晚夕常抱他在被窩裡睡,又不撒尿屎在衣服上,呼之即至,揮之即去,婦人常喚他是雪賊。每日不吃牛肝乾魚,只吃生肉,調養的十分肥壯,毛內可藏一雞蛋。甚是愛惜他,終日在房裡用紅絹裹肉,令貓撲而撾食。這日也是合當有事,官哥兒心中不自在,連日吃劉婆子藥,畧覺好些。李瓶兒與他穿上紅段衫兒,安頓在外間炕上頑耍,迎春守着,奶子便在旁吃飯。不料這雪獅子正蹲在護炕上,看見官哥兒在炕上,穿着紅衫兒一動動的頑耍,只當平日哄喂他肉食一般,猛然望下一跳,將官哥兒身上皆抓破了。只聽那官哥兒「呱」的一聲,倒嚥了一口氣,就不言語了,手脚俱風搐起來。慌的奶子丟下飯碗,摟抱在懷,只顧唾噦,與他收驚。那貓還來趕着他要撾,被迎春打出外邊去了。如意兒實承望孩子搐過一陣好了,誰想只顧常連,一陣不了一陣搐起來。忙使迎春後邊請李瓶兒去,說:「哥兒不好了,風搐着哩,娘快去!」那李瓶兒不聽便罷,聽了,正是:

\begin{myquote} 
驚損六葉連肝肺,唬壞三毛七孔心。
\end{myquote} 

連月娘慌的兩步做一步,逕撲到房中。見孩子搐的兩隻眼直徃上弔,通不見黑眼睛珠兒,口中白沫流出,咿咿猶如小雞叫,手足皆動。一見心中猶如刀割相侵,連忙摟抱起來,臉搵着他嘴兒,大哭道:「我的哥哥,我出去好好兒,怎麼就搐起來?」迎春與奶子,悉把被五娘房裡貓所唬一節說了。那李瓶兒越發哭起來,說道:「我的哥哥,你緊不可公婆意,今日你只當脫不了,打這條路兒去了!」{\meipi{以瓶兒之忍耐,到此時亦忍耐不住,怨恨極矣。}}月娘聽了,一聲兒沒言語,一面叫將金蓮來,問他說:{\pangpi{多此一問。}}「是你屋裡的貓唬了孩子?」金蓮問:「是誰說的?」月娘指着:「是奶子和迎春說來。」金蓮道:「你看這老婆子這等張嘴!俺貓在屋裡好好兒的臥着不是。你每怎的把孩子唬了,沒的賴人起來。瓜兒只揀軟處捏,俺每這屋裡是好纏的!」月娘道:「他的貓怎得來這屋裡?」迎春道:「每常也來這邊屋裡走跳。」金蓮接過來道:「早時你說,每常怎的不撾他?可哥今日兒就撾起來?你這丫頭也跟着他恁張眉瞪眼兒,六說白道的。將就些兒罷了,怎的要把弓兒扯滿了?可哥兒俺每自恁沒時運來。」{\pangpi{惹他開口。}}於是使性子抽身徃房裡去了。看官聽說:潘金蓮見李瓶兒有了官哥兒,西門慶百依百隨,要一奉十,故行此陰謀之事,馴養此貓,必欲唬死其子,使李瓶兒寵衰,教西門慶復親於己。就如昔日屠岸賈養神獒害趙盾丞相一般。{\meipi{此亦在有意無意間,未必如所言者之甚也。}}正是:

\begin{myquote} 
花枝葉底猶藏刺,人心怎保不懷毒。
\end{myquote} 

月娘衆人見孩子只顧搐起來,一面熬薑湯灌他,一面使來安兒快叫劉婆去。不一時,劉婆子來到,看了脈息,只顧跌脚,說道:「此遭驚唬重了,難得過了。快熬燈心薄荷金銀湯。」取出一丸金箔丸來,向鍾兒內研化。牙關緊閉,月娘連忙拔下金簪兒來,撬開口,灌下去。劉婆道:「過得來便罷。如過不來,告過主家奶奶,必須要灸幾醮纔好。」月娘道:「誰敢耽?必須等他爹來問了不敢。灸了,惹他來家喓喝。」李瓶兒道:「大娘救他命罷!若等來家,只恐遲了。若是他爹罵,等我承當就是了。」月娘道:「孩兒是你的孩兒,隨你灸,我不敢張主,」當下,劉婆子把官哥兒眉攢、脖根、兩手關尺並心口,共灸了五醮,放他睡下。那孩子昏昏沉沉,直睡到日暮時分西門慶來家還不醒。那劉婆見西門慶來家,月娘與了他五錢銀子,一溜烟從夾道內出去了。

西門慶歸到上房,月娘把孩子風搐不好對西門慶說了,西門慶連忙走到前邊來看視,見李瓶兒哭的眼紅紅的,問:「孩兒怎的風搐起來?」李瓶兒滿眼落淚,只是不言語。問丫頭、奶子,都不敢說。西門慶又見官哥手上皮兒去了,灸的滿身火艾,心中焦燥,又走到後邊問月娘。月娘隱瞞不住,只得把金蓮房中貓驚唬之事說了:「劉婆子剛纔看,說是急驚風,若不針灸,難過得來。若等你來,只恐怕遲了。他娘母子自主張,叫他灸了孩兒身上五醮,纔放下他睡了。這半日還未醒。」西門慶不聽便罷,聽了此言,三屍暴跳,五臟氣冲,怒從心上起,惡向膽邊生,直走到潘金蓮房中,不繇分說,尋着雪獅子,提着脚走向穿廊,望石臺基輪起來只一摔,只聽响喨一聲,腦漿迸萬朵桃花,滿口牙零噙碎玉。正是:

\begin{myquote} 
不在陽間擒鼠耗,卻歸陰府作狸仙。
\end{myquote} 

潘金蓮見他拏出貓去摔死了,坐在炕上風紋也不動。待西門慶出了門,口裡喃喃吶吶罵道:「賊作死的強盜,把人粧出去殺了纔是好漢!一個貓兒礙着你𠳹屎?亡神也似走的來摔死了。他到陰司裡,明日還問你要命,你慌怎的?賊不逢好死變心的強盜!」{\meipi{西門慶正在氣頭上,又不敢明嚷,又不能暗忍,明嚷恐討沒趣,暗忍又恐人笑,等其去後,卻牢牢叨叨作絮語,妙得其情。}}

西門慶走到李瓶兒房裡,因說奶子、迎春:「我教你好看着孩兒,怎的教貓唬了他,把他手也撾了!又信劉婆子那老淫婦,平白把孩子灸的恁樣的。若好便罷,不好,把這老淫婦拏到衙門裡,與他兩拶!」李瓶兒道:「你看孩兒緊自不得命,你又是恁樣的。孝順是醫家,他也巴不得要好哩。」李瓶兒只指望孩兒好來,不料被艾火把風氣反於內,變為慢風,內裡抽搐的腸肚兒皆動,尿屎皆出,大便屙出五花顏色,眼目忽睜忽閉,終朝只是昏沉不省,奶也不吃了。李瓶兒慌了,到處求神問卜打卦,皆有兇無吉。月娘瞞着西門慶又請劉婆子來家跳神,又請小兒科太醫來看。都用接鼻散試之:若吹在鼻孔內打鼻涕,還看得;若無鼻涕出來,則看陰騭守他罷了。於是吹下去,茫然無知,並無一個噴涕出來。越發晝夜守着哭涕不止,連飲食都減了。

看看到八月十五日將近,月娘因他不好,連自家生日都囘了不做,親戚內眷,就送禮來也不請。家中止有吳大妗子、楊姑娘並大師父來相伴。那薛姑子和王姑子兩個,在印經處爭分錢不平,又使性兒,彼此互相揭調。十四日,賁四同薛姑子催討,將經卷挑將米,一千五百卷都完了。李瓶兒又與了一弔錢買紙馬香燭。十五日同陳敬濟早徃岳廟裡進香紙,把經看着都散施盡了,走來囘李瓶兒話。喬大戶家,一日一遍使孔嫂兒來看,又舉薦了一個看小兒的鮑太醫來看,說道:「這個變成天弔客忤,治不得了。」白與了他五錢銀子,打發去了。灌下藥去也不受,還吐出了。只是把眼合着,口中咬的牙格支支響。李瓶兒通衣不解帶,晝夜抱在懷中,眼淚不乾的只是哭。西門慶也不徃那裡去,每日衙門中來家,就進來看孩兒。

那時正値八月下旬天氣,李瓶兒守着官哥兒睡在床上,桌上點着銀燈,丫鬟養娘都睡熟了。覷着滿窓月色,更漏沉沉,果然愁腸萬結,離思千端。正是:人逢喜事精神爽,悶來愁腸瞌睡多。但見:

\begin{myquote} 
銀河耿耿,玉漏迢迢。穿窓皓月耿寒光,透戶涼風吹夜氣。樵樓禁鼓,一更未盡一更敲;別院寒砧,千搗將殘千搗起。畫簷前叮噹鐵馬,敲碎思婦情懷;銀臺上閃爍燈光,偏照佳人長嘆。一心只想孩兒好,誰料愁來睡夢多。
\end{myquote} 

當下,李瓶兒臥在床上,似睡不睡,夢見花子虛從前門外來,身穿白衣,恰似活時一般。見了李瓶兒,厲聲罵道:「潑賊淫婦,你如何抵盜我財物與西門慶?如今我告你去也。」被李瓶兒一手扯住他衣袖,央及道:「好哥哥,你饒恕我則個!」花子虛一頓,撒手驚覺,卻是南柯一夢。{\meipi{國家將亡,必有妖孽,好時偏不夢見。}}醒來,手裡扯着卻是官哥兒的衣衫袖子。連噦了幾口道:「恠哉!恠哉!」聽一聽更鼓,正打三更三點。李瓶兒唬的渾身冷汗,毛髮皆豎。

到次日,西門慶進房來,就把夢中之事告訴一遍。西門慶道:「知道他死到那裡去了!此是你夢想舊境。只把心來放正着,休要理他。如今我使小厮拏轎子接了吳銀兒來,與你做個伴兒。再把老馮叫來伏侍兩日。」玳安打院裡接了吳銀兒來。

那消到日西時分,那官哥兒在奶子懷裡只搐氣兒了。慌的奶子叫李瓶兒:「娘,你來看哥哥,這黑眼睛珠兒只徃上翻,口裡氣兒只有出來的,沒有進去的。」這李瓶兒走來抱到懷中,一面哭起來,叫丫頭:「快請你爹去!你說孩子待斷氣也。」可哥常峙節又走來說話,{\meipi{常峙節不先不後偏到此時來,眞若有窮鬼使之者然。}}告訴房子兒尋下了,門面兩間,二層,大小四間,只要三十五兩銀子。西門慶聽見後邊官哥兒重了,就打發常峙節起身,說:「我不送你罷,改日我使人拏銀子和你看去。」急急走到李瓶兒房中。月娘衆人都在房裡瞧着,那孩子在他娘懷裡一口口搐氣兒。{\meipi{情景逼眞。}}西門慶不忍看他,走到明間椅子上坐着,只長吁短嘆。那消半盞茶時,官哥兒嗚呼哀哉,斷氣身亡。時八月廿三日申時也,只活了一年零兩個月。合家大小放聲号哭。那李瓶兒撾耳撓腮,一頭撞在地下,哭的昏過去。半日方纔甦省,摟着他大放聲哭叫道:「我的沒救星兒,心疼殺我了!寧可我同你一答兒裡死了罷,我也不久活在世上了。我的拋閃殺人的心肝,撇的我好苦也!」那奶子如意兒和迎春在旁,哭的言不得,動不得。西門慶即令小厮收拾前廳西廂房乾淨,放下兩條寬櫈,要把孩子連枕蓆被褥擡出去那裡挺放。那李瓶兒倘在孩兒身上,兩手摟抱着,那裡肯放!口口聲聲直叫:「沒救星的冤家!嬌嬌的兒!生揭了我的心肝去了!撇的我枉費辛苦,乾生受一場,再不得見你了,我的心肝!……」月娘衆人哭了一囘,在旁勸他不住。西門慶走來,見他把臉抓破了,滾的寶髻蓬鬆,烏雲散亂,便道:「你看蠻的!他既然不是你我的兒女,幹養活他一場,他短命死了,哭兩聲丟開罷了,{\meipi{畢竟男子漢轉念快。}}如何只顧哭了去!又哭不活他,你的身子也要緊。如今擡出去,好叫小厮請陰陽來看。這是甚麼時候?」月娘道:「這個也有申時前後。」玉樓道:「我頭裡怎麼說來?他管情還等他這個時候纔去。原是申時生,還是申時死。日子又相同,都是二十三日,只是月分差些。圓圓的一年零兩個月。」李瓶兒見小厮每伺候兩旁要擡他,又哭了,說道:「慌擡他出去怎麼的?大媽媽,你伸手摸摸,他身上還熱哩!」{\pangpi{婦人癡語酷肖。}}叫了一聲:「我的兒嚛!你教我怎生割捨的你去?坑得我好苦也!……」一頭又撞倒在地下,哭了一囘。衆小厮纔把官哥兒擡出,停在西廂房內。月娘向西門慶計較:「還對親家那裡並他師父廟裡說聲去。」西門慶道,「他師父廟裡,明早去罷。」一面使玳安徃喬大戶家說了,一面使人請了徐陰陽來批書。又拏出十兩銀子與賁四,教他快擡了一付平頭杉板,令匠人隨即攢造了一具小棺槨兒,就要入殮。喬宅那裡一聞來報,喬大戶娘子隨即坐轎子來,進門就哭。月娘衆人又陪着大哭了一場,告訴前事一遍。不一時,陰陽徐先生來到,看了,說道:「哥兒還是正申時永逝。」月娘分付出來,教與他看看黑書。徐先生將陰陽祕書瞧了一囘,說道:「哥兒生於政和丙申六月廿三日申時,卒於政和丁酉八月廿三日申時。月令丁酉,日干壬子,犯天地重䘮,本家要忌忌哭聲,親人不忌。入殮之時,蛇、龍、鼠、兔四生人,避之則吉。又黑書上云:壬子日死者,上應寶瓶宮,下臨齊地。他前生曾在兗州蔡家作男子,曾倚力奪人財物,吃酒落魄,不敬天地六親,橫事牽連,遭氣寒之疾,久臥床蓆,穢汙而亡。今生為小兒,亦患風癇之疾。十日前被六畜驚去魂魄,又犯土司太歲,先亡攝去魂魄,托生徃鄭州王家為男子,後作千戶,壽六十八歲而終。」

須臾,徐先生看了黑書,請問老爹,明日出去或埋或化,西門慶道:「明日如何出得!擱三日,念了經,到五日出去,墳上埋了罷。」徐先生道:「二十七日丙辰,合家本命都不犯,宜正午時掩土。」批畢書,一面就收拾入殮,已有三更天氣。李瓶兒哭着徃房中,尋出他幾件小道衣、道髻、鞋襪之類,替他安放在棺槨內,釘了長命釘,合家大小又哭了一場,打發陰陽去了。

次日,西門慶亂着,也沒徃衙門中去。夏提刑打聽得知,早晨衙門散時,就來弔問。又差人對吳道官廟裡說知,到三日,請報恩寺八衆僧人在家誦經。吳道官廟裡並喬大戶家,俱備折卓三牲來祭奠。吳大舅、沈姨夫、門外韓姨夫、花大舅都有三牲祭卓來燒紙。應伯爵、謝希大、溫秀才、常峙節、韓道國、甘出身、賁第傳、李智、黃四都鬪了分資,晚夕來與西門慶伴宿。打發僧人去了,叫了一起提偶的,先在哥兒靈前祭畢,然後,西門慶在大廳上放卓席管待衆人。

那日院中李桂姐、吳銀兒並鄭月兒三家,都有人情來上紙。李瓶兒思想官哥兒,每日黃懨懨,連茶飯兒都懶待吃,題起來只是哭涕,把喉音都哭啞了。西門慶怕他思想孩兒,尋了拙智,白日裡分付奶子、丫鬟和吳銀兒相伴他,不離左右。晚夕,西門慶一連在他房中歇了三夜,枕上百般解勸。薛姑子夜間又替他念《楞嚴經》、《解冤咒》,勸他:「休要哭了。他不是你的兒女,都是宿世冤家債主。《陀羅經》上不說的好:昔日有一婦人,生產孩兒三遍,俱不過兩歲而亡,婦人悲啼不已。抱兒江邊,不忍拋棄。感得觀世音菩薩化作一僧,謂此婦人曰:『不用啼哭,此非你兒,是你生前冤家。三度托生,皆欲殺汝。你若不信,我交你看。』將手一指,其兒遂化作一夜叉之形,向水中而立,報言:『汝曾殺我來,我特來報冤。今因汝常持《佛頂心陀羅經》,善神日夜擁護,所以殺汝不得。我已蒙觀世音菩薩受度了,從今永不與汝為冤。』道畢,遂沉水中不見。不該我貧僧說,你這兒子,必是宿世冤家,托來你蔭下,化目化財,要惱害你身。為你捨了此《佛頂心陀羅經》一千五百卷,有此功行,他害你不得,故此離身。{\meipi{一邊捨經而一邊人死,似難再言因果矣。而薛姑反以人死為捨經一報,說得有源有委,利嘴之可長如此,尼僧之利嘴如此。}}到明日再生下來,纔是你兒女。」{\pangpi{此一轉更妙。}}李瓶兒聽了,終是愛緣不斷。但題起來,輒流涕不止。

須臾過了五日,到廿七日早晨,顧了八名青衣白帽小童,大紅銷金棺與幡幢、雪蓋、玉梅、雪柳圍隨,前首大紅銘旌,題着「西門塚男之樞」。吳道官廟裡,又差了十二衆青衣小道童兒來,遶棺轉咒生神玉章,動清樂送殯。衆親朋陪西門慶穿素服走至大街東口,將及門上,纔上頭口。西門慶恐怕李瓶兒到墳上悲痛,不叫他去。只是吳月娘、李嬌兒、孟玉樓、潘金蓮、大姐,家裡五頂轎子,陪喬親家母、大妗子和李桂兒、鄭月兒、吳舜臣媳婦鄭三姐徃墳頭去,留下孫雪娥、吳銀兒並兩個姑子在家與李瓶兒做伴兒。李瓶兒見不放他去,見棺材起身,送出到大門首,趕着棺材大放聲,一口一聲只叫:「不來家虧心的兒嚛!」叫的連聲氣破了。不防一頭撞在門底下,把粉額磕傷,金釵墜地,慌的吳銀兒與孫雪娥向前搊扶起來,勸歸後邊去了。到了房中,見炕上空落落的,只有他耍的那壽星博浪鼓兒還掛在床頭上,想將起來,{\meipi{記瓶兒初進門時何等冷落,尚歡喜忍耐,今雖子死,實無減於舊,遂淒涼痛苦如此,何人心之不能平耶。}}拍了桌子,又哭個不了。

吳銀兒在旁,拉着他手勸說道:「娘少哭了,哥哥已是拋閃你去了,那裡再哭得活!你須自解自嘆,休要只顧煩惱。」雪娥道:「你又年少青春,愁到明日養不出來也怎的?這裡墻有縫,壁有眼,俺每不好說的。他使心用心,反累已身。他將你孩子害了,教他一還一報,問他要命。不知你我被他活埋了幾遭了!只要漢子常守着他便好,到人屋裡睡一夜兒,他就氣生氣死。早是前者,你每都知道,漢子等閑不到我後邊,纔到了一遭兒,你看他就背地裡唧喳成一塊,對着他姐兒每說我長道我短。俺每也不言語,每日洗眼兒看着他。這個淫婦,到明日還不知怎麼死哩!」{\meipi{三句語就到自己心事,積恨之深可想。}}李瓶兒道:「罷了,我也惹了一身病在這裡,不知在今日明日死,和他也爭執不得了,隨他罷!」

正說着,只見奶子如意兒向前跪下,哭道:「小媳婦有句活,不敢對娘說,今日哥兒死了,乃是小媳婦沒造化。只怕徃後爹與大娘打發小媳婦出去,小媳婦男子漢又沒了,那裡投奔?」李瓶兒見他這般說,又心中傷痛起來,便道:「恠老婆,孩子便沒了,我還沒死哩!總然我到明日死了,你恁在我手下一場,我也不教你出門。徃後你大娘生下哥兒小姐來,交你接了奶,就是一般了。你慌亂的是甚麼?」那如意兒方纔不言語了。李瓶兒良久又悲慟哭起來,雪娥與吳銀兒兩個又解勸說道:「你肚中吃了些甚麼,只顧哭了去!」一面叫綉春後邊拏了飯來,擺在桌上,陪他吃。那李瓶兒怎生嚥下去!只吃了半甌兒,就丟下不吃了。西門慶在墳上,叫徐先生畫了穴,把官哥兒就埋在先頭陳氏娘懷中,抱孫葬了。

那日喬大戶井衆親戚都有祭祀,就在新蓋捲棚管待飲酒一日。來家,李瓶兒與月娘、喬大戶娘子、大妗子磕着頭又哭了。向喬大戶娘子說道:「親家,誰似奴養的孩兒不氣長,短命死了。既死了,累你家姐姐做了望門寡,勞而無功,親家休要笑話。」喬大戶娘子說道:「親家怎的這般說話?孩兒每各人壽數,誰人保的後來的事!常言先親後不改。親家每又不老,徃後愁沒子孫?須要慢慢來。親家也少要煩惱了。」說畢,作辭囘家去了。西門慶在前廳教徐先生灑掃,各門上都貼闢非黃符。死者煞高三丈,向東北方而去,遇日遊神冲囘不出,斬之則吉,親人不忌。西門慶拏出一疋大布、二兩銀子謝了徐先生,管待出門。晚夕入李瓶兒房中陪他睡。夜間百般言語溫存。見官哥兒的戲耍物件都還在跟前,恐怕這瓶兒看見思想煩惱,都令迎春拏到後邊去了。正是:

\begin{myquote} 
思想嬌兒晝夜啼,寸心如割命懸絲。\\世間萬般哀苦事,除非死別共生離。
\end{myquote} 

