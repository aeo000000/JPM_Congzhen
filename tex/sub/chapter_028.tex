\includepdf[pages={55,56},fitpaper=false]{tst.pdf}
\chapter*{第二十八囘 陳敬濟徼倖得金蓮 西門慶糊塗打鐵棍}
\addcontentsline{toc}{chapter}{第二十八囘 陳敬濟徼倖得金蓮 西門慶糊塗打鐵棍}
\markboth{{\titlename}卷之三}{第二十八囘 陳敬濟徼倖得金蓮 西門慶糊塗打鐵棍}


詩曰:

幾日深閨繡得成,看來便覺可人情。一灣暖玉淩波小,兩瓣秋蓮落地輕。南陌踏青春有跡,西廂立月夜無聲。看花又濕蒼苔露,晒向窓前趁晚晴。

話說西門慶扶婦人到房中,脫去上下衣裳,赤着身子,婦人止着紅紗抹胸兒。{\pangpi{又一種態度。}}兩個並肩疊股而坐,重斟盃酌。西門慶一手摟過他粉頸,一遞一口和他吃酒,極盡溫存之態。睨視婦人雲鬟斜軃,酥胸半露,嬌眼乜斜,猶如沉酒楊妃一般,{\meipi{寫得嬌倩如生。}}纖手不住只向他腰裡摸弄那話。那話因驚,銀托子還帶在上面,軟叮噹毛都魯的累垂偉長。西門慶戲道:「你還弄他哩,都是你頭裡唬出他風病來了。」婦人問:「怎的風病。」西門慶道:「既不是瘋病,如何這軟癱熱化,起不來了,你還不下去央及他央及兒哩。」婦人笑瞅了他一眼。一面蹲下身子去,枕着他一隻腿,取過一條褲帶兒來,把那話拴住,用手提着,說道:「你這厮!頭裡那等頭睜睜,股睜睜,把人奈何昏昏的,這咱你推風症,裝佯死兒。」{\meipi{分明穢語,閱來但見其風騷,不見其穢,可謂化腐臭為神奇矣。}}提弄了一囘,放在粉臉上偎晃良久,然後將口吮之,又用舌尖挑砥其蛙口。那話登時暴怒起來,裂瓜頭凹眼睜圓,落腮鬍挺身直豎。西門慶亦發坐在枕頭上,令婦人馬爬在紗帳內,盡着吮咂,以暢其美。俄爾淫思益熾,復與婦人交接。婦人哀告道:「我的達達,你饒了奴罷,又要捉弄奴也!」是夜,二人淫樂為之無度。有詞為證:

戰酣樂極,雲雨歇,嬌眼乜斜。手持玉莖猶堅硬,告纔郎將就些些。滿飲金盃頻勸,兩情似醉如痴。

一夜晚景題過。到次日,西門慶徃外邊去了。婦人約飯時起來,換睡鞋,尋昨日脚上穿的那雙紅鞋,左來右去少一隻。問春梅,春梅說:「昨日我和爹搊扶着娘進來,秋菊抱娘的鋪蓋來。」婦人叫了秋菊來問。秋菊道:「我昨日沒見娘穿着鞋進來。」婦人道:「你看胡說!我沒穿鞋進來,莫不我精着脚進來了?」秋菊道:「娘你穿着鞋,怎的屋裡沒有?」{\meipi{秋菊蠢不必言,然金蓮醜態亦得他搶白一番方快。}}婦人罵道:「賊奴才,還裝憨兒!無過只在這屋裡,你替我老實尋是的!」這秋菊三間屋裡,床上床下,到處尋了一遍,那裡討那隻鞋來?婦人道:「端的我這屋裡有鬼,攝了我這隻鞋去了。連我脚上穿的鞋都不見了,要你這奴才在屋裡做甚麼!」秋菊道:「倒只怕娘忘記落在花園裡,沒曾穿進來。」婦人道:「敢是㒲昏了,{\pangpi{自道。}}我鞋穿在脚上沒穿在脚上,我不知道?」叫春梅:「你跟着這奴才,徃花園裡尋去。尋出來便罷,若尋不出來,叫他院子裡頂石頭跪着。」這春梅真個押着他,花園到處並葡萄架跟前,尋了一遍兒,那裡得來!正是:

都被六丁收拾去,蘆花明月竟難尋。

兩個尋了一遍囘來,春梅罵道:「奴才,你『媒人婆迷了路兒——沒的說了』,『王媽媽賣了磨——推不的了』。」秋菊道:「不知甚麼人偷了娘的這隻鞋去了,我沒曾見娘穿進屋裡去。敢是你昨日開花園門放了那個,拾了娘的這隻鞋去了。」被春梅一口稠唾沫噦了去,罵道:「賊見鬼的奴才,又攪纏起我來了!六娘叫門,我不替他開?可哥兒的就放進人來了?你抱着娘的鋪蓋就不經心瞧瞧,還敢說嘴兒!」一面押他到屋裡,囘婦人說沒有鞋。婦人叫採出他院子裡跪着。秋菊把臉哭䘮下水來,說:「等我再徃花園裡尋一遍,尋不着隨娘打罷。」春梅道:「娘休信他。花園裡地也掃得乾乾淨淨的,就是針也尋出來,那裡討鞋來?」秋菊道:「等我尋不出來,教娘打就是了。你在旁戳舌兒怎的!」婦人向春梅道:「也罷,你跟着這奴才,看他那裡尋去!」這春梅又押着他,在花園山子底下,各處花池邊,松墻下,尋了一遍,沒有。他也慌了,被春梅兩個耳刮子,就拉囘來見婦人。秋菊道:「還有那個雪洞裡沒尋哩。」春梅道:「那藏春塢是爹的暖房兒,娘這一向又沒到那裡。我看尋不出來和你答話!」於是押着他,到於藏春塢雪洞內。正面是張坐床,旁邊香几上都尋到,沒有。又向書篋內尋,春梅道:「這書篋內都是他的拜帖紙,娘的鞋怎的到這裡?沒的摭溜子捱工夫兒!{\meipi{尋得無因,卻用此語庇護。}}翻的他恁亂騰騰的,惹他看見又是一場兒,你這歪刺骨可死的成了!」良久,只見秋菊說道:「這不是娘的鞋!」{\meipi{又為蕙蓮作餘波。}}在一個紙包內,裹着些棒兒香與排草,取出來與春梅瞧:「可怎的有了,剛纔就調唆打我!」春梅看見,果是一隻大紅平底鞋兒,說道:「是娘的,怎生得到這書篋內?好蹊蹺的事!」於是走來見婦人。婦人問:「有了我的鞋,端的在那裡?」春梅道:「在藏春塢,爹暖房書篋內尋出來,和些拜帖子紙、排草、安息香包在一處。」婦人拏在手內,取過他的那隻來一比,都是大紅四季花段子白綾平底繡花鞋兒,綠提根兒,藍口金兒。惟有鞋上鎖線兒差些,一隻是紗綠鎖線,一隻是翠藍鎖線,不仔細認不出來。婦人登在脚上試了試,尋出來這一隻比舊鞋略緊些,方知是來旺兒媳婦子的鞋:「不知幾時與了賊強人,不敢拏到屋裡,悄悄藏放在那裡。不想又被奴才翻將出來。」看了一囘,說道:「這鞋不是我的。奴才,快與我跪着去!」分咐春梅:「拏塊石頭與他頂着。」那秋菊哭起來,說道:「不是娘的鞋,是誰的鞋?我饒替娘尋出鞋來,還要打我;若是再尋不出來,不知還怎的打我哩!」婦人罵道:「賊奴才,休說嘴!」春梅一面掇了塊大石頭頂在他頭上。婦人又另換了一雙鞋穿在脚上,嫌房裡熱,分咐春梅把粧臺放在玩花樓上,梳頭去了,不在話下。

卻說陳敬濟早晨從鋪子裡進來尋衣服,走到花園角門首。小鐵棍兒在那裡正頑着,見陳敬濟手裡拏着一副銀網巾圈兒,便問:「姑夫,你拏的甚麼?與了我耍子罷。」敬濟道:「此是人家當的網巾圈兒,來贖,我尋出來與他。」那小猴子笑嘻嘻道:「姑夫,你與了我耍子罷,我換與你件好物件兒。」敬濟道:「傻孩子,此是人家當的。你要,我另尋一副兒與你耍子。你有甚麼好物件,拏來我瞧。」那猴子便向腰裡掏出一隻紅繡花鞋兒與敬濟看。敬濟便問:「是那裡的?」那猴子笑嘻嘻道:「姑夫,我對你說了罷!我昨日在花園裡耍子,看見俺爹弔着俺五娘兩隻腿兒,在葡萄架兒底下,搖搖擺擺。{\pangpi{諧甚。}}落後俺爹進去了,我尋俺春梅姑娘要菓子吃,在葡萄架底下拾了這隻鞋。」敬濟接在手裡:曲是天邊新月,紅如退瓣蓮花,把在掌中,恰剛三寸。就知是金蓮脚上之物,便道:「你與了我,明日另尋一對好圈兒與你耍子。」猴子道:「姑夫你休哄我,我明日就問你要哩。」敬濟道:「我不哄你。」那猴子一面笑的耍去了。

這敬濟把鞋褪在袖中,自己尋思「我幾次戲他,他口兒且是活,及到中間,又走滾了。不想天假其便,此鞋落在我手裡。今日我着實撩逗他一番,不怕他不上帳兒。」正是:

時人不用穿針線,那得工夫送巧來?

陳敬濟袖着鞋,逕徃潘金蓮房來。轉過影壁,只見秋菊跪在院內,便戲道:「小大姐,為甚麼來?投充了新軍,又掇起石頭來了?」{\meipi{開口便令人解頤。}}金蓮在樓上聽見,便叫春梅問道:「是誰說他掇起石頭來了?乾淨這奴才沒頂着?」春梅道:「是姑夫來了。秋菊頂着石頭哩。」婦人便叫:「陳姐夫,樓上沒人,{\pangpi{冷甚。}}你上來。」這小夥兒打步撩衣上的樓來。只見婦人在樓上,前面開了兩扇窓兒,掛着湘簾,那裡臨鏡梳粧。這陳敬濟走到旁邊一個小杌兒坐下,看見婦人黑油般頭髮,手挽着梳,還拖着地兒,紅絲繩兒扎着一窩絲,纘上戴着銀絲鬏髻,還墊出一絲香雲,鬏髻內安着許多玫瑰花瓣兒,露着四髩,打扮的就是活觀音。{\meipi{寫得花光髩影,蕩人心魄。}}須臾,婦人梳了頭,掇過粧臺去,向面盤內洗了手,穿上衣裳,喚春梅拏茶來與姐夫吃。那敬濟只是笑,不做聲。{\meipi{眉眼俱有勾挑意,妙甚。}}婦人因問:「姐夫,笑甚麼?」敬濟道:「我笑你管情不見了些甚麼兒?」婦人道:「賊短命!我不見了,關你甚事?你怎的曉得?」敬濟道:「你看,我好心倒做了驢肝肺,你倒訕起我來。恁說,我去了。」抽身徃樓下就走。被婦人一把手拉住,說道:「恠短命,會張致的!來旺兒媳婦子死了,沒了想頭了,卻怎麼還認的老娘。」{\meipi{又插入醋語,竟一日不忘。}}因問:「你猜着我不見了甚麼物件兒?」這敬濟向袖中取出來,提着鞋拽靶兒,笑道:「你看這個是誰的?」婦人道:「好短命,原來是你偷拏了我的鞋去了!教我打着丫頭,遶地裡尋。」敬濟道:「你怎的到得我手裡?」婦人道:「我這屋裡再有誰來?敢是你賊頭鼠腦,偷了我這隻鞋去了。」敬濟道:「你老人家不害羞。我這兩日又不徃你屋裡來,我怎生偷你的?」婦人道:「好賊短命,等我對你爹說,你倒偷了我鞋,還說我不害羞。」敬濟道:「你只好拏爹來唬我罷了。」婦人道:「你好小膽兒,明知道和來旺兒媳婦子七個八個,你還調戲他,你幾時有些忌憚兒的!既不是你偷了我的鞋,這鞋怎落在你手裡?趁早實供出來,交還與我鞋,你還便宜。自古物見主,必索取。但道半個不字,教你死在我手裡。」敬濟道:「你老人家是個女番子,且是倒會的放刁。這裡無人,咱們好講:你既要鞋,拏一件物事兒,我換與你,不然天雷也打不出去。」婦人道:「好短命!我的鞋應當還我,教換甚物事兒與你?」敬濟笑道:「五娘,你拏你袖的那方汗巾兒賞與兒子,兒子與了你的鞋罷。」{\meipi{勾挑軟暱處,在西門慶之上。}}婦人道:「我明日另尋一方好汗巾兒,這汗巾兒是你爹成日眼裡見過,不好與你的。」敬濟道:「我不。別的就與我一百方也不算,我一心只要你老人家這方汗巾兒。」婦人笑道:「好個牢成久慣的短命!我也沒氣力和你兩個纏。」於是向袖中取出一方細撮穗白綾挑線鶯鶯燒夜香汗巾兒,上面連銀三字兒都掠與他。有詩為證:

郎君見妾下蘭堦,來索纖纖紅繡鞋。不管露泥藏袖裡,只言從此事堪諧。

這陳敬濟連忙接在手裡,與他深深的唱個喏。{\pangpi{妙用。}}婦人分咐:「好生藏着,休教大姐看見,{\pangpi{自逗出私情。}}他不是好嘴頭子。」敬濟道:「我知道。」一面把鞋遞與他,如此這般:「是小鐵棍兒昨日在花園裡拾的,今早拏着問我換網巾圈兒耍子。」如此這般,告訴了一遍。婦人聽了,粉面通紅,說道:「你看賊小奴才,把我這鞋弄的恁漆黑的!看我教他爹打他不打他。」敬濟道:「你弄殺我!打了他不打緊,敢就賴着我身上,是我說的。千萬休要說罷。」婦人道:「我饒了小奴才,除非饒了蠍子。」兩個正說在熱鬧處,忽聽小厮來安兒來尋:「爹在前廳請姐夫寫礼帖兒哩。」婦人連忙攛掇他出去了。下的樓來,教春梅取板子來,要打秋菊。秋菊不肯儻,說道:「尋將娘的鞋來,娘還要打我!」婦人把陳敬濟拏的鞋遞與他看,罵道:「賊奴才,你把那個當我的鞋,將這個放在那裡?」秋菊看見,把眼瞪了半日,說道:「可是作恠的勾當,怎生跑出娘三隻鞋來了?」婦人道:「好大膽奴才!你拏誰的鞋來搪塞我,倒說我是三隻脚的蟾?」不繇分說,教春梅拉倒,打了十下。打有秋菊抱股而哭,望着春梅道:「都是你開門,教人進來,收了娘的鞋,這囘教娘打我。」{\meipi{分明說得是,只覺其蠢,人情乎?}}春梅罵道:「你倒收拾娘鋪蓋,不見了娘的鞋,娘打了你這幾下兒,還敢抱怨人!早是這隻舊鞋,若是娘頭上的簪環不見了,你也推賴個人兒就是了?{\meipi{語雖憊懶,氣象卻好。}}娘惜情兒,還打的你少。若是我,外邊叫個小厮,辣辣的打上他二三十板,看這奴才怎麼樣的!」幾句罵得秋菊忍氣吞聲,不言語了。

且說西門慶叫了敬濟到前廳,封尺頭禮物,送賀千戶新陞了淮安提刑所掌刑正千戶。本衛親識,都與他送行在永福寺,不必細說。西門慶差了鉞安送去,廳上陪着敬濟吃了飯,歸到金蓮房中。這金蓮千不合萬不合,把小鐵棍兒拾鞋之事告訴一遍,說道:「都是你這沒才料的貨平白幹的勾當!教賊萬殺的小奴才把我的鞋拾了,拏到外頭,誰是沒瞧見。被我知道,要將過來了。你不打與他兩下,到明日慣了他。」西門慶就不問「誰告你說來」,一冲性子走到前邊。那小猴兒不知,正在石臺基頑耍,被西門慶揪住頂角,拳打脚踢,殺豬也似叫起來,方纔住了手。這小猴子躺在地下,死了半日,慌得來昭兩口子走來扶救,半日甦醒。見小厮鼻口流血,抱他到房裡慢慢問他,方知為拾鞋之事惹起事來。這一丈青氣忿忿的走到後邊廚下,指東罵西,一頓海罵道:「賊不逢好死的淫婦,王八羔子!我的孩子和你有甚冤仇?他纔十一二歲,曉的甚麼?知道𣭈也在那塊兒?平白地調唆打他恁一頓,打的鼻口中流血。假若死了,淫婦、王八兒也不好!稱不了你甚麼願!」廚房裡罵了,到前邊又罵,整罵了一二日還不定。因金蓮在房中陪西門慶吃酒,還不知。

晚夕上床宿歇,西門慶見婦人脚上穿着兩隻綠紬子睡鞋,大紅提根兒,因說道:「啊呀,如何穿這個鞋在脚?恠恠的不好看。」婦人道:「我只一雙紅睡鞋,倒吃小奴才將一隻弄油了,那裡再討第二雙來?」西門慶道:「我的兒,你到明日做一雙兒穿在脚上。你不知,我達達一心歡喜穿紅鞋兒,看着心裡愛。」婦人道:「恠奴才!可哥兒的來想起一件事來,我要說,又忘了。」因令春梅:「你取那隻鞋來與他瞧。」——「你認的這鞋是誰的鞋?」西門慶道:「我不知是誰的鞋。」婦人道:「你看他還打張雞兒哩!瞞着我,黃貓黑尾,你幹的好繭兒!來旺兒媳婦子的一隻臭蹄子,寶上珠也一般,收藏在藏春塢雪洞兒裡拜帖匣子內,攪着些字紙和香兒一處放着。甚麼稀罕物件,也不當家化化的!恠不的那賊淫婦死了,墮阿鼻地獄!」{\meipi{只是家常口頭語,說來偏妙。}}又指着秋菊罵道:「這奴才當我的鞋,又翻出來,教我打了幾下。」分咐春梅:「趁早與我掠出去!」春梅把鞋掠在地下,看着秋菊說道:「賞與你穿了罷!」那秋菊拾在手裡,說道:「娘這個鞋,只好盛我一個脚指頭兒罷了。」婦人罵道:「賊奴才,還教甚麼𣭈娘哩,他是你家主子前世的娘!不然,怎的把他的鞋這等收藏的嬌貴?到明日好傳代!沒廉恥的貨!」秋菊拏着鞋就徃外走,被婦人又叫囘來,分咐:「取刀來,等我把淫婦剁作幾截子,掠到毛司裡去!叫賊淫婦陰山背後,永世不得超生!」{\meipi{又一波。寫要強婦人邪心痴妒,入骨三分,疑有鬼神供其筆墨。}}因向西門慶道:「你看着越心疼,我越發偏剁個樣兒你瞧。」西門慶笑道:「恠奴才,丟開手罷了。我那裡有這個心!」婦人道:「你沒這個心,你就賭了誓。淫婦死的不知徃那去了,你還留着他的鞋做甚麼?早晚有省,好思想他。正經俺每和你恁一場,你也沒恁個心兒,還要人和你一心一計哩!」{\meipi{到此方結出大意。}}西門慶笑道:「罷了,恠小淫婦兒,偏有這些兒的!他就在時,也沒曾在你跟前行差了礼法。」於是摟過粉項來就親了個嘴,兩個雲雨做一處。正是:

動人春色嬌還媚,惹蝶芳心軟又濃。

有詩為證:

漫吐芳心說向誰?欲於何處寄相思?相思有盡情難盡,一日都來十二時。

