\includepdf[pages={129,130},fitpaper=false]{tst.pdf}
\chapter*{第六十五囘 願同穴一時䘮禮盛 守孤靈半夜口脂香}
\addcontentsline{toc}{chapter}{第六十五囘 願同穴一時䘮禮盛 守孤靈半夜口脂香}
\markboth{{\titlename}卷之七}{第六十五囘 願同穴一時䘮禮盛 守孤靈半夜口脂香}


詩曰:

\begin{myquote} 
湘皋烟草碧紛紛,淚灑東風憶細君。\\見說嫦娥能入月,虛疑神女解為雲。\\花陰晝坐閑金剪,竹裡遊春冷翠裙。\\留得丹青殘錦在,傷心不忍讀迴文。
\end{myquote} 

話說到十月二十八日,是李瓶兒二七,玉皇廟吳道官受齋,請了十六個道衆,在家中揚幡修建齋壇。又有安郎中來下書,西門慶管待來人去了。吳道官廟中擡了三牲祭禮來,又是一疋尺頭以為奠儀。道衆遶棺傳咒,吳道官靈前展拜。西門慶與敬濟囘禮,謝道:「師父多有破費,何以克當?」吳道官道:「小道甚是惶愧,本該助一經追薦夫人,奈力薄,粗祭表意而已。」西門慶命收了,打發擡盒人囘去。那日三朝轉經,演生神章,破九幽獄,對靈攝召,整做法事,不必細說。

第二日,先是門外韓姨夫家來上祭。那時孟玉樓兄弟孟銳做買賣來家,{\pangpi{伏。}}見西門慶這邊有䘮事,跟隨韓姨夫那邊來上祭,討了一分孝去,送了許多人事。西門慶叙禮,進入玉樓房中拜見。西門慶亦設席管待,俱不在言表。

那日午間,又是本縣知縣李拱極、縣丞錢斯成、主簿任良貴、典史夏恭基,又有陽谷縣知縣狄斯朽,共五員官,都鬪了分子,穿孝服來上紙帛弔問。西門慶備席在捲棚內管待,請了吳大舅與溫秀才相陪,三個小優兒彈唱。

正飲酒到熱鬧處,忽報:「管磚廠工部黃老爹來弔孝。」慌的西門慶連忙穿孝衣靈前伺侯,溫秀才又早迎接至大門外,讓至前廳,換了衣裳進來。家人手捧香燭紙疋金段到靈前,黃主事上了香,展拜畢,西門慶同敬濟下來還禮。黃主事道:「學生不知尊閫沒了,弔遲,恕罪,恕罪!」西門慶道:「學生一向欠恭,今又承老先生賜弔,兼辱厚儀,不勝感激。」叙畢禮,讓至捲棚上面坐下。西門慶與溫秀才下邊相陪,左右捧茶上來吃了。黃主事道:「昨日宋松原多致意先生,他也聞知令夫人作過,也要來弔問,爭奈有許多事情羈絆。他如今在濟州住劄。先生還不知,朝廷如今營建艮嶽,勑令太尉朱勔,徃江南湖湘採取花石綱,運船陸續打河道中來。頭一運將到淮上。又欽差殿前六黃太尉來迎取卿雲萬態奇峰,長二丈,闊數尺,都用黃氊蓋覆,張打黃旗,費數號船隻,由山東河道而來。況河中沒水,起八郡民夫牽挽。官吏倒懸,民不聊生。{\meipi{說來若深知其為勞民傷財之事,而萬萬不願身為之者,究竟身卒為之,而勞民傷財特甚。古今具臣已為小人所為,而猶不自服其為小人,皆此類也。}}宋道長督率州縣,事事皆親身經歷,案牘如山,晝夜勞苦,通不得閑。況黃太尉不久自京而至,宋道長說,必須率三司官員,要接他一接。想此間無可相熟者,委托學生來,敬煩尊府做一東,要請六黃大尉一飯,未審尊意允否?」因喚左右:「叫你宋老爹承差上來。」有二青衣官吏跪下,氊包內捧出一對金段、一根沉香、兩根白蠟、一分綿紙。黃主事道:「此乃宋公致賻之儀。那兩封,是兩司八府官員辦酒分資——兩司官十二員、府官八員,計二十二分,共一百零六兩。」交與西門慶:「有勞盛使一備何如?」西門慶再三辭道:「學生有服在家,奈何,奈何?」因問:「迎接在於何時?」黃主事道:「還早哩,也得到出月半頭。黃太監京中還未起身。」西門慶道:「學生十月十二日纔發引。既是宋公祖與老先生分付,敢不領命!但這分資決不敢收。該多少桌席,只顧分付,學生無不畢具。」黃主事道:「四泉此意差矣!松原委托學生來煩瀆,此乃山東一省各官公禮,又非松原之己出,何得見卻?如其不納,學生即囘松原,再不敢煩瀆矣!」{\pangpi{又挾制一句,妙。}}西門慶聽了此言,說道:「學生權且領下。」因令玳安、王經接下去。問備多少桌席,黃主事道:「六黃備一張吃看大桌面,宋公與兩司都是平頭桌席,以下府官散席而已。承應樂人,自有差撥伺候,府上不必再叫。」說畢,茶湯兩換,作辭起身。

西門慶款留,黃主事道:「學生還要到尚柳塘老先生那裡拜拜,他昔年曾在學生敝處作縣令,然後轉成都府推官。如今他令郎兩泉,又與學生鄉試同年。」{\meipi{熱鬧中不廢冷案,文情如空谷幽蘭,芳香自吐。}}西門慶道:「學生不知老先生與尚兩泉相厚,兩泉亦與學生相交。」黃主事起身,西門慶道:「煩老先生多致意宋公祖,至期寒舍拱候矣。」黃主事道:「臨期,松原還差人來通報先生,亦不可太奢。」

西門慶道,「學生知道。」送出大門,上馬而去。那縣中官員,聽見黃主事帶領巡按上司人來,唬的都躲在山子下小捲棚內飲酒,分付手下把轎馬藏過一邊。{\meipi{絕平處皆是奇思,極俗事亦有畫意。}}當時,西門慶囘到捲棚與衆官相見,具說宋巡按率兩司八府來,央煩出月迎請六黃太尉之事。衆官悉言:「正是州縣不勝憂苦。這件事,欽差若來,凡一應秪迎、廩餼、公宴、器用、人夫,無不出於州縣,州縣必取之於民,公私困極,莫此為甚。我輩還望四泉於上司處美言提拔,足見厚愛。」言訖,都不久坐,告辭起身而去。

話休饒舌。到李瓶兒三七,有門外永福寺道堅長老,領十六衆上堂僧來念經,穿雲錦袈裟,戴毗盧帽,大鈸大鼓,甚是齊整。{\meipi{一味點綴,炫人耳目。}}十月初八日是四七,請西門外寶慶寺趙喇嘛,亦十六衆,來念番經,結壇,跳沙,灑花米,行香,口誦眞言。齋供都用牛乳茶酪之類,懸掛都是九醜天魔變相,身披纓絡琉璃,項掛髑髏,口咬嬰兒,坐跨妖魅,腰纏蛇螭,或四頭八臂,或手執戈戟,朱髮藍面,醜惡莫比。午齋以後,就動葷酒。西門慶那日不在家,同陰陽徐先生徃墳上破土開壙去了,後晌方囘。晚夕,打發喇嘛散了。次日,推運山頭酒米、桌面餚品一應所用之物,又委付主管夥計,庄上前後搭棚,墳內穴邊又起三間罩棚。先請附近地隣來,大酒大肉管待。臨散,皆肩背項負而歸,{\meipi{鋪敍處蓋欲極其盛而言之。}}俱不必細說。十一日白日,先是歌郎並鑼鼓地弔來靈前叅靈,弔《五鬼鬧判》、《張天師着鬼迷》、《鍾馗戲小鬼》、《老子過函關》、《六賊鬧彌陀》、《雪裡梅》、《莊周夢蝴蝶》、《天王降地水火風》、《洞賓飛劍斬黃龍》、《趙太祖千里送荊娘》,各樣百戲弔罷,堂客都在簾內觀看。叅罷靈去了,內外親戚都來辭靈燒紙,大哭一場。

到次日發引,先絕早擡出名旌、各項幡亭紙紮,僧道、鼓手、細樂、人役都來伺候。西門慶預先問帥府周守備討了五十名巡捕軍士,都帶弓馬,全裝結束。留十名在家看守,四十名在材邊擺馬道,分兩翼而行。衙門裡又是二十名排軍打路,照管冥器。墳頭又是二十名把門,管收祭祀。那日官員士夫、親隣朋友來送殯者,車馬喧呼,填街塞巷。本家並親眷轎子也有百十餘頂,三院鴇子粉頭小轎也有數十。徐陰陽擇定辰時起棺,西門慶留下孫雪娥並二女僧看家,平安兒同兩名排軍把前門。女婿陳敬濟跪在柩前摔盆,{\pangpi{活悔氣。}}六十四人上扛,有仵作一員官立於增架上敲響板,指撥擡材人上肩。先是請了報恩寺僧官來起棺,轉過大街口望南走。兩邊觀看的人山人海。那日正値晴明天氣,果然好殯。但見:

\begin{myquote}
和風開綺陌,細雨潤芳塵,東方曉日初陞,北陸殘烟乍斂。鼕鼕嚨嚨,花䘮鼓不住聲喧;叮叮噹噹,地弔鑼連宵振作。銘旌招颭,大書九尺紅羅;起火軒天,冲散半天黃霧。猙猙獰獰,開路鬼斜担金斧;忽忽洋洋,險道神端秉銀戈。逍逍遙遙八洞仙,龜鶴遶定;窈窈窕窕四毛女,虎鹿相隨。熱熱鬧鬧採蓮船,撒科打諢;長長大大高蹺漢,貫甲頂盔。清清秀秀小道童,一十六衆,都是霞衣道髻,動一派之仙音;肥肥胖胖大和尚,二十四個,個個都是雲錦袈裟,轉五方之法事。一十二座大絹亭,亭亭皆綠舞紅飛;二十四座小絹亭,座座盡珠圍翠遶。左勢下,天倉與地庫相連;右勢下,金山與銀山作隊。掌醢廚,列八珍之礶;香燭亭,供三獻之儀。六座百花亭,現千團錦綉;一乘引魂轎,紮百結黃絲。這邊把花與雪柳爭輝,那邊寶蓋與銀幢作隊。金字幡,銀字幡,緊護棺輿;白絹繖,綠絹繖,同圍增架。功布招颭,孝眷聲哀。打路排軍,執欖杆前後呼擁;迎䘮神會,耍武藝左右盤旋。賣解猶如鷹鷂,走馬好似猿猴。豎肩樁,打斤斗,隔肚穿錢,金雞獨立。人人喝彩,個個爭誇。扶肩擠背,不辨賢愚;挨睹並觀,那分貴賤!張三蠢胖,只把氣籲;李四矮矬,頻將脚跕。白頭老叟,盡將柺棒拄髭鬚;綠鬢佳人,也帶兒童來看殯。
\end{myquote}

吳月娘與李嬌兒等本家轎子十餘頂,一字兒緊跟材後。西門慶總冠孝服同衆親朋在材後,陳敬濟緊扶棺輿,走出東街口。西門慶具禮,請玉皇廟吳道官來懸眞。身穿大紅五彩鶴氅,頭戴九陽雷巾,脚登丹舄,手執牙笏,坐在四人肩輿上,迎殯而來。將李瓶兒大影捧於手內,陳敬濟跪在前面,那殯停住了。衆人聽他在上高聲宣念:

\begin{myquote}[\markfont]
恭惟

故錦衣西門恭人李氏之靈,存日陽年二十七歲,元命辛未相,正月十五日午時受生,大限於政和七年九月十七日丑時分身故。伏以尊靈,名家秀質,綺閣嬌姝。稟花月之儀容,蘊蕙蘭之佳氣。鬱德柔婉,賦性溫和。配我西君,克諧伉儷。處閨門而賢淑,資琴瑟以好和。曾種藍田,尋嗟楚畹。正宜享福百年,可惜春光三九。{\pangpi{句巧。}}嗚呼!明月易缺,好物難全。善類無常,修短有數。今日棺輿載道,丹旆迎風,良夫躃踴於柩前,孝眷哀矜於巷陌。離別情深而難已,音容日遠以日忘。某等謬忝冠簪,愧領玄教。愧無新垣平之神術,{\pangpi{用得當。}}恪遵玄元始之遺風。徒展崔巍鏡裡之容,難返莊周夢中之蝶。漱甘露而沃瓊漿,超知識登於紫府;披百寶而面七眞,引淨魄出於冥途。一心無掛,四大皆空。苦,苦,苦!氣化清風形歸土。一靈眞性去弗囘,改頭換面無遍數。

衆聽末後一句:咦!精爽不知何處去,眞容留與後人看。
\end{myquote}

吳道官念畢,端坐轎上,那轎捲坐退下去了。這裡鼓樂喧天,哀聲動地,殯纔起身,迤邐出南門。衆親朋陪西門慶,走至門上方乘馬,陳敬濟扶柩,到於山頭五里原。原來坐營張團練,帶領二百名軍,同劉、薛二內相,又早在墳前高阜處搭帳房,吹響器,打銅鑼銅鼓,迎接殯到,{\meipi{以瓶兒之為人,在西門慶妻妾中似不應獲早死之報,不知早死正瓶兒之福。知此方可論因果報應。}}看着裝燒冥器紙紮,烟焰漲天。棺輿到山下扛,徐先生率仵作,依羅經弔向,巳時祭告后土方隅後,纔下葬掩土。西門慶易服,備一對尺頭禮,請帥府周守備點主。衛中官員並親朋夥計,皆爭拉西門慶遞酒,鼓樂喧天,烟火匝地,熱鬧豐盛,不必細說。吃畢,後晌囘靈,吳月娘坐魂轎,抱神主、魂幡,陳敬濟扶靈床,鼓手細樂十六衆小道童兩邊吹打。吳大舅並喬大戶、吳二舅、花大舅、沈姨夫、孟二舅、應伯爵、謝希大、溫秀才、衆主管夥計,都陪着西門慶進城,堂客轎子壓後,到家門首燎火而入。

李瓶兒房中安靈已畢,徐先生前廳祭神灑掃,各門戶皆貼闢非黃符。謝徐先生一疋尺頭、五兩銀子出門,各項人役打發散了。又拏出二十弔錢來,五弔賞巡捕軍人,五弔與衙門中排軍,十弔賞營里人馬。拏帖兒囘謝周守備、張團練、夏提刑,俱不在話下。西門慶還要留喬大戶、吳大舅衆人坐,衆人都不肯,作辭起身。來保進說:「搭棚在外伺候,明日來拆棚。」西門慶道:「棚且不消拆,亦發過了你宋老爹擺酒日子來拆罷。」打發搭綵匠去了。後邊花大娘子與喬大戶娘子衆堂客,還等着安畢靈,哭了一場,方纔去了。

西門慶不忍遽捨,晚夕還來李瓶兒房中,要伴靈宿歇。見靈床安在正面,大影掛在旁邊,靈床內安着半身,裡面小錦被褥,床几、衣服、粧奩之類,無不畢具,下邊放着他的一對小小金蓮,{\pangpi{寫出傷心。}}桌上香花燈燭、金碟樽俎,般般供養,西門慶大哭不止。令迎春就在對面炕上搭鋪,到夜半,對着孤燈,半窓斜月,翻覆無寐,長吁短嘆,思想佳人。有詩為證:

\begin{myquote} 
短嘆長吁對鎖窓,舞鸞孤影寸心傷。\\蘭枯楚畹三秋雨,楓落吳江一夜霜。\\夙世已違連理願,此生難覓返魂香。\\九泉果有精靈在,地下人間兩斷腸。
\end{myquote} 

白日間供養茶飯,西門慶俱親看着丫鬟擺下,他便對面和他同吃。舉起筯兒來:「你請些飯兒!」{\pangpi{更寫得傷心。}}行如在之禮。丫鬟養娘都忍不住掩淚而哭。奶子如意兒,無人處常在跟前遞茶遞水,挨挨搶搶,掐掐捏捏,插話兒應答,那消三夜兩夜。{\meipi{「那消」二字,甚言此道感人之易。}}這日,西門慶因請了許多官客堂客,墳上煖墓來家,陪人吃得醉了。進來,迎春打發歇下。到夜間要茶吃,叫迎春不應,如意兒便來遞茶。因見被拖下炕來,接過茶盞,用手扶被,西門慶一時興動,摟過脖子就親了個嘴,遞舌頭在他口內。老婆就咂起來,一聲兒不言語。西門慶令脫去衣服上炕,兩個摟在被窩內,不勝歡娛,雲雨一處。老婆說:「既是爹擡舉,娘也沒了,小媳婦情願不出爹家門,隨爹收用便了。」西門慶便叫:「我兒,你只用心伏侍我,愁養活不過你來!」這老婆聽了,枕蓆之間,無不奉承,顛鸞倒鳳,隨手而轉,{\pangpi{談何容易。}}把西門慶歡喜的要不的。

次日,老婆早晨起來,與西門慶拏鞋脚,疊被褥,就不靠迎春,極盡殷勤,無所不至。西門慶開門尋出李瓶兒四根簪兒來賞他,老婆磕頭謝了。迎春知收用了他,兩個打成一路。老婆自恃得寵,脚跟已牢,無復求告於人,就不同徃日,打扮喬模喬樣,在丫鬟夥內,說也有,笑也有。{\pangpi{淺人徃徃如此。}}早被潘金蓮看在眼裡。

早晨,西門慶正陪應伯爵坐的,忽報宋御史差人來送賀黃太尉一桌金銀酒器:兩把金壺、兩副金臺盞、十副小銀鍾、兩副銀折盂、四副銀賞鍾;兩疋大紅彩蟒、兩疋金段、十罈酒、兩牽羊。傳報:「太尉船隻已到東昌地方,煩老爹這裡早備酒席,準在十八日迎請。」西門慶收入明白,與了來人一兩銀子,用手本打發囘去。隨即兌銀與賁四、來興兒,定桌面,粘菓品,買辦整理,不必細說。因向伯爵說:「自從他不好起,到而今,我再沒一日兒心閑。剛剛打發䘮事出去了,又鑽出這等勾當來,教我手忙脚亂。」{\pangpi{分明快心事,卻作埋怨說,酷肖。}}伯爵道:「這個哥不消抱怨,你又不曾兜攬他,他上門兒來央煩你。雖然你這席酒替他陪幾兩銀子,到明日,休說朝廷一位欽差殿前大太尉來咱家坐一坐,只這山東一省官員,並巡撫巡按、人馬散級,也與咱門戶添許多光輝。」{\meipi{數語擅艷,幾垂天下之涎。以灌夫之意氣,而猶以丞相過竇嬰為榮,未免此見,則士之恠人薰灼者有幾,何況伯爵。}}西門慶道:「不是此說,我承望他到二十已外也罷,不想十八日就迎接,忒促急促忙。這日又是他五七,我已與了吳道官寫法銀子去了,如何又改!不然,雙頭火杖都擠在一處,怎亂得過來?」應伯爵道:「這個不打緊,我算來,嫂子是九月十七日沒了,此月二十一日正是五七。你十八日擺了酒,二十日與嫂子念經也不遲。」西門慶道:「你說的是,我就使小厮囘吳道官改日子去。」伯爵道:「哥,我又一件:東京黃眞人,朝廷差他來泰安州進金鈴吊掛御香,建七晝夜羅天大醮,如今在廟裡住。趁他未起身,倒好教吳道官請他那日來做高功,領行法事。咱圖他個名聲,也好看。」{\pangpi{一味好名。}}西門慶道:「都說這黃眞人有利益,{\pangpi{想到實處,畢竟情深。}}請他到好,爭奈吳道官齋日受他祭禮,出殯又起動他懸眞,道童送殯,沒的酬謝他,教他念這個經兒,表意而已。今又請黃眞人主行,卻不難為他?」伯爵道:「齋一般還是他受,只教他請黃眞人做高功就是了。哥只多費幾兩銀子,為嫂子,沒曾為了別人。」{\meipi{事從其心所欲,直言不避,賊甚。}}西門慶一面教陳敬濟寫帖子,又多封了五兩銀子,教他早請黃眞人,改在二十日念經,二十四衆道士,水火練度一晝夜。即令玳安騎頭口去了。

西門慶打發伯爵去訖,進入後邊。只見吳月娘說:「賁四嫂買了兩個盒兒,他女兒長姐定與人家,來磕頭。」西門慶便問:「誰家?」賁四娘子領他女兒,穿着大紅段襖兒、黃紬裙子,戴着花翠,插燭向西門慶磕了四個頭。月娘在旁說:「咱也不知道,原來這孩子與了夏大人房裡擡舉,昨日纔相定下。{\meipi{亦是冷案,似乎可省,然細觀首尾,方知其妙。}}這二十四日就娶過門,只得了他三十兩銀子。論起來,這孩子倒也好身量,不相十五歲,到有十六七歲的。多少時不見,就長的成成的。」西門慶道:「他前日在酒席上和我說,要擡舉兩個孩子學彈唱,不知你家孩子與了他。」於是教月娘讓至房內,擺茶留坐。落後,李嬌兒、孟玉樓、潘金蓮、孫雪娥、大姐都來見禮陪坐。臨去,月娘與了一套重絹衣服、一兩銀子,李嬌兒衆人都有與花翠、汗巾、脂粉之類。晚上,玳安囘話:「吳道官收了銀子,知道了。黃眞人還在廟裡住,過二十頭纔囘東京去。十九日早來鋪設壇場。」

西門慶次日,家中廚役落作治辦酒席,務要齊整,大門上紮七級彩山,廳前五級彩山。十七日,宋御史差委兩員縣官來觀看筵席:廳正面,屏開孔雀,地匝氍毹,都是錦綉桌幃,粧花椅甸。黃太尉便是肘件大飯簇盤、定勝方糖,吃看大插桌;觀席兩張小插桌,是巡撫、巡按陪坐;兩邊布按三司,有桌席列坐。其餘八府官,都在廳外棚內兩邊,只是五菓五菜平頭桌席。看畢,西門慶待茶,起身囘話去了。

到次日,撫按率領多官人馬,早迎到船上,張打黃旗「欽差」二字,捧着勑書在頭裡走,地方統制、守禦、都監、團練,各衛掌印武官,皆戎服甲冑,各領所部人馬,圍隨儀杖,擺數裡之遠。黃太尉穿大紅五彩雙掛繡蟒,坐八擡八簇銀頂暖轎,張打茶褐傘。後邊名下執事人役跟隨無數,皆駿騎咆哮,如萬花之燦錦,隨鼓吹而行。黃土塾道,雞犬不聞,樵採遁跡。人馬過東平府,進清河縣,縣官黑壓壓跪於道旁迎接,左右喝叱起去。隨路傳報,直到西門慶門首。教坊鼓樂,聲震雲霄,兩邊執事人役皆青衣排伏,雁翅而列。西門慶青衣冠冕,望塵拱伺。{\pangpi{隱約其詞。}}良久,人馬過盡,太尉落轎進來,後面撫按率領大小官員,一擁而入。到於廳上,又是箏、𥱧、方晌、雲璈、龍笛、鳳管,細樂响動。為首就是山東巡撫都御史侯蒙、巡按監察御史宋喬年叅見,大尉還依禮答之。其次就是山東左布政龔共、左叅政何其高、右布政陳四箴、右叅政季侃廷、叅議馮廷鵠、右叅議汪伯彥、廉使趙訥、採訪使韓文光、提學副使陳正彙、兵備副使雷啟元等兩司官叅見,太尉稍加優禮。及至東昌府徐崧、東平府胡師文、兗州府淩雲翼、徐州府韓邦奇、濟南府張叔夜、青州府王士奇、登州府黃甲、萊州府葉遷等八府官行廳叅之禮,太尉答以長揖而已。至於統制、制置、守禦、都監、團練等官,太尉則端坐。各官聽其發放,外邊伺候。然後,西門慶與夏提刑上來拜見獻茶,{\meipi{此時各官禮貌已如此,而為西門慶諱不可,誇不可,最難下筆。此只以拜見獻茶一混,又若誇,又若諱,絕妙躲閃之法。}}侯巡撫、宋巡按向前把盞,下邊動鼓樂,來與太尉簪金花,捧玉斝,彼此酬飲。遞酒已畢,太尉正席坐下,撫按下邊主席,其餘官員並西門慶等,各依次第坐了。教坊伶官遞上手本奏樂,一應彈唱隊舞,各有節次,極盡聲容之盛。當筵搬演《裴晉公還帶記》,一折下來,廚役割獻燒鹿、花豬、百寶攢湯、大飯燒賣。又有四員伶官,箏、𥱧、琵琶、箜篌,上來清彈小唱。唱畢,湯未兩陳,樂已三奏。下邊跟從執事人等,宋御史差兩員州官,在西門慶捲棚內,自有桌席管待。守禦、都監等官,西門慶都安在前邊客位,自有坐處。黃太尉令左右,拏十兩銀子來賞賜各項人役,隨即看轎起身。衆官再三款留不住,即送出大門。鼓樂笙簧迭奏,兩街儀衛喧闐,清蹕傳道,人馬森列。多官俱上馬遠送,太尉悉令免之,舉手上轎而去。{\meipi{席終賓主不交一言,寫出勢分所臨,元無情義,徒以套禮尊拱而已。}}宋御史、候巡撫分付都監以下軍衛有司,直護送至皇船上來囘話。桌面器皿,答賀羊酒,具手本差東平府知府胡師文與守禦周秀,親送到船所,交付明白。囘至廳上,拜謝西門慶說:「今日負累取擾,深感,深感!分資有所不足,容當奉補。」西門慶慌躬身施禮道:「卑職重承教愛,累辱盛儀,日昨又蒙賻禮,蝸居卑陋,猶恐有不到處,萬望公祖諒宥,幸甚!」宋御史謝畢,即令左右看轎,與候巡撫一同起身,兩司八府官員皆拜辭而去。各項人役,一鬨而散。西門慶囘至廳上,將伶官樂人賞以酒食,俱令散了,止留下四名官身小優兒伺候。廳內外各官桌面,自有本官手下人領,{\pangpi{絕不漏空。}}不題。

西門慶見天色尚早,收拾家伙停當,攢下四張桌席,使人請吳大舅、應伯爵、謝希大、溫秀才、傅自新、甘出身、韓道國、賁四、崔本及女婿陳敬濟,從五更起來,各項照管辛苦,坐飲三盃。不一時,衆人來到,擺上酒來飲酒。伯爵道:「哥,今日黃太尉坐了多大一囘?歡喜不歡喜?」韓道國道:「今日六黃老公公見咱家酒席齊整,無個不歡喜的。巡撫、巡按兩位甚是知感不盡,謝了又謝。」伯爵道:「若是第二家擺這席酒也成不的,也沒咱家恁大地方,也沒府上這些人手。今日少說也有上千人進來,都要管待出去。哥就陪了幾兩銀子,咱山東一省也響出名去了。」溫秀才道:「學生宗主提學陳老先生,也在這裡預席。」{\meipi{秀才便講宗師,若誇矣。而不知其為聞見之陋,可發一笑。}}西門慶問其名,溫秀才道:「名陳正彙者,乃諫垣陳了翁先生乃郎,本貫河南鄄城縣人,十八歲科舉,中壬辰進士,今任本處提學副使,極有學問。」西門慶道:「他今年纔二十四歲。」正說着,湯飯上來。衆人吃畢,西門慶叫上四個小優兒,問道:「你四人叫甚名字?」答道:「小的叫周採、梁鐸、馬眞、韓畢。」伯爵道:「你不是韓金釧兒一家?」韓畢跪下說道:「金釧兒、玉釧兒是小的妹子。」西門慶因想起李瓶兒來:「今日擺酒,就不見他。」分付小優兒:「你們拏樂器過來,唱個『洛陽花,梁園月』我聽。」韓畢與周採一面搊箏撥阮,唱道:

\begin{myquote} 
{\markfont{〔普天樂〕}}洛陽花,梁園月。好花須買,皓月須賒。花倚欄杆看爛熳開,月曾把酒問團圞夜。月有盈虧,花有開謝。想人生最苦離別。花謝了,三春近也;月缺了,中秋到也;人去了,何日來也?{\meipi{此曲詞旨甚悲,雖歡時亦不堪讀。}}
\end{myquote} 

唱畢,應伯爵見西門慶眼裡酸酸的,便道:「哥教唱此曲,莫非想起過世嫂子來?」西門慶看見後邊上菓碟兒,叫:「應二哥,你只嗔我說,有他在,就是他經手整定。從他沒了,隨着丫鬟撮弄,你看相甚模樣?好應口菜也沒一根我吃!」{\meipi{字字從深情中流出,卻妙在一字不切,若切便淺。}}溫秀才道:「這等盛設,老先生中餽也不謂無人,足可以勾了。」伯爵道:「哥休說此話。你心間疼不過,便是這等說,恐一時冷淡了別的嫂子們心。」{\meipi{一語同一意,而口角各肖其人,化工之手。}}這裡酒席上說話,不想潘金蓮在軟壁後聽唱,聽見西門慶說此話,走到後邊,一五一十告訴月娘。月娘道:「隨他說去就是了,你如今卻怎樣的?前日他在時,即許下把綉春教伏侍李嬌兒,他到睜着眼與我叫,說:『死了多少時,就分散他房裡丫頭!』教我就一聲兒再沒言語。這兩日憑着他那媳婦子和兩個丫頭,狂的有些樣兒?我但開口,就說咱們擠撮他。」金蓮道:「這老婆這兩日有些別改模樣,只怕賊沒廉恥貨,鎭日在那屋裡,纏了這老婆也不見的。我聽見說,前日與了他兩對簪子,老婆戴在頭上,拏與這個瞧,拏與那個瞧。」月娘道:「『豆芽菜兒——有甚捆兒!』」衆人背地裡都不喜歡。正是:

\begin{myquote} 
遺蹤堪入時人眼,多買胭脂畫牡丹。
\end{myquote} 

