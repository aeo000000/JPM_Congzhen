\part*{{\titlename}卷之四}
\addcontentsline{toc}{part}{{\titlename}卷之四}


\includepdf[pages={61,62},fitpaper=false]{tst.pdf}
\chapter*{第三十一囘 琴童兒藏壺搆釁 西門慶開宴為歡}
\addcontentsline{toc}{chapter}{第三十一囘 琴童兒藏壺搆釁 西門慶開宴為歡}
\markboth{{\titlename}卷之四}{第三十一囘 琴童兒藏壺搆釁 西門慶開宴為歡}


詩曰:

\begin{myquote}
幽情憐獨夜,花事復相催。\\欲使春心醉,先教玉友來。\\濃香猶帶膩,紅暈漸分腮。\\莫醒沉酣恨,朝雲逐夢囘。
\end{myquote}

話說西門慶,次日使來保提刑所下文書。一面使人做官帽,又喚趙裁裁剪尺頭,攢造圓領,又叫許多匠人,釘了七八條帶。不說西門慶家中熱亂,且說吳典恩那日走到應伯爵家,把做驛丞之事,再三央及伯爵,要問西門慶錯銀子,上下使用,許伯爵十兩銀子相謝,說着跪在地下。慌的伯爵拉起,說道:「此是成人之美,大官人攜帶你得此前程,也不是尋常小可。」因問:「你如今所用多少勾了?」吳典恩道:「不瞞老兄說,我家活人家,一文錢也沒有。到明日上任叅官贄見之禮,連擺酒,並治衣類鞍馬,少說也得七八十兩銀子。如今我寫了一紙文書在此,也沒敢下數兒。望老兄好歹扶持小人,事成恩有重報。」伯爵看了文書,因說:「吳二哥,你借出這七八十兩銀子來也不勾使。依我,取筆來寫上一百兩。恆是看我面,不要你利錢,你且得手使了。到明日做了官,慢慢陸續還他也不遲。俗語說得好:『借米下得鍋,討米下不得鍋。』哄了一日是兩晌。」{\pangpi{先開賴債門。}}吳典恩聽了,謝了又謝。於是把文書上塡寫了一百兩之數。兩個吃了茶,一同起身,來到西門慶門首。平安兒通報了,二人進入裡面,見有許多裁縫匠人七手八脚做生活。西門慶和陳敬濟在穿廊下,看着寫見官手本揭帖,見二人,作揖讓坐。伯爵問道:「哥的手本劄付,下了不曾?」西門慶道:「今早使小价徃提刑府下劄付去了。還有東平府並本縣手本,如今正要叫賁四去下。」說畢,畫童兒拏上茶來。吃畢茶,那應伯爵並不提吳主管之事,{\pangpi{有竅。}}走下來且看匠人釘帶。西門慶見他拏起帶來看,就賣弄說道:「你看我尋的這幾條帶如何?」伯爵極口稱赞誇獎道:「虧哥那裡尋的,都是一條賽一條的好帶,難得這般寬大。別的倒也罷了,自這條犀角帶並鶴頂紅,就是滿京城拏着銀子也尋不出來。不是面獎,就是東京衛主老爺,玉帶金帶空有,也沒這條犀角帶。這是水犀角,不是旱犀角。旱犀角不値錢。水犀角號作通天犀。你不信,取一碗水,把犀角放在水內,分水為兩處,此為無價之寶。」{\meipi{先只奉承,暢其歡心,心一歡,便容易打入,絕妙騙法。}}因問:「哥,你使了多少銀子尋的?」西門慶道:「你們試估估價値。」伯爵道:「這個有甚行款,我每怎麼估得出來!」西門慶道:「我對你說了罷,此帶是大街上王昭宣府裡的帶。昨日一個人聽見我這裡要,巴巴來對我說。我着賁四拏了七十兩銀子,再三囘了來。他家還張致不肯,定要一百兩。」伯爵道:「難得這等寬樣好看。哥,你明日系出去,甚是霍綽。就是你同僚間,見了也愛。」誇美了一囘,坐下。西門慶便向吳主管問道:「你的文書下了不曾?」{\meipi{捱西門慶先開口,尤妙。}}伯爵道:「吳二哥正要下文書,今日巴巴的央我來激煩你。蒙你照顧他徃東京押生辰担,雖是太師與了他這個前程,就是你擡舉他一般,也是他各人造化。說不的,一品至九品都是朝廷臣子。但他告我說,如今上任,見官擺酒,並治衣服之類,共要許多銀子使,那處活變去?一客不煩二主,沒奈何,哥看我面,{\pangpi{又插入情分。}}有銀子借與他幾兩,率性賙濟了這些事兒。他到明日做上官,就啣環結草也不敢忘了哥大恩!休說他舊在哥門下出入,就是外京外府官吏,哥也不知拔濟了多少。不然,你教他那裡區處去?」{\meipi{稱恩頌德,說得人快甚,不由不借。哄騙財主,非此等口嘴不能。}}因說道:「吳二哥,你拏出那符兒來,{\pangpi{好口角。}}與你大官人瞧。」這吳典恩連忙向懷中取出,遞與西門慶觀看。見上面借一百兩銀子,中人就是應伯爵,每月利行五分。西門慶取筆把利錢抹了,說道:「既是應二哥作保,你明日只還我一百兩本錢就是了。我料你上下也得這些銀子攪纏。」於是把文書收了。纔待後邊取銀子去,忽有夏提刑拏帖兒差了一名寫字的,拏手本三班送了二十名排軍來答應,{\pangpi{是武官行徑。}}就問討上任日期,討問字型大小,衙門同僚具公禮來賀。西門慶教陰陽徐先生擇定七月初二日辰時到任,拏帖兒囘夏提刑,賞了寫字的五錢銀子。正打發出門去了,只見陳敬濟拏着一百兩銀子出來,教與吳主管,說:「吳二哥,你明日只還我本錢便了。」那吳典恩一面接了銀在手,叩頭謝了。西門慶道:「我不留你坐罷,你家中執你的事去。留下應二哥,我還和你說句話兒。」那吳典恩拏着銀子,歡喜出門。看官聽說:後來西門慶死了,家中時敗勢衰,吳月娘守寡,被平安兒偸盜出解當庫頭面,在南瓦子裡宿娼,被吳驛丞拏住,教他指攀吳月娘與玳安有姦,要羅織月娘出官,恩將仇報。此係後事,表過不題。正是:

\begin{myquote}
不結子花休要種,無義之人不可交。
\end{myquote}

那時賁四徃東平府並本縣下了手本來囘話,西門慶留他和應伯爵,陪陰陽徐先生擺飯。正吃着飯,只見吳大舅來拜望,徐先生就起身。良久,應伯爵也作辭出門,來到吳主管家。吳典恩早封下十兩保頭錢,雙手遞與伯爵,磕下頭去。伯爵道:「若不是我那等取巧說着,會勝不肯與借與你。」{\pangpi{寔寔虧他。}}吳典恩酬謝了伯爵,治辦官帶衣類,擇日見官上任不題。那時本縣正堂李知縣,會了四衙同僚,差人送羊酒賀禮來,又拏帖兒送了一名小郎來答應。年方一十八歲,本貫蘇州府常熟縣人,喚名小張松。原是縣中門子出身,生得清俊,面如傅粉,齒白唇紅;又識字會寫,善能歌唱南曲;穿着青綃直綴,涼鞋淨襪。西門慶一見小郎伶俐,滿心歡喜,就拏拜帖囘覆李知縣,留下他在家答應,改喚了名字叫作書童兒。與他做了一身衣服,新鞋新帽,不教他跟馬,教他專管書房,收禮帖,拏花園門鑰匙。祝實念又舉保了一個十四歲小厮來答應,亦改名棋童,每日派定和琴童兒兩個背書袋、夾拜帖匣跟馬。到了上任日期,在衙門中擺大酒席桌面,出票拘集三院樂工承應吹打彈唱。此時李銘也夾在中間來了,後堂飲酒,日暮時分散歸。每日騎着大白馬,頭戴烏紗,身穿五彩灑線揉頭獅子補子員領,四指大寬萌金茄楠香帶,粉底皁靴,排軍喝道,張打着大黑扇,前呼後擁,何止十數人跟隨,在街上搖擺。{\meipi{鋪敍中隱隱寫出小人負日乘光景。}}上任囘來,先拜本府縣帥府都監,並清河左右衛同僚官,然後新朋隣舍,何等榮耀施為!家中收禮接帖子,一日不斷。正是:

\begin{myquote}
白馬紅纓色色新,不來親者強來親。\\時來頑鐵生光彩,運去良金不發明。
\end{myquote}

西門慶自從到任以來,每日坐提刑院衙門中,陞廳畫卯,問理公事。光陰迅速,不覺李瓶兒坐褥一月將滿。吳大妗子、二妗子、楊姑娘、潘姥姥、吳大姨、喬大戶娘子,許多親隣堂客女眷,都送禮來,與官哥兒做彌月。院中李桂姐、吳銀兒見西門慶做了提刑所千戶,家中又生了子,亦送大禮,坐轎子來慶賀。西門慶那日在前邊大廳上擺設筵席,請堂客飲酒。春梅、迎春、玉簫、蘭香都打扮起來,在席前斟酒執壺。

原來西門慶每日從衙門中來,只到外邊廳上就脫了衣服,教書童疊了,安在書房中,止帶着冠帽進後邊去。到次日起來,旋使丫鬟來書房中取。{\pangpi{徃徃自開端。}}新近收拾大廳西廂房一間做書房,內安床几、桌椅、屏幃、筆硯、琴書之類。書童兒晚夕只在床脚踏板上鋪着鋪睡。西門慶或在那房裡歇,早晨就使出那房裡丫鬟來前邊取衣服。取來取去,不想這小郎本是門子出身,生的伶俐清俊,與各房丫頭打牙犯嘴慣熟,於是暗和上房裡玉簫兩個嘲戲上了。那日也是合當有事,這小郎正起來,在窓戶臺上擱着鏡兒梳頭,拏紅繩紮頭髮。不料玉簫推開門進來,看見說道:「好賊囚,你這咱還描眉畫眼的,爹吃了粥便出來。」書童也不理,只顧紮包髻兒。玉簫道:「爹的衣服疊了,在那裡放着哩?」書童道:「在床南頭安放着哩。」玉簫道:「他今日不穿這一套。分咐我教問你要那件玄色匾金補子、絲布員領、玉色襯衣穿。」書童道:「那衣服在廚櫃裡。我昨日纔收了,今日又要穿他。姐,你自開門取了去。」那玉簫且不拏衣服,走來跟前看着他紮頭,戲道:「恠賊囚,也象老婆般拏紅繩紮着頭兒,梳的𩬆虛籠籠的!」{\meipi{騷丫頭意態宛然。}}因見他白滾紗漂白布汗褂兒上繫着一個銀紅紗香袋兒,一個綠紗香袋兒,就說道:「你與我這個銀紅的罷!」書童道:「人家個愛物兒,你就要。」玉簫道:「你小厮家帶不的這銀紅的,只好我帶。」{\pangpi{自認丫頭。}}書童道:「早是這個罷了,倘是個漢子兒,你也愛他罷?」{\meipi{愛香袋正是愛漢子。}}被玉簫故意向他肩膀上擰了一把,說道:「賊囚,你『夾道賣門神——看出來的好畫兒』。」不繇分說,把兩個香袋子等不的解,都揪斷系兒,放在袖子內。{\pangpi{寫出賤相。}}書童道:「你好不尊貴,把人的帶子也揪斷。」被玉簫發訕,一拳一把,戲打在身上。打的書童急了,說:「姐,你休鬼混我,待我紮上這頭髮着!」玉簫道:「我且問你,沒聽見爹今日徃那去?」書童道:「爹今日與縣中華主簿老爹送行,在皇庄薛公公那裡擺酒,來家只怕要下午時分,又聽見會下應二叔,今日兌銀子,要買對門喬大戶家房子,那裡吃酒罷了。」玉簫道:「等住囘,你休徃那去了,我來和你說話。」書童道:「我知道。」玉簫於是與他約會下,纔拏衣服徃後邊去了。

少頃,西門慶出來,就叫書童,分咐:「在家,別徃那去了,先寫十二個請帖兒,都用大紅紙封套,二十八日請官客吃慶官哥兒酒;教來興兒買辦東西,添廚役茶酒,預備桌面齊整;玳安和兩名排軍送帖兒,叫唱的;留下琴童兒在堂客面前管酒。」分咐畢,西門慶上馬送行去了。吳月娘衆姊妹,請堂客到齊了,先在捲棚擺茶,然後大廳上屏開孔雀,褥隱芙蓉,上坐。席間叫了四個妓女彈唱。果然西門慶到午後時分來家,家中安排一食盒酒菜,邀了應伯爵和陳敬濟,兌了七百兩銀子,徃對門喬大戶家成房子去了。堂客正飲酒中間,只見玉簫拏下一銀執壺酒並四個梨、一個柑子,逕來廂房中送與書童兒吃。推開門,不想書童兒不在裡面,恐人看見,連壺放下,就出來了。可霎作恠,琴童兒正在上邊看酒,冷眼睃見玉簫進書房裡去,半日出來,只知有書童兒在裡邊,三不知叉進去瞧。不想書童兒外邊去,不曾進來,一壺熱酒和菓子還放在床底下。這琴童連忙把菓子藏在袖裡,將那一壺酒,影着身子,一直提到李瓶兒房裡。只見奶子如意兒和綉春在屋裡看哥兒。琴童進門就問:「姐在那裡?」綉春道:「他在上邊與娘斟酒哩。你問他怎的?」琴童兒道:「我有個好的兒,教他替我收着。」綉春問他甚麼,他又不拏出來。正說着,迎春從上邊拏下一盤子燒鵝肉、一碟玉米麵玫瑰果餡蒸餅兒與奶子吃,看見便道:「賊囚,你在這裡笑甚麼,不在上邊看酒?」那琴童方纔把壺從衣裳底下拏出來,教迎春:「姐,你與我收了。」迎春道:「此是上邊篩酒的執壺,你平白拏來做甚麼?」琴童道:「姐,你休管他。此是上房裡玉簫,和書童兒小厮,七個八個,偸了這壺酒和些柑子、梨,送到書房中與他吃。我趕眼不見,戲了他的來。你只與我好生收着,隨問甚麼人來抓尋,休拏出來。我且拾了白財兒着!」因把梨和柑子掏出來與迎春瞧,迎春道:「等住囘抓尋壺反亂,你就承當?」琴童道:「我又沒偸他的壺。各人當場者亂,隔壁心寬,管我腿事!」說畢,揚長去了。迎春把壺藏放在裡間桌子上,不題。

至晚,酒席上人散,查收家伙,少了一把壺。玉簫徃書房中尋,那裡得來!問書童,說:「我外邊有事去,不知道。」那玉簫就慌了,一口推在小玉身上。小玉罵道:「㒲昏了你這淫婦!我後邊看茶,你抱着執壺,在席間與娘斟酒。這囘不見了壺兒,你來賴我!」向各處都抓尋不着。良久,李瓶兒到房來,迎春如此這般告訴:「琴童兒拏了一把進來,教我替他收着。」李瓶兒道:「這囚根子,他做甚麼拏進來?後邊為這把壺好不反亂,玉簫推小玉,小玉推玉簫,急得那大丫頭賭身發咒,只是哭。你趁早還不快送進去哩,遲迴管情就賴在你這小淫婦兒身上。」那迎春方纔取出壺,送入後邊來。後邊玉簫和小玉兩個,正嚷到月娘面前。月娘道:「賊臭肉,還敢嚷些甚麼?你每管着那一門兒?把壺不見了!」玉簫道:「我在上邊跟着娘送酒,他守着銀器家伙。不見了,如今賴我。」小玉道:「大妗子要茶,我不徃後邊替他取茶去?你抱着執壺兒,怎的不見了?敢『屁股大——弔了心』也怎的?」月娘道:「今日席上再無閑雜人,怎的不見了東西?等住囘你主子來,沒這壺,管情一家一頓。」正亂着,只見西門慶自外來,問:「因甚嚷亂?」月娘把不見壺一節說了一遍。西門慶道:「慢慢尋就是了,平白嚷的是些甚麼?」潘金蓮道:「若是吃一遭酒,不見了一把,不嚷亂,你家是王十萬!頭醋不酸,到底兒薄。」{\meipi{金蓮歡時,譏刺無一字不韻趣動人;一至瓶兒生子後,便強口硬舌,愈排低愈使人愛,愈爭寵愈使人憎。一味心忙情急無忌憚矣。作者傳神至此。}}看官聽說:金蓮此話,譏諷李瓶兒首先生孩子,滿月就不見了壺,也是不吉利。西門慶明聽見,只不做聲。只見迎春送壺進來。玉簫便道:「這不是壺有了。」月娘問迎春:「這壺端的徃那裡來?」迎春悉把琴童從外邊拏到我娘屋裡收着,不知在那裡來。月娘因問:「琴童兒那奴才,如今在那裡?」玳安道:「他今日該獅子街房子裡上宿去了。」金蓮在旁不覺鼻子裡笑了一聲。西門慶便問:「你笑怎的?」金蓮道:「琴童兒是他家人,放壺他屋裡,想必要瞞昧這把壺的意思。要叫我,使小厮如今叫將那奴才來,老實打着,問他個下落。不然,頭裡就賴着他那兩個,正是走殺金剛坐殺佛!」西門慶聽了,心中大怒,睜眼看着金蓮,{\pangpi{畫。}}說道:「依着你恁說起來,莫不李大姐他愛這把壺?既有了,丟開手就是了,只管亂甚麼!」那金蓮把臉羞的飛紅了,便道:「誰說姐姐手裡沒錢。」說畢,走過一邊使性兒去了。西門慶就有陳敬濟進來說話。

金蓮和孟玉樓站在一處,罵道:「恁不逢好死,三等九做賊強盜!這兩日作死也怎的?自從養了這種子,恰似生了太子一般,見了俺每如同生剎神一般,越發通沒句好話兒說了,行動就睜着兩個𣭈窟礲喓喝人。誰不知姐姐有錢,明日慣的他每小厮丫頭養漢做賊,把人㒲遍了,也休要管他!」說着,只見西門慶與陳敬濟說了一囘話,就徃前邊去了。孟玉樓道:「你還不去,他管情徃你屋裡去了。」金蓮道:「可是他說的,有孩子屋裡熱鬧,俺每沒孩子的屋裡冷清。」正說着,只見春梅從外走來。玉樓道:「我說他徃你屋裡去了,你還不信,這不是春梅叫你來了。」一面叫過春梅來問。春梅道:「我來問玉簫要汗巾子來。」{\pangpi{諧甚。}}玉樓問道:「你爹在那裡?」春梅道:「爹徃六娘房裡去了。」這金蓮聽了,心上如攛上把火相似,罵道:「賊強人,到明日永世千年,就跌折脚,也別要進我那屋裡!踹踹門檻兒,教那牢拉的囚根子把懷子骨𢱉折了!」玉樓道:「六姐,你今日怎的下恁毒口咒他?」金蓮道:「不是這等說,賊三寸貨強盜,那鼠腹雞腸的心兒,只好有三寸大一般。都是你老婆,無故只是多有了這點尿胞種子罷了,難道怎麼樣兒的!做甚麼恁擡一個滅一個,把人躧到泥裡!」正是:

\begin{myquote}
大風颳倒梧桐樹,自有旁人說短長。
\end{myquote}

這裡金蓮使性兒不題。且說西門慶走到前邊,薛太監差了家人,送了一罈內酒、一牽羊、兩疋金段、一盤壽桃、一盤壽麵、四樣嘉餚,一者祝壽,二者來賀。西門慶厚賞來人,打發去了。到後邊,有李桂姐、吳銀兒兩個拜辭要家去。西門慶道:「你每兩個再住一日兒,到二十八日,我請許多官客,有院中雜耍扮戲的,教你二位只管遞酒。」桂姐道:「既留下俺每,我教人家去囘媽聲,放心些。」於是把兩人轎子都打發去了,不在話下。

次日,西門慶在大廳上錦屏羅列,綺席鋪陳,請官客飲酒。因前日在皇庄見管磚廠劉公公,故與薛內相都送了禮來。西門慶這裡發柬請他,又邀了應伯爵、謝希大兩個相陪。從飯時,二人衣帽齊整,又早先到了。西門慶讓他捲棚內待茶。伯爵因問:「今日,哥席間請那幾客?」西門慶道:「有劉、薛二內相,帥府周大人,都監荊南江,敝同僚夏提刑,團練張總兵,衛上范千戶,吳大哥,吳二哥。喬老便今日使人來囘了不來。連二位通只數客。」說畢,適有吳大舅、二舅到,作了揖,同坐下,左右放桌兒擺飯。吃畢,應伯爵因問:「哥兒滿月抱出來不曾?」西門慶道:「也是因衆堂客要看,房下說且休教孩兒出來,恐風試着他,他奶子說不妨事。教奶子用被裹出來,他大媽屋裡走了遭,應了個日子兒,就進屋去了。」伯爵道:「那日嫂子這裡請去,房下也要來走走,百忙裡舊疾又舉發了,起不得炕兒,心中急的要不的。如今趁人未到,哥倒好說聲,抱哥兒出來,俺每同看一看。」西門慶一面分咐後邊:「慢慢抱哥兒出來,休要唬着他。對你娘說,大舅、二舅在這裡,和應二爹、謝爹要看一看。」月娘教奶子如意兒用紅綾小被兒裹的緊緊的,送到捲棚角門首,玳安兒接抱到捲棚內。衆人觀看,官哥兒穿着大紅段毛衫兒,生的面白唇紅,甚是富態,都誇獎不已。吳大舅、二舅與希大每人袖中掏出一方錦段兜肚,上帶着一個小銀墜兒;惟應伯爵是一柳五色線,上穿着十數文長命錢。教與玳安兒好生抱囘房去,休要驚唬哥兒,說道:「相貌端正,天生的就是個戴紗帽胚胞兒。」{\pangpi{雖油嘴,卻妙。}}西門慶大喜,作揖謝了。說話中間,忽報劉公公、薛公公來了。慌的西門慶穿上衣,儀門迎接。二位內相坐四人轎,穿過肩蟒,纓槍排隊,喝道而至。西門慶先讓至大廳上拜見,叙禮接茶。落後周守備、荊都監、夏提刑等衆武官都是錦綉服,藤棍大扇,軍牢喝道。須臾都到了門首,黑壓壓的許多伺候。裡面鼓樂喧天,笙歌迭奏。西門慶迎入,與劉、薛二內相相見。廳正面設十二張桌席。西門慶就把盞讓坐。劉、薛二內再三讓遜道:「還有列位。」只見周守備道:「二位老太監齒德俱尊。常言:三歲內宦,居冠王公之上。這個自然首坐,何消泛講。」彼此讓遜了一囘。薛內相道:「劉哥,既是列位不肯,難為東家,咱坐了罷。」於是羅圈唱了個喏,打了恭,劉內相居左,薛內相居右,每人膝下放一條手巾,兩個小厮在旁打扇,就坐下了。其次者纔是周守備、荊都監衆人。須臾堦下一派簫韶,動起樂來。當日這筵席,說不盡食烹異品,果獻時新。須臾酒過五巡,湯陳三獻,教坊司俳官簇擁一段笑樂院本上來。正是:

\begin{myquote}
百寶粧腰帶,珍珠絡臂韝。\\笑時能近眼,舞罷錦纏頭。
\end{myquote}

笑樂院本扮完下去,就是李銘、吳惠兩個小優兒上來彈唱。一個ち箏,一個琵琶。周守備先舉手讓兩位內相,說:「老太監分咐,賞他二人唱那套詞兒?」劉太監道:「列位請先。」周守備道:「老太監,自然之理,不必過謙。」劉太監道:「兩個子弟唱個『嘆浮生有如一夢裡』。」{\pangpi{何異說法。}}周守備道:「老太監,此是歸隱嘆世之辭,今日西門慶大人喜事,又是華誕,唱不的。」劉太監又道:「你會唱『雖不是八位中紫綬臣,管領的六宮中金釵女』?」{\pangpi{此題不即不離,尤切。}}周守備道:「此是《陳琳抱粧盒》雜記,今日慶賀,唱不的。」薛太監道:「你叫他二人上來,等我分咐他。你記的《普天樂》『想人生最苦是離別』?」{\meipi{觀者只知老太監三曲題懵語可笑,不知作者借老太監懵語一笑,嘆盡西門慶之終身事業矣,細心玩味自見。}}夏提刑大笑道:「老太監,此是離別之詞,越發使不的。」薛太監道:「俺每內官的營生,只曉的答應萬歲爺,不曉得詞曲中滋味,憑他每唱罷。」{\pangpi{刻畫處,入骨三分。}}夏提刑終是金吾執事人員,倚仗他刑名官,遂分咐:「你唱套《三十腔》。今日是你西門老爹加官進祿,又是好日子,又是弄璋之喜,宜該唱這套。」薛內相問:「怎的是弄璋之喜?」{\pangpi{趣。}}周守備道:「二位老太監,此日又是西門大人公子彌月之辰,俺每同僚都有薄禮慶賀。」薛內相道:「這等——」因向劉太監道:「劉家,咱每明日都補禮來慶賀。」西門慶謝道:「學生生一豚犬,不足為賀,到不必老太監費心。」說畢,喚玳安裡邊叫出吳銀兒、李桂姐,席前遞酒。兩個唱的打扮出來,花枝招展,望上插燭也似磕了四個頭兒,起來執壺斟酒,逐一敬奉。兩個樂工,又唱一套新詞,歌喉宛轉,眞有遶梁之聲。當夜前歌後舞,錦簇花攢,直飲至更餘時分,薛內相方纔起身,說道:「生等一者過蒙盛情,二者又値喜慶,不覺留連暢飲,十分擾極,學生告辭。」西門慶道:「盃茗相邀,得蒙光降,頓使蓬蓽增輝,幸再寬坐片時,以畢餘興。」

衆人俱出位說道:「生等深擾,酒力不勝。」各躬身施禮相謝。西門慶再三款留不住,只得同吳大舅、二舅等,一齊送至大門。一派鼓樂喧天,兩邊燈火燦爛,前遮後擁,喝道而去。正是,得多少:

\begin{myquote}
歌舞歡娛嫌日短,故燒高燭照紅粧。
\end{myquote}

