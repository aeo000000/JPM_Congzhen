\chapter*{後記}
\addcontentsline{toc}{chapter}{後記}
\markboth{\titlename}{後記}

本來想著,花費如許精力做完這本書,是該要寫點什麼。但臨到提筆時,卻字斟句酌,數易其稿{\kaishu(此處表達有誤,我原本是習慣爬格子的。其實說是爬格子,我向來卻討厭把字寫在格子裡,這大概與我「筆走龍蛇」「飄飄乎如馮虛御風」的字有關,因為感覺手寫下來才能做到心手相應。但是古人說情隨事遷,倒是十分有道理的,現在我也可以做到鍵盤所打和心中所想相契合了)}而不就。既怕貽笑大方,又怕玷辱斯文,直覺著「斯文斯文,原來腹內空空」(西遊中沙和尚語)。思慮再三,索性胡圈亂點,草草按下這幾行方塊字了事,權充作一篇文章。一如古人訪問名山、遊歷名勝,上者留下一段佳話典故、中者留下幾點詩詞墨跡、下者只能留下「某某到此一遊」了。

是為後記。

\begin{quotation}
\raggedleft{丙申季冬中浣\rightquadmargin}
\end{quotation}
