\includepdf[pages={171,172},fitpaper=false]{tst.pdf}
\chapter*{第八十六囘 雪娥唆打陳敬濟 金蓮解渴王潮兒}
\addcontentsline{toc}{chapter}{第八十六囘 雪娥唆打陳敬濟 金蓮解渴王潮兒}
\markboth{{\titlename}卷之九}{第八十六囘 雪娥唆打陳敬濟 金蓮解渴王潮兒}


詩曰:

\begin{myquote}
雨打梨花倍寂寥,幾囘腸斷淚珠拋。\\睽違一載猶三載,情緒千絲與萬條。\\好句每從秋裡得,離魂多自夢中消。\\香羅重解知何日,辜負巫山幾暮朝。
\end{myquote}

話說潘金蓮自從春梅去後,房中納悶,不題。單表陳敬濟,次日早飯時出去,假作討帳,騎頭口到於薛嫂兒家。薛嫂兒正在屋裡,一面讓進來坐。敬濟拴了頭口,進房坐下,點茶吃了。薛嫂故意問:「姐夫來有何話說?」敬濟道:「我徃前街討帳,竟到這裡。昨晚大小姐出來了,和他說句話兒。」薛嫂故作喬張致,{\meipi{雖是起發,卻說得婉款,不似王婆一味死狠。}}說:「好姐夫,昨日你家丈母好不分付我,因為你每通同作弊,弄出醜事來,纔把他打發出門,教我防範你們,休要與他會面說話。你還不趁早去哩,只怕他一時使將小厮來看見,到家學了,又是一場兒。倒沒的弄的我也上不的門。」那敬濟便笑嘻嘻袖中拏出一兩銀子來:「權作一茶,你且收了,改日還謝你。」那薛嫂見錢眼開,便道:「好姐夫,自恁沒錢使,將來謝我!只是我去年臘月,你鋪子當了人家兩付扣花枕頂,將有一年來,本利該八錢銀子,你尋與我罷。」{\meipi{討添得妙,妙在近情。}}敬濟道:「這箇不打緊,明日就尋與你。」

這薛嫂兒一面請敬濟裡間房裡去,與春梅厮見,一面叫他媳婦金大姐定菜兒,「我去買茶食點心。」又打了一壺酒,並肉鮓之類,教他二人吃。這春梅看見敬濟,說道:「姐夫,你好人兒,就是箇弄人的劊子手!把俺娘兒兩箇弄的上不上下不下,出醜惹人嫌,到這步田地。」{\meipi{說離散處,語似恨而實苦。}}敬濟道:「我的姐姐,你既出了他家門,我在他家也不久了。『妻兒趙迎春,各自尋投奔』。你教薛媽媽替你尋箇好人家去罷,我『醃韭菜——已是入不的畦」了。我徃東京俺父親那裡去計較了囘來,把他家女兒休了,只要我家寄放的箱子。」說畢,不一時,薛嫂買將茶食酒菜來,放炕桌兒擺了,兩箇做一處飲酒叙話。薛嫂也陪他吃了兩盞,一遞一句,說了囘月娘心狠:「宅裡恁箇出色姐兒出來,通不與一件兒衣服簪環。就是徃人家上主兒去,裝門面也不好看。還要舊時原價。就是清水,這碗裡傾倒那碗內,也拋撒些兒。原來這等夾腦風。臨時出門,倒虧了小玉丫頭做了箇分上,教他娘拏了兩件衣服與他。不是,徃人家相去,拏甚麼做上蓋?」比及吃得酒濃時,薛嫂教他媳婦金大姐抱孩子,躲去人家坐的,教他兩箇在裡間自在坐箇房兒。{\pangpi{偏照顧得到。}}正是:

\begin{myquote}
雲淡淡天邊鸞鳳,水沉沉波底鴛鴦。\\寫成今世不休書,結下來生歡喜帶。
\end{myquote}

兩箇幹訖一度作別,比時難割難捨。{\pangpi{後相思種子。}}薛嫂恐怕月娘使人來瞧,連忙攛掇敬濟出港,騎上頭口來家。遲不上兩日,敬濟又稍了兩方銷金汗巾,兩雙膝褲與春梅,又尋枕頂出來與薛嫂兒。又拏銀子打酒,在薛嫂兒房內正和春梅吃酒,不想月娘使了來安小厮來催薛嫂兒:「怎的還不上主兒?」看見頭口拴在門首,來安兒到家學了舌,說:「姐夫也在那裡來。」月娘聽了,心中大怒,使人一替兩替叫了薛嫂兒去,盡力數說了一遍,道:「你領了奴才去,今日推明日,明日推後日,只顧不上緊替我打發,好窩藏着養漢掙錢兒與你家使。若是你不打發,把丫頭還與我領了來,我另教馮媽媽子賣,你再休上我門來。」這薛嫂兒聽了,到底還是媒人的嘴,說道:「天麼天麼!你老人家恠我差了。我趕着增福神着棍打?你老人家照顧我,怎不打發?昨日也領着走了兩三箇主兒,都出不上,你老人家要十六兩原價,俺媒人家那裡有這些銀子陪上。」{\meipi{不獨洗清,還要趁勢壓價錢,狠心利嘴。}}月娘又道:「小厮說陳家種子今日在你家和丫頭吃酒來。」薛嫂慌道:「耶嚛!耶嚛!又是一場兒。還是去年臘月,當了人家兩付枕頂,在咱獅子街鋪內,銀子收了,今日姐夫送枕頂與我。我讓他吃茶,他不吃,忙忙就上頭口來了。幾時進屋裡吃酒來!原來咱家這大官兒,恁快搗謊駕舌!」月娘吃他一篇,說的不言語了,說道:「我只怕一時被那種子設念隨邪,差了念頭。」薛嫂道:「我是三歲小孩兒?豈可恁些事兒不知道。你那等分付了我,我長吃好,短吃好?他在那裡也沒的久停久坐,與了我枕頭,茶也沒吃就來了。幾曾見咱家小大姐面兒來!萬物也要箇真實,你老人家就上落我起來。既是如此,如今守備周老爺府中,要他圖生長,只出十二兩銀子。看他若添到十三兩上,我兌了銀子來罷。說起來,守備老爺前者在咱家酒席上,也曾見過小大姐來。因他會這幾套唱,好模樣兒,纔出這幾兩銀子。又不是女兒,其餘別人出不上。」薛嫂當下和月娘砸死了價錢。

次日,早把春梅收拾打扮,粧點起來,戴着圍髮雲髻兒,滿頭珠翠,穿上紅段襖兒,藍段裙子,脚上雙彎尖翹翹,一頂轎子送到守備府中。周守備見了春梅生的模樣兒,比舊時越又紅又白,身段兒不短不長,一雙小脚兒,滿心歡喜,{\meipi{守備見春梅,只一歡喜,便不說完,其味直如春欖,且隱隱接去無痕。}}就兌出五十兩一錠元寶來,這薛嫂兒拏出家,鑿下十三兩銀子,徃西門慶家交與月娘,另外又拏出一兩來,說:「是周爺賞我的喜錢,你老人家這邊不與我些兒?」那吳月娘免不過,只得又秤出五錢銀子與他,恰好他還禁了三十七兩五錢銀子。十箇九箇媒人,都是如此撰錢養家。

卻表陳敬濟見賣了春梅,又不得徃金蓮那邊去,見月娘凡事不理他,門戶都嚴禁,到晚夕親自出來,打燈籠前後照看,上了鎖,方纔睡去,因此弄不得手脚。敬濟十分急了,先和西門大姐嚷了兩場,{\pangpi{必至之情。}}淫婦前淫婦後罵大姐:「我在你家做女婿,不道的雌飯吃,吃傷了!你家收了我許多金銀箱籠,你是我老婆,不顧贍我,反說我雌你家飯吃!我白吃你家飯來?」罵的大姐只是哭涕。

十一月念七日,孟玉樓生日。玉樓安排了幾碗酒菜點心,好意教春鴻拏出前邊鋪子,教敬濟陪傅夥計吃。月娘便攔說:「他不是才料。休要理他。{\pangpi{亦太冷。}}要與傅夥計,自與傅夥計自家吃就是了,不消叫他。」玉樓不肯。{\pangpi{厚道。}}春鴻拏出來,擺在水櫃上。一大壺酒都吃了,不勾,又使來安兒後邊要去。傅夥計便說:「姐夫不消要酒去了,這酒勾了,我也不吃了。」敬濟不肯,定要來安要去。等了半晌,來安兒出來,囘說沒了酒了。這陳敬濟也有半酣酒兒在肚內,又使他要去,那來安不動。又另拏錢,打了酒來吃着。罵來安兒:「賊小奴才兒,你別要慌!你主子不待見我,連你這奴才每也欺負我起來了,使你使兒不動。我與你家做女婿,不道的酒肉吃傷了,有爹在怎麼行來?今日爹沒了,就改變了心腸,把我來不理,都亂來擠撮我。我大丈母聽信奴才言語,凡事托奴才,不托我。繇他,我好耐驚耐怕兒!」傅夥計勸道:「好姐夫,快休舒言。不敬奉姐夫,再敬奉誰?想必後邊忙。怎不與姐夫吃?你罵他不打緊,墻有縫,壁有耳,恰似你醉了一般。」敬濟道:「老夥計,你不知道,我酒在肚裡,事在心頭。俺丈母聽信小人言語,罵我一篇是非。就算我㒲了人,人沒㒲了我?好不好我把這一屋子裡老婆都刮剌了,到官也只是後丈母通奸,論箇不應罪名。如今我先把你家女兒休了,然後一紙狀子告到官。再不,東京萬壽門進一本,你家見收着我家許多金銀箱籠,都是楊戩應沒官賍物。好不好把你這幾間業房子都抄沒了,老婆便當官辦賣。我不圖打魚,只圖混水耍子。{\meipi{敬濟又恨又急,又沒法奈何。又欺月娘孤寡,故無忌憚如此。然妙在語語是少年不經事,市井油滑狂妄之談。}}會事的把俺女婿收籠着,照舊看待,還是大家便益。」傅夥計見他話頭兒來的不好,說道:「姐夫,你原來醉了。王十九,只吃酒,且把散話革起。」這敬濟眼瞅着傅夥計,罵道:「老賊狗,怎的說我散話!揭跳我醉了,吃了你家酒來?我不才是他家女婿嬌客,你無故只是他家行財,你也擠撮我起來!我教你這老狗別要慌,你這幾年撰的俺丈人錢勾了,飯也吃飽了,心裡要打夥兒把我疾發了去,要奪權兒做買賣,好禁錢養家。{\meipi{語雖妄誕,然而胸中無聊極矣。}}我明日本狀也帶你一筆。教他打官司!」{\pangpi{妙語。}}那傅夥計最是箇小膽兒的人,見頭勢不好,穿上衣裳,悄悄徃家一溜烟走了。小厮收了家活,後邊去了,敬濟倒在炕上睡下,一宿晚景題過。

次日,傅夥計早辰進後邊,見月娘把前事具訴一遍,哭哭啼啼,要告辭家去,交割帳目,不做買賣了。月娘便勸道:「夥計,你只安心做買賣,休要理那潑才料,如臭屎一般丟着他。當初你家為官事投到俺家來權住着,{\meipi{一「權」字已有逐客之意。}}有甚金銀財寶?也只是大姐幾件粧奩,隨身箱籠。你家老子便躲上東京去了,那時恐怕小人不足,教俺家晝夜耽心。你來時纔十六七歲,黃毛團兒也一般。也虧在丈人家養活了這幾年,調理的諸般買賣兒都會。今日翅膀毛兒乾了,反恩將仇報,一掃帚掃的光光的。小孩兒家說話欺心,恁沒天理,到明日只天照看他!夥計,你自安心做你買賣,休理他便了。他自然也羞。」一面把傅夥計安撫住了不題。

一日,也是合當有事,印子鋪擠着一屋裡人贖討東西。只見奶子如意兒,抱着孝哥兒送了一壺茶來與傅夥計吃,放在桌上。孝哥兒在奶子懷裡,哇哇的只管哭。這陳敬濟對着那些人,作耍當真說道:「我的哥哥,乖乖兒,你休哭了。」向衆人說:「這孩子倒相我養的,依我說話,教他休哭,他就不哭了。」{\meipi{明弄風,放刀撒潑,冀月娘畏而重之,或可與金蓮重合,是痴心,卻是下着。}}那些人就呆了。如意兒說:「姐夫,你說的好妙話兒,越發叫起兒來了,看我進房裡說不說。」這陳敬濟趕上踢了奶子兩脚,戲罵道:「恠賊邋遢,你說不是!我且踢箇响屁股兒着。」{\pangpi{語趣。}}那奶子抱孩子走到後邊,如此這般向月娘哭說:「姐夫對衆人將哥兒這般言語發出來。」這月娘不聽便罷,聽了此言,正在鏡臺邊梳着頭,半日說不出話來,徃前一撞,就昏倒在地,不省人事。但見:

\begin{myquote}
荊山玉損,可惜西門慶正室夫妻;寶鑑花殘,枉費九十日東君匹配。花容掩淡,猶如西園芍藥倚朱欄;檀口無言,一似南海觀音來入定。小園昨日春風急,吹折江梅就地花。
\end{myquote}

慌了小玉,叫將家中大小,扶起月娘來炕上坐的。孫雪娥跳上炕,撅救了半日,舀薑湯灌下去,半日甦醒過來。月娘氣堵心胸,只是哽咽,哭不出聲來。奶子如意兒對孟玉樓、孫雪娥,將敬濟對衆人將哥兒戲言之事,說了一遍:「我好意說他,又趕着我踢了兩脚,把我也氣的發昏在這裡。」雪娥扶着月娘,待的衆人散去,悄悄在房中對月娘說:「娘也不消生氣,氣的你有些好歹,越發不好了。這小厮因賣了春梅,不得與潘家那淫婦弄手脚,纔發出話來。如今一不做,二不休,大姐已是嫁出女,如同賣出田一般,咱顧不得他這許多。常言養蝦蟆得水蠱兒病,只顧教那小厮在家裡做甚麼!明日哄撰進後邊,下老實打與他一頓,即時趕離門,教他家去。然後叫將王媽媽子來,把那淫婦教他領了去,變賣嫁人,如同狗臭尿,掠將出去,一天事都沒了。平空留着他在家裡做甚麼!到明日,沒的把咱們也扯下水去了。」{\meipi{雪娥雖未免公報私仇,然為此時計,亦未有善於此者。}}月娘道:「你說的也是。」當下計議已定了。

到次日,飯時已後,月娘埋伏了丫鬟媳婦七八箇人,各拏短棍棒槌。使小厮來安兒請進陳敬濟來後邊,只推說話。把儀門關了,教他當面跪下,問他:「你知罪麼?」那陳敬濟也不跪,轉把臉兒高揚,佯佯不採。月娘大怒,於是率領雪娥並來興兒媳婦、來昭妻一丈青、中秋兒、小玉、綉春衆婦人,七手八脚,按在地下,拏棒槌短棍打了一頓。西門大姐走過一邊,也不來救。打的這小夥兒急了,把褲子脫了,露出那直豎一條棍來。{\meipi{這等想頭,不知從何處得來,想有色鬼附其腕。}}唬的衆婦人看見,卻丟下棍棒亂跑了。月娘又是那惱,又是那笑,口裡罵道:「好箇沒根基的王八羔子!」敬濟口中不言,心中暗道:「若不是我這箇法兒,怎得脫身。」於是扒起來,一手兜着褲子,徃前走了。月娘隨令小厮跟隨,教他算帳,交與傅夥計。敬濟自知也立脚不定,一面收拾衣服鋪蓋,也不作辭,使性兒一直出離西門慶家,徑徃他母舅張團練家,他舊房子自住去了。正是:

\begin{myquote}
唯有感恩並積恨,萬年千載不生塵。
\end{myquote}

潘金蓮在房中,聽見打了陳敬濟,趕離出門去了,越發憂上加憂,悶上添悶。一日,月娘聽信雪娥之言,使玳安兒去叫了王婆來。那王婆自從他兒子王潮跟淮上客人,拐了起車的一百兩銀子來家,得其發跡,也不賣茶了,買了兩箇驢兒,安了盤磨,一張羅櫃,開起磨房來。聽見西門慶宅裡叫他,連忙穿衣就走,到路上問玳安說:「我的哥哥,幾時沒見你,又早籠起頭去了,有了媳婦兒不曾?」玳安道:「還不曾有哩。」王婆子道:「你爹沒了,你家誰人請我做甚麼?莫不是你五娘養了兒子了,請我去抱腰?」玳安道:「俺五娘倒沒養兒子,倒養了女婿。{\pangpi{妙語。}}俺大娘請你老人家,領他出來嫁人。」王婆子道:「天麼,天麼,你看麼!我說這淫婦,死了你爹,怎守的住。只當狗改不了吃屎,{\pangpi{趣語。}}就弄硶兒來了。就是你家大姐那女婿子?他姓甚麼?」{\pangpi{伏脈。}}玳安道:「他姓陳,名喚陳敬濟。」王婆子道:「想着去年,我為何老九的事,去央煩你爹。到宅內,你爹不在,賊淫婦,他就沒留我房裡坐坐兒,折針也迸不出箇來,{\pangpi{又夾出來時貪想,妙。}}只叫丫頭倒一鍾清茶我吃了,出來了。我只道千年萬歲在他家,如何今日也還出來!好箇浪蹄子淫婦,休說我是你箇媒主,替你作成了恁好人家,就是閑人進去,也不該那等大意。」玳安道:「為他和俺姐夫在家裡炒嚷作亂,昨日差些兒沒把俺大娘氣殺了哩。俺姐夫已是打發出去了,只有他老人家,如今教你領他去哩。」王婆子道:「他原是轎兒來,少不得還叫頂轎子。他也有箇箱籠來,這裡少不的也與他箇箱子兒。」玳安道:「這箇少不的,俺大娘自有箇處。」

兩箇說話間,到了門首。進入月娘房裡,道了萬福坐下,丫鬟拏茶吃了。月娘便道:「老王,無事不請你來。」悉把潘金蓮如此這般,上項說了一遍:「今來是是非人,去是是非者。一客不煩二主,還起動你領他出去,或聘嫁,或打發,叫他吃自在飯去罷。我男子漢已是沒了,招攬不過這些人來。說不的當初死鬼為他丟了許多錢底那話了,就打他恁箇銀人兒也有。{\pangpi{雖恨語,亦不宜。}}如今隨你聘嫁,多少兒交得來,我替他爹念箇經兒,也是一場勾當。」王婆道:「你老人家,是稀罕這錢的?只要把禍害離了門就是了。{\meipi{只一語便打到心上,把銀子抹過,真利嘴。}}我知道,我也不肯差了。」又道:「今日好日,就出去罷。又一件,他當初有箇箱籠兒,有頂轎兒來,也少不的與他頂轎兒坐了去。」月娘道:「箱子與他一箇,轎子不容他坐。」小玉道:「俺奶奶氣頭上便是這等說,到臨岐,少不的顧頂轎兒。不然街坊人家看着,拋頭露面的,不吃人笑話?」月娘不言語了,一面使丫鬟綉春,前邊叫金蓮來。{\meipi{便下一「叫」,傷心字。}}這金蓮一見王婆子在房裡,就睜了,向前道了萬福,坐下。王婆子開言便道:「你快收拾了。剛纔大娘說,教我今日領你出去哩。」金蓮道:「我漢子死了多少時兒,我為下甚麼非,作下甚麼歹來?如何平空打發我出去?」王婆道:「你休稀里打哄,做啞裝聾!自古蛇鑽窟窿蛇知道,各人幹的事兒,各人心裡明。金蓮你休呆裡撒奸,說長道短,我手裡使不的巧語花言,幫閑鑽懶。自古沒箇不散的筵席,『出頭椽兒先朽爛』,人的名兒,樹的影兒。『蒼蠅不鑽沒縫兒蛋』,你休把養漢當飯,我如今要打發你上陽關。」{\meipi{小人於世,並不肯讓人一刻,全人半點,當下闢面便來,可畏,可悲,可嘆。}}金蓮見勢頭不好,料難久住,便也發話道:「你打人休打臉,罵人休揭短!有勢休要使盡了,趕人不可趕上。我在你家做老婆,也不是一日兒,怎聽奴才淫婦戳舌,便這樣絕情絕義的打發我出去!我去不打緊,只要大家硬氣,守到老沒箇破字兒纔好。」{\pangpi{還不饒人。}}

當下金蓮與月娘亂了一囘。月娘到他房中,打點與了他兩箇箱子,一張抽替桌兒,四套衣服,幾件釵梳簪環,一床被褥。其餘他穿的鞋脚,都填在箱內。把秋菊叫到後邊來,一把鎖就把房門鎖了。{\pangpi{月娘亦做得出。}}金蓮穿上衣服,拜辭月娘,在西門慶靈前大哭了一囘。{\meipi{衆妾散去,獨金蓮辭靈大哭,可見情之所鍾,雖無情人,亦不能絕。}}又走到孟玉樓房中,也是姊妹相處一場,一旦分離,兩箇落了一囘眼淚。玉樓瞞着月娘,悄悄與了他一對金碗簪子,一套翠藍段襖、紅裙子,說道:「六姐,奴與你離多會少了,你看箇好人家,徃前進了罷。自古道:『千里長篷,也沒箇不散的筵席。』你若有了人家,使箇人來對我說聲,奴徃那裡去,順便到你那裡看你去,也是姐妹情腸。」{\meipi{玉樓雖是安慰金蓮,然隱隱情見乎詞矣。}}於是灑淚而別。臨出門,小玉送金蓮,悄悄與了金蓮兩根金頭簪兒。金蓮道:「我的姐姐,你倒有一點人心兒在我。」王婆又早顧人把箱籠桌子擡的先去了。獨有玉樓、小玉送金蓮到門首,坐了轎子纔囘。正是:

\begin{myquote}
世上萬般哀苦事,無非死別共生離。
\end{myquote}

卻說金蓮到王婆家,王婆安插他在裡間,晚夕同他一處睡。他兒子王潮兒,也長成一條大漢,籠起頭去了,還未有妻室,外間支着床睡。這潘金蓮次日依舊打扮,喬眉喬眼在簾下看人。無事坐在炕上,不是描眉畫眼,就是彈弄琵琶。王婆不在,就和王潮兒鬪葉兒、下棋。那王婆自去掃麵,喂養驢子,不去管他。

朝來暮去,又把王潮兒刮剌上了。晚間等的王婆子睡着了,婦人推下炕溺尿,走出外間床上,和王潮兒兩箇幹,搖的床子一片响聲。{\meipi{金蓮於此味,老的少的、村的俏的、貴的賤的,皆有所遇,可謂備嘗之矣。}}被王婆子醒來聽見,問那裡响。王潮兒道:「是櫃底下貓捕老鼠响。」王婆子睡夢中,喃喃吶吶,口裡說道:「只因有這些麩麵在屋裡,引的這紮心的半夜三更耗爆人,不得睡。」良久,又聽見動旦,搖的床子格支支响,王婆又問那裡响。王潮道:「是貓咬老鼠,鑽在炕洞下嚼的响。」{\meipi{眼前景,心上事,偏道得出。}}婆子側耳,果然聽見貓在炕洞裡咬的响,方纔不言語了。婦人和小厮幹完事,依舊悄悄上炕睡去了。有幾句雙關,說得這老鼠好:

\begin{myquote}
你身軀兒小,膽兒大,嘴兒尖,忒潑皮。見了人藏藏躲躲,耳邊廂叫叫唧唧,攪混人半夜三更不睡。不行正人倫,偏好鑽穴隙。更有一樁兒不老實,到底改不的偷饞抹嘴。
\end{myquote}

有日,陳敬濟打聽得潘金蓮出來,還在王婆家聘嫁,因提着兩弔銅錢,走到王婆家來。婆子正在門前掃驢子撒的糞。這敬濟向前深深地唱箇喏。婆子問道:「哥哥,你做甚麼?」敬濟道:「請借裡邊說話。」王婆便讓進裡面。敬濟便道:「動問西門大官人宅內,有一位娘子潘六姐,在此出嫁?」王婆便道:「你是他甚麼人?」那敬濟嘻嘻笑道:「不瞞你老人家說,我是他兄弟,他是我姐姐。」那王婆子眼上眼下,打量他一囘,說:「他有甚兄弟,我不知道,你休哄我。你莫不是他家女婿姓陳的,在此處撞蠓子,{\pangpi{活賊。}}我老娘手裡放不過。」敬濟笑向腰裡解下兩弔銅錢來,放在面前,說:「這兩弔錢權作王奶奶一茶之費,教我且見一面,改日還重謝你老人家。」婆子見錢,越發喬張致起來,便道:「休說謝的話。他家大娘子分付將來,不許教閑雜人來看他。咱放倒身說話,你既要見這雌兒一面,與我五兩銀子,見兩面與我十兩。你若娶他,便與我一百兩銀子,我的十兩媒人錢在外。{\meipi{終是老手,刀刀見血。}}我不管閑帳。你如今兩串錢兒,打水不渾的,做甚麼?」敬濟見這虔婆口硬,不收錢,又向頭上拔下一對金頭銀脚簪子,重五錢,殺雞扯腿跪在地下,說道:「王奶奶,你且收了,容日再補一兩銀子來與你,不敢差了。且容我見他一面,說些話兒則箇。」那婆子於是收了簪子和錢,分付:「你進去見他,說了話就與我出來。不許你涎眉睜目,只顧坐着。{\pangpi{斷得趣甚。}}所許那一兩頭銀子,明日就送來與我。」於是掀簾,放敬濟進裡間。婦人正坐在炕上,看見敬濟,便埋怨他道:「你好人兒!弄的我前不着村,後不着店,有上稍,沒下稍,出醜惹人嫌。你就影兒也不來看我看兒了。我娘兒們好好的,拆散的你東我西,皆是為誰來?」說着,扯住敬濟,只顧哭泣。王婆又嗔哭,恐怕有人聽見。敬濟道:「我的姐姐,我為你剮皮剮肉,你為我受氣耽羞,怎不來看你?昨日到薛嫂兒家,已知春梅賣在守備府裡去了,纔打聽知你出離了他家門,在王奶奶這邊聘嫁。今日特來見你一面,和你計議。咱兩箇恩情難捨,拆散不開,如之奈何?我如今要把他家女兒休了,問他要我家先前寄放金銀箱籠。他若不與我,我東京萬壽門一本一狀進下來,那裡他雙手奉與我還是遲了。我暗地裡假名托姓,一頂轎子娶到你家去,咱兩箇永遠團圓,做上箇夫妻,有何不可?」婦人道:「現今王乾娘要一百兩銀子,你有這些銀子與他?」敬濟道:「如何要這許多?」婆子說道:「你家大丈母說,當初你家爹,為他打箇銀人兒也還多,定要一百兩銀子,少一絲毫也成不的。」敬濟道:「實不瞞你老人家說,我與六姐打得熱了,拆散不開,看你老人家下顧,退下一半兒來,五六十兩銀子也罷,我徃母舅那裡典上兩三間房子,娶了六姐家去,也是春風一度。你老人家少轉些兒罷。」婆子道:「休說五六十兩銀子,八十兩也輪不到你手裡了。昨日湖州販紬絹何官人,{\pangpi{劈空扭來作脈,妙甚。}}出到七十兩;大街坊張二官府,如今見在提刑院掌刑,使了兩箇節級來,出到八十兩上,拏着兩封銀子來兌,還成不的,都囘去了。你這小孩兒家,空口來說空話,倒還敢奚落老娘,老娘不道的吃傷了哩!」當下一直走出街上,大喓喝說:「誰家女婿要娶丈母,還來老娘屋裡放屁!」{\meipi{又赫他一陣,是降小夥兒手段。}}敬濟慌了,一手扯進婆子來,雙膝跪下央及:「王奶奶噤聲,我依王奶奶價值一百兩銀子罷。爭奈我父親在東京,我明日起身徃東京取銀子去。」婦人道:「你既為我一場,休與乾娘爭執,上緊取去,只恐來遲了,別人娶了奴去,就不是你的人了。」敬濟道:「我顧頭口連夜兼程,多則半月,少則十日就來了。」婆子道:「常言先下米先食飯,我的十兩銀子在外,休要少了,{\pangpi{一步緊一步。}}我先與你說明白着。」敬濟道:「這箇不必說,恩有重報,不敢有忘。」說畢,敬濟作辭出門,到家收拾行李,次日早顧頭口,上東京取銀子去。此這去,正是:

\begin{myquote}
青龍與白虎同行,吉兇事全然未保。
\end{myquote}

