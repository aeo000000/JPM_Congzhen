\chapter*{金瓶梅序}
\addcontentsline{toc}{chapter}{東吴弄珠客《金瓶梅序》}
\markboth{\titlename}{金瓶梅序}


《金瓶梅》,穢書也。袁石公亟稱之,亦自寄其牢騷耳,非有取於《金瓶梅》也。然作者亦自有意,蓋為世戒,非為世勸也。如諸婦多矣,而獨以潘金蓮、李瓶兒、春梅命名者,亦楚《檮杌》之意也。蓋金蓮以姦死,瓶兒以孽死,春梅以淫死,較諸婦為更慘耳。借西門慶以描畫世之大淨,應伯爵以描繪世之小丑,諸淫婦以描畫世之醜婆、淨婆,令人讀之汗下。蓋為世戒,非為世勸也。余嘗曰:「讀《金瓶梅》而生憐憫心者,菩薩也;生畏懼心者,君子也;生歡喜心者,小人也;生傚法心者,乃禽獸耳。」余友人褚孝秀偕一少年同赴歌舞之筵,衍至霸王夜宴,少年垂涎曰:「男兒何可不如此!」褚孝秀曰:「也只為這烏江設此一着耳。」同座聞之,嘆為有道之言。若有人識得此意,方許他讀《金瓶梅》也。不然,石公幾為導淫宣慾之尤矣。奉勸世人,勿為西門之後車可也。

\begin{quotation}
\raggedleft{東吳弄珠客題\rightquadmargin}
\end{quotation}

