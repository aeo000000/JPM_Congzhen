\chapter*{製作說明}
\addcontentsline{toc}{chapter}{製作說明}
\markboth{\titlename}{製作說明}


◎本書是根據齊魯書社1989年齊煙、王汝梅校點出版的《新刻繡像批評金瓶梅》足本作為校對底本。此版本是《金瓶梅》在大陸出版唯一的未刪減版本,出版後,一時洛陽紙貴。目前因多種原因,導致實體版已炒至上萬元,而且大多數正版由個人收藏,有價無市,電子版更是稀缺。本人以夢梅齋製作的PDF為基礎,對照秦修容1998年中華書局出版的《會評會校金瓶梅》版本,修正了缺字若干,加入胡也佛繪《金瓶梅祕戲圖》、戴敦邦繪《金瓶梅人物譜》61張、《金瓶梅全書圖》40張、《日本內閣文庫藏本·新鐫繡像批評金瓶梅·插圖》200張。
後老而彌堅劉仁軌以松影拂雲製作的mobi文件為基礎,仿照原刊本格式,重新進行了豎行排版,將原電子文本的簡體重新繁化,採用康熙字典字體,以求儘量在電子設備再現原刊本的樣貌。原刊本各卷時常簡繁互用(如:礼—禮、响—響),一字多形(如:個—箇、由—繇),幷包含大量異體字和生僻字。種種非標準情況都儘量在本書內予以還原。其本人對照掃描刊本,進行了校對修正補充。本矢量版PDF以而彌堅劉仁軌製作的豎排電子檔為基礎,僅做少量修改。採用{\LaTeX}製版。

◎繪者:

戴敦邦(1938年-),中國著名國畫家,擅人物,工寫兼長,多以古典題材及古裝人物入畫,所作氣魄宏大,筆墨雄健豪放,形象生動傳神,畫風雅俗共賞,主要作品《水滸人物一百零八圖》、《戴敦邦水滸人物譜》、《紅樓夢人物百圖》、《戴敦邦新繪紅樓夢》、《戴敦邦古典文學名著畫集》等;

胡也佛(1908-1980)本名國華,工書畫,學宗仇十洲,擅作仕女,間寫宋元一路山水,雋逸過人。

◎版本:新刻繡像批評《金瓶梅》

◎出版社:齊魯書社

◎出版時間:1989年

◎模板原始程式碼設計:林某人/林姓匹夫

◎製作說明頁原始程式碼設計:雲軒閣閣主

◎拓展字集提供並指導:萌哦萌

◎插圖《人物譜》攝影者:vc2270

◎校對排版:細雨如煙

◎特別申明:此書為金學交流使用,切勿用於商業。版權歸原作者所有。

\begin{quotation}
\raggedleft\small\kaishu\color{gray}{庚子歲正月製書於米國山景城}
\end{quotation}

{\insertauthorlogo}