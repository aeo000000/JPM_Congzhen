\includepdf[pages={179,180},fitpaper=false]{tst.pdf}
\chapter*{第九十囘 來旺偷拐孫雪娥 雪娥受辱守備府}
\addcontentsline{toc}{chapter}{第九十囘 來旺偷拐孫雪娥 雪娥受辱守備府}
\markboth{{\titlename}卷之九}{第九十囘 來旺偷拐孫雪娥 雪娥受辱守備府}


詩曰:

\begin{myquote}
菟絲附蓬麻,引蔓原不長。\\失身與狂夫,不如棄道旁。\\暮夜為儂好,席不暖儂牀。\\昏來辰一別,無乃太匆忙。\\行將濱死地,老痛迫中腸。
\end{myquote}

話說吳大舅領着月娘等一簇男女,離了永福寺,順着大樹長堤前來。玳安又早在杏花酒樓下邊,人烟熱鬧,揀高阜去處,幕天席地設下酒餚,等候多時了。遠遠望月娘衆人轎子驢子到了,問道:「如何這咱纔來?」月娘又把永福寺中遇見春梅告訴一遍。不一時斟上酒來。衆人坐下正飲酒,只見樓下香車繡轂徃來,人烟喧雜。月娘衆人躧着高阜,把眼觀看,只見人山人海圍着,都看教師走馬耍解。

原來是本縣知縣相公兒子李衙內,名喚李拱璧,年約三十餘歲,見為國子上舍,一生風流博浪,懶習詩書,專好鷹犬走馬,打毬蹴踘,常在三瓦兩巷中走,人稱他為「李棍子」。那日穿着一弄兒輕羅軟滑衣裳,頭戴金頂纏棕小帽,脚踏乾黃靴,同廊吏何不韋帶領二三十好漢,拏彈弓、吹筒、毬棒,在於杏花村大酒樓下,看教師李貴走馬賣解,豎肩樁、隔肚帶,輪槍舞棒,做各樣技藝頑耍,引了許多男女圍着烘笑。那李貴諢名為山東夜叉,頭帶萬字巾,身穿紫窄衫,銷金裏肚,坐下銀鬃馬,手執朱紅杆明槍,背插招風令字旗,在街心扳鞍上馬,徃來賣弄手段。這李衙內正看處,忽擡頭看見一簇婦人在高阜處飲酒,內中一箇長挑身材婦人,不覺心搖目蕩,觀之不足,看之有餘,口中不言,心內暗道:「不知是誰家婦女,有男子漢沒有?」{\meipi{玉樓嫁人意說出便傷貞淑,只看無意中暗暗逗露,處賢者以禮也。}}一面叫過手下答應的小張閑架兒來,悄悄分付:「你去那高坡上,打聽那三箇穿白的婦人是誰家的。訪得的實,告我知道。」那小張閑應諾,雲飛跑去。不多時,走到跟前附耳低言囘報說:「如此這般,是縣門前西門慶家妻小。一箇年老的姓吳,是他妗子;一箇五短身材,是他大娘子吳月娘;那箇長挑身材,有白麻子的,是第三箇娘子,姓孟,名玉樓;如今都守寡在家。」這李衙內聽了,獨看上孟玉樓,重賞小張閑,不在話下。

吳月娘和大舅衆人觀看了半日,見日色銜山,令玳安收拾了食盒,上轎騎驢,一徑囘家。有詩為證:

\begin{myquote}
柳底花陰壓路塵,一囘遊賞一囘新。\\有緣千里來相會,無緣對面不相親。
\end{myquote}

這裡月娘衆人囘家不題。卻說那日,孫雪娥與西門大姐在家,午後時分無事,都出大門首站立。也是天假其便,不想一箇搖驚閨的過來——那時賣脂粉、花翠生活,磨鏡子,都搖驚閨。大姐說:「我鏡子昏了。」使平安兒:「叫住那人,與我磨鏡子。」那人放下担兒,說道:「我不會磨鏡子,我只賣些金銀生活,首飾花翠。」站立在門前,只顧眼上眼下看着雪娥。雪娥便道:「那漢子,你不會磨鏡子,去罷,只顧看我怎的!」{\meipi{雪娥與來旺,情人也,曾間別幾何時,而便不復認矣,蠢甚。}}那人說:「雪姑娘,大姑娘,不認的我了?」大姐道:「眼熟,急忙想不起來。」那人道:「我是爹手裡出去的來旺兒。」雪娥便道:「你這幾年在那裡來?出落得恁胖了。」來旺兒道:「我離了爹門,到原籍徐州,家裡閑着沒營生,投跟了箇老爹上京來做官。不想到半路里,他老爺兒死了,丁憂家去了。我便投在城內顧銀鋪,學會了此銀行手藝,各樣生活。這兩日行市遲,顧銀鋪教我挑副担兒,出來街上發賣些零碎。看見娘每在門首,不敢來相認,恐怕踅門了戶的。今日不是你老人家叫住,還不敢相認。」雪娥道:「原來是你。教我只顧認了半日,白想不起。既是舊兒女,怕怎的?」因問:「你担兒裡賣的是甚麼生活?挑進裡面,等俺每看一看。」那來旺兒一面把担兒挑入裡邊院子裡來。開啟箱子,用篋兒托出幾樣首飾來:金銀鑲嵌不等,打造得十分奇巧。大姐與雪娥看了一囘,問來旺兒:「你還有花翠,拏出來。」那來旺兒又取一盒子各樣大翠𩬆花,翠翹滿冠,並零碎草蟲生活來。大姐揀了他兩對𩬆花。這孫雪娥便留了他一對翠鳳,一對柳穿金魚兒。大姐便稱出銀子來與他。雪娥兩樣生活,欠他一兩二錢銀子,約下他:「明日早來取罷。今日你大娘不在家,和你三娘和哥兒都徃墳上與你爹燒紙去了。」來旺道:「我去年在家裡,就聽見人說爹死了。大娘生了哥兒,怕不的好大了。」雪娥道:「你大娘孩兒如今纔周半兒。一家兒大大小小,如寶上珠一般,全看他過日子哩。」

說話中間,來昭妻一丈青出來,傾了盞茶與他吃,那來旺兒接了茶,與他唱了箇喏。來昭也在跟前,同叙了囘話。分付:「你明日來見見大娘。」那來旺兒挑担出門。到晚上,月娘衆人轎子來家。雪娥、大姐、衆人丫頭接着,都磕了頭。玳安跟盒担走不上,顧了匹驢兒騎來家,打發擡盒人去了。月娘告訴雪娥、大姐,說今日寺裡遇見春梅一節:「原來他把潘家的就葬在寺後首,俺每也不知。他來替他娘燒紙,誤打誤撞遇見他。娘兒每又認了囘親。先是寺里長老擺齋吃了。落後他又教伴當擺上他家的四五十攢盒,各樣菜蔬下飯,篩酒上來,通吃不了。他看見哥兒,又與了他一對簪兒,好不和氣。起解行三坐五,坐着大轎子,許多跟隨。又且是出落的比舊時長大了好些,越發白胖了。」{\meipi{月娘口角津津,只以誤遇為幸,認親為榮,與簪為厚,全不以賣去為愧,亦大可笑。}}吳大妗子道:「他倒也不改常忘舊。那時在咱家時,我見他比衆丫鬟行事兒正大,說話兒沉穩,就是箇才料兒。你看今日福至心靈,恁般造化。」{\meipi{徃日不聞有此言,差甚。}}孟玉樓道:「姐姐沒問他,我問他來。果然半年沒洗換,身上懷着喜事哩。也只是八九月裡孩子,守備好不喜歡哩。薛嫂兒說的倒不差。」說了一囘,雪娥題起:「今日娘不在,我和大姐在門首,看見來旺兒。原來他又在這裡學會了銀匠,挑着担兒賣金銀生活花翠。俺每就不認得了,買了他幾枝花翠,他問娘來,我說徃墳上燒紙去了。」月娘道:「你怎的不教他等着我來家?」雪娥道:「俺每教他明日來。」正坐着說話,只見奶子如意兒向前對月娘說:「哥兒來家這半日,只是昏睡不醒,口中出冷氣,身上湯燒火熱的。」這月娘聽見慌了,向炕上抱起孩兒來,口搵着口兒,果然出冷汗,渾身發熱,罵如意兒:「好淫婦,此是轎子冷了孩兒了。」如意兒道:「我拏小被兒裹的緊緊的,怎得凍着?」月娘道:「再不是抱了徃那死鬼墳上,唬了他來了。那等分付教你休抱他去,你不依,浪着抱的去了。」如意兒道:「早小玉姐姐看着,只抱了他那裡看看就來了,幾時唬着他來!」月娘道:「別要說嘴,看那看兒便怎的?卻把他唬了。」急忙叫來安兒:「快請劉婆子去。」不一時,劉婆來到。看了脈息,摸了身上,說:「着了些涼寒,撞見邪祟了。」{\pangpi{一猜便猜至小心上。}}留了兩服硃砂丸,用薑湯灌下去。分付奶子抱着他,熱炕上睡到半夜,出了些冷汗,身上纔涼了。於是管待劉婆子吃了茶,與了他三錢銀子,叫他明日還來看看。一家子慌的要不的,起起倒倒,整亂了半夜。

卻說來旺,次日依舊挑將生活担兒,來到西門慶門首,與來昭唱喏,說:「昨日雪姑娘留下我些生活,許下今日教我來取銀子,就見見大娘。」來昭道:「你且去着,改日來。昨日大娘來家,哥兒不好,叫醫婆、太醫看下藥,整亂了一夜,好不焦心,今日纔好些,那得工夫稱銀子與你。」正說着,只見月娘、玉樓、雪娥送出劉婆子,來到大門首,看見來旺兒。那來旺兒扒在地下,與月娘、玉樓磕下兩箇頭。月娘道:「幾時不見你,就不來這裡走走。」來旺兒悉將前事說了一遍,「要來不好來的。」月娘道:「舊兒女人家,怕怎的?你爹又沒了。當初只因潘家那淫婦,一頭放火,一頭放水,架的舌,把箇好媳婦兒生生逼勒的弔死了,將有作沒,把你墊發了去。今日天也不容,他徃那去了!」{\meipi{月娘一味以誠待人,雖不失為好人,然禍亂皆此好人釀成也,世亦何貴有此好人哉!}}來旺兒道:「也說不的,只是娘心裡明白就是了。」說了囘話,月娘問他:「賣的是甚樣生活?拏出來瞧。」揀了他幾件首飾,該還他三兩二錢銀子,都用等子稱了與他。叫他進入儀門裡面,{\pangpi{又引賊入室。}}分付小玉取一壺酒來,又是一盤點心,教他吃。那雪娥在廚上一力攛掇,又熱了一大碗肉出來與他。吃的酒飯飽了,磕頭出門。月娘、玉樓衆人歸到後邊去。{\pangpi{絕不防嫌。}}雪娥獨自悄悄和他說話:「你常常來走着,怕怎的!奴有話教來昭嫂子對你說。我明日晚夕,在此儀門裡紫墻兒跟前耳房內等你。」兩箇遞了眼色,這來旺兒就知其意,說:「這儀門晚夕關不關?」雪娥道:「如此這般,你來先到來昭屋裡,等到晚夕,踩着梯櫈,越過墻,順着遮墻,我這邊接你下來。咱二人會合一囘,還有細話與你說。」

這來旺得了此話,正是歡從額起,喜向腮生,作辭雪娥,挑担兒出門。正是:不着家神,弄不得家鬼。有詩為證:

\begin{myquote}
閑來無事倚門闌,偶遇多情舊日緣。\\對人不敢高聲語,故把秋波送幾番。
\end{myquote}

這來旺兒歡喜來家,一宿無話。到次日,也不挑担兒出來賣生活,慢慢踅來西門慶門首,等來昭出來與他唱喏。那來昭便說:「旺哥稀罕,好些時不見你了。」來旺兒笑道:「不是也不來,裡邊雪姑娘少我幾錢生活銀,討討。」來昭一面把來旺兒讓到房裡坐下。來旺兒道:「嫂子怎不見?」來昭道:「你嫂子今日後邊上竈哩。」那來旺兒拏出一兩銀子,遞與來昭,{\meipi{好歹俱要此物向前。}}說:「這銀子取壺酒來,和哥嫂吃。」來昭道:「何消這許多。」即叫他兒子鐵棍兒過來。那鐵棍弔起頭去——十五歲了,拏壺出來,打了一大注酒,使他後邊叫一丈青來。不一時,一丈青蓋了一錫鍋熱飯,一大碗雜熬下飯,兩碟菜蔬,說道:「好呀,旺官兒在這裡。」來昭便拏出銀子與一丈青瞧,說:「兄弟破費,要打壺酒咱兩口兒吃。」一丈青笑道:「無功消受,怎生使得?」一面放了炕桌,讓來旺炕上坐。擺下酒菜,把酒來斟。來旺兒先傾頭一盞,遞與來昭,次遞一盞與一丈青,深深唱喏,說:「一向不見哥嫂,這盞水酒孝順哥嫂。」一丈青便說:「哥嫂不道酒肉吃傷了!你對真人休說假話。裡邊雪姑娘昨日已央及達知我了,你兩箇舊情不斷,托俺每兩口兒如此這般周全你。你休推睡裡夢裡,要知山下路,須問過來人。你若入港相會,有東西出來,休要獨吃,須把些汁水教我呷一呷,俺替你每須耽許多利害。」那來旺便跪下說:「只望哥嫂周全,並不敢有忘。」說畢,把酒吃了一囘。一丈青徃後邊和雪娥答了話出來,對他說,約定晚上來,來昭屋裡窩藏,待夜裡關上儀門,後邊人歇下,越墻而過,於中取事。有詩為證:

\begin{myquote}
報應本無私,影响皆相似。\\要知禍福因,但看所為事。
\end{myquote}

這來旺得了此言,囘來家,巴不到晚,踅到來昭屋裡,打酒和他兩口兒吃。至更深時分,更無一人覺的,直待的大門關了,後邊儀門上了拴,家中大小歇息定了,彼此都有箇暗號兒,只聽墻內雪娥咳嗽之聲。這來旺兒踏着梯櫈,黑暗中扒過粉墻,雪娥那邊用櫈子接着。兩箇就在西耳房堆馬鞍子去處,兩箇相摟相抱,雲雨做一處。彼此都是曠夫寡婦,慾心如火。那來旺兒纓槍強壯,盡力弄了一囘,樂極精來,一泄如注。幹畢,雪娥遞與他一包金銀首飾,幾兩碎銀子,兩件段子衣服,分付:「明日晚夕你再來,我還有些細軟與你。你外邊尋下安身去處。徃後這家中過不出好來,不如和你悄悄出去,外邊尋下房兒,成其夫婦。你又會銀行手藝,愁過不得日子?」{\meipi{所算亦是。既有此算,何不凜明月娘,擇一夫嫁之為正大也。}}來旺兒便說:「如今東門外細米巷,有我箇姨娘,有名收生的屈老娘。你那裡曲彎小巷,倒避眼,咱兩箇投奔那裡去。遲些時,看無動靜,我帶你徃原籍家裡,買幾畝地種去也好。」兩箇商量已定。這來旺就作別雪娥,依舊扒過墻來,到來昭屋裡。等至天明,開了大門,挨身出去。到黃昏時分,又來門首,踅入來昭屋裡。晚夕依舊跳過墻去,兩箇幹事。朝來暮徃,非止一日,也抵盜了許多細軟東西,金銀器皿,衣服之類。來昭兩口子也得抽分好些肥己,俱不必細說。

一日,後邊月娘看孝哥兒出花兒,心中不快,睡得早。這雪娥房中使女中秋兒,原是大姐使的,因李嬌兒房中元宵兒被敬濟要了,月娘就把中秋兒與了雪娥,把元宵兒伏侍大姐。那一日,雪娥打發中秋兒睡下,房裡打點一大包釵環頭面,裝在一箇匣內,用手帕蓋了頭,隨身衣服,約定來旺兒在來昭屋裡等候,兩箇要走。來昭便說:「不爭你走了,我看守大門,管放水鴨兒!若大娘知道,問我要人怎的?不如你每打房上去,就躧破些瓦,還有蹤跡。」來旺兒道:「哥也說得是。」雪娥又留一箇銀折盂,一根金耳斡,一件青綾襖,一條黃綾裙,謝了他兩口兒。直等五更鼓,月黑之時,隔房扒過去。來昭夫婦又篩上兩大鐘煖酒,與來旺、雪娥吃,說:「吃了好走,路上壯膽些。」吃到五更時分,每人拏着一根香,躧着梯子,打發兩箇扒上房去,一步一步把房上瓦也跳破許多。比及扒到房簷跟前,街上人還未行走,聽巡捕的聲音,這來旺兒先跳下去,後卻教雪娥躧着他肩背,接摟下來。兩箇徃前邊走,到十字路口上,被巡捕的攔住,{\meipi{私奔乃千古才子佳人偶為奇事,豈愚夫愚婦所可效也。雪娥、來旺宜敗也。}}便問:「徃那裡去的男女?」雪娥便唬慌了手脚。這來旺兒不慌不忙,把手中官香彈了一彈,說道:「俺是夫婦二人,前徃城外岳廟裡燒香,起的早了些,長官勿恠。」那人問:「背的包袱內是甚麼?」來旺兒道:「是香燭紙馬。」那人道:「既是兩口兒岳廟燒香,也是好事,你快去罷。」這來旺兒得不的一聲,拉着雪娥,徃前飛走。走到城下,城門纔開。打人鬧裡挨出城去,轉了幾條街巷。

原來細米巷在箇僻靜去處,住着不多幾家人家,都是矮房低廈。到於屈姥姥家,屈姥姥還未開門。叫了半日,屈姥姥纔起來開了門,見來旺兒領了箇婦人來。原來來旺兒本姓鄭,名喚鄭旺,說:「這婦人是我新尋的妻小。姨娘這裡有房子,且借一間,寄住些時,再尋房子。」遞與屈姥姥三兩銀子,教買柴米。那屈姥姥得了銀子,只得留下。他兒子屈鐺,因見鄭旺夫妻二人,帶着許多金銀首飾東西,夜晚見財起意,就掘開房門偷盜出來去耍錢,致被捉獲,具了事件,拏去本縣見官。李知縣見系賊賍之事,賍物見在,即差人押着屈鐺到家,把鄭旺、孫雪娥一條索子都拴了。那雪娥唬的臉蠟黃也似黃了,換了滲淡衣裳,帶着眼紗,把手上戒指都勒下來打發了公人,押去見官。當下烘動了一街人觀看,有認得的,說是西門慶家小老婆,今被這走出的小厮來旺兒——改名鄭旺通姦,拐盜財物在外居住。又被這屈鐺掏摸了,今事發見官。當下一箇傳十箇,十箇傳百箇,路上行人口似飛。{\meipi{凡西門慶壞事必盛為播揚者,以其作書懲創之大意故耳。}}

月娘家中自從雪娥走了,房中中秋兒見箱內細軟首飾都沒了,衣服丟的亂三攪四,報與月娘。月娘吃了一驚,便問中秋兒:「你跟着他睡,走了,你豈不知?」中秋兒便說:「他要便晚夕悄悄偷走出外邊,半日方囘,不知詳細。」月娘又問來昭:「你看守大門,人出去你怎不曉的?」來昭便說:「大門每日上鎖,莫不他飛出去!」落後看見房上瓦躧破許多,方知越房而去了。又不敢使人躧訪,只得按納含忍。不想本縣知縣當堂理問這件事,先把屈鐺夾了一頓,追出金頭面四件,銀首飾三件,金環一雙,銀鍾二箇,碎銀五兩,衣服二件,手帕一箇,匣一箇。向鄭旺名下追出銀三十兩,金碗簪一對,金仙子一件,戒指四箇。向雪娥名下追出金挑心一件,銀鐲一付,金鈕五付,銀簪四對,碎銀一包。屈姥姥名下追出銀三兩。就將來旺兒問擬奴婢因奸盜取財物,屈鐺系竊盜,俱系雜犯死罪,準徒五年,賍物入官。雪娥孫氏系西門慶妾,與屈姥姥當下都當官拶了一拶。屈姥姥供明放了。雪娥責令本縣差人到西門慶家,教人遞領狀領孫氏。那吳月娘叫吳大舅來商議:「已是出醜,平白又領了來家做甚麼?{\pangpi{有主意。}}沒的玷汙了家門,與死的裝幌子。」打發了差人錢,囘了知縣話。知縣拘將官媒人來,當官辨賣。

卻說守備府中,春梅打聽得知,說西門慶家中孫雪娥如此這般,被來旺兒拐出,盜了財物去在外居住,事發到官,如今當官辨賣。這春梅聽見,要買他來家上竈,要打他嘴,以報平昔之仇。對守備說:「雪娥善能上竈,會做的好茶飯湯水,買來家中伏侍。」這守備即差張勝、李安。拏貼兒對知縣說。知縣自恁要做分上,只要八兩銀子官價。交完銀子,領到府中,先見了大奶奶並二奶奶孫氏,次後到房中來見春梅。春梅正在房裡縷金床上,錦帳之中,纔起來。{\pangpi{寫出富貴驕奢之態。}}手下丫鬟領雪娥見面。那雪娥見是春梅,不免低頭進見。望上倒身下拜,磕了四箇頭。這春梅把眼瞪一瞪,{\pangpi{畫。}}喚將當直的家人媳婦上來,「與我把這賤人撮去了鬏髻,剝了上蓋衣裳,打入廚下,與我燒火做飯。」這雪娥聽了,暗暗叫苦。自古世間打墻板兒翻上下,掃米卻做管倉人。既在他簷下,怎敢不低頭?孫雪娥到此地步,只得摘了髻兒,換了艷服,滿臉悲慟,徃廚下去了。有詩為證:

\begin{myquote}
布袋和尚到明州,策杖芒鞋任處遊。\\饒你化身千百億,一身還有一身愁。
\end{myquote}

