\includepdf[pages={185,186},fitpaper=false]{tst.pdf}
\chapter*{第九十三囘 王杏庵義恤貧兒 金道士孌淫少弟}
\addcontentsline{toc}{chapter}{第九十三囘 王杏庵義恤貧兒 金道士孌淫少弟}
\markboth{{\titlename}卷之十}{第九十三囘 王杏庵義恤貧兒 金道士孌淫少弟}


詩曰:

\begin{myquote}
堦前潛制淚,衆裡自嫌身。\\氣味如中酒,情懷似別人。\\暖風張樂席,晴日看花塵。\\盡是添愁處,深居乞過春。
\end{myquote}

話說陳敬濟,自從西門大姐死了,被吳月娘告了一狀,打了一場官司出來,唱的馮金寶又歸院中去了,剛刮剌出箇命兒來。{\pangpi{足矣。}}房兒也賣了,本錢兒也沒了,頭面也使了,家伙也沒了。又說陳定在外邊打發人,剋落了錢,把陳定也攆去了。家中日逐盤費不周,坐吃山空,不時徃楊大郎家中,問他這半船貨的下落。一日,來到楊大郎門首,叫聲:「楊大郎在家不在?」不想楊光彥拐了他半船貨物,一向在外,賣了銀兩,四散躲閃。及打聽得他家中弔死了老婆,他丈母縣中告他,坐了半箇月監,這楊大郎就驀地來家住着。聽見敬濟上門叫他,問貨船下落,一徑使兄弟楊二風出來,反問敬濟要人:「你把我哥哥叫的外面做買賣,這幾箇月通無音信,不知拋在江中,推在河內,害了性命,你倒還來我家尋貨船下落?人命要緊,你那貨物要緊?」這楊二風平昔是箇刁徒潑皮,耍錢搗子,胳膊上紫肉橫生,胸前上黃毛亂長,是一條直率光棍。走出來一把扯住敬濟,就問他要人。那敬濟慌忙掙開手跑出囘家來。這楊二風故意拾了塊三尖瓦楔,將頭顱鑽破,血流滿面,趕將敬濟來,罵道:「我㒲你娘眼!我見你家甚麼銀子來?你來我屋裡放屁,吃我一頓好拳頭。」那敬濟金命水命,走投無命,奔到家,把大門關閉如鐵桶相似,繇着楊二風牽爹娘,罵父母,拏大磚砸門,只是鼻口內不敢出氣兒。又況纔打了官司出來,夢條繩蛇也害怕,只得含忍過了。正是:

\begin{myquote}
嫩草怕霜霜怕日,惡人自有惡人磨。
\end{myquote}

不消幾時,把大房賣了,找了七十兩銀子,典了一所小房,在僻巷內居住。落後兩箇丫頭,賣了一箇重喜兒,只留着元宵兒和他同鋪歇。又過了不上半月,把小房倒騰了,卻去賃房居住。陳安也走了,家中沒營運,元宵兒也死了,止是單身獨自,家伙桌椅都變賣了,只落得一貧如洗。{\meipi{富貴家子弟,父兄死後,你不讀書,任聰明乖巧,亦必流落至此,非異事也。}}未幾,房錢不給,鑽入冷鋪記憶體身。花子見他是箇富家勤兒,生得清俊,叫他在熱炕上睡,與他燒餅兒吃。有當夜的過來教他頂火夫,打梆子搖鈴。那時正值臘月,殘冬時分,天降大雪,弔起風來,十分嚴寒。這陳敬濟打了囘梆子,打發當夜的兵牌過去,不免手提鈴串了幾條街巷。又是風雪,地下又踏着那寒氷,凍得聳肩縮背,戰戰兢兢。臨五更雞叫,只見箇病花子躺在墻底下,恐怕死了,總甲分付他看守着,尋了把草叫他烤。這敬濟支更一夜,沒曾睡,就𢱉下睡着了。不想做了一夢,夢見那時在西門慶家,怎生受榮華富貴,和潘金蓮勾搭,頑耍戲謔,從睡夢中就哭醒來。衆花子說:「你哭怎的?」這敬濟便道:「你衆位哥哥,我的苦楚,你怎得知?頻年困苦痛妻亡,身上無衣口絕糧。馬死奴逃房又賣,隻身獨自在他鄉。朝依肆店求遺饌,暮宿庄園倚敗墻。只有一條身後路,冷鋪之中去打梆。」

陳敬濟晚夕在冷鋪存身,白日間街頭乞食。清河縣城內有一老者,姓王名宣,字廷用,年六十餘歲,家道殷實,為人心慈,仗義疏財,專一濟貧拔苦,好善敬神。所生二子,皆當家成立。長子王幹,襲祖職為牧馬所掌印正千戶;次子王震,充為府學庠生。老者門首搭了箇主管,開着箇解當鋪兒。每日豐衣足食,閑散無拘,在梵宇聽經,琳宮講道。無事在家門首施藥救人,拈素珠念佛。因後園中有兩株杏樹,道號為杏庵居士。一日,杏庵頭戴重簷幅巾,身穿水合道服,在門首站立。只見陳敬濟打他門首過,向前扒在地下磕了箇頭。忙的杏庵還禮不迭,說道:「我的哥,你是誰?老拙眼昏,不認的你。」這敬濟戰戰兢兢,站立在旁邊說道:「不瞞你老人家,小人是賣松槁陳洪兒子。」老者想了半日,說:「你莫不是陳大寬的令郎麼?」{\meipi{陳洪號到此點出,冷甚。}}因見他衣服襤褸,形容憔悴,說道:「賢姪,你怎的弄得這般模樣?」便問:「你父親、母親可安麼?」敬濟道:「我爹死在東京,我母親也死了。」杏庵道:「我聞得你在丈人家住來?」敬濟道:「家外父死了,外母把我攆出來。他女兒死了,告我到官,打了一場官司。把房兒也賣了,有些本錢兒,都吃人坑了,一向閑着沒有營生。」杏庵道:「賢姪,你如今在那裡居住?」敬濟半日不言語,說:「不瞞你老人家說,如此如此。」{\pangpi{吞吐妙甚。}}杏庵道:「可憐,賢姪你原來討吃哩。想着當初,你府上那樣根基人家。我與你父親相交,賢姪,你那咱還小哩,纔紮着總角上學堂,怎就流落到此地位?可傷,可傷。你還有甚親家?也不看顧你看顧兒。」敬濟道:「正是。俺張舅那裡,一向也久不上門,不好去的。」問了一囘話,老者把他讓到裡面客位裡,令小厮放桌兒,擺出點心嗄飯來,教他盡力吃了一頓。見他身上單寒,拏出一件青布綿道袍兒,一頂氊帽,又一雙毡襪、綿鞋,又秤一兩銀子,五百銅錢,遞與他,分付說:「賢姪,這衣服鞋襪與你身上,那銅錢與你盤纏,賃半間房兒住;這一兩銀子,你拏着做上些小買賣兒,也好餬口過日子,強如在冷鋪中,學不出好人來。{\pangpi{正色說趣語,妙。}}每月該多少房錢,來這裡,老拙與你。」這陳敬濟扒在地下磕頭謝了,說道:「小姪知道。」拏着銀錢,出離了杏庵門首。也不尋房子,也不做買賣,把那五百文錢,每日只在酒店面店以了其事。{\pangpi{自然之理。}}那一兩銀子,搗了些白銅頓礶,在街上行使。{\pangpi{人貧智短,信然。}}吃巡邏的當土賊拏到該坊節級處,一頓拶打,使的罄盡,還落了一屁股瘡。不消兩日,把身上綿衣也輸了,襪兒也換嘴來吃了,依舊原在街上討吃。

一日,又打王杏庵門首所過,杏庵正在門首,只見敬濟走來磕頭,身上衣襪都沒了,止戴着那氊帽,精脚靸鞋,凍的乞乞縮縮。老者便問:「陳大官,做的買賣如何?房錢到了,來取房錢來了?」那陳敬濟半日無言可對。問之再三,方說如此這般,都沒了。老者便道:「阿呀,賢姪,你這等就不是過日子的道理。你又拈不的輕,負不的重,但做了些小活路兒,不強如乞食,免教人恥笑,有玷你父祖之名。你如何不依我說?」一面又讓到裡面,教安童拏飯來與他吃飽了。又與了他一條夾褲,一領白布衫,一雙裹脚,一弔銅錢,一斗米:「你拏去務要做上了小買賣,賣些柴炭、豆兒、瓜子兒,也過了日子,強似這等討吃。」

這敬濟口雖答應,拏錢米在手,出離了老者門,那消幾日,熟食肉面,都在冷鋪內和花子打夥兒都吃了。耍錢,又把白布衫、夾褲都輸了。大正月裡,又抱着肩兒在街上走,不好來見老者,走在他門首房山墻底下,向日陽站立。老者冷眼看見他,不叫他。他挨挨搶搶,又到根前扒在地下磕頭。{\meipi{飢寒似為廉恥而忍,而廉恥終捱不過飢寒。死生之際,君子小人之間難言哉!}}老者見他還依舊如此,說道:「賢姪,這不是常策。咽喉深似海,日月快如梭,無底坑如何填得起?你進來,我與你說,有一箇去處,又清閑,又安得你身,只怕你不去。」敬濟跪下哭道:「若得老伯見憐,不拘那裡,但安下身,小的情願就去。」杏庵道:「此去離城不遠,臨清馬頭上,有座晏公廟。那裡魚米之鄉,舟船輻輳之地,錢糧極廣,清幽瀟灑。廟主任道士,與老拙相交極厚,他手下也有兩三箇徒弟徒孫。我備分禮物,把你送與他做箇徒弟出家,學些經典吹打,與人家應福,也是好處。」敬濟道:「老伯看顧,可知好哩。」杏庵道:「既然如此,你去,明日是箇好日子,你早來,我送你去。」敬濟去了。這王老連忙叫了裁縫來,就替敬濟做了兩件道袍,一頂道髻,鞋襪俱全。

次日,敬濟果然來到。王老教他空屋裡洗了澡,梳了頭,戴上道髻,裡外換了新襖新褲,上蓋表絹道衣,下穿雲履毡襪,備了四盤羹菓,一罈酒,一疋尺頭,封了五兩銀子。他便乘馬,顧了一匹驢兒與敬濟騎着,安童、喜童跟隨,兩箇人担了盒担,出城門,徑徃臨清馬頭晏公廟來。止七十里,一日路程。比及到晏公廟,天色已晚,王老下馬,進入廟來。只見青松鬱鬱,翠柏森森,兩邊八字紅墻,正面三間朱戶,端的好座廟宇。但見:

\begin{myquote}
山門高聳,殿閣稜層。高懸勑額金書,彩畫出朝入相。{\pangpi{入情。}}五間大殿,塑龍王一十二尊;兩下長廊,刻水族百千萬衆。旗竿淩漢,帥字招風。四通八達,春秋社禮享依時;雨順風調,河道民間皆祭賽。萬年香火威靈在,四境官民仰賴安。
\end{myquote}

山門下早有小童看見,報入方丈,任道士忙整衣出迎。王杏庵令敬濟和禮物且在外邊伺候。不一時,任道士把杏庵讓入方丈松鶴軒叙禮,說:「王老居上,怎生一向不到敝廟隨喜?今日何幸,得蒙下顧。」杏庵道:「只因家中俗冗所羈,久失拜望。」叙禮畢,分賓主而坐,小童獻茶。茶罷,任道士道:「老居士,今日天色已晚,你老人家不去罷了。」分付把馬牽入後槽喂息。杏庵道:「沒事不登三寶殿。老拙敬來有一事干瀆,未知尊意肯容納否?」任道士道:「老居士有何見教?只顧分付,小道無不領命。」杏庵道:「今有故人之子,姓陳,名敬濟,年方二十四歲。生的資格清秀,倒也伶俐。只是父母去世太早,自幼失學。若說他父祖根基,也不是無名少姓人家,有一分家當,只因不幸遭官事沒了,無處栖身。老拙念他乃尊舊日相交之情,欲送他來貴宮作一徒弟,未知尊意如何?」任道士便道:「老居士分付,小道怎敢違阻?奈因小道命蹇,手下雖有兩三箇徒弟,都不省事,沒一箇成立的,小道常時惹氣,未知此人誠實不誠實?」杏庵道:「這箇小的,不瞞尊師說,只顧放心,一味老實本分,膽兒又小,{\pangpi{何以見得?}}所事兒伶範,堪可作一徒弟。」{\meipi{為敬濟則得矣,道士晦氣,奈何。}}任道士問:「幾時送來?」杏庵道:「見在山門外伺候。還有些薄禮,伏乞笑納。」慌的任道士道:「老居幹何不早說?」一面道:「有請。」於是擡盒人擡進禮物。任道士見帖兒上寫着:「謹具粗段一端,魯酒一樽,豚蹄一副,燒鴨二隻,樹菓二盒,白金五兩。知生王宣頓首拜。」連忙稽首謝道:「老居士何以見賜許多重禮,使小道卻之不恭,受之有愧。」只見陳敬濟頭戴金梁道髻,身穿青絹道衣,脚下雲履淨襪,腰繫絲縧,生的眉清目秀,齒白唇紅,面如傅粉,走進來向任道士倒身下拜,拜了四雙八拜。任道士因問他:「多少青春?」敬濟道:「屬馬,交新春二十四歲了。」任道士見他果然伶俐,取了他箇法名,叫做陳宗美。原來任道士手下有兩箇徒弟,大徒弟姓金,名宗明;二徒弟姓徐,名宗順。他便叫陳宗美。王杏庵都請出來,見了禮數。一面收了禮物,小童掌上燈來,放卓兒,先擺飯,後吃酒。餚品盃盤,堆滿桌上,無非是雞蹄鵝鴨魚肉之類。王老吃不多酒,徒弟輪番勸勾幾巡,王老不勝酒力,告辭。房中自有床鋪,安歇一宿。

到次日清晨,小童舀水淨面,梳洗盥漱畢,任道士又早來遞茶。不一時,擺飯,又吃了兩盃酒,喂飽頭口,與了擡盒人力錢。王老臨起身,叫過敬濟來分付:「在此好生用心習學經典,聽師父指教。我常來看你,按季送衣服鞋襪來與你。」又向任道士說:「他若不聽教訓,一任責治,老拙並不護短。」一面背地又囑付敬濟:「我去後,你要洗心改正,習本等事業。你若再不安分,我不管你了。」{\pangpi{情景如畫。}}那敬濟應諾道:「兒子理會了。」王老當下作辭任道士,出門上馬,離晏公廟,囘家去了。敬濟自此就在晏公廟做了道士。因見任道士年老赤鼻,身體魁偉,聲音洪亮,一部髭髯,能談善飲,只專迎賓送客。凡一應大小事,都在大徒弟金宗明手裡。那時,朝廷運河初開,臨清設二閘,以節水利。不拘官民,船到閘上,都來廟裡,或求神福,或來祭願,或設卦與笤,或做好事。也有布施錢米的,也有餽送香油紙燭的,也有留松蒿蘆蓆的。這任道士將常署裡多餘錢糧,都令家下徒弟在馬頭上開設錢米鋪,賣將銀子來,積攢私囊。

他這大徒弟金宗明,也不是箇守本分的。年約三十餘歲,常在娼樓包占樂婦,是箇酒色之徒。手下也有兩箇清潔年少徒弟,同鋪歇臥,日久絮繁。{\pangpi{名言。}}因見敬濟生的齒白唇紅,面如傅粉,清俊乖覺,眼裡說話,就纏他同房居住。晚夕和他吃半夜酒,把他灌醉了,在一鋪歇臥。初時兩頭睡,便嫌敬濟脚臭,叫過一箇枕頭上睡。睡不多囘,又說他口氣噴着,令他弔轉身子,屁股貼着肚子。那敬濟推睡着,不理他。他把那話弄得硬硬的,直豎一條棍,抹了些唾津在頭上,徃他糞門裡只一頂。原來敬濟在冷鋪裡,被花子飛天鬼侯林兒弄過的,眼子大了,那話不覺就進去了。{\meipi{演大了,討便宜如此。}}這敬濟口中不言,心內暗道:「這厮合敗。他討得十方便宜多了,把我不知當做甚麼人兒。與他箇甜頭兒,且教他在我手內納些錢鈔。」一面故意聲叫起來。這金宗明恐怕老道士聽見,連忙掩住他口,說:「好兄弟,噤聲!隨你要的,我都依你。」敬濟道:「你既要勾搭我,我不言語,須依我三件事。」{\pangpi{金蓮傳授。}}宗明道:「好兄弟,休說三件,就是十件事,我也依你。」敬濟道:「第一件,你既要我,不許你再和那兩箇徒弟睡;第二件,大小房門鑰匙,我要執掌;第三件,隨我徃那裡去,你休嗔我。你都依了我,我方依你此事。」金宗明道:「這箇不打緊,我都依你。」當夜兩箇顛來倒去,整狂了半夜。這陳敬濟自幼風月中撞,甚麼事不知道。當下被底山盟,枕邊海誓,淫聲艷語,摳吮舔品,{\pangpi{大才而小用矣。}}把這金宗明哄得歡喜無盡。到第二日,果然把各處鑰匙都交與他手內,就不和那兩箇徒弟在一處,每日只同他一鋪歇臥。

一日兩,兩日三,這金宗明便再三稱赞他老實。任道士聽信,又替他使錢討了一張度牒。自此以後,凡事並不防範。這陳敬濟因此常拏着銀錢徃馬頭上游玩,看見院中架兒陳三兒說:「馮金寶兒他鴇子死了,他又賣在鄭家,叫鄭金寶兒。如今又在大酒樓上趕趁哩,你不看他看去?」這小夥兒舊情不改,拏着銀錢,跟定陳三兒,徑徃馬頭大酒樓上來。此不來倒好,若來,正是:五百載冤家來聚會,數年前姻眷又相逢。有詩為證:

\begin{myquote}
人生莫惜金縷衣,人生莫負少年時。\\有花欲折須當折,莫待無花空折枝。
\end{myquote}

原來這座酒樓乃是臨清第一座酒樓,名喚謝家酒樓。裡面有百十座閣兒,周圍都是綠欄杆,就緊靠着山岡,前臨官河,極是人烟鬧熱去處,舟船徃來之所。怎見得這座酒樓齊整?但見:

\begin{myquote}
雕簷映日,畫棟飛雲。綠欄杆低接軒窓,翠簾櫳高懸戶牖。吹笙品笛,盡都是公子王孫;執盞擎盃,擺列着歌嫗舞女。消磨醉眼,依青天萬疊雲山;勾惹吟魂,翻瑞雪一河烟水。樓畔綠楊啼野鳥,門前翠柳系花驄。
\end{myquote}

這陳三兒引敬濟上樓,到一箇閣兒裡坐下。便叫店小二打抹春臺,安排一分上品酒菓下飯來擺着,使他下邊叫粉頭去了。須臾,只見樓梯响,馮金寶上來,手中拏着箇厮鑼兒,見了敬濟,深深道了萬福。常言情人見情人,不覺簇地兩行淚下。{\meipi{寫得默然有慘色,妙。}}正是:

\begin{myquote}
數聲嬌語如鶯囀,一串珍珠落線頭。
\end{myquote}

敬濟一見,便拉他一處坐,問道:「姐姐,你一向在那裡來?不見你。」這馮金寶收淚道:「自從縣中打斷出來,我媽着了驚唬,不久得病死了,把我賣在鄭五媽家。這兩日子弟稀少,不免又來在臨清馬頭上趕趁酒客。昨日聽見陳三兒說你在這裡開錢鋪,要見你一見。不期今日會見一面。可不想殺我也!」說畢,又哭了。敬濟取出袖中帕兒,替他抹了眼淚,說道:「我的姐姐,你休煩惱。我如今又好了,{\pangpi{試看敬濟數箇「我好了」,大有儆醒。}}自從打出官司來,家業都沒了,投在這晏公廟,做了道士。師父甚是托我,徃後我常來看你。」因問:「你如今在那裡安下?」金寶便道:「奴就在這橋西灑家店劉二那裡。有百十房子,四外衏䘕窠子,妓女都在那裡安下,白日裡便是這各酒樓趕趁。」說着,兩箇挨身做一處飲酒。陳三兒燙酒上樓,拏過琵琶來。金寶彈唱了箇曲兒與敬濟下酒,名《普天樂》:

\begin{myquote}
淚雙垂,垂雙淚。三盃別酒,別酒三盃。鸞鳳對拆開,折開鸞鳳對。嶺外斜暉看看墜,看看墜,嶺外暉。天昏地暗,徘徊不捨,不捨徘徊。
\end{myquote}

兩人吃得酒濃時,未免解衣雲雨,下箇房兒。這陳敬濟一向不曾近婦女,久渴的人,今得遇金寶,盡力盤桓,尤雲殢雨,未肯即休。須臾事畢,各整衣衫。敬濟見天色晚了,與金寶作別,與了金寶一兩銀子,與了陳三兒三百文銅錢,囑付:「姐姐,我常來看你,咱在這搭兒裡相會。你若想我,使陳三兒叫我去。」{\pangpi{痴語。}}下樓來,又打發了店主人謝三郎三錢銀子酒錢。敬濟囘廟中去了。馮金寶送至橋邊方囘。正是:

\begin{myquote}
盼穿秋水因錢鈔,哭損花容為鄧通。
\end{myquote}

