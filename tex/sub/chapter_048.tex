\includepdf[pages={95,96},fitpaper=false]{tst.pdf}
\chapter*{第四十八囘 弄私情戲贈一枝桃 走捷徑探歸七件事}
\addcontentsline{toc}{chapter}{第四十八囘 弄私情戲贈一枝桃 走捷徑探歸七件事}
\markboth{{\titlename}卷之五}{第四十八囘 弄私情戲贈一枝桃 走捷徑探歸七件事}


詞曰:

碧桃花下,紫簫吹罷。驀然一點心驚,卻把那人牽掛,向東風淚灑。東風淚灑,不覺暗沾羅帕,恨如天大。那冤家既是無情去,囘頭看怎麼!

——右調《桂枝香》

話說安童領着書信,辭了黃通判,徑徃山東大道而來。打聽巡按御史在東昌府住紮,姓曾,雙名孝序,乃都御史曾布之子,新中乙未科進士,極是個清廉正氣的官。這安童自思:「我若說下書的,門上人決不肯放。不如等放告牌出來,我跪門進去,連狀帶書呈上。老爹見了,必然有個決斷。」於是早把狀子寫下,揣在懷裡,在察院門首等候多時。只聽裡面打的雲板響,開了大門,曾御史坐廳。頭面牌出來,大書告親王、皇親、駙馬、勢豪之家;{\meipi{數語凜然,應使朝廷側目。}}第二面牌出來,告都、布、按並軍衛有司官吏;第三面牌出來,纔是百姓戶婚田土詞訟之事。這安童就隨狀牌進去,待把一應事情發放淨了,方走到丹墀上跪下。兩邊左右問是做甚麼的,這安童方纔把書雙手舉得高高的呈上。只聽公座上曾御史叫:「接上來!」慌的左右吏典下來把書接上去,安放於書案上。曾公拆開觀看,端的上面寫着甚言詞?書曰:

寓都下年教生黃端肅書奉

大柱史少亭曾年兄先生大人門下:違越光儀,倏忽一載。知己難逢,勝遊易散。此心耿耿,常在左右。去秋忽報瑤章,開軸啟函,捧誦之間而神遊恍惚,儼然長安對面時也。未幾,年兄省親南旋,復聞德音,知年兄按巡齊魯,不勝欣慰。叩賀,叩賀。惟年兄忠孝大節,風霜貞操,砥礪其心,耿耿在廊廟,歷歷在士論。今茲出巡,正當摘發官邪,以正風紀之日。區區愛念,尤所不能忘者矣。竊謂年兄平日抱可為之器,當有為之年,值聖明有道之世,老翁在家康健之時,當乘此大展才猷,以振揚法紀,勿使舞文之吏以撓其法,而奸頑之徒以逞其欺。胡乃如東平一府,而有撓大法如苗青者,抱大冤如苗天秀者乎?生不意聖明之世而有此魍魎。年兄巡歷此方,正當分理冤滯,振刷為之一清可也。去伴安童,持狀告訴,幸察,不宣。仲春望後一日具。

這曾御史覽書已畢,便問:「有狀沒有?」左右慌忙下來問道:「老爺問你有狀沒有。」這安童向懷中取狀遞上。曾公看了,取筆批:「仰東平府府官,從公查明,驗相屍首,連卷詳報。」喝令安童東平府伺候。這安童連忙磕頭起來,從便門放出。這裡曾公將批詞連狀裝在封套內,鈐了關防,差人齎送東平府來。府尹胡師文見了上司批下來,慌得手脚無措,即調委陽谷縣縣丞狄斯彬——本貫河南舞陽人氏,為人剛方不要錢,問事糊突,人都號他做狄混。先是這狄縣丞徃清河縣城西河邊過,忽見馬頭前起一陣旋風,團團不散,只隨着狄公馬走。狄縣丞道:「恠哉!」便勒住馬,令左右公人:「你隨此旋風,務要跟尋個下落。」{\meipi{此處甚不混。}}那公人真個跟定旋風而來,七八將近新河口而止,走來囘覆了狄公話。狄公即拘集里老,用鍬掘開岸上數尺,見一死屍,宛然頸上有一刀痕。命仵作檢視明白,問其前面是那裡。公人稟道:「離此不遠就是慈惠寺。」縣丞即拘寺中僧行問之,皆言:「去冬十月中,本寺因放水燈兒,見一死屍從上流而來,漂入港裡。長老慈悲,故收而埋之。不知為何而死。」縣丞道:「分明是汝衆僧謀殺此人,埋於此處。想必身上有財帛,故不肯寔說。」於是不繇分說,先把長老一箍兩拶,一夾一百敲,餘者衆僧都是二十板,{\meipi{至此大混。然原情察理,不無有之,非刻意做官者不為也。}}俱令收入獄中。報與曾公,再行檢視。各僧皆稱冤不服。曾公尋思道:「既是此僧謀死,屍必棄於河中,豈反埋於岸上?又說干礙人衆,此有可疑。」因令將衆僧收監。將近兩月,不想安童來告此狀。即令委官押安童前至屍所,令其認視。安童見屍大哭道:「正是我的主人,被賊人所傷,刀痕尚在。」於是檢驗明白,囘報曾公,即把衆僧放囘。一面查刷卷宗,復提出陳三、翁八審問,俱執稱苗青主謀之情。曾公大怒,差人行牌,星夜徃揚州提苗青去了。一面寫本叅劾提刑院兩員問官受賍賣法。正是:

汙吏賍官濫國刑,曾公判刷雪冤情。雖然號令風霆肅,夢裡輸贏總未真。

話分兩頭,卻表王六兒自從得了苗青幹事的那一百兩銀子、四套衣服,與他漢子韓道國就白日不閑,一夜沒的睡,計較着要打頭面,治簪環,喚裁縫來裁衣服,從新抽銀絲鬏髻。用十六兩銀子,又買了個丫頭——名喚春香——使喚,早晚教韓道國收用,不題。{\meipi{乞兒路撿一金,便手足無措,韓氏夫婦較猶能位置者。}}一日,西門慶到韓道國家,王六兒接着。裡面吃茶畢,西門慶徃後邊淨手去,看見隔壁月臺,問道:「是誰家的?」王六兒道:「是隔壁樂三家月臺。」西門慶分付王六兒:「如何教他遮住了這邊風水?你對他說,若不與我即便拆了,我教地方分付他。」這王六兒與韓道國說:「隣舍家,怎好與他說的。」韓道國道:「咱不如瞞着老爹,買幾根木植來,咱這邊也搭起個月臺來。上面晒醬,下邊不拘做馬坊,做個東淨,也是好處。」老婆道:「呸!賊沒算計的。比時搭月臺,不如買些磚瓦來,蓋上兩間廈子卻不好?」韓道國道:「蓋兩間廈子,不如蓋一層兩間小房罷。」於是使了三十兩銀子,又蓋兩間平房起來。

西門慶差玳安兒擡了許多酒、肉、燒餅來,與他家犒賞匠人。那條街上誰人不知。夏提刑得了幾百兩銀子在家,把兒子夏承恩——年十八歲——幹入武學肄業,做了生員。{\meipi{生員徃徃繇此,可嘆。}}每日邀結師友,習學弓馬。西門慶約會劉薛二內相、周守備、荊都監、張團練、合衛官員,出人情與他掛軸文慶賀,俱不必細說。

西門慶因墳上新蓋了山子捲棚房屋,自從生了官哥,並做了千戶,還沒徃墳上祭祖。叫陰陽徐先生看了,從新立了一座墳門,砌的明堂神路,門首栽桃桺,周圍種松柏,兩邊疊成坡峰。清明日上墳,要更換錦衣牌匾,宰豬羊,定桌面。三月初六日清明,預先發柬,請了許多人,搬運了東西、酒米、下飯菜蔬,叫的樂工、雜耍、扮戲的。小優兒是李銘、吳惠、王柱、鄭奉;唱的是李桂姐、吳銀兒、韓金釧,董嬌兒。官客請了張團練、喬大戶、吳大舅、吳二舅、花大舅、沈姨夫、應伯爵、謝希大、傅夥計、韓道國、雲理守、賁第傳並女婿陳敬濟等,約二十餘人。堂客請了張團練娘子、張親家母、喬大戶娘子、朱臺官娘子、尚舉人娘子、吳大妗子、二妗子、楊姑娘、潘姥姥、花大妗子、吳大姨、孟大姨、吳舜臣媳婦鄭三姐、崔本妻段大姐,並家中吳月娘、李嬌兒,孟玉樓、潘金蓮、李瓶兒、孫雪娥、西門大姐、春梅、迎春、玉簫、蘭香、奶子如意兒抱着官哥兒,裡外也有二十四五頂轎子。先是月娘對西門慶說:「孩子且不消教他徃墳上去罷。一來還不曾過一週,二者劉婆子說,這孩子囟門還未長滿,膽兒小。這一到墳上路遠,只怕唬着他。{\meipi{真心實愛。}}依着我不教他去,留下奶子和老馮在家和他做伴兒,只教他娘母子一個去罷。」西門慶不聽,便道:「此來為何?他娘兒兩個不到墳前與祖宗磕個頭兒去!你信那婆子老淫婦胡說,可哥就是孩子囟門未長滿,教奶子用被兒裹着,在轎子裡按的孩兒牢牢的,怕怎的?」那月娘便道:「你不聽人說,隨你。」從清早晨,堂客都從家裡取齊,起身上了轎子,無辭。出南門,到五里外祖墳上,遠遠望見青松鬱鬱,翠柏森森,新蓋的墳門,兩邊坡峰上去,周圍石墻,當中甬道,明堂、神臺、香爐、燭臺都是白玉石鑿的。墳門上新安的牌匾,大書「錦衣武略將軍西門氏先塋」。墳內正面土山環抱,林樹交枝。西門慶穿大紅冠帶,擺設豬羊祭品桌席祭奠。官客祭畢,堂客纔祭。響器鑼鼓,一齊打起來。那官哥兒唬的在奶子懷裡磕伏着,只倒嚥氣,不敢動一動兒。月娘便叫:「李大姐,你還不教奶子抱了孩子徃後邊去哩,你看唬的那腔兒!{\meipi{處處寫出月娘根心生色,一片菩薩熱念。}}我說且不教孩兒來罷,恁強的貨,只管教抱了他來。你看唬的那孩兒這模樣!」李瓶兒連忙下來,分付玳安:「且叫把鑼鼓住了。」連忙攛掇掩着孩兒耳朵,快抱了後邊去了。

須臾祭畢,徐先生念了祭文,燒了紙。西門慶邀請官客在前客位。月娘邀請堂客在後邊捲棚內,逰花園進去,兩邊松墻竹徑,周圍花草,一望無際。正是:

桃紅桺綠鶯梭織,都是東君造化成。

當下,扮戲的在捲棚內扮與堂客們瞧,四個小優兒在前廳官客席前彈唱。四個唱的,輪番遞酒。春梅、玉簫、蘭香、迎春四個,都在堂客上邊執壺斟酒,就立在大姐桌頭,同吃湯飯點心。吃了一囘,潘金蓮與玉樓、大姐、李桂姐、吳銀兒同徃花園裡打了囘鞦韆。

原來捲棚後邊,西門慶收拾了一明兩暗三間房兒。裡邊鋪陳床帳,擺放桌椅、梳籠、抿鏡、粧臺之類,預備堂客來上墳,在此梳粧歇息,糊的猶如雪洞般乾淨,懸掛的書畫,琴棋瀟灑。奶子如意兒看守官哥兒,正在那灑金床炕上鋪着小褥子兒睡,迎春也在旁和他頑耍。只見潘金蓮獨自從花園驀地走來,手中拈着一枝桃花兒,{\meipi{意致便別,韻甚,媚甚。}}看見迎春便道:「你原來這一日沒在上邊伺候。」迎春道:「有春梅、蘭香、玉簫在上邊哩,俺娘叫我下邊來看哥兒,就拏了兩碟下飯點心與如意兒吃。」奶子見金蓮來,就抱起官哥兒來。金蓮便戲他說道:「小油嘴兒,頭裡見打起鑼鼓來,唬的不則聲,原來這等小膽兒。」於是一面解開藕絲羅襖兒,接過孩兒抱在懷裡,與他兩個嘴對嘴親嘴兒。

忽有陳敬濟掀簾子走入來,看見金蓮逗孩子頑耍,便也逗那孩子。金蓮道:「小道士兒,你也與姐夫親個嘴兒。」可霎作恠,那官哥兒便嘻嘻望着他笑。{\meipi{是天緣,走天緣。}}敬濟不繇分說,把孩子就摟過來,一連親了幾個嘴。金蓮罵道:「恠短命,誰家親孩子,把人的髩都抓亂了!」敬濟笑戲道:「你還說,早時我沒錯親了哩。」{\meipi{雖說不親錯,卻正恨不得親錯耳。}}金蓮聽了,恐怕奶子瞧科,便戲發訕,將手中拏的扇子倒過柄子來,向他身上打了一下,打的敬濟鯽魚般跳。罵道:「恠短命,誰和你那等調嘴調舌的!」敬濟道:「不是,你老人家摸量惜些情兒。人身上穿着恁單衣裳,就打恁一下!」金蓮道:「我平白惜甚情兒?今後惹着我,只是一味打。」{\meipi{「今後」二字,「惹着我」三字,隱隱開門揖盜,愛殺,愛殺。}}如意兒見他頑的訕,連忙把官哥兒接過來抱着,金蓮與敬濟兩個還戲謔做一處。金蓮將那一枝桃花兒做了一個圈兒,悄悄套在敬濟帽子上。{\meipi{調處亦是當情,只一桃花圈出自金蓮手,便饒風韻。}}走出去,正值孟玉樓和大姐、桂姐三個從那邊來。大姐看見,便問:「是誰幹的營生?」敬濟取下來去了,一聲兒也沒言語。堂客前戲文扮了四大折。但見:

窓外日光彈指過,席前花影座間移。

看看天色晚來,西門慶分付賁四,先把擡轎子的每人一碗酒、四個燒餅、一盤子熟肉,分散停當,然後,纔把堂客轎子起身。官家騎馬在後,來興兒與廚役慢慢的擡食盒煞後。玳安、來安、畫童、棋童兒跟月娘衆人轎子,琴童並四名排軍跟西門慶馬。奶子如意兒獨自坐一頂小轎,懷中抱着哥兒,用被裹得緊緊的進城。月娘還不放心,又使囘畫童兒來,叫他跟定着奶子轎子,恐怕進城人亂。{\meipi{如此留心,誰人到得。吾謂月娘去《螽斯》之化不遠。}}且說月娘轎子進了城,就與喬家那邊衆堂客轎子分路,來家先下轎進去,半日西門慶、陳敬濟纔到家下馬。只見平安兒迎門就稟說:「今日掌刑夏老爹,親自下馬到廳,問了一遍去了。落後又差人問了兩遍。不知有甚勾當。」{\meipi{閑閑下此數語,隱出緊急情繇,多少波瀾。}}西門慶聽了,心中猶豫。到於廳上,只見書童兒在旁接衣服。西門慶因問:「今日你夏老爹來,留下甚麼話來?」書童道:「他也沒說出來,只問爹徃那去了:『使人請去,我有句要緊話兒說。』小的便道:『今日都徃墳上燒紙去了,至晚纔來。』夏老爹說:『我到午上還來。』落後又差人來問了兩遭,小的說:『還未來哩!』」西門慶心下轉道:「卻是甚麼?」正疑惑之間,只見平安來報:「夏老爹來了。」那時已有黃昏時分,只見夏提刑便衣坡巾,兩個伴當跟隨。下馬到於廳上叙礼,說道:「長官今日徃宝庄去來?」西門慶道:「今日先塋祭掃,不知長官下降,失迎,恕罪,恕罪!」夏提刑道:「有一事敢來報與長官知道。」因說:「咱們徃那邊客位內坐去罷。」西門慶令書童開捲棚門,請徃那裡說話,左右都令下去。夏提刑道:「今朝縣中李大人到學生那裡,如此這般,說大巡新近有叅本上東京,長官與學生俱在叅例。學生令人抄了個底本在此,與長官看。」西門慶聽了,大驚失色,急接過底報來燈下觀看,端的上面寫着甚言:

巡按山東監察御史曾孝序一本:叅劾貪肆不職武官,乞賜罷黜,以正法紀事:臣聞巡搜四方,省察風俗,乃天子巡狩之事也;彈壓官邪,振揚法紀,乃御史糾政之職也。昔《春秋》載天王巡狩,而萬邦懷保,民風協矣,王道彰矣,四民順矣,聖治明矣。臣自去年奉命巡按山東齊魯之邦,一年將滿,歷訪方面有司文武官員賢否,頗得其實。茲當差滿之期,敢不循例甄別,為我皇上陳之!除叅劾有司方面官員,另具疏上請。叅照山東提刑所掌刑金吾衛正千戶夏延齡,闒茸之材,貪鄙之行,久於物議,有玷班行。昔者典牧皇畿,大肆科擾,被屬官陰發其私。今省理山東刑獄,復着狼貪,為同僚之箝制。縱子承恩冒籍武舉,倩人代考,而士風掃地矣。信家人夏壽監索班錢,被軍騰詈而政事不可知乎!接物則奴顏婢膝,時人有丫頭之稱;問事則依違兩可,群下有木偶之誚。理刑副千戶西門慶,本系市井棍徒,夤緣陞職,濫冒武功,菽麥不知,一丁不識。縱妻妾嬉遊街巷,而帷薄為之不清;攜樂婦而酣飲市樓,官箴為之有玷。至於包養韓氏之婦,恣其歡淫,而行檢不修;受苗青夜賂之金,曲為掩飾,而賍跡顯着。此二臣者,皆貪鄙不職,久乖清議,一刻不可居任者也。伏望聖明垂聽,勑下該部,再加詳查。如果臣言不謬,將延齡等亟賜罷斥,則官常有賴而俾聖德永光矣。

西門慶看了一遍,唬的面面相覷,默默不言。夏提刑道:「長官,似此如何計較?」西門慶道:「常言『兵來將擋,水來土掩』。事到其間,道在人為。少不的你我打點禮物,早差人上東京央及老爺那裡去。」於是,夏提刑急急作辭,到家拏了二百兩銀子、兩把銀壺。西門慶這裡是金鑲玉寶石鬧粧一條、三百兩銀子。夏家差了家人夏壽,西門慶這裡是來保,將禮物打包端正,西門慶寫了一封書與翟管家,兩個早顧了頭口,星夜徃東京幹事去了,不題。

且表官哥兒自從墳上來家,夜間只是驚哭,不肯吃奶。但吃下奶去就吐了。慌的李瓶兒走來告訴月娘,月娘道:「我那等說,還未到一週的孩子,且休帶他出城門去。濁漒貨,他生死不依,只說:『今日墳上祭祖為甚麼來?不教他娘兒兩個走走!』只象那裡攙了分兒一般,睜着眼和我兩個叫。如今卻怎麼好?」{\meipi{不聽好言,宜乎有此。}}李瓶兒正沒法兒擺佈。況西門慶又因巡按叅了,和夏提刑在前邊說話,徃東京打點幹事,心上不遂,家中孩子又不好。月娘使小厮叫婆子來看,又請小兒科太醫,開門闔戶,亂了一夜。劉婆子看了說:「哥兒着了些驚氣入肚,又路上撞見五道將軍。不打緊,買些紙兒退送退送就好了。」又留了兩服硃砂丸藥兒,用薄荷燈心湯送下去,那孩兒方纔寧貼睡了一覺,不驚哭吐奶了。只是身上熱還未退,李瓶兒連忙拏出一兩銀子,教劉婆子備紙去。後又帶了他老公,還和一個師婆來,在捲棚內與哥兒燒紙跳神。那西門慶早五更打發來保、夏壽起身,就亂着和夏提刑徃東平府胡知府那裡,打聽提苗青訊息去了。吳月娘聽見劉婆說孩子路上着了驚氣,甚是抱怨如意兒,{\meipi{病根還在金蓮調戲,筆意隱然卻不說出,妙手。}}說他:「不用心看孩兒,想必路上轎子裡唬了他了。不然,怎的就不好起來?」如意兒道:「我在轎子裡,將被兒包得緊緊的,又沒[]着他。娘叫畫童兒來跟着轎子,他還好好的,我按着他睡。只進城七八到家門首,我只覺他打了個冷戰,到家就不吃奶,哭起來了。」{\meipi{因劉婆數語,奶子便得藉口,自是恆情。}}按下這裡家中燒紙,與孩子下神。

且說來保、夏壽一路攢行,只六日就趕到東京城內。到太師府內見了翟管家,將兩家禮物交割明白。翟謙看了西門慶書信,說道:「曾御史叅本還未到哩,{\meipi{本尚未行,而打點先到,的真神手!}}你且住兩日。如今老爺新近條陳了七件事,旨意還未曾下來。待行下這個本去,曾御史本到,等我對老爺說,交老爺閣中只批與他『該部知道』。我這裡差人再拏帖兒分付兵部余尚書,把他的本只不覆上來。交你老爹只顧放心,管情一些事兒沒有。」於是把二人管待了酒飯,還歸到客店安歇,等聽訊息。一日,蔡太師條陳本,聖旨準下來了。來保央府中門吏暗暗抄了個邸報,帶囘家與西門慶瞧,不在話下。

一日,等的翟管家寫了囘書,與了五兩盤纏,與夏壽取路囘山東清河縣。來到家中,西門慶正在家耽心不下,那夏提刑一日一遍來問信。聽見來保二人到了,叫至後邊問他端的。來保對西門慶悉把上項事情訴說一遍,道:「翟爹看了爹的書,便說:『此事不打緊,教你爹放心。見今巡按也滿了,另點新巡按下來了。況他的叅本還未到,等他本上時,等我對老爺說了,隨他本上叅的怎麼重,只批「該部知道」。老爺這裡再拏帖兒分付兵部餘尚書,只把他的本立了案不覆上去,隨他有撥天關本事也無妨。』」西門慶聽了,方纔心中放下。因問:「他的本怎還不到?」來保道:「俺們一去時,晝夜馬上行去,只五日就趕到京中,可知在他頭裡。俺每囘來,見路上一簇响鈴驛馬,背着黃色袱,插着兩根雉尾、兩面牙旗,怕不就是巡按衙門進送實封纔到了。」西門慶道:「得他的本上的遲,事情就停當了。我只怕去遲了。」來保道:「爹放心,管情沒事。小的不但幹了這件事,又打聽得兩樁好事來,報爹知道。」西門慶問道:「端的何事?」來保道:「太師老爺新近條陳了七件事,旨意已是準行。如今老爺親家戶部侍郎韓爺題準事例:在陝西等三邊開引種鹽,各府州郡縣設立義倉,官糶糧米。令民間上上之戶赴倉上米,討倉鈔,派給鹽引支鹽。舊倉鈔七分,新倉鈔三分。咱舊時和喬親家爹,高陽關上納的那三萬糧倉鈔,派三萬鹽引,戶部坐派。如今蔡狀元又點了兩淮巡鹽,不日離京,倒有好些利息。」{\meipi{既遶鉅萬,復悉錙銖,來保亦可兒也。}}西門慶聽言問道:「真個有此事?」來保道:「爹不信,小的抄了個邸報在此。」向書篋中取出來與西門慶觀看。因見上面許多字樣,前邊叫了陳敬濟來念與他聽。陳敬濟念到中間,只要結住了,還有幾個眼生字不認的。旋叫了書童兒來念。那書童倒還是門子出身,蕩蕩如流水不差,直念到底。端的上面奏着那七件事:

崇政殿大學士吏部尚書魯國公蔡京一本:為陳愚見,竭愚衷,收人才,臻實效,足財用,便民情,以隆聖治事:{\meipi{此疏條理井然,使實心行之,當亦有利,孰得以其人而忽其言乎。}}

第一曰:罷科舉取士,悉繇學校陞貢。竊謂教化淩夷,風俗頹敗,皆繇取士不得真才,而教化無以仰賴。《書》曰:「天生斯民,作之君,作之師。」漢舉孝廉,唐興學校,我國家始制考貢之法,各執偏陋,以致此輩無真才,而民之司牧何以賴焉?今皇上寤寐求才,宵旰圖治。治在於養賢,養賢莫如學校。今後取士,悉遵古繇學校陞貢。其州縣發解礼闈,一切罷之。每歲考試上舍則差知貢舉,亦如礼闈之式。仍立八行取士之科。八行者,謂孝、友、睦、姻、任、恤、忠、和也。士有此者,即免試,率相補太學上舍

二曰:罷講議財利司。竊惟國初定製,都堂置講議財利司。蓋謂人君節浮費,惜民財也。今陛下即位以來,不寶遠物,不勞逸民,躬行節儉以自奉。蓋天下亦無不可返之俗,亦無不可節之財。惟當事者以俗化為心,以禁令為信,不忽其初,不弛其後,治隆俗美,豐亨豫大,又何講議之為哉?悉罷。

三曰:更鹽鈔法。竊惟鹽鈔,乃國家之課以供邊備者也。今合無遵復祖宗之製鹽法者。詔雲中、陝西、山西三邊,上納糧草,關領舊鹽鈔,易東南淮浙新鹽鈔。每鈔折派三分,舊鈔搭派七分。今商人照所派產鹽之地,下場支鹽。亦如茶法,赴官秤驗,納息請批引,限日行鹽之處販賣。如遇過限,並行拘收;別買新引增販者,俱屬私鹽。如此則國課日增,而邊儲不乏矣。

四曰:制錢法。竊謂錢貨,乃國家之血脈,貴乎流通而不可淹滯。如有厄阻淹滯不行者,則小民何以變通,而國課何以仰賴矣?自晉末鵝眼錢之後,至國初瑣屑不堪,甚至雜以鉛鐵夾錫。邊人販於虜,因而鑄兵器,為害不小,合無一切通行禁之也。以陛下新鑄大錢崇甯、大觀通寶,一以當十,庶小民通行,物價不致於踴貴矣。

五曰:行結糶俵糴之法。竊惟官糴之法,乃賑恤之義也。近年水旱相仍,民間就食,上始下賑恤之詔。近有戶部侍郎韓侶題覆欽依:將境內所屬州縣各立社會,行結糶俵糴之法。保之於黨,黨之於里,里之於鄉,倡之結也。每鄉編為三戶,按上上、中中、下下。上戶者納糧,中戶者減半,下戶者退派糧數關支,謂之俵糶。如此則斂散便民之法得以施行,而皇上可廣不費之仁矣。惟責守令核切舉行,其關係蓋匪細矣。

六曰:詔天下州郡納免夫錢。竊惟我國初,寇亂未定,悉令天下軍徭丁壯,集於京師,以供運餽,以壯國勢。今承平日久,民各安業,合頒詔行天下州郡,每歲上納免夫錢,每名折錢三十貫,解赴京師,以資邊餉之用。庶兩得其便,而民力少蘇矣。

七曰:置提舉御前人船所。竊惟陛下自即位以來,無聲色犬馬之奉。所尚花石,皆山林間物,乃人之所棄者。但有司奉行之過,因而致擾,有傷聖治。陛下節其浮濫,仍請作御前提舉人船所。凡有用悉出內帑,差官取之,庶無擾於州郡。伏乞聖裁。{\meipi{數語微露侫吻,入下俱見憂民之忠。}}

奉旨曰:卿言深切時艱,朕心嘉悅,足見忠猷,都依擬行。該部知道。

西門慶聽了,又看了翟管家書信,已知禮物交得明白。蔡狀元見朝,又點了兩淮巡鹽,不日徃此經過,心中不勝歡喜。一面打發夏壽囘家:「報與你老爹知道。」一面賞了來保五兩銀子、兩瓶酒、一方肉,囘房歇息,不在話下。正是:

樹大招風風損樹,人為名高名䘮身。

有詩為證:

得失榮枯命裡該,皆因年月日時栽。胸中有志終須至,囊內無財莫論才。

