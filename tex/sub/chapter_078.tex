\includepdf[pages={155,156},fitpaper=false]{tst.pdf}
\chapter*{第七十八囘 林太太鴛幃再戰 如意兒莖露獨嘗}
\addcontentsline{toc}{chapter}{第七十八囘 林太太鴛幃再戰 如意兒莖露獨嘗}
\markboth{{\titlename}卷之八}{第七十八囘 林太太鴛幃再戰 如意兒莖露獨嘗}


詞曰:

\begin{myquote}
鳳髻金泥帶,龍紋玉掌梳。去來窓下笑來扶,愛道畫眉深淺入時無?弄筆偎人久,描花試手初。等閑含笑問狂夫,笑問歡情不減舊時麼?
\end{myquote}

話說西門慶陪大舅飲酒,至晚囘家。到次日,荊都監早晨騎馬來拜謝,說道:「昨日見旨意下來,下官不勝歡喜,足見老翁愛厚,費心之至,實為啣結難忘。」說畢,茶湯兩換,荊都監起身,因問:「雲大人到幾時請俺們吃酒?」西門慶道:「近節這兩日也是請不成,直到正月間罷了。」送至大門,上馬而去。西門慶宰了一口鮮豬,兩罈浙江酒,一疋大紅絨金豸員領,一疋黑青粧花紵絲員領,一百菓餡金餅,謝宋御史。就差春鴻拏貼兒,送到察院去。門吏人報進去,宋御史喚至後廳火房內,賞茶吃。等寫了囘帖,又賞了春鴻三錢銀子。來見西門慶,拆開觀看,上寫着:

\begin{myquote}[\markfont]
「兩次造擾華府,悚愧殊甚。今又辱承厚貺,何以克當?外令親荊子事,已具本矣,想已知悉。連日渴仰丰標,容當面悉。使旋謹謝。

侍生宋喬年拜大錦衣西門先生大人門下。」
\end{myquote}

宋御史隨即差人,送了一百本歷日,四萬紙,一口豬來囘禮。一日,上司行下文書來,令吳大舅本衛到任管事。西門慶拜去,就與吳大舅三十兩銀子,四疋京段,交他上下使用。到二十四日,封了印來家,又備羊酒花紅軸文,邀請親朋,等吳大舅從衛中上任囘來,迎接到家,擺大酒席與他作賀。又是何千戶東京家眷到了,西門慶寫月娘名字,送茶過去。到二十六日,玉皇廟吳道官十二個道衆,在家與李瓶兒念百日經,整做法事,大吹大打,各親朋都來送茶,請吃齋供,至晚方散,俱不在言表。

至廿七日,西門慶打發各家送禮,應伯爵、謝希大、常峙節、傅夥計、甘夥計、韓道國、賁第傳、崔本,每家半口豬,半腔羊,一罈酒,二包米,一兩銀子,院中李桂姐、吳銀兒、鄭愛月兒,每人一套衣服,三兩銀子。{\meipi{如此財主,儘自不俗。}}吳月娘又與庵裡薛姑子打齋,令來安兒送香油、米麵、銀錢去,不在言表。

看看到年除之日,牕梅表月,簷雪滾風,{\pangpi{幽致。}}竹爆千門萬戶,家家貼春勝,處處掛桃符。西門慶燒了紙,又到於李瓶兒房,靈前祭奠。祭畢,置酒於後堂,合家大小歡樂。手下家人小厮並丫頭媳婦,都來磕頭。西門慶與吳月娘,俱有手帕、汗巾、銀錢賞賜。

到次日,重和元年新正月元旦,西門慶早起冠冕,穿大紅,天地上燒了紙,吃了點心,備馬就拜巡按賀節去了。月娘與衆婦人早起來,施朱傅粉,插花插翠,錦裙繡襖,羅襪弓鞋,粧點妖嬈,打扮可喜,都來月娘房裡行禮。那平安兒與該日節級,在門首接拜貼,上門簿,答應徃來官長士夫。玳安與王經穿着新衣裳,新靴新帽,在門首踢毽子,放炮𤍤,磕瓜子兒。衆夥計主管,伺候見節者,不計其數,都是陳敬濟一人管待。{\meipi{畫出新年光景。}}約晌午,西門慶徃府縣拜了人囘來,剛下馬,招宣府王三官兒衣巾着來拜。到廳上拜了西門慶四雙八拜,然後請吳月娘見。西門慶請到後邊,與月娘見了,出來前廳留坐。纔拏起酒來吃了一盞,只見何千戶來拜。西門慶就叫陳敬濟管待陪王三官兒,他便徃捲棚內陪何千戶坐去了。王三官吃了一囘,告辭起身。陳敬濟送出大門,上馬而去。落後又是荊都監、雲指揮、喬大戶,皆絡繹而至。西門慶待了一日人,已酒帶半酣,至晚打發人去了,囘到上房歇了一夜。到次早,又出去賀節,至晚歸來,家中已有韓姨夫、應伯爵、謝希大、常峙節、花子繇來拜。陳敬濟陪在廳上坐的。西門慶到了,見畢禮,重新擺上酒來飲酒。韓姨夫與花子繇隔門,先去了。剩下伯爵、希大、常峙節,坐個定光油兒不去。{\pangpi{常態,也不足多恠。}}又撞見吳二舅來了,見了禮,又徃後邊拜見月娘,出來一處坐的。直吃到掌燈已後方散。

西門慶已吃的酩酊大醉,送出伯爵,等到門首衆人去了。西門慶見玳安在旁站立,捏了一把手。玳安就知其意,說道:「他屋裡沒人。」這西門慶就撞入他房內。老婆早已在門裡迎接進去。{\meipi{賁四嫂與王六兒一般夥計娘子,而巧拙遂分厚薄。}}兩個也無閑話,走到裡間,脫衣解帶就幹起來。原來老婆好並着腿幹,兩隻手𢵞着,只教西門慶攮他心子。那浪水熱熱一陣流出來,把床褥皆濕。西門慶龜頭蘸了藥,攮進去,兩手扳着腰,只顧揉搓,麈柄盡入至根,不容毫髮,婦人瞪目,口中只叫「親爺。」那西門慶問他:「你小名叫甚麼?說與我。」老婆道:「奴娘家姓葉,排行五姐。」西門慶口中喃喃吶吶,就叫「葉五兒」不絕。那老婆原是奶子出身,與賁四私通,被拐出來,佔為妻子。今年三十二歲,甚麼事兒不知道!口裡如流水連叫「親爺」不絕,情濃一泄如注。西門慶扯出麈柄要抹,婦人攔住:「休抹,等淫婦下去,替你吮淨了罷。」西門慶滿心歡喜,婦人真個蹲下身子,雙手捧定那話,吮咂得乾乾淨淨,纔繫上褲子。因問西門慶:「他怎的去恁些時不來?」西門慶道:「我這裡也盼他哩。只怕京中你夏老爹留住他使。」又與了老婆二、三兩銀子盤纏,因說:「我待與你一套衣服,恐賁四知道不好意思。不如與你些銀子兒,你自家治買罷。」開門送出來。玳安又早在鋪子裡掩門等候。西門慶便徃後邊去了。

看官聽說,自古上梁不正則下梁歪,原來賁四老婆先與玳安有姦,這玳安剛打發西門慶進去了,因傅夥計又沒在鋪子裡上宿,他與平安兒打了兩大壺酒,就在老婆屋裡吃到有二更時分,平安在鋪子裡歇了,他就和老婆在屋裡睡了一宿。有這等的事!正是:

\begin{myquote}
滿眼風流滿眼迷,殘花何事濫如泥。\\拾琴暫息商陵操,惹得山禽遶樹啼。
\end{myquote}

卻說賁四老婆晚夕同玳安睡了,因對他說:「我一時依了爹,只怕隔壁韓嫂兒傳嚷的後邊知道,也似韓夥計娘子,一時被你娘們說上幾句,羞人答答的,怎好相見?」玳安道:「如今家中,除了俺大娘和五娘不言語,別的不打緊。俺大娘倒也罷了,只是五娘快出尖兒。你依我,節間買些甚麼兒,進去孝順俺大娘。別的不稀罕,他平昔好吃蒸酥,你買一錢銀子菓餡蒸酥、一盒好大壯瓜子送進去。這初九日是俺五娘生日,你再送些禮去,梯己再送一盒瓜子與俺五娘。管情就掩住許多口嘴。」{\meipi{金蓮於財、色二者,無所不愛,然亦有以其不甚愛而愛其所最愛者。色不可自主,而財則亦其樂得也。}}這賁四老婆真個依着玳安之言,第二日趕西門慶不在家,玳安就替他買了盒子,掇進月娘房中。月娘便道:「是那裡的?」玳安道:「是賁四嫂子送與娘吃的。」月娘道:「他男子漢又不在家,那討個錢來,又交他費心。」連忙收了,又囘出一盒饅頭,一盒菓子,說:「上覆他,多謝了。」

那日西門慶拜人囘家,早又玉皇廟吳道官來拜,在廳上留坐吃酒。剛打發吳道官去了,西門慶脫了衣服,使玳安:「你騎了馬,問聲文嫂兒去:『俺爹今日要來拜拜太太。』看他怎的說?」玳安道:「爹,不消去,頭裡文嫂兒騎着驢子打門首過去了。他說明日初四,王三官兒起身徃東京,與六黃公公磕頭去了。太太說,交爺初六日過去見節,他那裡伺候。」西門慶便道:「他真個這等說來?」玳安道:「莫不小的敢說謊!」這西門慶就入後邊去了。剛到上房坐下,忽來安兒來報:「大舅來了。」只見吳大舅冠冕着,束着金帶,進入後堂,先拜西門慶,說道:「我吳鎧多蒙姐夫擡舉看顧,又破費姐夫,多謝厚禮。昨日姐夫下降,我又不在家,失迎。今日敬來與姐夫磕個頭兒,恕我遲慢之罪。」說着,磕下頭去。西門慶慌忙頂頭相還,說道:「大舅恭喜,至親何必計較。」拜畢,月娘出來與他哥磕頭。慌的大舅忙還半禮,說道:「姐姐,兩禮兒罷,哥哥嫂嫂不識好歹,常來擾害你兩口兒。你哥老了,看顧看顧罷。」月娘道:「一時有不到處,望哥耽帶便了。」吳大舅道:「姐姐沒的說,累你兩口兒還少哩?」拜畢,西門慶留吳大舅坐,說道:「這咱晚了,料大舅也不拜人了,寬了衣裳,咱房裡坐罷。」不想孟玉樓與潘金蓮兩個都在屋裡,聽見嚷吳大舅進來,連忙走出來,與大舅磕頭。磕了頭,徑徃各人房裡去了。西門慶讓大舅房內坐的,騎火盆安放桌兒,擺上菜兒來。小玉、玉簫都來與大舅磕頭。月娘用小金鑲鍾兒,斟酒遞與大舅,西門慶主位相陪。吳大舅讓道:「姐姐你也來坐的。」月娘道:「我就來。」又徃裡間房內,拏出數樣配酒的菓菜來。飲酒之間,西門慶便問:「大舅的公事都停當了?」吳大舅道:「蒙姐夫擡舉,衛中任便到了,上下人事,倒也都周給的七八。只有屯所裡未曾去到到任。明日是個好日期,衛中開了印,來家整理些盒子,須得擡到屯所裡到任,行牌拘將那屯頭來叅見,分付分付。前官丁大人壞了事情,已被巡撫侯爺叅劾去了。如今我接管承行,須要振刷在冊花戶,警勵屯頭,務要把這舊管新增開報明白,到明日秋糧夏稅,纔好下屯徵收。」西門慶道:「通共約有多少屯田?」吳大舅道:「太祖舊例,為養兵省轉輸之勞,纔立下這屯田。那時只是上納秋糧,後吃宰相王安石立青苗法,增上這夏稅。而今濟州管內,除了拋荒、葦場、港隘,通共二萬七千頃屯地。每頃秋稅夏稅只徵收一兩八錢,不上五百兩銀子。到年終總傾銷了,徃東平府交納,轉行招商,以備軍糧馬草作用。」西門慶又問:「還有羨餘之利?」吳大舅道:「雖故還有些拋零人戶不在冊者,鄉民頑滑,若十分徵緊了,等秤斛斗量,恐聲口致起公論。」西門慶道:「若是多寡有些兒也罷,難道說全徵?」吳大舅道:「不瞞姐夫說,若會管此屯,見一年也有百十兩銀子。到年終,人戶們還有些雞鵝豕米相送,那個是各人取覓,不在數內的。只是多賴姐伕力量扶持。」西門慶道:「得勾你老人家攪給,也盡我一點之心。」說了囘,月娘也走來旁邊陪坐,三人飲酒。到掌燈已後,吳大舅纔起身去了。西門慶就在金蓮房中歇了一夜。到次日早徃衙門中開印,陞廳畫卯,發放公事。先是雲理守家發貼兒,初五日請西門慶併合衛官員吃慶官酒。次日,何千戶娘子藍氏下貼兒,初六日請月娘姊妹相會。

且說那日西門慶同應伯爵、吳大舅三人起身到雲理守家。原來旁邊又典了人家一所房子,三間客位內擺酒,叫了一起吹打鼓樂迎接,都有桌面,吃至晚夕來家。巴不到次日,月娘徃何千戶家吃酒去了。西門慶打選衣帽齊整,騎馬帶眼紗,玳安、琴童跟隨,午後時分,徑來王招宣府中拜節。王三官兒不在,送進貼兒去。文嫂兒又早在那裡,接了貼兒,連忙報與林太太說,出來,請老爺後邊坐。轉過大廳,到於後邊,掀起明簾,只見裡邊氍毹匝地,簾幕垂紅。

少頃,林氏穿着大紅通袖袍兒,珠翠盈頭,與西門慶見畢禮數,留坐待茶,分付:「大官,把馬牽於後槽喂養。」茶罷,讓西門慶寬衣房內坐,說道:「小兒從初四日徃東京與他叔岳父六黃太尉磕頭去了,只過了元宵纔來。」西門慶一面喚玳安,脫去上蓋,裡邊穿着白綾襖子,天青飛魚氅衣,十分綽耀。婦人房裡安放桌席。須臾,丫鬟拏酒菜上來,盃盤羅列,餚饌堆盈,酒泛金波,茶烹玉蕊。婦人玉手傳盃,秋波送意,猜枚擲骰,笑語烘春。話良久,意洽情濃;飲多時,目邪心蕩。看看日落黃昏,又早高燒銀燭。玳安、琴童自有文嫂兒管待,等閑不過這邊來。婦人又倒扣角門,僮僕誰敢擅入。酒酣之際,兩人共入裡間房內,掀開繡帳,關上窓戶,輕剔銀缸,忙掩朱戶。男子則解衣就寢,婦人即洗牝上床,枕設寶花,被翻紅浪。原來西門慶帶了淫器包兒來,安心要鏖戰這婆娘,早把胡僧藥用酒吃在腹中,那話上使着雙托子,在被窩中,架起婦人兩股,縱麈柄入牝中,舉腰展力,一陣掀騰鼓搗,連聲响喨。婦人在下,沒口叫親達達如流水。正是:招海旌幢秋色裡,擊天鼙鼓月明中。但見:

\begin{myquote}
迷魂陣罷,攝魄旗開。迷魂陣上,閃出一員酒金剛,色魔王能爭慣戰;攝魂旗下,擁一個粉骷髏,花狐狸百媚千嬌。這陣上,撲鼕鼕,鼓震春雷;那陣上,鬧挨挨,麝蘭靉靆。這陣上,復溶溶,被翻紅浪精神健;那陣上,刷剌剌,帳控銀鉤情意乖。這一個急展展,二十四解任徘徊;那一個忽剌剌,一十八滾難掙扎。鬪良久,汗浸浸,釵橫鬢亂;戰多時,喘吁吁,枕側衾歪。頃刻間,腫眉𦣘眼;霎時下,肉綻皮開。
\end{myquote}

正是:

\begin{myquote}
幾番鏖戰貪淫婦,不是今番這一遭。
\end{myquote}

當下西門慶就在這婆娘心口與陰戶燒了兩炷香,許下明日家中擺酒,使人請他同三官兒娘子去看燈耍子。這婦人一段身心已被他拴縛定了,於是滿口應承都去。西門慶滿心歡喜,起來與他留連痛飲,至二更時分,把馬從後門牽出,作別囘家。正是:

\begin{myquote}
盡日思君倚畫樓,相逢不捨又頻留。\\劉郎莫謂桃花老,浪把輕紅逐水流。
\end{myquote}

西門慶到家,有平安攔門稟說:「今日有薛公公家差人送請貼兒,請爹早徃門外皇庄看春。又是雲二叔家送了五個貼兒,請五位娘吃節酒。」西門慶聽了,進入月娘房來。只見孟玉樓、潘金蓮都在房內坐的。月娘從何千戶家赴了席來家,正坐着說話。見西門慶進來,連忙道了萬福。因問:「你今日徃那裡,這咱纔來?」西門慶沒得說,只說:「我在應二哥家留坐。」月娘便說起今日何千戶家酒席上事:「原來何千戶娘子年還小哩,今年纔十八歲,生的燈上人兒也似,一表人物,好標緻,知今博古,見我去,恰似會了幾遍,好不喜洽。嫁了何大人二年光景,房裡到使着四個丫頭,兩個養娘,兩房家人媳婦。」西門慶道:「他是內府生活所藍太監姪女兒,嫁與他陪了好少錢兒!」月娘道:「明日雲夥計家,又請俺每吃節酒,送了五個貼兒業,端的去不去?」西門慶說:「他既請你每,都去走走罷。」月娘道:「留雪姐在家罷,只怕大節下,一時有個人客闖將來,他每沒處撾撓。」西門慶道:「也罷,留雪姐在家裡,你每四個去罷。明日薛太監請我看春,我也懶待去。這兩日春氣發也怎的,只害這腰腿疼。」月娘道:「你腰腿疼只怕是痰火,問任醫官討兩服藥吃不是,只顧挨着怎的?」西門慶道:「不妨事,由他。一發過了這兩日吃,心淨些。」因和月娘計較:「到明日燈節,咱少不的置席酒兒,請請何大人娘子。連周守備娘子,荊南崗娘子,張親家母,雲二哥娘子,連王三官兒母親,和大妗子、崔親家母,這幾位元都會會。也只在十二三,掛起燈來。還叫王皇親家那起小厮扮戲耍一日。去年還有賁四在家,紮幾架烟火放,今年他東京去了,只顧不見來,卻教誰人看着紮?」那金蓮在旁插口道:「賁四去了,他娘子兒紮也是一般。」{\meipi{心痛病人,便一句說話吃不起。}}這西門慶就瞅了金蓮道:「這個小淫婦兒,三句話就說下道兒去了。」那月娘、玉樓也不採顧,就罷了。因說道:「那王三官兒娘,咱每與他沒會過,人生面不熟,怎麼好請他?只怕他也不肯來。」西門慶道:「他既認我做親,咱送個貼兒與他,來不來,隨他就是了。」月娘又道:「我明日不徃雲家去罷,懷着個臨月身子,只管徃人家撞來撞去的,交人家唇齒。」玉樓道:「怕怎的,你身子懷的又不顯,怕還不是這個月的孩子,不妨事。大節下自恁散心,去走走兒纔好。」說畢,西門慶吃了茶,就徃後邊孫雪娥房裡去了。那潘金蓮見他徃雪娥房中去,叫了大姐,也就徃前邊去了。西門慶到於雪娥房中,交他打腿捏身上,捏了半夜。{\meipi{吾惱如雪娥者,不得歡娛而反勞碌。}}一宿晚景題過。

到次日早晨,只見應伯爵走來,對西門慶說:「昨日雲二嫂送了個貼兒,今日請房下陪衆嫂子坐。家中舊時有幾件衣服兒,都倒塌了。大正月不穿件好衣服,惹的人家笑話。敢來上覆嫂子,有上蓋衣服,借約兩套兒,頭面簪環,借約幾件兒,交他穿戴了去。」西門慶令王經:「你裡邊對你大娘說去。」伯爵道:「應寶在外邊拏着氊包並盒兒哩。哥哥,累你拏進去,就包出來罷。」那王經接氊包進去,良久抱出來,交與應寶,說道:「裡面兩套上色段子織金衣服,大小五件頭面,一雙環兒。」{\meipi{此一去有得來否?}}應寶接的去了。西門慶陪伯爵吃茶,說道:「今日薛內相又請我門外看春,怎麼得工夫去?吳親家廟裡又送貼兒,初九日年例打醮,也是去不成,教小婿去罷了。這兩日不知酒多了也怎的,只害腰疼,懶待動旦。」伯爵道:「哥,你還是酒之過,濕痰流注在這下部,也還該忌忌。」西門慶道:「這節間到人家,誰肯輕放了你,怎麼忌的住?」

正說着,只見玳安拏進盒兒來,說道:「何老爹家差人送請貼兒來,初九日請吃節酒。」西門慶道:「早是你看着,人家來請,你怎不去?」於是看盒兒內,放着三個請貼兒,一個雙紅僉兒,寫着「大寅丈四泉翁老先生大人」,一個寫「大都閫吳老先生大人」,一個寫着「大鄉望應老先生大人」,俱是「侍教生何永壽頓首拜」。玳安說:「他說不認的,教咱這裡轉送送兒去。」伯爵一見便說:「這個卻怎樣兒的?我還沒送禮兒去與他,怎好去?」西門慶道:「我這裡替你封上分帕禮兒,你差應寶早送去就是了。」一面令王經:「你封二錢銀子,一方手帕,寫你應二爹名字,與你應二爹。」因說:「你把這請貼兒袖了去,省的我又教人送。」只把吳大舅的差來安兒送去了。須臾,王經封了帕禮遞與伯爵。伯爵打恭說道:「又多謝哥,我後日早來會你,咱一同起身。」

說畢,作辭去了。午間,吳月娘等打扮停當,一頂大轎,三頂小轎,後面又帶着來爵媳婦兒惠元,收疊衣服,一頂小轎兒,四名排軍喝道,琴童、春鴻、棋童、來安四個跟隨,徃雲指揮家來吃酒。正是:

\begin{myquote}
翠眉雲鬢畫中人,嬝娜宮腰迥出塵。\\天上嫦娥元有種,嬌羞釀出十分春。
\end{myquote}

不說月娘衆人吃酒去了。且說西門慶分付大門上平安兒:「隨問甚麼人,只說我不在。有貼兒接了就是了。」那平安經過一遭,那裡再敢離了左右,只在門首坐的。但有人客來望,只囘不在家。西門慶因害腿疼,猛然想起任醫官與他延壽丹,用人乳吃。{\meipi{此非延壽丹,乃催命藥也。}}於是來到李瓶兒房中,叫迎春拏菜兒,篩酒來吃。迎春打發了,就走過隔壁,和春梅下棋去了。要茶要水,自有如意兒打發。西門慶見丫鬟不在屋裡,就在炕上斜靠着。露出那話,帶着銀托子,教他用口吮咂。一面斟酒自飲,因呼道:「章四兒,我的兒,你用心替達達咂,我到明日,尋出件好粧花段子比甲兒來,你正月十二日穿。」老婆道:「看他可憐見。」咂弄勾一頓飯時,西門慶道:「我兒,我心裡要在你身上燒炷香兒。」老婆道:「隨爹揀着燒。」西門慶令他關上房門,把裙子脫了,仰臥在炕上。西門慶袖內還有燒林氏剩下的三個燒酒浸的香馬兒,撇去他抹胸兒,一個坐在他心口內,一個坐在他小肚兒底下,一個安在他𣭈蓋子上,用安息香一齊點着,那話下邊便插進牝中,低着頭看着拽,只顧僅沒其稜,徃來送進不已。又取過鏡臺來旁邊照看,{\pangpi{好看。}}須臾,那香燒到肉根前,婦人蹙眉齧齒,忍其疼痛,口裡顫聲柔語,哼成一塊,沒口子叫:「達達,爹爹,罷了我了,好難忍他。」西門慶便叫道:「章四兒淫婦,你是誰的老婆?」婦人道:「我是爹的老婆。」西門慶教與他:「你說是熊旺的老婆,今日屬了我的親達達了。」{\meipi{如此作情語,抵見其俗耳,有何妙處?然出自西門慶口中,固妙。}}那婦人迴應道:「淫婦原是熊旺的老婆,今日屬了我的親達達了。」西門慶又問道:「我會㒲不會?」婦人道:「達達會㒲𣭈。」兩個淫聲艷語,無般言語不說出來。西門慶那話粗大,撐得婦人牝中滿滿,徃來出入,帶的花心紅如鸚鵡舌,黑似蝙蝠翅,翻覆可愛。西門慶於是把他兩股扳抱在懷內,四體交匝,兩廂迎湊,那話盡沒至根,不容毫髮,婦人瞪目失聲,淫水流下,西門慶情濃樂極,精邈如泉湧。正是:

\begin{myquote}
不知已透春訊息,但覺形骸骨節镕。
\end{myquote}

西門慶燒了老婆身上三處香,開門尋了一件玄色段子粧花比甲兒與他。至晚,月娘衆人來家,對西門慶說:「原來雲二嫂也懷着個大身子,俺兩今日酒席上都遞了酒,說過,到明日兩家若分娩了,若是一男一女,兩家結親做親家;若都是男子,同堂攻書;若是女兒,拜做姐妹,一處做針指,來徃親戚耍子。應二嫂做保證。」西門慶聽的笑了。

話休饒舌。到第二日,卻是潘金蓮上壽。西門慶早起徃衙門中去了,分付小厮每擡出燈來,收拾揩抹乾淨,各處張掛。叫來興買鮮菓,叫小優晚夕上壽。潘金蓮早晨打扮出來,花粧粉抹,翠袖朱唇,走來大廳上。看見玳安與琴童站在高櫈上掛燈,因笑嘻嘻說道:「我道是誰在這裡,原來是你每掛燈哩。」琴童道:「今日是五娘上壽,爹分付叫俺每掛了燈,明日娘生日好擺酒。晚夕小的每與娘磕頭,娘已定賞俺每哩。」婦人道:「要打便有,要賞可沒有。」琴童道:「耶嚛,娘怎的沒打不說話,行動只把打放在頭裡,小的每是娘的兒女,娘看顧看顧兒便好,如何只說打起來。」婦人道:「賊囚,別要說嘴,你好生仔細掛那燈,沒的例兒撦兒的,拏不牢弔將下來。前日年裡,為崔本來,說你爹大白裡不見了,險了險赦了一頓打,沒曾打,這遭兒可打的成了。」琴童道:「娘只說破話,小的命兒薄薄的,又唬小的。」{\meipi{琴童嘴兒盡滑。}}玳安道:「娘也會打聽,這個話兒娘怎得知?」婦人道:「宮外有株松,宮內有口鐘。鐘的聲兒,樹的影兒,我怎麼有個不知道的?昨日可是你爹對你大娘說,去年有賁四在家,還紮了幾架烟火放,今年他不在家,就沒人會紮。吃我說了兩句:『他不在家,左右有他老婆會紮,教他紮不是!』」{\meipi{賁四老婆還不如五娘會咂。}}玳安道:「娘說的甚麼話,一個夥計家,那裡有此事!」婦人道:「甚麼話?檀木靶,有此事,真個的。畫一道兒,只怕㒲過界兒去了。」琴童道:「娘也休聽人說,只怕賁四來家知道。」婦人道:「可不瞞那王八哩。我只說那王八也是明王八,恠不的他徃東京去的放心,丟下老婆在家,料莫他也不肯把𣭈閑着。賊囚根子們,別要說嘴,打夥兒替你爹做牽頭,引上了道兒,你每好圖躧狗尾兒。說的是也不是?敢說我知道?嗔道賊淫婦買禮來,與我也罷了,又送蒸酥與他大娘,另外又送一大盒瓜子兒與我,要買住我的嘴頭子,他是會養漢兒。我就猜沒別人,就知道是玳安這賊囚根子,替他鋪謀定計。」玳安道:「娘屈殺小的。小的平白管他這勾當怎的?小的等閑也不徃他屋裡去。娘也少聽韓囘子老婆說話,他兩個為孩子好不嚷亂。常言『要好不能勾,要歹登時就』;『房倒壓不殺人,舌頭倒壓殺人』;『聽者有,不聽者無』。論起來,賁四娘子為人和氣,在咱門首住着,家中大小沒曾惡識了一個人。誰不在他屋裡討茶吃,莫不都養着?倒沒處放。」金蓮道:「我見那水眼淫婦,矮着個靶子,像個半頭磚兒也是的,把那水濟濟眼擠着,七八拏杓兒舀。好個恠淫婦!他和那韓道國老婆,那長大摔瓜的淫婦,我不知怎的,掐了眼兒不待見他。」{\pangpi{此是妒心所使。}}正說着,只見小玉走來說:「俺娘請五娘,潘姥姥來了,要轎子錢哩。」金蓮道:「我在這裡站着,他從多咱進去了?」琴童道:「姥姥打夾道里進去的。一來的轎子,該他六分銀子。」金蓮道:「我那得銀子?來人家來,怎不帶轎子錢兒走!」一面走到後邊,見了他娘,只顧不與他轎子錢,只說沒有。{\meipi{金蓮小氣,不獨在色上着脚,即財上亦十分鄭重,可見四者之慾,一齊都到。}}月娘道:「你與姥姥一錢銀子,寫帳就是了。」金蓮道:「我是不惹他,他的銀子都有數兒,只教我買東西,沒教我打發轎子錢。」坐了一囘,大眼看小眼,外邊挨轎的催着要去。

玉樓見不是事,向袖中拏出一錢銀子來,打發擡轎的去了。不一時,大妗子、二妗子、大師父來了,月娘擺茶吃了。潘姥姥歸到前邊他女兒房內來,被金蓮盡力數落了一頓,說道:「你沒轎子錢,誰教你來?恁出醜㓦劃的,教人家小看!」潘姥姥道:「姐姐,你沒與我個錢兒,老身那討個錢兒來?好容易籌辦了這分禮兒來。」婦人道:「指望問我要錢,我那裡討個錢兒與你?你看七個窟窿到有八個眼兒等着在這裡。今後你看有轎子錢便來他家來,沒轎子錢別要來。料他家也沒少你這個窮親戚!休要做打嘴的獻世包!『關王賣豆腐——人硬貨不硬』。我又聽不上人家那等𣭈聲顙氣。前日為你去了,和人家大嚷大鬧的,{\pangpi{冤得奇。}}你知道也怎的?驢糞毬兒面前光,卻不知裡面受悽惶。」幾句說的潘姥姥嗚嗚咽咽哭起來了。春梅道:「娘今日怎的,只顧說起姥姥來了。」一面安撫老人家,在裡邊炕上坐的,連忙點了盞茶與他吃。潘姥姥氣的在炕上睡了一覺,只見後邊請吃飯,纔起來徃後邊去了。

西門慶從衙門中來家,正在上房擺飯,忽有玳安拏進貼兒來說:「荊老爹陞了東南統制,來拜爹。」西門慶見貼兒上寫:「新東南統制兼督漕運總兵官荊忠頓首拜。」慌的西門慶連忙穿衣,冠帶迎接出來。只見都總制穿着大紅麒麟補服、渾金帶進來,後面跟着許多僚掾軍牢。一面讓至大廳上叙禮畢,分賓主而坐,茶湯上來。荊統制說道:「前日陞官勑書纔到,還未上任,徑來拜謝老翁。」西門慶道:「老總兵榮擢恭喜,大才必有大用,自然之道。吾輩亦有光矣,容當拜賀。」一面請寬尊服,少坐一飯。即令左右放卓兒,荊統制再三致謝道:「學生奉告老翁,一家尚未拜,還有許多薄冗,容日再來請教罷。」便要起身,西門慶那裡肯放,隨令左右上來,寬去衣服,登時打抹春臺,收拾酒菓上來。獸炭頓燒,煖簾低放。金壺斟玉液,翠盞貯羊羔,纔斟上酒來,只見鄭春、王相兩個小優兒來到,扒在面前磕頭。西門慶道:「你兩個如何這咱纔來?」問鄭春:「那一個叫甚名字?」鄭春道:「他喚王相,是王桂的兄弟。」

西門慶即令拏樂器上來彈唱。須臾,兩個小優歌唱了一套「霽景融和」。左右拏上兩盤攢盒點心嗄飯,兩瓶酒,打發馬上人等。荊統制道:「這等就不是了。學生叨擾,下人又蒙賜饌,何以克當?」即令上來磕頭。西門慶道:「一二日房下還要潔誠請尊正老夫人賞燈一叙,望乞下降。在座者惟老夫人、張親家夫人、同僚何天泉夫人,還有兩位舍親,再無他人。」荊統制道:「若老夫人尊票制,賤荊已定趨赴。」又問起:「周老總兵怎的不見陞轉?」荊統制道:「我聞得周菊軒也只在三月間有京榮之轉。」西門慶道:「這也罷了。」坐不多時,荊統制告辭起身,西門慶送出大門,看着上馬喝道而去。晚夕,潘金蓮上壽,後廳小優彈唱,遞了酒,西門慶便起身徃金蓮房中去了。月娘陪着大妗子、潘姥姥、女兒郁大姐、兩個姑子在上房坐的飲酒。潘金蓮便陪西門慶在他房內,從新又安排上酒來,與西門慶梯己遞酒磕頭。{\meipi{專在此處用功夫。}}落後潘姥姥來了,金蓮打發他李瓶兒這邊歇臥。他陪着西門慶自在飲酒,頑耍做一處。

卻說潘姥姥到那邊屋裡,如意、迎春讓他熱炕上坐着。先是姥姥看明間內靈前,供擺着許多獅仙五老定勝桌席,旁邊掛着他影,因向前道了個問訊,說道:「姐姐好處生天去了。」進來坐在炕上,向如意兒、迎春道:「你娘勾了。官人這等費心追薦,受這般大供養,勾了。他是有福的。」{\meipi{語有含蓄。}}如意兒道:「前日娘的百日,請姥姥,怎的不來?門外花大妗子和大妗子都在這裡來,十二個道士念經,好不大吹大打,揚幡道場,水火練度,晚上纔去了。」潘姥姥道:「幫年逼節,丟着個孩子在家,我來家中沒人,所以就不曾來。今日你楊姑娘怎的不見?」如意兒道:「姥姥還不知道,楊姑娘老病死了,從年裡俺娘念經就沒來,俺娘們都徃北邊與他上祭去來。」潘姥姥道:「可傷,他大如我,我還不曉的他老人家沒了。嗔道今日怎的不見他。」說了一囘,如意兒道:「姥姥,有鍾甜酒兒,你老人家用些兒。」一面叫:「迎春姐,你放小卓兒在炕上,篩甜酒與姥姥吃盃。」不一時取到。飲酒之間,婆子又題起李瓶兒來:「你娘好人,有仁義的姐姐,熱心腸兒。{\meipi{瓶兒身後論定,所謂自有旁人說短長也。}}我但來這裡,沒曾把我老娘當外人看承,一到就是熱茶熱水與我吃,還只恨我不吃。晚間和我坐着說話兒,我臨家去,好歹包些甚麼兒與我拏了去,再不曾空了我。不瞞你姐姐每說,我身上穿的這披襖兒,還是你娘與我的。{\meipi{及其老也,戒之在得。}}正經我那冤家,半分折針兒也迸不出來與我。我老身不打誑語,阿彌陀佛,{\pangpi{妙。}}水米不打牙。他若肯與我一個錢兒,我滴了眼睛在地。你娘與了我些甚麼兒,他還說我小眼薄皮,愛人家的東西。想今日為轎子錢,你大包家拏着銀子,就替老身出幾分便怎的?咬定牙兒只說沒有,到教後邊西房裡姐姐,拏出一錢銀子來,打發擡轎的去了。歸到屋裡,還數落了我一頓,到明日有轎子錢,便教我來,沒轎子錢,休叫我上門走。我這去了不來了。{\pangpi{讖。}}來到這裡沒的受他的氣。隨他去,有天下人心狠,不似俺這短壽命。姐姐你每聽着我說,老身若死了,他到明日不聽人說,還不知怎麼收成結果哩!{\meipi{老者之言,每多奇中,以其見多識明之故。}}想着你從七歲沒了老子,我怎的守你到如今,從小兒交你做針指,徃餘秀才家上女學去,替你怎麼纏手纏脚兒的,你天生就是這等聰明伶俐,到得這步田地?他把娘喝過來斷過去,不看一眼兒。」如意兒道:「原來五娘從小兒上學來,嗔道恁題起來就會識字深。」潘姥姥道:「他七歲兒上女學,上了三年,字仿也曾寫過,甚麼詩詞歌賦唱本上字不認的!」

正說着,只見打的角門子響,如意兒道:「是誰叫門?」使綉春:「你瞧瞧去。」那綉春走來說:「是春梅姐姐來了。」如意兒連忙捏了潘姥姥一把手,{\pangpi{妙。}}就說道:「姥姥悄悄的,春梅來了。」潘姥姥道:「老身知道他與我那冤家一條腿兒。」只見春梅進來,見衆人陪着潘姥姥吃酒,說道:「我來瞧瞧姥姥來了。」如意兒讓他坐,這春梅把裙子摟起,一屁股坐在炕上。迎春便挨着他坐,如意坐在右邊炕頭上,潘姥姥坐在當中。因問:「你爹和你娘睡了不曾?」春梅道:「剛纔打發他兩個睡下了。我來這邊瞧瞧姥姥,有幾樣菜兒,一壺兒酒,取過來和姥姥坐的。」因央及綉春:「你那邊教秋菊掇了來,我已是攢下了。」綉春去了,不一時,秋菊用盒兒掇着菜兒,綉春提了一錫壺金華酒來。春梅分付秋菊:「你徃房裡看去,若叫我,來這裡對我說。」秋菊去了。一面擺酒在炕卓上,都是燒鴨、火腿、海味之類,堆滿春臺。綉春關上角門,走進在旁邊陪坐,於是篩上酒來。春梅先遞了一鍾與潘姥姥,然後遞如意兒與迎春、綉春。又將護衣碟兒內,每樣揀出,遞與姥姥衆人吃,說道:「姥姥,這個都是整菜,你用些兒。」那婆子道:「我的姐姐,我老身吃。」因說道:「就是你娘,從來也沒費恁個心兒,管待我管待兒。姐姐,你倒有惜孤愛老的心,你到明日管情一步好一步。敢是俺那冤家,沒人心沒人義,幾遍為他心齷齪,我也勸他,就扛的我失了色。今日早是姐姐你看着,我來你家討冷飯來了,你下老實那等扛我!」春梅道:「姥姥,罷,你老人家只知其一,不知其二。

俺娘是爭強不伏弱的性兒。比不的六娘,銀錢自有,他本等手裡沒錢,你只說他不與你。別人不知道,我知道。{\meipi{千古相知。}}想俺爹雖是有的銀子放在屋裡,俺娘正眼兒也不看他的。若遇着買花兒東西,明公正義問他要。不恁瞞瞞藏藏的,教人看小了他,怎麼張着嘴兒說人!他本沒錢,姥姥恠他,就虧了他了。莫不我護他?也要個公道。」{\meipi{好隱諷。}}如意兒道:「錯恠了五娘。自古親兒骨肉,五娘有錢,不孝順姥姥,再與誰?常言道,『要打看娘面』,『千朵桃花一樹兒生』。{\pangpi{扯得妙。}}到明日你老人家黃金入櫃,五娘他也沒個貼皮貼肉的親戚,就如死了俺娘樣兒。」婆子道:「我有今年沒明年,知道今日死明日死?我也不恠他。」春梅見婆子吃了兩鍾酒,韶刀上來,便叫迎春:「二姐,你拏骰盆兒來,咱每擲個骰兒,搶紅耍子兒罷。」不一時,取了四十個骰兒的骰盆來。春梅先與如意兒擲,擲了一囘,又與迎春擲,都是賭大鐘子。你一盞,我一鍾。須臾,竹葉穿心,桃花上臉,把一錫瓶酒吃的罄淨。{\meipi{一般大量,豈安得歟?}}迎春又拏上半罈麻姑酒來,也都吃了。約莫到二更時分,那潘姥姥老人家熬不的,又早前靠後仰,打起盹來,方纔散了。

春梅便歸這邊來,推了推角門,開着,進入院內。只見秋菊正在明間板壁縫兒內,倚着春櫈兒,聽他兩個在屋裡行房,怎的作聲喚,口中呼叫甚麼。正聽在熱鬧,不防春梅走到根前,向他腮頰上盡力打了個耳刮子,罵道:「賊少死的囚奴,你平白在這裡聽甚麼?」打的秋菊睜睜的,說道:「我這裡打盹,誰聽甚麼來,你就打我?」不想房裡婦人聽見,便問春梅,他和誰說話。春梅道:「沒有人,我使他關門,他不動。」於是替他摭過了。秋菊揉着眼,關上房門。春梅走到炕上,摘頭睡了。正是:

\begin{myquote}
鶬鶊有意留殘景,杜宇無情戀晚暉。
\end{myquote}

一宿晚景題過。次日,潘金蓮生日,有傅夥計、甘夥計、賁四娘子、崔本媳婦、段大姐、吳舜臣媳婦、鄭三姐、吳二妗子,都在這裡。西門慶約會吳大舅、應伯爵,整衣冠,尊瞻視,騎馬喝道,徃何千戶家赴席。那日也有許多官客,四個唱的,一起雜耍,周守備同席飲酒。至晚囘家,就在前邊和如意兒歇了。

到初十日,發貼兒請衆官娘子吃酒,月娘便問西門慶說:「趁着十二日看燈酒,把門外的孟大姨和俺大姐,也帶着請來坐坐,省的教他知道惱,請人不請他。」西門慶道:「早是你說。」分付陳敬濟:「再寫兩個貼,差琴童兒請去。」這潘金蓮在旁,聽着多心,走到屋裡,一面攛掇潘姥姥就要起身。月娘道:「姥姥你慌去怎的?再消住一日兒是的。」金蓮道:「姐姐,大正月裡,他家裡丟着孩子,沒人看,教他去罷。」慌的月娘裝了兩個盒子點心茶食,又與了他一錢轎子錢,管待打發去了。金蓮因對着李嬌兒說:「他明日請他有錢的大姨兒來看燈吃酒,一個老行貨子,觀眉觀眼的,不打發去了,平白教他在屋裡做甚麼?待要說是客人,沒好衣服穿。待要說是燒火的媽媽子,又不像。倒沒的教我惹氣。」因西門慶使玳安兒送了兩個請書兒,徃招宣府,一個請林太太,一個請王三官兒娘子黃氏。又使他院中早叫李桂兒、吳銀兒、鄭愛月兒、洪四兒四個唱的,李銘、吳惠、鄭奉三個小優兒。不想那日賁四從東京來家,梳洗頭臉,打選衣帽齊整,來見西門慶磕頭。遞上夏指揮囘書。西門慶問道:「你如何這些時不來?」賁四具言在京感冒打寒一節,「直到正月初二日,纔收拾起身囘來,夏老爹多上覆老爹,多承看顧。」西門慶照舊還把鑰匙教與他管絨線鋪。另開啟一間,教吳二舅開鋪子賣紬絹,到明日松江貨舡到,都卸在獅子街房內,同來保發賣。且叫賁四叫花兒匠在家攢造兩架烟火,十二日要放與堂客看。

只見應伯爵領了李三見西門慶,先道外面承攜之事。坐下吃畢茶,方纔說起:「李三哥今有一宗買賣與你說,你做不做?」西門慶道:「甚麼買賣?」李三道:「你東京行下文書,天下十三省,每省要幾萬兩銀子的古器。咱這東平府,坐派着二萬兩,批文在巡按處,還未下來。如今大街上張二官府,破二百兩銀子幹這宗批要做,都看有一萬兩銀子尋。小人會了二叔,敬來對老爹說。老爹若做,張二官府拏出五千兩來,老爹拏出五千兩來,兩家合着做這宗買賣。左右沒人,這邊是二叔和小人與黃四哥,他那邊還有兩個夥計,二分八利錢。未知老爹意下何如?」西門慶問道:「是甚麼古器?」李三道:「老爹還不知,如今朝廷皇城內新蓋的艮嶽,改為壽嶽,上面起蓋許多亭臺殿閣,又建上清寶籙宮、會真堂、璿神殿,又是安妃娘娘梳粧閣,都用着這珍禽奇獸,周彝商鼎,漢篆秦爐,宣王石鼓,歷代銅鞮,仙人掌承露盤,並希世古董玩器擺設,好不大興工程,好少錢糧!」{\meipi{土木珍玩之費如此,安得不民窮盜起。}}西門慶聽了,說道:「比是我與人家打夥而做,不如我自家做了罷,敢量我拏不出這一二萬銀子來?」{\pangpi{大口氣。}}李三道:「得老爹全做又好了,俺每就瞞着他那邊了。左右這邊二叔和俺每兩個,再沒人。」伯爵道:「哥,家裡還添個人兒不添?」西門慶道:「到根前再添上賁四,替你們走跳就是了。」西門慶又問道:「批文在那裡?」李三道:「還在巡按上邊,沒發下來哩。」西門慶道:「不打緊,我差人寫封書,封些禮,問宋松原討將來就是了。」李三道:「老爹若討去,不可遲滯,自古兵貴神速,先下米的先吃飯,誠恐遲了,行到府裡。吃別人家幹的去了。」西門慶笑道:「不怕他,就行到府裡,我也還教宋松原拏囘去。就是胡府尹,我也認的。」於是留李三、伯爵同吃了飯,約會:「我如今就寫書,明日差小价去。」李三道:「又一件,宋老爹如今按院不在這裡了,從前日起身徃兗州府盤查去了。」西門慶道:「你明日就同小价徃兗州府走遭。」李三道:「不打緊,等我去,來囘破五六日罷了。老爹差那位管家,等我會下,有了書,教他徃我那裡歇,明日我同他好早起身。」西門慶道:「別人你宋老爹不信的,他常喜的是春鴻,叫春鴻、來爵兩個去罷。」於是叫他二人到面前,會了李三,晚夕徃他家宿歇。伯爵道:「這等纔好,事要早幹,高材疾足者先得之。」於是與李三吃畢飯,告辭而去。西門慶隨即教陳敬濟寫了書,又封了十兩葉子黃金在書帕內,與春鴻、來爵二人。分付:「路上仔細,若討了批文,即便早來。若是行到府裡,問你宋老爹討張票,問府裡要。」來爵道:「爹不消分付,小的曾在充州答應過徐叅議,小的知道。」於是領了書禮,打在身邊,徑徃李三家去了。

不說十一日來爵、春鴻同李三早顧了長行頭口,徃兗州府去了。卻說十二日,西門慶家中請各堂客飲酒。那日在家不出門,約下吳大舅、謝希大、常峙節四位,晚夕來在捲棚內賞燈飲酒。王皇親家小厮,從早辰就挑了箱子來了,等堂客到,打銅鑼鼓迎接。周守備娘子有眼疾不得來,差人來囘。止是荊統制娘子、張團練娘子、雲指揮娘子,並喬親家母、崔親家母、吳大姨、孟大姨,都先到了。只有何千戶娘子、王三官母親林太太並王三官娘子不見到。{\meipi{林氏食言。}}西門慶使排軍、玳安、琴童兒來囘催邀了兩三遍,又使文嫂兒催邀。午間,只見林氏一頂大轎,一頂小轎跟了來。見了禮,請西門慶拜見,問:「怎的三官娘子不來?」林氏道:「小兒不在,家中沒人。」拜畢下來。止有何千戶娘子,直到晌午半日纔來,坐着四人大轎,一個家人媳婦坐小轎跟隨,排軍擡着衣箱,又是兩個青衣人緊扶着轎扛,到二門裡纔下轎。前邊鼓樂吹打迎接,吳月娘衆姊妹迎至儀門首。西門慶悄悄在西廂房,放下簾來偷瞧,見這藍氏年約不上二十歲,生的長挑身材,打扮的如粉粧玉琢,頭上珠翠堆滿,鳳翹雙插,身穿大紅通袖五彩粧花四獸麒麟袍兒,繫着金鑲碧玉帶,下襯着花錦藍裙,兩邊禁步叮咚,麝蘭撲鼻。但見:

\begin{myquote}
儀容嬌媚,體態輕盈。姿性兒百伶百俐,身段兒不短不長。細彎彎兩道蛾眉,直侵入鬢;滴流流一雙鳳眼,來徃踅人。嬌聲兒似囀日流鶯,嫩腰兒似弄風楊柳。端的是綺羅隊裡生來,卻厭豪華氣象,珠翠叢中長大,那堪雅淡梳粧。開遍海棠花,也不問夜來多少;標殘楊柳絮,竟不知春意如何。輕移蓮步,有蕊珠仙子之風流;款蹙湘裙,似水月觀音之態度。{\meipi{畫出嫣媚情態如見。}}
\end{myquote}

正是:

\begin{myquote}
比花花解語,比玉玉生香。
\end{myquote}

這西門慶不見則已,一見魂飛天外,魄䘮九霄,未曾體交,精魄先失。少頃,月娘等迎接進入後堂,相見叙禮已畢,請西門慶拜見。西門慶得了這一聲,連忙整衣冠行禮,恍若瓊林玉樹臨凡,神女巫山降下,躬身施禮,心搖目蕩,不能禁止。{\meipi{聞此一請,如聽將軍令矣,惜乎西門非秀才耳。}}拜見畢下來,月娘先請在捲棚內擺過茶,然後大廳吹打,安席上坐,各依次序,當下林太太上席。戲文扮的是《小天香半夜朝元記》。唱的兩折下來,李桂姐、吳銀兒、鄭月兒、洪四兒四個唱的上去,彈唱燈詞。

西門慶在捲棚內,自有吳大舅、應伯爵、謝希大、常峙節、李銘、吳惠、鄭奉三個小優兒彈唱、飲酒,不住下來大廳格子外徃裡觀覷。看官聽說,明月不常圓,彩雲容易散,樂極悲生,否極泰來,自然之理。西門慶但知爭名奪利,縱意奢淫,殊不知天道惡盈,鬼錄來追,死限臨頭。{\meipi{熱鬧時忽下莊語,如火炕中一盆氷雪水。}}到晚夕堂中點起燈來,小優兒彈唱。還未到起更時分,西門慶陪人坐的,就在席上齁齁的打起睡來。伯爵便行令猜枚鬼混他,說道:「哥,你今日沒高興,怎的只打睡?」西門慶道:「我昨日沒曾睡,不知怎的,今日只是沒精神,要打睡。」只見四個唱的下來,伯爵教洪四兒與鄭月兒兩個彈唱,吳銀兒與李桂姐遞酒。

正耍在熱鬧處,忽玳安來報:「王太太與何老爹娘子起身了。」{\meipi{掃興。}}西門慶就下席來,黑影裡走到二門裡首,偷看他上轎。月娘衆人送出來,前邊天井內看放烟火。藍氏已換了大紅遍地金貂鼠皮襖,林太太是白綾襖兒,貂鼠披風,帶着金釧玉佩。家人打燈籠,簇擁上轎而去。這西門慶正是餓眼將穿,饞涎空咽,恨不能就要成雙。見藍氏去了,悄悄從夾道進來。當時沒巧不成語,姻緣會湊,可霎作恠,來爵兒媳婦見堂客散了,正從後邊歸來,開房門,不想頂頭撞見西門慶,沒處藏躲。原來西門慶見媳婦子生的喬樣,安心已久,雖然不及來旺妻宋氏風流,也頗充得過第二。於是乘着酒興兒,雙關抱進他房中親嘴。這老婆當初在王皇親家,因是養主子,被家人不忿攘鬧,打發出來,今日又撞着這個道路,如何不從了?{\meipi{積祖是孝順媳婦兒。}}一面就遞舌頭在西門慶口中。兩個解衣褪褲,就按在炕沿子上,掇起腿來,被西門慶就聳了個不亦樂乎。{\pangpi{何等敏捷。}}正是:未曾得遇鶯娘面,且把紅娘去解饞。有詩為證:

\begin{myquote}
燈月交光浸玉壺,分得清光照綠珠。\\莫道使君終有婦,教人桑下覓羅敷。
\end{myquote}
