\includepdf[pages={107,108},fitpaper=false]{tst.pdf}
\chapter*{第五十四囘 應伯爵隔花戲金釧 任醫官垂帳診瓶兒}
\addcontentsline{toc}{chapter}{第五十四囘 應伯爵隔花戲金釧 任醫官垂帳診瓶兒}
\markboth{{\titlename}卷之六}{第五十四囘 應伯爵隔花戲金釧 任醫官垂帳診瓶兒}


詞曰:

\begin{myquote}
美酒斗十千,更對花前。芳樽肯放手中間?起舞酬花花不語,似解人憐。不醉莫言還,請看枝間。已飄零一片減嬋娟。花落明年猶自好,可惜朱顏。

\raggedleft{——右調《浪淘沙》\rightquadmargin}
\end{myquote}

卻說王姑子和李瓶兒、吳月娘,商量來日起經頭停當,月娘便拏了些應用物件送王姑子去,又教陳敬濟來分付道:「明日你李家丈母拜經保佑官哥,你早去礼拜礼拜。」敬濟推道:「爹明日要去門外花園吃酒,留我店裡照管,着別人去罷。」原來敬濟聽見應伯爵請下了西門慶,便想要乘機和潘金蓮弄鬆,因此推故。月娘見說照顧生意,便不違拗他,放他出去了,便着書童礼拜。調撥已定,單待明日起經。

且說西門慶和應伯爵、常峙節談笑多時,只見琴童來囘話道:「唱的叫了。吳銀兒有病去不的,韓金釧兒答應了,明日早去。」西門慶道:「吳銀兒既病,再去叫董嬌兒罷。」常峙節道:「郊外飲酒,有一個儘勾了,不消又去叫。」說畢,各各別去,不在話下。

次日黎明,西門慶起身梳洗畢,月娘安排早飯吃了,便乘轎徃觀音庵起經。書童、玳安跟隨而行。王姑子出大門迎接,西門慶進庵來,北面皈依叅拜。但見:

\begin{myquote}
金仙建化,啟第一之真乘;玉偈演音,集三千之妙利。寶花座上,裝成莊嚴世界;惠日光中,現出歡喜慈悲。香烟繚遶,直透九霄;仙鶴盤旋,飛來秪樹。訪問緣繇,果然稀罕;但思福果,那惜金錢!
\end{myquote}

正是:

\begin{myquote}
辦個至誠心,何處皇天難感。\\願將大佛事,保祈殤子彭籛。
\end{myquote}

王姑子宣讀疏頭,西門慶聽了,平身更衣。王姑子捧出茶來,又拏些點心餅饊之物擺在桌上。西門慶不吃,單呷了口清茶,便上轎囘來,留書童礼拜。正是:

\begin{myquote}
願心酬畢喜匆匆,感謝靈神保佑功。\\更願皈依蓮座下,卻教關煞永亨通。
\end{myquote}

囘來,紅日纔半竿,應伯爵早同常峙節來請。西門慶笑道:「那裡有請吃早飯的?我今日雖無事故,也索下午纔好去。」應伯爵道:「原來哥不知,出城二十里,有個內相花園,極是華麗,且又幽深,兩三日也遊玩不到哩。因此要早去,盡這一日工夫,可不是好。」常峙節道:「今日哥既沒甚事故,應哥早邀,便索去休。」西門慶道:「既如此;常二哥和應二哥先行,我乘轎便到了。」應伯爵道:「專待哥來。」說罷,兩人出門,叫頭口前去,又轉到院內,立等了韓金釧兒坐轎子同去。應伯爵先一日已着火家來園內,殺雞宰鵝,安排筵席,又叫下兩個優童隨着去了。西門慶見三人去了多時,便乘轎出門,迤邐漸近。舉頭一看,但見:

\begin{myquote}
千樹濃陰,一灣流水。粉墻藏不謝之花,華屋掩長春之景。武陵桃放,漁人何處識迷津?庾嶺梅開,詞客此中尋好句。端的是天上蓬萊,人間閬苑。
\end{myquote}

西門慶赞歎不已道:「好景致!」下轎步人園來。應伯爵和常峙節出來迎接,園亭內坐的。先是韓金釧兒磕了頭,纔是兩個歌童磕頭。吃了茶,伯爵就要遞上酒來,西門慶道:「且住,你每先陪我去瞧瞧景致來。」一面立起身來,攙着韓金釧手兒同走。伯爵便引着,慢慢的步出迴廊,循朱闌轉過垂楊邊一曲荼䕷架,踅過太湖石、松鳳亭,來到奇字亭。亭後是遶屋梅花三十樹,中間探梅閣。閣上名人題詠極多,西門慶備細看了。又過牡丹臺,臺上數十種奇異牡丹。又過北是竹園,園左有聽竹館、鳳來亭,匾額都是名公手跡;右是金魚池,池上樂水亭,憑朱欄俯看金魚,卻象錦被也似一片浮在水面。西門慶正看得有趣,伯爵催促,{\pangpi{好頓挫。}}又登一個大樓,上寫「聽月樓」。樓上也有名人題詩對聯,也是刊板砂綠嵌的。下了樓,徃東一座大山,山中八仙洞,深幽廣闊。洞中有石棋盤,壁上鐵笛銅簫,似仙家一般。出了洞,登山頂一望,滿園都是見的。{\meipi{鋪敍園林丘壑,頗有別致,不似內臣家一味排偶富麗。}}西門慶走了半日,常峙節道:「恐怕哥勞倦了,且到園亭上坐坐,再走不遲。」西門慶道:「十分走不過一分,卻又走不得了。多虧了那些擡轎的,一日趕百來里多路。」{\pangpi{是俗人之奇想。}}大家笑了,讓到園亭裡,西門慶坐了上位,常峙節坐東,應伯爵坐西,韓金釧兒在西門慶側邊陪坐。大家送過酒來,西門慶道:「今日多有相擾,怎的生受!」伯爵道:「一盃水酒,哥說那裡話!」三人吃勾數盃,兩個歌童上來。西門慶看那歌童生得:粉塊捏成白麵,胭脂點就朱唇。綠糝糝,披幾寸青絲,{\pangpi{可愛。}}香馥馥,着滿身羅綺。秋波一轉,憑他鐵石心腸。檀板輕敲,遮莫金聲玉振。正是但得傾城與傾國,不論南方與北方。兩個歌童上來,拏着鼓板,合唱了一套時曲《字字錦》「群芳綻錦鮮」。唱的嬌喉婉轉,端的是遶梁之聲,西門慶稱赞不已。常峙節道:「恠他是男子,若是婦女,便無價了。」西門慶道:「若是婦女,咱也早叫他坐了,決不要他站着唱。」{\meipi{好男色者,見此語定不能平。}}伯爵道:「哥本是在行人,說的話也在行。」衆人都笑起來。三人又吃了數盃,伯爵送上令盆,斟一大鐘酒,要西門慶行令。西門慶道:「這便不消了。」伯爵定要行令,西門慶道:「我要一個風花雪月,第一是我,第二是常二哥,第三是主人,第四是釧姐。但說的出來,只吃這一盃。若說不出,罰一盃,還要講十個笑話。講得好便休;不好,從頭再講。如今先是我了。」拏起令鍾,一飲而盡,就道:「雲淡風輕近午天。{\pangpi{虧他說。}}如今該常二哥了。」常峙節接過酒來吃了,便道:「傍花隨桺過前川。{\pangpi{更虧他。}}如今該主人家了。」應伯爵吃了酒,呆登登講不出來。西門慶道:「應二哥請受罰。」伯爵道:「且待我思量。」又遲了一囘,被西門慶催逼得緊,便道:「泄漏春光有幾分。」西門慶大笑道:「好個說別字的,論起來,講不出該一盃,說別字又該一盃,共兩盃。」伯爵笑道:「我不信,有兩個『雪』字,便受罰了兩盃?」衆人都笑了,催他講笑話。伯爵說道:「一秀才上京,泊船在揚子江。到晚,叫艄公:『泊別處罷,這裡有賊。』艄公道:『怎的便見得有賊?』秀才道:『兀那碑上寫的不是江心賊?』艄公笑道:『莫不是江心賦,怎便識差了?』秀才道:『賦便賦,有些賊形。』」西門慶笑道:「難道秀才也識別字?」{\pangpi{太奉承秀才。}}常峙節道:「應二哥該罰十大盃。」伯爵失驚道:「卻怎的便罰十盃?」常峙節道:「你且自家去想。」原來西門慶是山東第一個財主,卻被伯爵說了「賊形」,可不罵他了!西門慶先沒理會,到被常峙節這句話提醒了。伯爵覺失言,取酒罰了兩盃,便求方便。西門慶笑道:「你若不該,一盃也不強你;若該罰時,卻饒你不的。」伯爵滿面不安。又吃了數盃,瞅着常峙節道:「多嘴!」{\pangpi{畫。}}{\meipi{諺云:言多語失。任伯爵乖人巧嘴,亦要說差,況不如伯爵者乎?此作者微意,若執伯爵必不如此,便失之矣。}}西門慶道:「再說來!」伯爵道:「如今不敢說了。」西門慶道:「胡亂取笑,顧不的許多,且說來看。」

伯爵纔安心,又說:「孔夫子西狩得麟,不能勾見,在家裡日夜啼哭。弟子恐怕哭壞了,尋個牯牛,滿身掛了銅錢哄他。那孔子一見便識破,道:『這分明是有錢的牛,卻怎的做得麟!』」說罷,慌忙掩着口跪下道:「小人該死了,寔是無心。」西門慶笑着道:「恠狗才,還不起來。」金釧兒在旁笑道:「應花子成年說嘴麻犯人,今日一般也說錯了。{\pangpi{道出。}}大爹,別要理他。」說的伯爵急了,走起來把金釧兒頭上打了一下,說道:「緊自常二那天殺的韶叨,還禁的你這小淫婦兒來插嘴插舌!」不想這一下打重了,把金釧疼的要不的,又不敢哭,肐䐐着臉,待要使性兒。西門慶笑罵道:「你這狗才,可成個人?嘲戲了我,反又打人,該得何罪?」伯爵一面笑着,摟了金釧說道:「我的兒,誰養的你恁嬌?輕輕蕩得一蕩兒就待哭,虧你挨那驢大的行貨子來!」金釧兒揉着頭,瞅了他一眼,罵道:「恠花子,你見來?沒的扯淡!敢是你家媽媽子倒挨驢的行貨來。」伯爵笑說道:「我怎不見?只大爹他是有名的潘、驢、鄧、小、閑,不少一件,你怎的賴得過?」又道:「哥,我還有個笑話兒,一發奉承了列位罷:一個小娘,因那話寬了,有人教道他:『你把生礬一塊,塞在裡邊,敢就緊了。』那小娘真箇依了他。不想那礬澁得疼了,不好過,肐䐐着立在門前。一個走過的人看見了,說道:『這小淫婦兒,倒象粧霸王哩!』這小娘正沒好氣,聽見了,便罵道:『恠囚根子,俺樊噲粧不過,誰這裡粧霸王哩!』」說畢,一座大笑,連金釧兒也撲扢的笑了。

少頃,伯爵飲過酒,便送酒與西門慶完令。西門慶道:「該釧姐了。」金釧兒不肯。常峙節道:「自然還是哥。」西門慶取酒飲了,道:「月殿雲梯拜洞仙。」令完,西門慶便起身更衣散步。伯爵一面叫擺上添換來,轉眼卻不見了韓金釧兒。伯爵四下看時,只見他走到山子那邊薔薇架兒底下,正打沙窩兒溺尿。伯爵看見了,連忙折了一枝花枝兒,輕輕走去,蹲在他後面,伸手去挑弄他的花心。韓金釧兒吃了一驚,尿也不曾溺完就立起身來,連褲腰都濕了。{\meipi{戲謔處俱千古韻事。}}不防常峙節從背後又影來,猛力把伯爵一推,撲的向前倒了一交,險些兒不曾濺了一臉子的尿。{\meipi{又深一層,尤妙。}}伯爵爬起來,笑罵着趕了打,西門慶立在那邊松陰下看了,笑的要不的。連韓金釧兒也笑的打跌道:「應花子,可見天理近哩!」於是重新入席飲酒。西門慶道:「你這狗才,剛纔把俺們都嘲了,如今也要你說個自己的本色。」伯爵連說:「有有有,一財主撒屁,幫閑道:『不臭。』財主慌的道:『屁不臭,不好了,快請醫人!』幫閑道:『待我聞聞滋味看。』假意兒把鼻一嗅,口一咂,{\pangpi{妙在此。}}道:『囘味略有些臭,還不妨。』」說的衆人都笑了。常峙節道:「你自得罪哥哥,怎的把我的本色也說出來?」衆人又笑了一場。伯爵又要常峙節與西門慶猜枚飲酒。韓金釧兒又彈唱着奉酒。衆人歡笑,不在話下。

且說陳敬濟探聽西門慶出門,便百般打扮的俊俏,一心要和潘金蓮弄鬼,又不敢造次,只在雪洞裡張看,還想婦人到後園來。等了半日不見來,耐心不過,就一直逕奔到金蓮房裡來,喜得沒有人看見。走到房門首,忽聽得金蓮嬌聲低唱了一句道:「莫不你纔得些兒,便將人忘記。」{\meipi{借唱一句作針線引入,何等幽細。}}已知婦人動情,便介面道:「我那敢忘記了你!」搶進來,緊緊抱住道:「親親,昨日丈母叫我去觀音庵礼拜,我一心放你不下,推事故不去。今日爹去吃酒了,我絕早就在雪洞裡張望。望得眼穿,並不見我親親的俊影兒。{\pangpi{趣語。}}因此,拚着死踅得進來。」金蓮道:「硶說嘴的,你且禁聲。墻有風,壁有耳,這裡說話不當穩便。」說未畢,窓縫裡隱隱望見小玉手拏一幅白絹,漸漸走近屋裡來,又忽地轉去了。金蓮忖道:「這恠小丫頭,要進房卻又跑轉去,定是忘記甚東西。」知道他要再來,慌教陳敬濟:「你索去休,這事不濟了。」敬濟沒奈何,一溜烟出去了。{\meipi{驚散得悠然有致。}}果然,小玉因月娘教金蓮描畫副裙拖送人,沒曾拏得花樣,因此又跑轉去。這也是金蓮造化,不該出醜。待的小玉拏了花樣進門,敬濟已跑去久了。金蓮接着絹兒,尚兀是手顫哩。

話分兩頭。再表西門慶和應伯爵、常峙節,三人吃的酩酊,方纔起身。伯爵再四留不住,忙跪着告道:「莫不哥還恠我那句話麼?可知道留不住哩。」西門慶笑道:「恠狗才,誰記着你話來!」伯爵便取個大甌兒,滿滿斟了一甌遞上來,西門慶接過吃了。常峙節又把些細果供上來,西門慶也吃了,便謝伯爵起身。與了金釧兒一兩銀子,叫玳安又賞了歌童三錢銀子,分付:「我有酒,也着人叫你。」說畢,上轎便行,兩個小厮跟隨。伯爵叫人家收過家活,打發了歌童,騎頭口同金釧兒轎子進城來,不題。

西門慶到家,已是黃昏時分,就進李瓶兒房裡歇了。次日,李瓶兒和西門慶說:「自從養了孩子,身上只是不淨。早晨看鏡子,兀那臉皮通黃了,飲食也不想,走動卻似閃肭了腿的一般。倘或有些山高水低,丟了孩子教誰看管?」西門慶見他掉下淚來,便道:「我去請任醫官來,看你脈息,吃些丸藥,管就好了。」便叫書童寫個帖兒,去請任醫官來。書童依命去了。

西門慶自來廳上,只見應伯爵早來謝勞。西門慶謝了相擾,兩人一處坐地說話。不多時,書童通報任醫官到,西門慶慌忙出迎,和應伯爵厮見,三人依次而坐。書童遞上茶來吃了,任醫官便動問:「府上是那一位貴恙?」西門慶道:「就是第六個小妾,身子有些不好,勞老先生仔細一看。」任醫官道:「莫不就是前日得哥兒的麼?」西門慶道:「正是。不知怎麼生起病來。」任醫官道:「且待學生進去看看。」說畢,西門慶陪任醫官進到李瓶兒屋裡,就床前坐下。叫丫頭把帳兒輕輕揭開一縫,先放出李瓶兒的右手來,用帕兒包着,擱在書上。任醫官道:「且待脈息定着。」定了一囘,然後把三個指頭按在脈上,{\pangpi{妙甚。}}自家低着頭,細玩脈息,{\pangpi{畫。}}多時纔放下。李瓶兒在帳縫裡慢慢的縮了進去。不一時,又把帕兒包着左手,捧將出來,擱在書上,任醫官也如此看了。看完了,便向西門慶道:「老夫人兩手脈都看了,卻斗膽要瞧瞧氣色。」西門道:「通家朋友,但看何妨。」就教揭起帳兒。{\meipi{費了半日工夫遮掩,卻又全體露出,寫藏頭露尾情景,真令人噴飯。}}任醫官一看,只見:臉上桃花紅綻色,眉尖桺葉翠含顰。那任醫官略看了兩眼,便對西門慶說:「夫人尊顏,學生已是望見了。大約沒有甚事,還要問個病源,纔是個望、聞、問、切。」西門慶就喚奶子。只見如意兒打扮的花花哨哨走過來,向任醫官道個萬福,把李瓶兒那口燥唇乾、睡炕不穩的病症,細細說了一遍。

那任醫官即便起身,打個恭兒道:「老先生,若是這等,學生保的沒事。大凡以下人家,他形神粗鹵,氣血強旺,可以隨分下藥,就差了些,也不打緊的。{\meipi{觀此則貧賤人有病,萬萬不可服藥。}}如宅上這樣大家,夫人這樣柔弱的形軀,怎容得一毫兒差池!正是藥差指下,延禍四肢。以此望、聞、問、切,一件兒少不得的。前日,王吏部的夫人也有些病症,看來卻與夫人相似。學生診了脈,問了病源,看了氣色,心下就明白得緊。到家查了古方,叅以己見,把那熱者涼之,虛者補之,停停當當,不消三四劑藥兒,登時好了。那吏部公也感小弟得緊,不論尺頭銀兩,加礼送來。那夫人又有梯己謝意,吏部公又送學生一個匾兒,鼓樂喧天,送到家下。匾上寫着『儒醫神術』四個大字。近日,也有幾個朋友來看,說道寫的是甚麼顏體,{\pangpi{奇想。}}一個個飛得起的。況學生幼年曾讀幾行書,因為家事消乏,就去學那岐黃之術。真正那『儒醫』兩字,一發道的着哩!」{\meipi{自一叙勢要貴人作聲價,是醫生常套,忽然拈出,令人捧腹不已。}}西門慶道:「既然不妨,極是好了。不瞞老先生說,家中雖有幾房,只是這個房下,極與學生契合。學生偌大年紀,近日得了小兒,全靠他扶養,怎生差池的!全仗老先生神術,與學生用心兒調治他速好,學生恩有重報。縱是咱們武職,比不的那吏部公,{\meipi{又襯一句,更肖。}}須索也不敢怠慢。」任醫官道:「老先生這樣相處,小弟一分也不敢望謝。就是那藥本,也不敢領。」西門慶聽罷,笑將起來道:「學生也不是吃白藥的。近日有個笑話兒講得好:有一人說道:『人家貓兒若是犯了癩的病,把烏藥買來,喂他吃了就好了。』旁邊有一人問:『若是狗兒有病,還吃甚麼藥?』那人應聲道:『吃白藥,吃白藥。』可知道白藥是狗吃的哩!」那任醫官拍手大笑道:「竟不知那寫白方兒的是什麼?」又大笑一囘。任醫官道:「老先生既然這等說,學生也止求一個匾兒罷。謝儀斷然不敢,不敢。」又笑了一囘,起身,大家打恭到廳上去了。正是:

\begin{myquote}
神方得自蓬萊監,脈訣傳從少室君。\\凡為採芝騎白鶴,時緣度世訪豪門。
\end{myquote}

