\chapter*{修訂後記}
\addcontentsline{toc}{chapter}{修訂後記}
\markboth{\titlename}{修訂後記}

這次修訂的事由,原在我的意料之中,正如我在前言和凡例中所說,錯誤是必然且眞眞實實地存在的,數目也不會少。但當一份勘誤顯示在我的屏幕上時,仍讓我吃驚不小──這眞的是我從頭到尾看完的書麼?卻為何有那麼多那麼明顯易發現的錯誤被我略過了呢?這使我更加明白了「一人之力有限,眾生智慧無窮」{\kaishu(前言中語。我在前言中這麼說,其實自謙的成分佔了四五成,所謂自謙者,實是一種自負,此亦我之毛病一也)},本著有錯必糾、有錯必改的原則,我將一二百條錯誤依著勘誤一一訂正完畢,不禁沾沾自喜,舊病登時復發了,又誇下海口:「{\kaishu(這本書會)}愈來愈趨近完美。我……堅信這一點」{\kaishu(凡例第九條)},誰想一記大耳刮子轉眼便狠狠地抽在了我的臉上──第二回描寫西門慶眼中的金蓮的詞句「紅紗膝褲扣鶯花,行坐處風吹裙袴」,袴字誤作跨,其之明顯為錯,想來即使不看原書也會發生懷疑,但它就靜悄悄地躺在那兒,不離不棄,不動如山。一日之內,嘗到兩次自己豎起來的旗子被砍的味道{\kaishu(所謂立flag)},誰比我慘耶?一笑。

\begin{quotation}
\raggedleft{丁酉仲秋戊辰\rightquadmargin}
\end{quotation}
